%% Generated by Sphinx.
\def\sphinxdocclass{report}
\documentclass[a5paper,9pt,spanish]{sphinxmanual}
\ifdefined\pdfpxdimen
   \let\sphinxpxdimen\pdfpxdimen\else\newdimen\sphinxpxdimen
\fi \sphinxpxdimen=.75bp\relax
\ifdefined\pdfimageresolution
    \pdfimageresolution= \numexpr \dimexpr1in\relax/\sphinxpxdimen\relax
\fi
%% let collapsible pdf bookmarks panel have high depth per default
\PassOptionsToPackage{bookmarksdepth=5}{hyperref}

\PassOptionsToPackage{warn}{textcomp}
\usepackage[utf8]{inputenc}
\ifdefined\DeclareUnicodeCharacter
% support both utf8 and utf8x syntaxes
  \ifdefined\DeclareUnicodeCharacterAsOptional
    \def\sphinxDUC#1{\DeclareUnicodeCharacter{"#1}}
  \else
    \let\sphinxDUC\DeclareUnicodeCharacter
  \fi
  \sphinxDUC{00A0}{\nobreakspace}
  \sphinxDUC{2500}{\sphinxunichar{2500}}
  \sphinxDUC{2502}{\sphinxunichar{2502}}
  \sphinxDUC{2514}{\sphinxunichar{2514}}
  \sphinxDUC{251C}{\sphinxunichar{251C}}
  \sphinxDUC{2572}{\textbackslash}
\fi
\usepackage{cmap}
\usepackage[T1]{fontenc}
\usepackage{amsmath,amssymb,amstext}
\usepackage{babel}



\usepackage{tgtermes}
\usepackage{tgheros}
\renewcommand{\ttdefault}{txtt}



\usepackage[Sonny]{fncychap}
\ChNameVar{\Large\normalfont\sffamily}
\ChTitleVar{\Large\normalfont\sffamily}
\usepackage{sphinx}
\sphinxsetup{%
        hmargin={1cm,1cm},
        vmargin={2cm,2cm},
        marginpar=1cm
    }
\fvset{fontsize=auto}
\usepackage{geometry}


% Include hyperref last.
\usepackage{hyperref}
% Fix anchor placement for figures with captions.
\usepackage{hypcap}% it must be loaded after hyperref.
% Set up styles of URL: it should be placed after hyperref.
\urlstyle{same}


\usepackage{sphinxmessages}
\setcounter{tocdepth}{1}



\title{Curso de programación Documentation}
\date{14 de noviembre de 2022}
\release{v 0.4.7}
\author{Andrés Vázquez}
\newcommand{\sphinxlogo}{\vbox{}}
\renewcommand{\releasename}{Versión}
\makeindex
\begin{document}

\ifdefined\shorthandoff
  \ifnum\catcode`\=\string=\active\shorthandoff{=}\fi
  \ifnum\catcode`\"=\active\shorthandoff{"}\fi
\fi

\pagestyle{empty}
\sphinxmaketitle
\pagestyle{plain}
\sphinxtableofcontents
\pagestyle{normal}
\phantomsection\label{\detokenize{index::doc}}


\sphinxAtStartPar
El presente curso esta orientado a comenzar a programar desde cero usando el lenguaje Python.

\sphinxstepscope


\chapter{¿Que es programar?}
\label{\detokenize{que:que-es-programar}}\label{\detokenize{que::doc}}
\sphinxAtStartPar
Programar es escribir las \sphinxstyleemphasis{instrucciones} necesarias para que una \sphinxstyleemphasis{computadora} realice alguna tarea.
Las \sphinxstyleemphasis{instrucciones} son diferentes según el entorno donde se usan o la finalidad que se busca.
Existen muchos lenguajes de programación que reflejan esa variedad.
Excede a este manual definir estrictamente que es una \sphinxstyleemphasis{computadora}. Diremos que al nombrarla incluimos a:
\begin{itemize}
\item {} 
\sphinxAtStartPar
La clásica computadora que podemos tener en nuestra casa u oficina.

\item {} 
\sphinxAtStartPar
Una portatil como las conocidas notebooks

\item {} 
\sphinxAtStartPar
Un teléfono celular

\end{itemize}

\sphinxAtStartPar
Podemos extender lo que entendemos por computadoras agregando:
\begin{itemize}
\item {} 
\sphinxAtStartPar
Un lavarropas moderno en tanto que tienen programas variados en un mini\sphinxhyphen{}computadora interna.

\item {} 
\sphinxAtStartPar
La alarma de una casa o auto que según diferentes acciones pre\sphinxhyphen{}establecidas dispara acciones programadas (luces y bocinas).

\end{itemize}

\sphinxAtStartPar
En general muchos electrodomésticos ya incluyen computadoras y por lo tanto programas.
Estas \sphinxstyleemphasis{instrucciones} organizadas con alguna finalidad específica conforman lo que denominamos \sphinxstyleemphasis{software}.


\section{Ejemplo}
\label{\detokenize{que:ejemplo}}
\sphinxAtStartPar
La computadora interna de un lavarropas que gestiona los dispositivos y conexiones internas podría tener estas instrucciones:
\begin{enumerate}
\sphinxsetlistlabels{\arabic}{enumi}{enumii}{}{.}%
\item {} 
\sphinxAtStartPar
Abrir el conducto de agua hasta que el sensor detecte que se llegó al nivel esperado

\item {} 
\sphinxAtStartPar
\textendash{} Si no se cumple en 20 segundos mostrar en pantalla codigo de error ERR01

\item {} 
\sphinxAtStartPar
Abrir el conducto de jabón líquido 10 segundos

\item {} 
\sphinxAtStartPar
Girar a velocidad normal el tambor 20 vueltas hacia la derecha

\item {} 
\sphinxAtStartPar
Girar a velocidad normal el tambor 20 vueltas hacia la izquierda

\item {} 
\sphinxAtStartPar
Abrir el desagote del tambor mientras gira durante 1 minuto

\item {} 
\sphinxAtStartPar
\textendash{} Si el sensor detecta que todavia hay agua mostrar en pantalla codigo de error ERR02

\item {} 
\sphinxAtStartPar
Abrir el conducto de agua durante 20 segundos (para enjuague)

\item {} 
\sphinxAtStartPar
Girar a velocidad normal el tambor 20 segundos

\item {} 
\sphinxAtStartPar
Abrir el desagote del tambor

\item {} 
\sphinxAtStartPar
\textendash{} Si el sensor detecta que todavia hay agua mostrar en pantalla codigo de error ERR03

\item {} 
\sphinxAtStartPar
Girar a velocidad rápida el tambor 2 minutos hacia la derecha

\item {} 
\sphinxAtStartPar
Girar a velocidad rápida el tambor 2 minutos hacia la izquierda

\item {} 
\sphinxAtStartPar
Destrabar la puerta del tambor, trabajo terminado.

\end{enumerate}

\sphinxstepscope


\chapter{¿Que hace un programador?}
\label{\detokenize{hace:que-hace-un-programador}}\label{\detokenize{hace::doc}}
\sphinxAtStartPar
El trabajo de un programador de software es en general
\sphinxstylestrong{escribir instrucciones en archivos de texto simples} en un lenguaje de programación específico.
Un programador conoce uno o más \sphinxstyleemphasis{lenguajes de programación} y puede desempeñarse con ellos en
múltiples entornos de trabajo.
Estos entornos incluyen:
\begin{itemize}
\item {} 
\sphinxAtStartPar
Una página web

\item {} 
\sphinxAtStartPar
Una aplicación para tu celular

\item {} 
\sphinxAtStartPar
Un sistema que funciona en una computadora estándar (como Word, Google Chrome, Excel o
cualquier programa que ves en el \sphinxstyleemphasis{escritorio} de tu computadora)

\item {} 
\sphinxAtStartPar
Micro\sphinxhyphen{}sistemas que sirvan de soporte a sistemas más grandes y que \sphinxstyleemphasis{no se ven}

\end{itemize}

\sphinxAtStartPar
Acotar estos entornos donde un software se ejecuta es difícil, son múltiples.

\noindent\sphinxincludegraphics{{hola1}.png}

\sphinxstepscope


\chapter{¿Donde escribimos nuestro código?}
\label{\detokenize{donde:donde-escribimos-nuestro-codigo}}\label{\detokenize{donde::doc}}
\sphinxAtStartPar
Como dijimos programar es escribir instrucciones (las denominaremos código o \sphinxstyleemphasis{código fuente}). Para esto podríamos usar algún
software de edición de texto pero existen herramientas específicas para esta tarea.

\sphinxAtStartPar
\sphinxhref{https://ide.atom.io/}{Atom}%
\begin{footnote}[1]\sphinxAtStartFootnote
\sphinxnolinkurl{https://ide.atom.io/}
%
\end{footnote}

\noindent\sphinxincludegraphics{{hola3}.png}

\sphinxAtStartPar
\sphinxhref{https://code.visualstudio.com/}{Visual Studio}%
\begin{footnote}[2]\sphinxAtStartFootnote
\sphinxnolinkurl{https://code.visualstudio.com/}
%
\end{footnote} (de Microsoft)

\noindent\sphinxincludegraphics{{hola2}.png}

\sphinxAtStartPar
A estos entornos de trabajo se los conoce \sphinxstyleemphasis{Entornos de desarrollo Integrado} o
IDE por sus siglas en inglés (Integrated Development Environment).
Estas herramientas proveen funcionalidades que simplifican el trabajo de un programador.

\sphinxAtStartPar
Es también posible ejecutar código Python línea a línea con la consola interactiva de Python.

\noindent\sphinxincludegraphics{{hola4}.png}

\noindent\sphinxincludegraphics{{hola5}.png}

\sphinxAtStartPar
La consola interactiva nos permite escribir código Python al mismo tiempo que se ejecuta.
Es ideal para probar pequeñas porciones de código. Una vez que la cerramos, todo lo que
hemos escrito se pierde.

\sphinxstepscope


\chapter{Escribiendo nuestro primer código}
\label{\detokenize{interactivo:escribiendo-nuestro-primer-codigo}}\label{\detokenize{interactivo::doc}}
\sphinxAtStartPar
Cuando programamos parte de nuestro trabajo es tomar datos de entrada,
procesarlos y finalmente transformarlos en datos de salida que necesitamos.
Esta es una parte muy importante de nuestro trabajo como programadores. Es por
esto que todos los lenguajes de programación incluyen formas de procesar datos
de diferentes tipos.

\begin{sphinxadmonition}{note}{tipos de datos}

\sphinxAtStartPar
Como permanentemente vamos a recibir, procesar y devolver datos es muy importante
conocer cuales son y de que herramientas dispone cada uno para resolver problemas
comunes.
\end{sphinxadmonition}

\sphinxAtStartPar
En Python (y en casi todos los lenguajes de programación) se utiliza el
operador \sphinxcode{\sphinxupquote{=}} como forma de asignar un valor (el de la izquierda) a una \sphinxstyleemphasis{variable}
(a la derecha). De tal forma el código.

\begin{sphinxVerbatim}[commandchars=\\\{\}]
\PYG{n}{a} \PYG{o}{=} \PYG{l+m+mi}{1}
\end{sphinxVerbatim}

\sphinxAtStartPar
quiere decir: \sphinxstyleemphasis{asignar el valor 1 a la variable llamada} \sphinxtitleref{a}.
Si mas adelante en el código usamos \sphinxtitleref{a} nos estaremos refiriendo a su contenido: \sphinxstyleemphasis{1}.
Entonces …

\begin{sphinxVerbatim}[commandchars=\\\{\}]
\PYG{n}{b} \PYG{o}{=} \PYG{n}{a} \PYG{o}{+} \PYG{l+m+mi}{1}
\end{sphinxVerbatim}

\sphinxAtStartPar
quiere decir \sphinxstyleemphasis{asignar el valor a+1 a la variable llamada} \sphinxcode{\sphinxupquote{b}}.
En este caso \sphinxcode{\sphinxupquote{b}} será igual a 2 (1+1).


\section{¿Qué es una variable?}
\label{\detokenize{interactivo:que-es-una-variable}}
\sphinxAtStartPar
Las \sphinxstyleemphasis{variables} son los instrumentos que usan los lenguajes de programación para
almacenar datos de diferentes tipos.
Deben tener un identificador o nombre (en los ejemplos anteriores \sphinxcode{\sphinxupquote{a}} y \sphinxcode{\sphinxupquote{b}}).

\sphinxAtStartPar
Estos identificadores deben ser letras, números y el símbolo \sphinxstyleemphasis{\_} (guión bajo) con estos límites:
\begin{itemize}
\item {} 
\sphinxAtStartPar
no puede tener espacios.

\item {} 
\sphinxAtStartPar
no empezar con un número (si puede usarse despues del primer caracter).

\item {} 
\sphinxAtStartPar
no puede ser una palabra que Python ya tiene reservado para otras funciones como \sphinxcode{\sphinxupquote{if}}, \sphinxcode{\sphinxupquote{for}}, etc.

\end{itemize}

\sphinxAtStartPar
De tal forma, los siguientes identificadores de variables son válidos:
\begin{itemize}
\item {} 
\sphinxAtStartPar
\sphinxcode{\sphinxupquote{nombre}}

\item {} 
\sphinxAtStartPar
\sphinxcode{\sphinxupquote{n3}}

\item {} 
\sphinxAtStartPar
\sphinxcode{\sphinxupquote{nombre\_y\_apellido}}

\item {} 
\sphinxAtStartPar
\sphinxcode{\sphinxupquote{\_dato\_privado}}

\item {} 
\sphinxAtStartPar
\sphinxcode{\sphinxupquote{f910293}}

\end{itemize}

\sphinxAtStartPar
y los sigiuentes no son válidos:
\begin{itemize}
\item {} 
\sphinxAtStartPar
\sphinxcode{\sphinxupquote{3n}} (no se puede empezar con números)

\item {} 
\sphinxAtStartPar
\sphinxcode{\sphinxupquote{while}}  (es una palabra reservada de Python)

\item {} 
\sphinxAtStartPar
\sphinxcode{\sphinxupquote{nombre apellido}} (no se pueden usar espacios)

\end{itemize}


\subsection{¿Cómo puede saber que tipo de datos almacena una variable?}
\label{\detokenize{interactivo:como-puede-saber-que-tipo-de-datos-almacena-una-variable}}
\sphinxAtStartPar
Python incluye una herramienta llamada \sphinxcode{\sphinxupquote{type}} que informa
que tipo de datos contiene una variable dada.

\begin{sphinxVerbatim}[commandchars=\\\{\}]
\PYG{n}{a} \PYG{o}{=} \PYG{l+m+mi}{5}
\PYG{n+nb}{type}\PYG{p}{(}\PYG{n}{a}\PYG{p}{)}
\PYG{c+c1}{\PYGZsh{} devuelve}
\PYG{c+c1}{\PYGZsh{} \PYGZlt{}class \PYGZsq{}int\PYGZsq{}\PYGZgt{}}
\end{sphinxVerbatim}

\sphinxAtStartPar
En este caso, la variable \sphinxcode{\sphinxupquote{a}} es del tipo \sphinxcode{\sphinxupquote{int}} (que veremos a continuación).
\sphinxstyleemphasis{Nota: La palabra \textasciigrave{}\textasciigrave{}class\textasciigrave{}\textasciigrave{} cobrará sentido más adelante}.

\sphinxAtStartPar
Los siguientes son ejemplo de uso de los tipos básicos de datos de los que
disponemos en Python.

\sphinxstepscope


\chapter{Números enteros \textless{}int\textgreater{}}
\label{\detokenize{int:numeros-enteros-int}}\label{\detokenize{int::doc}}
\sphinxAtStartPar
Desde la consola interactiva de Python podemos usar numeros enteros.
Estos son llamados \sphinxcode{\sphinxupquote{int}} en Python (vienen de \sphinxstyleemphasis{Integer} en inglés).
Con los enteros podemos hacer operaciones de cálculo básicas en Python.

\begin{sphinxVerbatim}[commandchars=\\\{\}]
\PYG{n}{edad} \PYG{o}{=} \PYG{l+m+mi}{31}
\PYG{n}{calculo} \PYG{o}{=} \PYG{l+m+mi}{291} \PYG{o}{+} \PYG{l+m+mi}{56} \PYG{o}{\PYGZhy{}} \PYG{l+m+mi}{12}
\PYG{c+c1}{\PYGZsh{} tambien es posible mostrar (imprimir) el resultado simplemente escribiendo}
\PYG{c+c1}{\PYGZsh{} el nombre de la variable que deseamos conocer}
\PYG{n}{calculo}
\PYG{l+m+mi}{355}
\PYG{c+c1}{\PYGZsh{} fuera de la consola interactiva, es necesario usar la funcion \PYGZdq{}print\PYGZdq{} para}
\PYG{c+c1}{\PYGZsh{} mostrar el valor contenido en una variable}
\PYG{n+nb}{print}\PYG{p}{(}\PYG{n}{calculo}\PYG{p}{)}
\PYG{l+m+mi}{355}

\PYG{n}{una\PYGZus{}multiplicacion} \PYG{o}{=} \PYG{l+m+mi}{3} \PYG{o}{*} \PYG{l+m+mi}{4}
\PYG{n}{potencia} \PYG{o}{=} \PYG{l+m+mi}{2} \PYG{o}{*}\PYG{o}{*} \PYG{l+m+mi}{3}  \PYG{c+c1}{\PYGZsh{} ** hace calculos de potencias, en este caso: 2 al cubo}
\PYG{n}{division} \PYG{o}{=} \PYG{l+m+mi}{8} \PYG{o}{/} \PYG{l+m+mi}{4}

\PYG{n}{resto} \PYG{o}{=} \PYG{l+m+mi}{5} \PYG{o}{\PYGZpc{}} \PYG{l+m+mi}{2}  \PYG{c+c1}{\PYGZsh{} (calcula el resto de la division, en este caso 5/2 es 2 con resto 1)}
\end{sphinxVerbatim}

\sphinxAtStartPar
Nota: Las palabras a la izquierda del signo igual son los nombres que elegimos
para nuestras variables. Son arbitrarios y no representan nada más que un nombre
interno, no tienen un significado especial para Python.

\sphinxAtStartPar
Es tambien posible usar las variables para hacer cálculos

\begin{sphinxVerbatim}[commandchars=\\\{\}]
\PYG{n}{unidades} \PYG{o}{=} \PYG{l+m+mi}{3}
\PYG{n}{precio} \PYG{o}{=} \PYG{l+m+mi}{11}
\PYG{n}{precio\PYGZus{}final} \PYG{o}{=} \PYG{n}{unidades} \PYG{o}{*} \PYG{n}{precio}

\PYG{c+c1}{\PYGZsh{} Mostrar resultado}
\PYG{n}{precio\PYGZus{}final}
\PYG{l+m+mi}{33}
\end{sphinxVerbatim}

\begin{sphinxadmonition}{note}{variables y objetos}

\sphinxAtStartPar
En Python, todo es un \sphinxstyleemphasis{objeto}. El concepto de \sphinxstyleemphasis{objeto} lo vamos a ver en profundidad más adelante.
Por lo pronto diremos que en Python una variable es un \sphinxstyleemphasis{objeto} de un tipo específico.
Este objeto tiene propiedades (también llamados \sphinxstyleemphasis{atributos}) y funciones.
\end{sphinxadmonition}

\sphinxAtStartPar
Por ejemplo, en el código anterior la variable \sphinxstyleemphasis{unidades} (definida en la linea que dice
\sphinxcode{\sphinxupquote{unidades = 3}}) es en realidad un \sphinxstyleemphasis{objeto} de tipo \sphinxcode{\sphinxupquote{int}}.

\sphinxAtStartPar
Los objetos de tipo \sphinxcode{\sphinxupquote{int}} no tienen muchas propiedades y funciones
A mono de ejemplo, la funcion \sphinxcode{\sphinxupquote{bit\_length}} permite saber el numero
de dígitos que este número tiene en su versión binaria.

\begin{sphinxVerbatim}[commandchars=\\\{\}]
\PYG{n}{unidades} \PYG{o}{=} \PYG{l+m+mi}{3}
\PYG{n}{unidades}\PYG{o}{.}\PYG{n}{bit\PYGZus{}length}\PYG{p}{(}\PYG{p}{)}
\PYG{l+m+mi}{2} \PYG{c+c1}{\PYGZsh{} sería = 11 en binario (2 digitos binarios)}
\PYG{c+c1}{\PYGZsh{} Es tambien posible asignar el resultado de esa funcion a una variable}
\PYG{n}{bits\PYGZus{}en\PYGZus{}unidades} \PYG{o}{=} \PYG{n}{unidades}\PYG{o}{.}\PYG{n}{bit\PYGZus{}length}\PYG{p}{(}\PYG{p}{)}
\PYG{c+c1}{\PYGZsh{} ver el resultado}
\PYG{n}{bits\PYGZus{}en\PYGZus{}unidades}
\PYG{l+m+mi}{2}
\end{sphinxVerbatim}

\sphinxAtStartPar
\sphinxstylestrong{Nota importante: La forma de llamar a las propiedades y funciones de un
objeto es usando el . (punto)}.
Se hace de la forma \sphinxcode{\sphinxupquote{objeto.propiedad}} o \sphinxcode{\sphinxupquote{objeto.funcion()}}.
\sphinxstylestrong{Nótese que para llamar a las funciones son necesarios los paréntesis.}

\sphinxAtStartPar
Si tenes curiosidad por conocer todas las propiedades y funciones de un \sphinxstyleemphasis{objeto}
(del tipo que sea) podes usar la funcion \sphinxcode{\sphinxupquote{\_\_dir\_\_()}}

\begin{sphinxVerbatim}[commandchars=\\\{\}]
\PYG{n}{unidades} \PYG{o}{=} \PYG{l+m+mi}{3}
\PYG{n}{unidades}\PYG{o}{.}\PYG{n+nf+fm}{\PYGZus{}\PYGZus{}dir\PYGZus{}\PYGZus{}}\PYG{p}{(}\PYG{p}{)}  \PYG{c+c1}{\PYGZsh{} tambien puede obtenerse esta lista con dir(unidades)}

\PYG{p}{[}\PYG{l+s+s1}{\PYGZsq{}}\PYG{l+s+s1}{\PYGZus{}\PYGZus{}repr\PYGZus{}\PYGZus{}}\PYG{l+s+s1}{\PYGZsq{}}\PYG{p}{,} \PYG{l+s+s1}{\PYGZsq{}}\PYG{l+s+s1}{\PYGZus{}\PYGZus{}hash\PYGZus{}\PYGZus{}}\PYG{l+s+s1}{\PYGZsq{}}\PYG{p}{,} \PYG{l+s+s1}{\PYGZsq{}}\PYG{l+s+s1}{\PYGZus{}\PYGZus{}getattribute\PYGZus{}\PYGZus{}}\PYG{l+s+s1}{\PYGZsq{}}\PYG{p}{,} \PYG{l+s+s1}{\PYGZsq{}}\PYG{l+s+s1}{\PYGZus{}\PYGZus{}lt\PYGZus{}\PYGZus{}}\PYG{l+s+s1}{\PYGZsq{}}\PYG{p}{,} \PYG{l+s+s1}{\PYGZsq{}}\PYG{l+s+s1}{\PYGZus{}\PYGZus{}le\PYGZus{}\PYGZus{}}\PYG{l+s+s1}{\PYGZsq{}}\PYG{p}{,} \PYG{l+s+s1}{\PYGZsq{}}\PYG{l+s+s1}{\PYGZus{}\PYGZus{}eq\PYGZus{}\PYGZus{}}\PYG{l+s+s1}{\PYGZsq{}}\PYG{p}{,}
 \PYG{l+s+s1}{\PYGZsq{}}\PYG{l+s+s1}{\PYGZus{}\PYGZus{}ne\PYGZus{}\PYGZus{}}\PYG{l+s+s1}{\PYGZsq{}}\PYG{p}{,} \PYG{l+s+s1}{\PYGZsq{}}\PYG{l+s+s1}{\PYGZus{}\PYGZus{}gt\PYGZus{}\PYGZus{}}\PYG{l+s+s1}{\PYGZsq{}}\PYG{p}{,} \PYG{l+s+s1}{\PYGZsq{}}\PYG{l+s+s1}{\PYGZus{}\PYGZus{}ge\PYGZus{}\PYGZus{}}\PYG{l+s+s1}{\PYGZsq{}}\PYG{p}{,} \PYG{l+s+s1}{\PYGZsq{}}\PYG{l+s+s1}{\PYGZus{}\PYGZus{}add\PYGZus{}\PYGZus{}}\PYG{l+s+s1}{\PYGZsq{}}\PYG{p}{,} \PYG{l+s+s1}{\PYGZsq{}}\PYG{l+s+s1}{\PYGZus{}\PYGZus{}radd\PYGZus{}\PYGZus{}}\PYG{l+s+s1}{\PYGZsq{}}\PYG{p}{,} \PYG{l+s+s1}{\PYGZsq{}}\PYG{l+s+s1}{\PYGZus{}\PYGZus{}sub\PYGZus{}\PYGZus{}}\PYG{l+s+s1}{\PYGZsq{}}\PYG{p}{,} \PYG{l+s+s1}{\PYGZsq{}}\PYG{l+s+s1}{\PYGZus{}\PYGZus{}rsub\PYGZus{}\PYGZus{}}\PYG{l+s+s1}{\PYGZsq{}}\PYG{p}{,}
 \PYG{l+s+s1}{\PYGZsq{}}\PYG{l+s+s1}{\PYGZus{}\PYGZus{}mul\PYGZus{}\PYGZus{}}\PYG{l+s+s1}{\PYGZsq{}}\PYG{p}{,} \PYG{l+s+s1}{\PYGZsq{}}\PYG{l+s+s1}{\PYGZus{}\PYGZus{}rmul\PYGZus{}\PYGZus{}}\PYG{l+s+s1}{\PYGZsq{}}\PYG{p}{,} \PYG{l+s+s1}{\PYGZsq{}}\PYG{l+s+s1}{\PYGZus{}\PYGZus{}mod\PYGZus{}\PYGZus{}}\PYG{l+s+s1}{\PYGZsq{}}\PYG{p}{,} \PYG{l+s+s1}{\PYGZsq{}}\PYG{l+s+s1}{\PYGZus{}\PYGZus{}rmod\PYGZus{}\PYGZus{}}\PYG{l+s+s1}{\PYGZsq{}}\PYG{p}{,} \PYG{l+s+s1}{\PYGZsq{}}\PYG{l+s+s1}{\PYGZus{}\PYGZus{}divmod\PYGZus{}\PYGZus{}}\PYG{l+s+s1}{\PYGZsq{}}\PYG{p}{,} \PYG{l+s+s1}{\PYGZsq{}}\PYG{l+s+s1}{\PYGZus{}\PYGZus{}rdivmod\PYGZus{}\PYGZus{}}\PYG{l+s+s1}{\PYGZsq{}}\PYG{p}{,}
 \PYG{l+s+s1}{\PYGZsq{}}\PYG{l+s+s1}{\PYGZus{}\PYGZus{}pow\PYGZus{}\PYGZus{}}\PYG{l+s+s1}{\PYGZsq{}}\PYG{p}{,} \PYG{l+s+s1}{\PYGZsq{}}\PYG{l+s+s1}{\PYGZus{}\PYGZus{}rpow\PYGZus{}\PYGZus{}}\PYG{l+s+s1}{\PYGZsq{}}\PYG{p}{,} \PYG{l+s+s1}{\PYGZsq{}}\PYG{l+s+s1}{\PYGZus{}\PYGZus{}neg\PYGZus{}\PYGZus{}}\PYG{l+s+s1}{\PYGZsq{}}\PYG{p}{,} \PYG{l+s+s1}{\PYGZsq{}}\PYG{l+s+s1}{\PYGZus{}\PYGZus{}pos\PYGZus{}\PYGZus{}}\PYG{l+s+s1}{\PYGZsq{}}\PYG{p}{,} \PYG{l+s+s1}{\PYGZsq{}}\PYG{l+s+s1}{\PYGZus{}\PYGZus{}abs\PYGZus{}\PYGZus{}}\PYG{l+s+s1}{\PYGZsq{}}\PYG{p}{,} \PYG{l+s+s1}{\PYGZsq{}}\PYG{l+s+s1}{\PYGZus{}\PYGZus{}bool\PYGZus{}\PYGZus{}}\PYG{l+s+s1}{\PYGZsq{}}\PYG{p}{,} \PYG{l+s+s1}{\PYGZsq{}}\PYG{l+s+s1}{\PYGZus{}\PYGZus{}invert\PYGZus{}\PYGZus{}}\PYG{l+s+s1}{\PYGZsq{}}\PYG{p}{,}
 \PYG{l+s+s1}{\PYGZsq{}}\PYG{l+s+s1}{\PYGZus{}\PYGZus{}lshift\PYGZus{}\PYGZus{}}\PYG{l+s+s1}{\PYGZsq{}}\PYG{p}{,} \PYG{l+s+s1}{\PYGZsq{}}\PYG{l+s+s1}{\PYGZus{}\PYGZus{}rlshift\PYGZus{}\PYGZus{}}\PYG{l+s+s1}{\PYGZsq{}}\PYG{p}{,} \PYG{l+s+s1}{\PYGZsq{}}\PYG{l+s+s1}{\PYGZus{}\PYGZus{}rshift\PYGZus{}\PYGZus{}}\PYG{l+s+s1}{\PYGZsq{}}\PYG{p}{,} \PYG{l+s+s1}{\PYGZsq{}}\PYG{l+s+s1}{\PYGZus{}\PYGZus{}rrshift\PYGZus{}\PYGZus{}}\PYG{l+s+s1}{\PYGZsq{}}\PYG{p}{,} \PYG{l+s+s1}{\PYGZsq{}}\PYG{l+s+s1}{\PYGZus{}\PYGZus{}and\PYGZus{}\PYGZus{}}\PYG{l+s+s1}{\PYGZsq{}}\PYG{p}{,} \PYG{l+s+s1}{\PYGZsq{}}\PYG{l+s+s1}{\PYGZus{}\PYGZus{}rand\PYGZus{}\PYGZus{}}\PYG{l+s+s1}{\PYGZsq{}}\PYG{p}{,}
 \PYG{l+s+s1}{\PYGZsq{}}\PYG{l+s+s1}{\PYGZus{}\PYGZus{}xor\PYGZus{}\PYGZus{}}\PYG{l+s+s1}{\PYGZsq{}}\PYG{p}{,} \PYG{l+s+s1}{\PYGZsq{}}\PYG{l+s+s1}{\PYGZus{}\PYGZus{}rxor\PYGZus{}\PYGZus{}}\PYG{l+s+s1}{\PYGZsq{}}\PYG{p}{,} \PYG{l+s+s1}{\PYGZsq{}}\PYG{l+s+s1}{\PYGZus{}\PYGZus{}or\PYGZus{}\PYGZus{}}\PYG{l+s+s1}{\PYGZsq{}}\PYG{p}{,} \PYG{l+s+s1}{\PYGZsq{}}\PYG{l+s+s1}{\PYGZus{}\PYGZus{}ror\PYGZus{}\PYGZus{}}\PYG{l+s+s1}{\PYGZsq{}}\PYG{p}{,} \PYG{l+s+s1}{\PYGZsq{}}\PYG{l+s+s1}{\PYGZus{}\PYGZus{}int\PYGZus{}\PYGZus{}}\PYG{l+s+s1}{\PYGZsq{}}\PYG{p}{,} \PYG{l+s+s1}{\PYGZsq{}}\PYG{l+s+s1}{\PYGZus{}\PYGZus{}float\PYGZus{}\PYGZus{}}\PYG{l+s+s1}{\PYGZsq{}}\PYG{p}{,} \PYG{l+s+s1}{\PYGZsq{}}\PYG{l+s+s1}{\PYGZus{}\PYGZus{}floordiv\PYGZus{}\PYGZus{}}\PYG{l+s+s1}{\PYGZsq{}}\PYG{p}{,}
 \PYG{l+s+s1}{\PYGZsq{}}\PYG{l+s+s1}{\PYGZus{}\PYGZus{}rfloordiv\PYGZus{}\PYGZus{}}\PYG{l+s+s1}{\PYGZsq{}}\PYG{p}{,} \PYG{l+s+s1}{\PYGZsq{}}\PYG{l+s+s1}{\PYGZus{}\PYGZus{}truediv\PYGZus{}\PYGZus{}}\PYG{l+s+s1}{\PYGZsq{}}\PYG{p}{,} \PYG{l+s+s1}{\PYGZsq{}}\PYG{l+s+s1}{\PYGZus{}\PYGZus{}rtruediv\PYGZus{}\PYGZus{}}\PYG{l+s+s1}{\PYGZsq{}}\PYG{p}{,} \PYG{l+s+s1}{\PYGZsq{}}\PYG{l+s+s1}{\PYGZus{}\PYGZus{}index\PYGZus{}\PYGZus{}}\PYG{l+s+s1}{\PYGZsq{}}\PYG{p}{,} \PYG{l+s+s1}{\PYGZsq{}}\PYG{l+s+s1}{\PYGZus{}\PYGZus{}new\PYGZus{}\PYGZus{}}\PYG{l+s+s1}{\PYGZsq{}}\PYG{p}{,}
 \PYG{l+s+s1}{\PYGZsq{}}\PYG{l+s+s1}{conjugate}\PYG{l+s+s1}{\PYGZsq{}}\PYG{p}{,} \PYG{l+s+s1}{\PYGZsq{}}\PYG{l+s+s1}{bit\PYGZus{}length}\PYG{l+s+s1}{\PYGZsq{}}\PYG{p}{,} \PYG{l+s+s1}{\PYGZsq{}}\PYG{l+s+s1}{to\PYGZus{}bytes}\PYG{l+s+s1}{\PYGZsq{}}\PYG{p}{,} \PYG{l+s+s1}{\PYGZsq{}}\PYG{l+s+s1}{from\PYGZus{}bytes}\PYG{l+s+s1}{\PYGZsq{}}\PYG{p}{,} \PYG{l+s+s1}{\PYGZsq{}}\PYG{l+s+s1}{as\PYGZus{}integer\PYGZus{}ratio}\PYG{l+s+s1}{\PYGZsq{}}\PYG{p}{,}
 \PYG{l+s+s1}{\PYGZsq{}}\PYG{l+s+s1}{\PYGZus{}\PYGZus{}trunc\PYGZus{}\PYGZus{}}\PYG{l+s+s1}{\PYGZsq{}}\PYG{p}{,} \PYG{l+s+s1}{\PYGZsq{}}\PYG{l+s+s1}{\PYGZus{}\PYGZus{}floor\PYGZus{}\PYGZus{}}\PYG{l+s+s1}{\PYGZsq{}}\PYG{p}{,} \PYG{l+s+s1}{\PYGZsq{}}\PYG{l+s+s1}{\PYGZus{}\PYGZus{}ceil\PYGZus{}\PYGZus{}}\PYG{l+s+s1}{\PYGZsq{}}\PYG{p}{,} \PYG{l+s+s1}{\PYGZsq{}}\PYG{l+s+s1}{\PYGZus{}\PYGZus{}round\PYGZus{}\PYGZus{}}\PYG{l+s+s1}{\PYGZsq{}}\PYG{p}{,} \PYG{l+s+s1}{\PYGZsq{}}\PYG{l+s+s1}{\PYGZus{}\PYGZus{}getnewargs\PYGZus{}\PYGZus{}}\PYG{l+s+s1}{\PYGZsq{}}\PYG{p}{,} \PYG{l+s+s1}{\PYGZsq{}}\PYG{l+s+s1}{\PYGZus{}\PYGZus{}format\PYGZus{}\PYGZus{}}\PYG{l+s+s1}{\PYGZsq{}}\PYG{p}{,}
 \PYG{l+s+s1}{\PYGZsq{}}\PYG{l+s+s1}{\PYGZus{}\PYGZus{}sizeof\PYGZus{}\PYGZus{}}\PYG{l+s+s1}{\PYGZsq{}}\PYG{p}{,} \PYG{l+s+s1}{\PYGZsq{}}\PYG{l+s+s1}{real}\PYG{l+s+s1}{\PYGZsq{}}\PYG{p}{,} \PYG{l+s+s1}{\PYGZsq{}}\PYG{l+s+s1}{imag}\PYG{l+s+s1}{\PYGZsq{}}\PYG{p}{,} \PYG{l+s+s1}{\PYGZsq{}}\PYG{l+s+s1}{numerator}\PYG{l+s+s1}{\PYGZsq{}}\PYG{p}{,} \PYG{l+s+s1}{\PYGZsq{}}\PYG{l+s+s1}{denominator}\PYG{l+s+s1}{\PYGZsq{}}\PYG{p}{,} \PYG{l+s+s1}{\PYGZsq{}}\PYG{l+s+s1}{\PYGZus{}\PYGZus{}doc\PYGZus{}\PYGZus{}}\PYG{l+s+s1}{\PYGZsq{}}\PYG{p}{,} \PYG{l+s+s1}{\PYGZsq{}}\PYG{l+s+s1}{\PYGZus{}\PYGZus{}str\PYGZus{}\PYGZus{}}\PYG{l+s+s1}{\PYGZsq{}}\PYG{p}{,}
 \PYG{l+s+s1}{\PYGZsq{}}\PYG{l+s+s1}{\PYGZus{}\PYGZus{}setattr\PYGZus{}\PYGZus{}}\PYG{l+s+s1}{\PYGZsq{}}\PYG{p}{,} \PYG{l+s+s1}{\PYGZsq{}}\PYG{l+s+s1}{\PYGZus{}\PYGZus{}delattr\PYGZus{}\PYGZus{}}\PYG{l+s+s1}{\PYGZsq{}}\PYG{p}{,} \PYG{l+s+s1}{\PYGZsq{}}\PYG{l+s+s1}{\PYGZus{}\PYGZus{}init\PYGZus{}\PYGZus{}}\PYG{l+s+s1}{\PYGZsq{}}\PYG{p}{,} \PYG{l+s+s1}{\PYGZsq{}}\PYG{l+s+s1}{\PYGZus{}\PYGZus{}reduce\PYGZus{}ex\PYGZus{}\PYGZus{}}\PYG{l+s+s1}{\PYGZsq{}}\PYG{p}{,} \PYG{l+s+s1}{\PYGZsq{}}\PYG{l+s+s1}{\PYGZus{}\PYGZus{}reduce\PYGZus{}\PYGZus{}}\PYG{l+s+s1}{\PYGZsq{}}\PYG{p}{,}
 \PYG{l+s+s1}{\PYGZsq{}}\PYG{l+s+s1}{\PYGZus{}\PYGZus{}subclasshook\PYGZus{}\PYGZus{}}\PYG{l+s+s1}{\PYGZsq{}}\PYG{p}{,} \PYG{l+s+s1}{\PYGZsq{}}\PYG{l+s+s1}{\PYGZus{}\PYGZus{}init\PYGZus{}subclass\PYGZus{}\PYGZus{}}\PYG{l+s+s1}{\PYGZsq{}}\PYG{p}{,} \PYG{l+s+s1}{\PYGZsq{}}\PYG{l+s+s1}{\PYGZus{}\PYGZus{}dir\PYGZus{}\PYGZus{}}\PYG{l+s+s1}{\PYGZsq{}}\PYG{p}{,} \PYG{l+s+s1}{\PYGZsq{}}\PYG{l+s+s1}{\PYGZus{}\PYGZus{}class\PYGZus{}\PYGZus{}}\PYG{l+s+s1}{\PYGZsq{}}\PYG{p}{]}
\end{sphinxVerbatim}

\sphinxAtStartPar
No te preocupes por esa larga lista y por todos esos guiones bajos,
gradualmete iremos comprendiendo de que se tratan.


\section{Tareas}
\label{\detokenize{int:tareas}}
\sphinxAtStartPar
Calcular cuantos segundos tiene un día definiendo las variables:
\begin{itemize}
\item {} 
\sphinxAtStartPar
\sphinxcode{\sphinxupquote{segundos\_en\_minuto}}

\item {} 
\sphinxAtStartPar
\sphinxcode{\sphinxupquote{minutos\_en\_hora}}

\item {} 
\sphinxAtStartPar
\sphinxcode{\sphinxupquote{horas\_en\_dia}}

\end{itemize}

\sphinxAtStartPar
Finalmente asignar el resultado a una variable llamada \sphinxcode{\sphinxupquote{segundos\_en\_dia}}

\sphinxstepscope


\chapter{Cadenadas de caracteres o \sphinxstyleemphasis{strings} \textless{}str\textgreater{}}
\label{\detokenize{str:cadenadas-de-caracteres-o-strings-str}}\label{\detokenize{str::doc}}
\sphinxAtStartPar
Las cadenas de caracteres o \sphinxstyleemphasis{strings} son el tipo de dato para almacenar textos.
Estos son llamados \sphinxcode{\sphinxupquote{str}} en Python.

\begin{sphinxVerbatim}[commandchars=\\\{\}]
\PYG{n}{nombre} \PYG{o}{=} \PYG{l+s+s2}{\PYGZdq{}}\PYG{l+s+s2}{Juana Velez}\PYG{l+s+s2}{\PYGZdq{}}
\PYG{c+c1}{\PYGZsh{} tambien es posible mostrar (imprimir) el contenido}
\PYG{n+nb}{print}\PYG{p}{(}\PYG{n}{nombre}\PYG{p}{)}
\PYG{n}{Juana} \PYG{n}{Velez}
\PYG{n+nb}{type}\PYG{p}{(}\PYG{n}{nombre}\PYG{p}{)}
\PYG{c+c1}{\PYGZsh{} devuelve \PYGZlt{}class \PYGZsq{}str\PYGZsq{}\PYGZgt{}}
\end{sphinxVerbatim}

\sphinxAtStartPar
Nota: Como los textos suelen naturalmente tener espacios es necesario
delimitar donde empiezan y terminan con las \sphinxcode{\sphinxupquote{"}} o \sphinxcode{\sphinxupquote{\textquotesingle{}}} (comillas
dobles o simples).

\sphinxAtStartPar
Si intentamos definir una variable de tipo \sphinxcode{\sphinxupquote{str}} sin comillas vamos a
recibir un error de sintaxis.

\begin{sphinxVerbatim}[commandchars=\\\{\}]
\PYG{n}{nombre} \PYG{o}{=} \PYG{n}{Juana} \PYG{n}{Velez}
  \PYG{n}{File} \PYG{l+s+s2}{\PYGZdq{}}\PYG{l+s+s2}{\PYGZlt{}stdin\PYGZgt{}}\PYG{l+s+s2}{\PYGZdq{}}\PYG{p}{,} \PYG{n}{line} \PYG{l+m+mi}{1}
    \PYG{n}{nombre} \PYG{o}{=} \PYG{n}{Juana} \PYG{n}{Velez}
               \PYG{o}{\PYGZca{}}
\PYG{n+ne}{SyntaxError}\PYG{p}{:} \PYG{n}{invalid} \PYG{n}{syntax}
\end{sphinxVerbatim}

\sphinxAtStartPar
Con los \sphinxstyleemphasis{strings} podemos hacer también algunas operaciones en Python.
La suma en \sphinxstyleemphasis{strings} (se llama \sphinxstyleemphasis{concatenar}) es posible:

\begin{sphinxVerbatim}[commandchars=\\\{\}]
\PYG{n}{nombre} \PYG{o}{=} \PYG{l+s+s2}{\PYGZdq{}}\PYG{l+s+s2}{Juana}\PYG{l+s+s2}{\PYGZdq{}}
\PYG{n}{apellido} \PYG{o}{=} \PYG{l+s+s2}{\PYGZdq{}}\PYG{l+s+s2}{Velez}\PYG{l+s+s2}{\PYGZdq{}}

\PYG{n}{nombre\PYGZus{}completo} \PYG{o}{=} \PYG{n}{nombre} \PYG{o}{+} \PYG{l+s+s2}{\PYGZdq{}}\PYG{l+s+s2}{ }\PYG{l+s+s2}{\PYGZdq{}} \PYG{o}{+} \PYG{n}{apellido}
\end{sphinxVerbatim}

\sphinxAtStartPar
\sphinxstylestrong{Nota: esta suma incluye tres strings, dos tienen nombre y otro es un
espacio definido directamente.}

\sphinxAtStartPar
La multiplicación tambien esta definida para \sphinxstyleemphasis{strings}:

\begin{sphinxVerbatim}[commandchars=\\\{\}]
\PYG{n}{letra} \PYG{o}{=} \PYG{l+s+s2}{\PYGZdq{}}\PYG{l+s+s2}{a}\PYG{l+s+s2}{\PYGZdq{}}
\PYG{n}{letra} \PYG{o}{*} \PYG{l+m+mi}{4}
\PYG{n}{aaaa}
\end{sphinxVerbatim}

\sphinxAtStartPar
Otras funciones disponibles para los \sphinxstyleemphasis{strings}:

\begin{sphinxVerbatim}[commandchars=\\\{\}]
\PYG{n}{nombre} \PYG{o}{=} \PYG{l+s+s2}{\PYGZdq{}}\PYG{l+s+s2}{Juana Velez}\PYG{l+s+s2}{\PYGZdq{}}
\PYG{c+c1}{\PYGZsh{} funcion lower \PYGZhy{}\PYGZgt{} pasar a minúsculas}
\PYG{n}{nombre}\PYG{o}{.}\PYG{n}{lower}\PYG{p}{(}\PYG{p}{)}
\PYG{l+s+s1}{\PYGZsq{}}\PYG{l+s+s1}{juana velez}\PYG{l+s+s1}{\PYGZsq{}}
\PYG{c+c1}{\PYGZsh{} funcion upper \PYGZhy{}\PYGZgt{} pasar a mayúsculas}
\PYG{n}{nombre}\PYG{o}{.}\PYG{n}{upper}\PYG{p}{(}\PYG{p}{)}
\PYG{l+s+s1}{\PYGZsq{}}\PYG{l+s+s1}{JUANA VELEZ}\PYG{l+s+s1}{\PYGZsq{}}
\PYG{c+c1}{\PYGZsh{} funcion format \PYGZhy{}\PYGZgt{} completar las llaves dentro de un string con}
\PYG{c+c1}{\PYGZsh{} valores definidos fuera}
\PYG{n}{saludo} \PYG{o}{=} \PYG{l+s+s2}{\PYGZdq{}}\PYG{l+s+s2}{Hola, }\PYG{l+s+si}{\PYGZob{}\PYGZcb{}}\PYG{l+s+s2}{\PYGZdq{}}\PYG{o}{.}\PYG{n}{format}\PYG{p}{(}\PYG{n}{nombre}\PYG{p}{)}
\PYG{c+c1}{\PYGZsh{} otra forma de hacer los mismo (se le llama \PYGZdq{}f strings\PYGZdq{})}
\PYG{n}{saludo} \PYG{o}{=} \PYG{l+s+sa}{f}\PYG{l+s+s2}{\PYGZdq{}}\PYG{l+s+s2}{Hola, }\PYG{l+s+si}{\PYGZob{}}\PYG{n}{nombre}\PYG{l+s+si}{\PYGZcb{}}\PYG{l+s+s2}{\PYGZdq{}}
\end{sphinxVerbatim}

\sphinxAtStartPar
Los objetos de tipo \sphinxcode{\sphinxupquote{str}} tienen muchas propiedades o funciones


\section{Tareas}
\label{\detokenize{str:tareas}}
\sphinxAtStartPar
Investigar, usar y describir para que sirven las siguientes funciones para objetos
\sphinxcode{\sphinxupquote{str}} en Python.
\begin{itemize}
\item {} 
\sphinxAtStartPar
\sphinxcode{\sphinxupquote{replace}}:

\item {} 
\sphinxAtStartPar
\sphinxcode{\sphinxupquote{capitalize}}:

\item {} 
\sphinxAtStartPar
\sphinxcode{\sphinxupquote{title}}:

\item {} 
\sphinxAtStartPar
\sphinxcode{\sphinxupquote{strip}}:

\end{itemize}

\sphinxAtStartPar
Se espera un archivo de Python con estas funciones en uso como ejemplo.


\section{Algunos ejemplos de uso}
\label{\detokenize{str:algunos-ejemplos-de-uso}}
\begin{sphinxVerbatim}[commandchars=\\\{\}]
\PYG{l+s+sd}{\PYGZdq{}\PYGZdq{}\PYGZdq{}}
\PYG{l+s+sd}{Opciones para concatenar strings con variables}
\PYG{l+s+sd}{\PYGZdq{}\PYGZdq{}\PYGZdq{}}

\PYG{n}{nombre} \PYG{o}{=} \PYG{l+s+s1}{\PYGZsq{}}\PYG{l+s+s1}{Pedro}\PYG{l+s+s1}{\PYGZsq{}}
\PYG{n}{pais} \PYG{o}{=} \PYG{l+s+s1}{\PYGZsq{}}\PYG{l+s+s1}{Chile}\PYG{l+s+s1}{\PYGZsq{}}

\PYG{n+nb}{print}\PYG{p}{(}\PYG{l+s+s2}{\PYGZdq{}}\PYG{l+s+s2}{Hello world }\PYG{l+s+si}{\PYGZob{}\PYGZcb{}}\PYG{l+s+s2}{ de }\PYG{l+s+si}{\PYGZob{}\PYGZcb{}}\PYG{l+s+s2}{!}\PYG{l+s+s2}{\PYGZdq{}}\PYG{o}{.}\PYG{n}{format}\PYG{p}{(}\PYG{n}{nombre}\PYG{p}{,} \PYG{n}{pais}\PYG{p}{)}\PYG{p}{)}
\PYG{c+c1}{\PYGZsh{} Hello world Pedro de Chile!}

\PYG{c+c1}{\PYGZsh{} valores enumerados}
\PYG{n+nb}{print}\PYG{p}{(}\PYG{l+s+s2}{\PYGZdq{}}\PYG{l+s+s2}{Valores enumerados. Hello world }\PYG{l+s+si}{\PYGZob{}0\PYGZcb{}}\PYG{l+s+s2}{ de }\PYG{l+s+si}{\PYGZob{}1\PYGZcb{}}\PYG{l+s+s2}{ (}\PYG{l+s+si}{\PYGZob{}0\PYGZcb{}}\PYG{l+s+s2}{\PYGZhy{}}\PYG{l+s+si}{\PYGZob{}1\PYGZcb{}}\PYG{l+s+s2}{)!}\PYG{l+s+s2}{\PYGZdq{}}\PYG{o}{.}\PYG{n}{format}\PYG{p}{(}\PYG{n}{nombre}\PYG{p}{,} \PYG{n}{pais}\PYG{p}{)}\PYG{p}{)}
\PYG{c+c1}{\PYGZsh{} Valores enumerados. Hello world Pedro de Chile (Pedro\PYGZhy{}Chile)!}

\PYG{c+c1}{\PYGZsh{} valores con nombre}
\PYG{n+nb}{print}\PYG{p}{(}\PYG{l+s+s2}{\PYGZdq{}}\PYG{l+s+s2}{Valores con nombre. Hello world }\PYG{l+s+si}{\PYGZob{}name\PYGZcb{}}\PYG{l+s+s2}{ de }\PYG{l+s+si}{\PYGZob{}country\PYGZcb{}}\PYG{l+s+s2}{!}\PYG{l+s+s2}{\PYGZdq{}}\PYG{o}{.}\PYG{n}{format}\PYG{p}{(}\PYG{n}{name}\PYG{o}{=}\PYG{n}{nombre}\PYG{p}{,} \PYG{n}{country}\PYG{o}{=}\PYG{n}{pais}\PYG{p}{)}\PYG{p}{)}
\PYG{c+c1}{\PYGZsh{} Valores con nombre. Hello world Pedro de Chile!}

\PYG{c+c1}{\PYGZsh{} Estilo C}
\PYG{n+nb}{print}\PYG{p}{(}\PYG{l+s+s2}{\PYGZdq{}}\PYG{l+s+s2}{Estilo C. Hello world }\PYG{l+s+si}{\PYGZpc{}s}\PYG{l+s+s2}{ de }\PYG{l+s+si}{\PYGZpc{}s}\PYG{l+s+s2}{ !}\PYG{l+s+s2}{\PYGZdq{}} \PYG{o}{\PYGZpc{}} \PYG{p}{(}\PYG{n}{nombre}\PYG{p}{,} \PYG{n}{pais}\PYG{p}{)}\PYG{p}{)}
\PYG{c+c1}{\PYGZsh{} Estilo C. Hello world Pedro de Chile !}

\PYG{c+c1}{\PYGZsh{} Nueva opcion desde python 3.6}
\PYG{n+nb}{print}\PYG{p}{(}\PYG{l+s+sa}{f}\PYG{l+s+s2}{\PYGZdq{}}\PYG{l+s+s2}{Hello world }\PYG{l+s+si}{\PYGZob{}}\PYG{n}{nombre}\PYG{l+s+si}{\PYGZcb{}}\PYG{l+s+s2}{ de }\PYG{l+s+si}{\PYGZob{}}\PYG{n}{pais}\PYG{l+s+si}{\PYGZcb{}}\PYG{l+s+s2}{!}\PYG{l+s+s2}{\PYGZdq{}}\PYG{p}{)}
\PYG{c+c1}{\PYGZsh{} Hello world Pedro de Chile!}
\end{sphinxVerbatim}

\sphinxstepscope


\chapter{Recibir datos del usuario \sphinxstyleliteralintitle{\sphinxupquote{input}}}
\label{\detokenize{input:recibir-datos-del-usuario-input}}\label{\detokenize{input::doc}}
\sphinxAtStartPar
Es posible detener la ejecución de tu programa para solicitar
al usuario de nuestro programa que ingrese datos.

\begin{sphinxVerbatim}[commandchars=\\\{\}]
\PYG{n}{nombre} \PYG{o}{=} \PYG{n+nb}{input}\PYG{p}{(}\PYG{l+s+s1}{\PYGZsq{}}\PYG{l+s+s1}{Ingresa tu nombre: }\PYG{l+s+s1}{\PYGZsq{}}\PYG{p}{)}
\PYG{n}{apellido} \PYG{o}{=} \PYG{n+nb}{input}\PYG{p}{(}\PYG{l+s+s1}{\PYGZsq{}}\PYG{l+s+s1}{Ingresa tu apellido: }\PYG{l+s+s1}{\PYGZsq{}}\PYG{p}{)}
\PYG{n+nb}{print}\PYG{p}{(}\PYG{l+s+sa}{f}\PYG{l+s+s1}{\PYGZsq{}}\PYG{l+s+s1}{Hola }\PYG{l+s+si}{\PYGZob{}}\PYG{n}{nombre}\PYG{l+s+si}{\PYGZcb{}}\PYG{l+s+s1}{ }\PYG{l+s+si}{\PYGZob{}}\PYG{n}{apellido}\PYG{l+s+si}{\PYGZcb{}}\PYG{l+s+s1}{!}\PYG{l+s+s1}{\PYGZsq{}}\PYG{p}{)}
\end{sphinxVerbatim}

\sphinxAtStartPar
La función \sphinxcode{\sphinxupquote{input}} devuelve como \sphinxstyleemphasis{string} lo que el usuario ingresa.
Si necesitaras un objeto de tipo \sphinxcode{\sphinxupquote{int}} (por ejemplo para hacer cálculos)
podes hacer la transformación con \sphinxcode{\sphinxupquote{int(variable\_string)}}.

\begin{sphinxVerbatim}[commandchars=\\\{\}]
\PYG{n+nb}{print}\PYG{p}{(}\PYG{l+s+s1}{\PYGZsq{}}\PYG{l+s+s1}{POTENCIAS}\PYG{l+s+s1}{\PYGZsq{}}\PYG{p}{)}
\PYG{n}{nro} \PYG{o}{=} \PYG{n+nb}{input}\PYG{p}{(}\PYG{l+s+s1}{\PYGZsq{}}\PYG{l+s+s1}{Ingresa un numero: }\PYG{l+s+s1}{\PYGZsq{}}\PYG{p}{)}
\PYG{n}{base} \PYG{o}{=} \PYG{n+nb}{int}\PYG{p}{(}\PYG{n}{nro}\PYG{p}{)}
\PYG{n+nb}{print}\PYG{p}{(}\PYG{l+s+sa}{f}\PYG{l+s+s1}{\PYGZsq{}}\PYG{l+s+si}{\PYGZob{}}\PYG{n}{base}\PYG{l+s+si}{\PYGZcb{}}\PYG{l+s+s1}{\(\sp{\text{2}}\) = }\PYG{l+s+si}{\PYGZob{}}\PYG{n}{base}\PYG{o}{*}\PYG{o}{*}\PYG{l+m+mi}{2}\PYG{l+s+si}{\PYGZcb{}}\PYG{l+s+s1}{\PYGZsq{}}\PYG{p}{)}
\PYG{n+nb}{print}\PYG{p}{(}\PYG{l+s+sa}{f}\PYG{l+s+s1}{\PYGZsq{}}\PYG{l+s+si}{\PYGZob{}}\PYG{n}{base}\PYG{l+s+si}{\PYGZcb{}}\PYG{l+s+s1}{\(\sp{\text{3}}\) = }\PYG{l+s+si}{\PYGZob{}}\PYG{n}{base}\PYG{o}{*}\PYG{o}{*}\PYG{l+m+mi}{3}\PYG{l+s+si}{\PYGZcb{}}\PYG{l+s+s1}{\PYGZsq{}}\PYG{p}{)}
\PYG{n+nb}{print}\PYG{p}{(}\PYG{l+s+sa}{f}\PYG{l+s+s1}{\PYGZsq{}}\PYG{l+s+si}{\PYGZob{}}\PYG{n}{base}\PYG{l+s+si}{\PYGZcb{}}\PYG{l+s+s1}{\(\sp{\text{4}}\) = }\PYG{l+s+si}{\PYGZob{}}\PYG{n}{base}\PYG{o}{*}\PYG{o}{*}\PYG{l+m+mi}{4}\PYG{l+s+si}{\PYGZcb{}}\PYG{l+s+s1}{\PYGZsq{}}\PYG{p}{)}
\PYG{n+nb}{print}\PYG{p}{(}\PYG{l+s+sa}{f}\PYG{l+s+s1}{\PYGZsq{}}\PYG{l+s+si}{\PYGZob{}}\PYG{n}{base}\PYG{l+s+si}{\PYGZcb{}}\PYG{l+s+s1}{\(\sp{\text{5}}\) = }\PYG{l+s+si}{\PYGZob{}}\PYG{n}{base}\PYG{o}{*}\PYG{o}{*}\PYG{l+m+mi}{5}\PYG{l+s+si}{\PYGZcb{}}\PYG{l+s+s1}{\PYGZsq{}}\PYG{p}{)}

\PYG{l+s+sd}{\PYGZdq{}\PYGZdq{}\PYGZdq{} Ejemplo}
\PYG{l+s+sd}{POTENCIAS}
\PYG{l+s+sd}{Ingresa un numero: 7}
\PYG{l+s+sd}{7\(\sp{\text{2}}\) = 49}
\PYG{l+s+sd}{7\(\sp{\text{3}}\) = 343}
\PYG{l+s+sd}{7\(\sp{\text{4}}\) = 2401}
\PYG{l+s+sd}{7\(\sp{\text{5}}\) = 16807}
\PYG{l+s+sd}{\PYGZdq{}\PYGZdq{}\PYGZdq{}}
\end{sphinxVerbatim}


\section{Tareas}
\label{\detokenize{input:tareas}}\begin{itemize}
\item {} 
\sphinxAtStartPar
¿Qué pasa si en el último ejemplo el usuario inserta una letra en lugar
de un número? ¿Por qué?.

\item {} 
\sphinxAtStartPar
Escribir un programa que le pida al usuario que ingrese los datos
necesarios y calcule el \sphinxhref{https://es.wikipedia.org/wiki/\%C3\%8Dndice\_de\_masa\_corporal}{índice de masa corporal}%
\begin{footnote}[3]\sphinxAtStartFootnote
\sphinxnolinkurl{https://es.wikipedia.org/wiki/\%C3\%8Dndice\_de\_masa\_corporal}
%
\end{footnote}.

\end{itemize}

\sphinxstepscope


\chapter{Entorno de desarrollo: Visual Studio Code}
\label{\detokenize{vscode:entorno-de-desarrollo-visual-studio-code}}\label{\detokenize{vscode::doc}}
\sphinxAtStartPar
La consola interactivo de Python es muy util para hacer pruebas pero si
queremos pasar al siguiente nivel y escribir código fuente más complejo
es necesario contar con un entorno de desarrollo con la posibilidad de
guardar nuestro trabajo y acceder a herramientas que lo simplifiquen.

\sphinxAtStartPar
Como desarrolladores de software vamos a pasar mucho tiempo escribiendo
texto y si bien un editor de texto simple sería suficiente vamos a usar
una herramienta hecha específicamente para esto.

\sphinxAtStartPar
Hay muchos productos disponibles y todos tienen funcionalidades similares.


\section{Visual Studio Code}
\label{\detokenize{vscode:visual-studio-code}}
\noindent\sphinxincludegraphics{{vscode1}.png}

\sphinxAtStartPar
Algunas funcionalidades interesantes:
\begin{itemize}
\item {} 
\sphinxAtStartPar
Estilos y coloreo del código.

\item {} 
\sphinxAtStartPar
Autocompletado: Ayuda para completar el código que estas escribiendo.

\item {} 
\sphinxAtStartPar
Posibilidad de ejecutar y depurar de código.

\item {} 
\sphinxAtStartPar
Extensión de la funcionalidad mediante plugins desarrollados por terceros.

\item {} 
\sphinxAtStartPar
Posibilidad de integrarte con sistemas de controles de versiones de
código (que veremos más adelante).

\end{itemize}

\sphinxstepscope


\chapter{Funciones en Python}
\label{\detokenize{fn:funciones-en-python}}\label{\detokenize{fn::doc}}
\sphinxAtStartPar
En este curso hemos nombrado y usado sin definir todavia a las \sphinxstyleemphasis{funciones}.
Una funcion en Python es una porción de código que cumple una tarea determinada
y opcionalmente devuelve un resultado.

\sphinxAtStartPar
Veamos un ejemplo:

\begin{sphinxVerbatim}[commandchars=\\\{\}]
\PYG{k}{def} \PYG{n+nf}{segundos\PYGZus{}en\PYGZus{}un\PYGZus{}dia}\PYG{p}{(}\PYG{p}{)}\PYG{p}{:}
    \PYG{n}{segundos\PYGZus{}en\PYGZus{}una\PYGZus{}hora} \PYG{o}{=} \PYG{l+m+mi}{60}
    \PYG{n}{minutos\PYGZus{}en\PYGZus{}una\PYGZus{}hora} \PYG{o}{=} \PYG{l+m+mi}{60}
    \PYG{n}{horas\PYGZus{}en\PYGZus{}un\PYGZus{}dia} \PYG{o}{=} \PYG{l+m+mi}{24}
    \PYG{n}{segundos\PYGZus{}en\PYGZus{}un\PYGZus{}dia} \PYG{o}{=} \PYG{n}{segundos\PYGZus{}en\PYGZus{}una\PYGZus{}hora} \PYG{o}{*} \PYG{n}{minutos\PYGZus{}en\PYGZus{}una\PYGZus{}hora} \PYG{o}{*} \PYG{n}{horas\PYGZus{}en\PYGZus{}un\PYGZus{}dia}
    \PYG{k}{return} \PYG{n}{segundos\PYGZus{}en\PYGZus{}un\PYGZus{}dia}

\PYG{n}{segundos} \PYG{o}{=} \PYG{n}{segundos\PYGZus{}en\PYGZus{}un\PYGZus{}dia}\PYG{p}{(}\PYG{p}{)}
\PYG{n+nb}{print}\PYG{p}{(}\PYG{n}{segundos}\PYG{p}{)}
\PYG{l+m+mi}{86400}
\end{sphinxVerbatim}


\section{Anatomía de una función simple}
\label{\detokenize{fn:anatomia-de-una-funcion-simple}}\begin{itemize}
\item {} 
\sphinxAtStartPar
\sphinxcode{\sphinxupquote{def}} es una palabra reservada de Python (no la podemos usar como nombre de variable)
que usamos para indicar que estamos definiendo una función.

\item {} 
\sphinxAtStartPar
Despues de \sphinxcode{\sphinxupquote{def}} agregamos el nombre de nuestra función. Se deben cumplir las mismas
reglas que para los nombres de variables (no pueden empezar con números, no pueden tener
espacios, etc).

\item {} 
\sphinxAtStartPar
Despues del nombre de la funcion colocamos parentesis (es obligatorio). En el futuro
vamos a usarlas para agregar los que se llaman \sphinxstyleemphasis{parámetros}. Por ahora solo es importante
no olvidar agregarlos.

\item {} 
\sphinxAtStartPar
Finalmente agregamos \sphinxstyleemphasis{:} (dos puntos) para indicar que terminamos de definir el encabezado
de la función y vamos a comenzar con el código.

\item {} 
\sphinxAtStartPar
El código de la función debe estar tabulado hacia la derecha. \sphinxstylestrong{Esta es una de las grandes
diferencias que Python tiene con los demás lenguajes de programación}. El código propio de
la función comienza tabulado y termina cuando el código vuelve a la izquierda. Muchos otros
lenguajes de programación usan las llaves \sphinxcode{\sphinxupquote{\{\}}} para delimitar donde empiezan y terminan
los bloques de código.

\item {} 
\sphinxAtStartPar
\sphinxcode{\sphinxupquote{return}} se usa para indicar cual es el valor que devolverá nuestra función cuando sea
llamada. En este caso es el resultado de un cálculo. No es obligario usarla, a veces simplemente
necesitamos procesar datos sin entregar resultados.

\item {} 
\sphinxAtStartPar
Una vez definida (y terminada anulando la tabulación y volviendo a la izquierda el código),
una función se puede llamar simplemente con su nombre y los paréntesis.

\end{itemize}


\section{Funciones con parámetros}
\label{\detokenize{fn:funciones-con-parametros}}
\sphinxAtStartPar
Muchas funciones procesan datos que ya conocemos pero en muchos casos necesitamos que nuestra
función procese datos que pueden variar. Para estos casos necesitamos darle a nuestra funcion
la posibilidad de agregar valores variables llamados \sphinxstyleemphasis{parámetros}.
Estos parámetros se incluyen dentro de los paréntesis del encabezado de nuestra funcion
\sphinxstylestrong{separados por comas}.

\sphinxAtStartPar
Ejemplo de una función con parámetros.

\begin{sphinxVerbatim}[commandchars=\\\{\}]
\PYG{k}{def} \PYG{n+nf}{sumar}\PYG{p}{(}\PYG{n}{a}\PYG{p}{,} \PYG{n}{b}\PYG{p}{)}\PYG{p}{:}
    \PYG{n+nb}{print}\PYG{p}{(}\PYG{l+s+sa}{f}\PYG{l+s+s1}{\PYGZsq{}}\PYG{l+s+s1}{Se llamo a la funcion sumar con }\PYG{l+s+si}{\PYGZob{}}\PYG{n}{a}\PYG{l+s+si}{\PYGZcb{}}\PYG{l+s+s1}{ y }\PYG{l+s+si}{\PYGZob{}}\PYG{n}{b}\PYG{l+s+si}{\PYGZcb{}}\PYG{l+s+s1}{\PYGZsq{}}\PYG{p}{)}
    \PYG{k}{return} \PYG{n}{a} \PYG{o}{+} \PYG{n}{b}

\PYG{n}{resultado} \PYG{o}{=} \PYG{n}{sumar}\PYG{p}{(}\PYG{l+m+mi}{10}\PYG{p}{,} \PYG{l+m+mi}{20}\PYG{p}{)}
\PYG{n+nb}{print}\PYG{p}{(}\PYG{l+s+sa}{f}\PYG{l+s+s1}{\PYGZsq{}}\PYG{l+s+s1}{Resultado de la suma: }\PYG{l+s+si}{\PYGZob{}}\PYG{n}{resultado}\PYG{l+s+si}{\PYGZcb{}}\PYG{l+s+s1}{\PYGZsq{}}\PYG{p}{)}
\PYG{n}{Resultado} \PYG{n}{de} \PYG{n}{la} \PYG{n}{suma}\PYG{p}{:} \PYG{l+m+mi}{30}

\PYG{n}{resultado} \PYG{o}{=} \PYG{n}{sumar}\PYG{p}{(}\PYG{l+m+mi}{7}\PYG{p}{,} \PYG{l+m+mi}{27}\PYG{p}{)}
\PYG{n+nb}{print}\PYG{p}{(}\PYG{l+s+sa}{f}\PYG{l+s+s1}{\PYGZsq{}}\PYG{l+s+s1}{Resultado de la suma: }\PYG{l+s+si}{\PYGZob{}}\PYG{n}{resultado}\PYG{l+s+si}{\PYGZcb{}}\PYG{l+s+s1}{\PYGZsq{}}\PYG{p}{)}
\PYG{n}{Resultado} \PYG{n}{de} \PYG{n}{la} \PYG{n}{suma}\PYG{p}{:} \PYG{l+m+mi}{34}

\PYG{n}{resultado} \PYG{o}{=} \PYG{n}{sumar}\PYG{p}{(}\PYG{l+s+s2}{\PYGZdq{}}\PYG{l+s+s2}{10}\PYG{l+s+s2}{\PYGZdq{}}\PYG{p}{,} \PYG{l+s+s2}{\PYGZdq{}}\PYG{l+s+s2}{20}\PYG{l+s+s2}{\PYGZdq{}}\PYG{p}{)}
\PYG{n+nb}{print}\PYG{p}{(}\PYG{l+s+sa}{f}\PYG{l+s+s1}{\PYGZsq{}}\PYG{l+s+s1}{Resultado de la suma: }\PYG{l+s+si}{\PYGZob{}}\PYG{n}{resultado}\PYG{l+s+si}{\PYGZcb{}}\PYG{l+s+s1}{\PYGZsq{}}\PYG{p}{)}
\PYG{n}{Resultado} \PYG{n}{de} \PYG{n}{la} \PYG{n}{suma}\PYG{p}{:} \PYG{l+m+mi}{1020}
\PYG{c+c1}{\PYGZsh{} Para pensar: ¿Que sucedió?)}

\PYG{c+c1}{\PYGZsh{} ¿Qué pasa si no completamos todos los parámetros esperados?}
\PYG{n}{sumar}\PYG{p}{(}\PYG{l+m+mi}{10}\PYG{p}{)}

\PYG{n}{Traceback} \PYG{p}{(}\PYG{n}{most} \PYG{n}{recent} \PYG{n}{call} \PYG{n}{last}\PYG{p}{)}\PYG{p}{:}
\PYG{n}{File} \PYG{l+s+s2}{\PYGZdq{}}\PYG{l+s+s2}{\PYGZlt{}stdin\PYGZgt{}}\PYG{l+s+s2}{\PYGZdq{}}\PYG{p}{,} \PYG{n}{line} \PYG{l+m+mi}{1}\PYG{p}{,} \PYG{o+ow}{in} \PYG{o}{\PYGZlt{}}\PYG{n}{module}\PYG{o}{\PYGZgt{}}
\PYG{n+ne}{TypeError}\PYG{p}{:} \PYG{n}{sumar}\PYG{p}{(}\PYG{p}{)} \PYG{n}{missing} \PYG{l+m+mi}{1} \PYG{n}{required} \PYG{n}{positional} \PYG{n}{argument}\PYG{p}{:} \PYG{l+s+s1}{\PYGZsq{}}\PYG{l+s+s1}{b}\PYG{l+s+s1}{\PYGZsq{}}

\PYG{c+c1}{\PYGZsh{} Obtenemos un error. Los parámetros asi como estan definidos son obligatorios.}
\end{sphinxVerbatim}

\sphinxAtStartPar
Nótese que dentros del código de la función los parametros son variables disponibles
para usar libremente. Fuera de ella (por ejemplo tratar de usar \sphinxcode{\sphinxupquote{a}} fuera de la
funcion \sphinxcode{\sphinxupquote{suma}}) no estan definidas y darán error. El alcance y validez de \sphinxcode{\sphinxupquote{a}}
es solo dentro de la función que la declaró como parámetro.


\section{Funciones con parámetros opcionales}
\label{\detokenize{fn:funciones-con-parametros-opcionales}}
\sphinxAtStartPar
En algunos casos no necesitamos que todos los parámetros sean obligatorios.
Muchas veces hay valores que son de uso más frecuente y el usuario no querrá
definirlos cada vez que llama a la función.
Para esto existen parámetros \sphinxstyleemphasis{opcionales}. Simplemente colocamos cual es el
valor predeterminado para los parámetros en el encabezado de la función y
es suficiente.
Tomemos el ejemplo de la funcion \sphinxstyleemphasis{potencia} y asumamos que en general los
usuarios querrán elevar números al cuadrado.

\begin{sphinxVerbatim}[commandchars=\\\{\}]
\PYG{k}{def} \PYG{n+nf}{potencia}\PYG{p}{(}\PYG{n}{a}\PYG{p}{,} \PYG{n}{b}\PYG{o}{=}\PYG{l+m+mi}{2}\PYG{p}{)}\PYG{p}{:}
    \PYG{n+nb}{print}\PYG{p}{(}\PYG{l+s+sa}{f}\PYG{l+s+s1}{\PYGZsq{}}\PYG{l+s+s1}{Se llamo a la funcion potencia con }\PYG{l+s+si}{\PYGZob{}}\PYG{n}{a}\PYG{l+s+si}{\PYGZcb{}}\PYG{l+s+s1}{ y }\PYG{l+s+si}{\PYGZob{}}\PYG{n}{b}\PYG{l+s+si}{\PYGZcb{}}\PYG{l+s+s1}{\PYGZsq{}}\PYG{p}{)}
    \PYG{k}{return} \PYG{n}{a} \PYG{o}{*}\PYG{o}{*} \PYG{n}{b}

\PYG{n}{resultado} \PYG{o}{=} \PYG{n}{potencia}\PYG{p}{(}\PYG{l+m+mi}{3}\PYG{p}{)}
\PYG{n+nb}{print}\PYG{p}{(}\PYG{l+s+sa}{f}\PYG{l+s+s1}{\PYGZsq{}}\PYG{l+s+s1}{Resultado de la potencia 3 ** 2: }\PYG{l+s+si}{\PYGZob{}}\PYG{n}{resultado}\PYG{l+s+si}{\PYGZcb{}}\PYG{l+s+s1}{\PYGZsq{}}\PYG{p}{)}

\PYG{n}{resultado} \PYG{o}{=} \PYG{n}{potencia}\PYG{p}{(}\PYG{l+m+mi}{3}\PYG{p}{,} \PYG{l+m+mi}{3}\PYG{p}{)}
\PYG{n+nb}{print}\PYG{p}{(}\PYG{l+s+sa}{f}\PYG{l+s+s1}{\PYGZsq{}}\PYG{l+s+s1}{Resultado de la potencia 3 ** 3: }\PYG{l+s+si}{\PYGZob{}}\PYG{n}{resultado}\PYG{l+s+si}{\PYGZcb{}}\PYG{l+s+s1}{\PYGZsq{}}\PYG{p}{)}

\PYG{c+c1}{\PYGZsh{} Es tambien posible cambiar el orden de los parametros al}
\PYG{c+c1}{\PYGZsh{} llamar a la función usando su nombre.}
\PYG{c+c1}{\PYGZsh{} Las siguientes opciones devolverán el mismo resultado.}
\PYG{n}{resultado} \PYG{o}{=} \PYG{n}{potencia}\PYG{p}{(}\PYG{l+m+mi}{3}\PYG{p}{,} \PYG{l+m+mi}{4}\PYG{p}{)}
\PYG{n}{resultado} \PYG{o}{=} \PYG{n}{potencia}\PYG{p}{(}\PYG{n}{a}\PYG{o}{=}\PYG{l+m+mi}{3}\PYG{p}{,} \PYG{n}{b}\PYG{o}{=}\PYG{l+m+mi}{4}\PYG{p}{)}
\PYG{n}{resultado} \PYG{o}{=} \PYG{n}{potencia}\PYG{p}{(}\PYG{n}{b}\PYG{o}{=}\PYG{l+m+mi}{4}\PYG{p}{,} \PYG{n}{a}\PYG{o}{=}\PYG{l+m+mi}{3}\PYG{p}{)}
\end{sphinxVerbatim}


\section{Tareas}
\label{\detokenize{fn:tareas}}\begin{itemize}
\item {} 
\sphinxAtStartPar
Un mago se para frente a su audiencia y les dice:
\begin{itemize}
\item {} 
\sphinxAtStartPar
\sphinxstylestrong{Piensen un numero cualquiera}

\item {} 
\sphinxAtStartPar
Ahora súmenle 2

\item {} 
\sphinxAtStartPar
Multipliquen el resultado por 2

\item {} 
\sphinxAtStartPar
Resten al resultado el numero que pensaron al inicio

\item {} 
\sphinxAtStartPar
Sumen al resultado 8

\item {} 
\sphinxAtStartPar
Resten otra vez el numero pensado al principio.

\end{itemize}

\sphinxAtStartPar
Finalmente les pregunta a todos el resultado final y \sphinxstylestrong{aplauden}

\sphinxAtStartPar
Tarea:
Crear una funcion que haga paso a paso los calculos del mago para llegar al resultado final.
Ejecutar la función para 5 numeros distintos y observar los resultados para entender que paso (editado)

\item {} 
\sphinxAtStartPar
Crear una funcion que calcule y devuelva la superficie de un rectangulo
dados (como parámetros) los valores de sus lados. Usar esta funcion con
valores ingresados por el usuario con \sphinxcode{\sphinxupquote{input}}.

\item {} 
\sphinxAtStartPar
Crear una funcion que dados un nombre y un apellido imprima en pantalla
\sphinxstyleemphasis{«Hola NOMBRE APELLIDO!»}. La función no debe devolver ningun valor.

\item {} 
\sphinxAtStartPar
Crear una función que dado un texto pasado como parámetro, devuelva
el mismo texto pero con todas las vocales cambiadas por un asterisco.

\item {} 
\sphinxAtStartPar
Crear una funcion que dada una temperatura en grados Celsius devuelva
el equivalente en grados Fahrenheit. Usar esta funcion con
valores ingresados por el usuario con \sphinxcode{\sphinxupquote{input}}.

\end{itemize}

\sphinxAtStartPar
\sphinxstylestrong{En todos los casos usar la función para asegurarse
que funciona como es esperado.}


\section{Algunos ejemplos de uso}
\label{\detokenize{fn:algunos-ejemplos-de-uso}}
\begin{sphinxVerbatim}[commandchars=\\\{\}]

\PYG{k}{def} \PYG{n+nf}{sumar}\PYG{p}{(}\PYG{n}{a}\PYG{p}{,} \PYG{n}{b}\PYG{p}{)}\PYG{p}{:}
    \PYG{n+nb}{print}\PYG{p}{(}\PYG{l+s+sa}{f}\PYG{l+s+s1}{\PYGZsq{}}\PYG{l+s+s1}{Se llamo a la funcion sumar con }\PYG{l+s+si}{\PYGZob{}}\PYG{n}{a}\PYG{l+s+si}{\PYGZcb{}}\PYG{l+s+s1}{ y }\PYG{l+s+si}{\PYGZob{}}\PYG{n}{b}\PYG{l+s+si}{\PYGZcb{}}\PYG{l+s+s1}{\PYGZsq{}}\PYG{p}{)}
    \PYG{k}{return} \PYG{n}{a} \PYG{o}{+} \PYG{n}{b}

\PYG{n}{resultado} \PYG{o}{=} \PYG{n}{sumar}\PYG{p}{(}\PYG{l+m+mi}{10}\PYG{p}{,} \PYG{l+m+mi}{20}\PYG{p}{)}
\PYG{n+nb}{print}\PYG{p}{(}\PYG{l+s+sa}{f}\PYG{l+s+s1}{\PYGZsq{}}\PYG{l+s+s1}{Resultado de la suma: }\PYG{l+s+si}{\PYGZob{}}\PYG{n}{resultado}\PYG{l+s+si}{\PYGZcb{}}\PYG{l+s+s1}{\PYGZsq{}}\PYG{p}{)}
\end{sphinxVerbatim}

\begin{sphinxVerbatim}[commandchars=\\\{\}]

\PYG{k}{def} \PYG{n+nf}{potencia}\PYG{p}{(}\PYG{n}{a}\PYG{p}{,} \PYG{n}{b}\PYG{o}{=}\PYG{l+m+mi}{2}\PYG{p}{)}\PYG{p}{:}
    \PYG{n+nb}{print}\PYG{p}{(}\PYG{l+s+sa}{f}\PYG{l+s+s1}{\PYGZsq{}}\PYG{l+s+s1}{Se llamo a la funcion potencia con }\PYG{l+s+si}{\PYGZob{}}\PYG{n}{a}\PYG{l+s+si}{\PYGZcb{}}\PYG{l+s+s1}{ y }\PYG{l+s+si}{\PYGZob{}}\PYG{n}{b}\PYG{l+s+si}{\PYGZcb{}}\PYG{l+s+s1}{\PYGZsq{}}\PYG{p}{)}
    \PYG{k}{return} \PYG{n}{a} \PYG{o}{*}\PYG{o}{*} \PYG{n}{b}

\PYG{n}{resultado} \PYG{o}{=} \PYG{n}{potencia}\PYG{p}{(}\PYG{l+m+mi}{3}\PYG{p}{)}
\PYG{n+nb}{print}\PYG{p}{(}\PYG{l+s+sa}{f}\PYG{l+s+s1}{\PYGZsq{}}\PYG{l+s+s1}{Resultado de la potencia 3 ** 2: }\PYG{l+s+si}{\PYGZob{}}\PYG{n}{resultado}\PYG{l+s+si}{\PYGZcb{}}\PYG{l+s+s1}{\PYGZsq{}}\PYG{p}{)}

\PYG{n}{resultado} \PYG{o}{=} \PYG{n}{potencia}\PYG{p}{(}\PYG{l+m+mi}{3}\PYG{p}{,} \PYG{l+m+mi}{3}\PYG{p}{)}
\PYG{n+nb}{print}\PYG{p}{(}\PYG{l+s+sa}{f}\PYG{l+s+s1}{\PYGZsq{}}\PYG{l+s+s1}{Resultado de la potencia 3 ** 3: }\PYG{l+s+si}{\PYGZob{}}\PYG{n}{resultado}\PYG{l+s+si}{\PYGZcb{}}\PYG{l+s+s1}{\PYGZsq{}}\PYG{p}{)}
\end{sphinxVerbatim}

\begin{sphinxVerbatim}[commandchars=\\\{\}]

\PYG{k}{def} \PYG{n+nf}{raiz}\PYG{p}{(}\PYG{n}{numero}\PYG{p}{,} \PYG{n}{raiz}\PYG{o}{=}\PYG{l+m+mi}{2}\PYG{p}{)}\PYG{p}{:}
    \PYG{n+nb}{print}\PYG{p}{(}\PYG{l+s+sa}{f}\PYG{l+s+s1}{\PYGZsq{}}\PYG{l+s+s1}{Se llamo a la funcion raiz con }\PYG{l+s+si}{\PYGZob{}}\PYG{n}{numero}\PYG{l+s+si}{\PYGZcb{}}\PYG{l+s+s1}{ y }\PYG{l+s+si}{\PYGZob{}}\PYG{n}{raiz}\PYG{l+s+si}{\PYGZcb{}}\PYG{l+s+s1}{\PYGZsq{}}\PYG{p}{)}
    \PYG{k}{return} \PYG{n}{numero} \PYG{o}{*}\PYG{o}{*} \PYG{p}{(}\PYG{l+m+mi}{1}\PYG{o}{/}\PYG{n}{raiz}\PYG{p}{)}

\PYG{n}{resultado} \PYG{o}{=} \PYG{n}{raiz}\PYG{p}{(}\PYG{n}{numero}\PYG{o}{=}\PYG{l+m+mi}{64}\PYG{p}{)}
\PYG{n+nb}{print}\PYG{p}{(}\PYG{l+s+sa}{f}\PYG{l+s+s1}{\PYGZsq{}}\PYG{l+s+s1}{Resultado de la raiz cuadrada de 64: }\PYG{l+s+si}{\PYGZob{}}\PYG{n}{resultado}\PYG{l+s+si}{\PYGZcb{}}\PYG{l+s+s1}{\PYGZsq{}}\PYG{p}{)}

\PYG{n}{resultado} \PYG{o}{=} \PYG{n}{raiz}\PYG{p}{(}\PYG{n}{raiz}\PYG{o}{=}\PYG{l+m+mi}{3}\PYG{p}{,} \PYG{n}{numero}\PYG{o}{=}\PYG{l+m+mi}{27}\PYG{p}{)}
\PYG{n+nb}{print}\PYG{p}{(}\PYG{l+s+sa}{f}\PYG{l+s+s1}{\PYGZsq{}}\PYG{l+s+s1}{Resultado de la raiz cúbica de 27: }\PYG{l+s+si}{\PYGZob{}}\PYG{n}{resultado}\PYG{l+s+si}{\PYGZcb{}}\PYG{l+s+s1}{\PYGZsq{}}\PYG{p}{)}
\end{sphinxVerbatim}

\begin{sphinxVerbatim}[commandchars=\\\{\}]

\PYG{k}{def} \PYG{n+nf}{mi\PYGZus{}function}\PYG{p}{(}\PYG{o}{*}\PYG{n}{args}\PYG{p}{,} \PYG{o}{*}\PYG{o}{*}\PYG{n}{kwargs}\PYG{p}{)}\PYG{p}{:}
    \PYG{n+nb}{print}\PYG{p}{(}\PYG{l+s+sa}{f}\PYG{l+s+s1}{\PYGZsq{}}\PYG{l+s+s1}{Argumentos sin nombre: }\PYG{l+s+si}{\PYGZob{}}\PYG{n}{args}\PYG{l+s+si}{\PYGZcb{}}\PYG{l+s+s1}{\PYGZsq{}}\PYG{p}{)}
    \PYG{n+nb}{print}\PYG{p}{(}\PYG{l+s+sa}{f}\PYG{l+s+s1}{\PYGZsq{}}\PYG{l+s+s1}{Argumentos con nombre: }\PYG{l+s+si}{\PYGZob{}}\PYG{n}{kwargs}\PYG{l+s+si}{\PYGZcb{}}\PYG{l+s+s1}{\PYGZsq{}}\PYG{p}{)}


\PYG{n}{mi\PYGZus{}function}\PYG{p}{(}\PYG{l+m+mi}{3}\PYG{p}{,} \PYG{l+s+s2}{\PYGZdq{}}\PYG{l+s+s2}{algo}\PYG{l+s+s2}{\PYGZdq{}}\PYG{p}{,} \PYG{n}{nombre}\PYG{o}{=}\PYG{l+s+s1}{\PYGZsq{}}\PYG{l+s+s1}{Luis}\PYG{l+s+s1}{\PYGZsq{}}\PYG{p}{,} \PYG{n}{edad}\PYG{o}{=}\PYG{l+m+mi}{91}\PYG{p}{)}

\PYG{c+c1}{\PYGZsh{} Argumentos sin nombre: (3, \PYGZsq{}algo\PYGZsq{})}
\PYG{c+c1}{\PYGZsh{} Argumentos con nombre: \PYGZob{}\PYGZsq{}nombre\PYGZsq{}: \PYGZsq{}Luis\PYGZsq{}, \PYGZsq{}edad\PYGZsq{}: 91\PYGZcb{}}
\end{sphinxVerbatim}

\sphinxstepscope


\chapter{Funciones incluidas en Python}
\label{\detokenize{builtins:funciones-incluidas-en-python}}\label{\detokenize{builtins::doc}}
\sphinxAtStartPar
Así como nosotros definimos nuestras propias funciones, Python incluye
algunas funciones por defecto. Ya hemos usado una de ella en nuestros códigos de
ejemplo: \sphinxcode{\sphinxupquote{print}}.

\sphinxAtStartPar
A estas funciones incorporadas y disponibles en Python se las conoce como \sphinxstyleemphasis{built\sphinxhyphen{}ins}.

\sphinxAtStartPar
La función \sphinxcode{\sphinxupquote{print}} simplemente \sphinxstyleemphasis{imprime} en nuestra terminal cualquier
valor que se le pase como parámetro.

\sphinxAtStartPar
Veamos algunos ejemplos:

\begin{sphinxVerbatim}[commandchars=\\\{\}]
\PYG{n+nb}{print}\PYG{p}{(}\PYG{l+s+s1}{\PYGZsq{}}\PYG{l+s+s1}{Hola}\PYG{l+s+s1}{\PYGZsq{}}\PYG{p}{)}
\PYG{n}{nombre} \PYG{o}{=} \PYG{l+s+s1}{\PYGZsq{}}\PYG{l+s+s1}{Juan}\PYG{l+s+s1}{\PYGZsq{}}
\PYG{n+nb}{print}\PYG{p}{(}\PYG{n}{nombre}\PYG{p}{)}
\end{sphinxVerbatim}

\sphinxAtStartPar
Pero \sphinxcode{\sphinxupquote{print}} no es la única función disponible, hay muchas.

\sphinxAtStartPar
Nota: La lista de todas las funciones built\sphinxhyphen{}ins de Python está disponible
\sphinxhref{https://docs.python.org/3/library/functions.html}{aquí}%
\begin{footnote}[4]\sphinxAtStartFootnote
\sphinxnolinkurl{https://docs.python.org/3/library/functions.html}
%
\end{footnote}.

\sphinxAtStartPar
Algunos ejemplos:

\begin{sphinxVerbatim}[commandchars=\\\{\}]
\PYG{c+c1}{\PYGZsh{} abs \PYGZhy{}\PYGZgt{} obtener el valor abosoluto de un numero}
\PYG{n+nb}{abs}\PYG{p}{(}\PYG{o}{\PYGZhy{}}\PYG{l+m+mi}{3}\PYG{p}{)}
\PYG{l+m+mi}{3}
\PYG{c+c1}{\PYGZsh{} len \PYGZhy{}\PYGZgt{} obtener el largo de un objeto. No disponible para cualquier}
\PYG{c+c1}{\PYGZsh{} objeto. En el caso de los strings, cuenta las letras}
\PYG{n+nb}{len}\PYG{p}{(}\PYG{l+s+s1}{\PYGZsq{}}\PYG{l+s+s1}{hola}\PYG{l+s+s1}{\PYGZsq{}}\PYG{p}{)}
\PYG{l+m+mi}{4}
\PYG{n}{nombre} \PYG{o}{=} \PYG{l+s+s1}{\PYGZsq{}}\PYG{l+s+s1}{Victor}\PYG{l+s+s1}{\PYGZsq{}}
\PYG{n+nb}{len}\PYG{p}{(}\PYG{n}{victor}\PYG{p}{)}
\PYG{l+m+mi}{6}
\PYG{c+c1}{\PYGZsh{} type \PYGZhy{}\PYGZgt{} devuelve el tipo de un objeto}
\PYG{n+nb}{type}\PYG{p}{(}\PYG{l+s+s1}{\PYGZsq{}}\PYG{l+s+s1}{hola}\PYG{l+s+s1}{\PYGZsq{}}\PYG{p}{)}
\PYG{c+c1}{\PYGZsh{} devuelve \PYGZlt{}class \PYGZsq{}str\PYGZsq{}\PYGZgt{}}
\PYG{n+nb}{type}\PYG{p}{(}\PYG{n}{nombre}\PYG{p}{)}
\PYG{c+c1}{\PYGZsh{} devuelve \PYGZlt{}class \PYGZsq{}str\PYGZsq{}\PYGZgt{}}
\PYG{c+c1}{\PYGZsh{} max y min \PYGZhy{}\PYGZgt{} devuelven el elemento maximo y minimo de una lista de elementos}
\PYG{n+nb}{max}\PYG{p}{(}\PYG{l+m+mi}{3}\PYG{p}{,} \PYG{l+m+mi}{5}\PYG{p}{,} \PYG{l+m+mi}{15}\PYG{p}{,} \PYG{l+m+mi}{1}\PYG{p}{)}
\PYG{l+m+mi}{15}
\PYG{n+nb}{min}\PYG{p}{(}\PYG{l+m+mi}{3}\PYG{p}{,} \PYG{l+m+mi}{5}\PYG{p}{,} \PYG{l+m+mi}{15}\PYG{p}{,} \PYG{l+m+mi}{1}\PYG{p}{)}
\PYG{l+m+mi}{1}
\PYG{c+c1}{\PYGZsh{} en caso de strings, max y min resuelven ordenando alfabéticamente.}
\PYG{n+nb}{max}\PYG{p}{(}\PYG{l+s+s2}{\PYGZdq{}}\PYG{l+s+s2}{hola}\PYG{l+s+s2}{\PYGZdq{}}\PYG{p}{,} \PYG{l+s+s2}{\PYGZdq{}}\PYG{l+s+s2}{chau}\PYG{l+s+s2}{\PYGZdq{}}\PYG{p}{)}
\PYG{l+s+s1}{\PYGZsq{}}\PYG{l+s+s1}{hola}\PYG{l+s+s1}{\PYGZsq{}}
\end{sphinxVerbatim}


\section{Tareas}
\label{\detokenize{builtins:tareas}}\begin{itemize}
\item {} 
\sphinxAtStartPar
Escribir una función que dadas tres palabras devuelva el
largo total de todas ellas juntas. Por ejemplo (si la función se llamara
\sphinxcode{\sphinxupquote{largo\_total}}) la llamada \sphinxcode{\sphinxupquote{largo\_total(\textquotesingle{}hola\textquotesingle{}, \textquotesingle{}chau\textquotesingle{}, \textquotesingle{}tercera\textquotesingle{})}}
debe devolver 15.

\item {} 
\sphinxAtStartPar
Escribir una función que dados dos números devuelva el valor
absoluto del menor de ambos.

\end{itemize}

\sphinxAtStartPar
\sphinxstylestrong{En todos los casos usar la función para asegurarse
que funciona como es esperado.}

\sphinxstepscope


\chapter{Booleanos \textless{}bool\textgreater{} + operadores de comparación + \sphinxstyleliteralintitle{\sphinxupquote{ìf}}}
\label{\detokenize{bool:booleanos-bool-operadores-de-comparacion-if}}\label{\detokenize{bool::doc}}
\sphinxAtStartPar
El tipo de datos \sphinxcode{\sphinxupquote{\textless{}bool\textgreater{}}} (nombrados así en memoria de matemático inglés
George Boole) solo usa dos valores: \sphinxcode{\sphinxupquote{True}} y \sphinxcode{\sphinxupquote{False}} (\sphinxstyleemphasis{verdadero} y \sphinxstyleemphasis{falso}).
Es el tipo de datos más simple y uno de los más usados.
Se puede asignar directamente o puede ser el resultado de una comparación.
Los operadores de comparación en Python son:
\begin{itemize}
\item {} 
\sphinxAtStartPar
\sphinxcode{\sphinxupquote{==}}: Compara si dos variables valen lo mismo.

\item {} 
\sphinxAtStartPar
\sphinxcode{\sphinxupquote{!=}}: Compara si dos variables son distintas. En general en signo \sphinxcode{\sphinxupquote{!}}
se usa para negar. Podemos pensar a este operador como \sphinxstyleemphasis{no igual}.

\item {} 
\sphinxAtStartPar
\sphinxcode{\sphinxupquote{\textgreater{}}} y \sphinxcode{\sphinxupquote{\textless{}}}: Compara si una variable es mayor o menor que otra.

\end{itemize}

\sphinxAtStartPar
Para los números estas comparaciones son fáciles de intuir pero pueden significar
diferentes cosas para diferentes tipos de datos. Como vimos antes en las
funciones \sphinxcode{\sphinxupquote{max}} y \sphinxcode{\sphinxupquote{min}} usadas para \sphinxstyleemphasis{strings}, las comparaciones son alfabeticas.
Es por esto que en Python podemos decir que “alpha” \textless{} “beta” es \sphinxcode{\sphinxupquote{True}}.

\sphinxAtStartPar
Veamos algunos ejemplo con código.

\begin{sphinxVerbatim}[commandchars=\\\{\}]
\PYG{c+c1}{\PYGZsh{} \PYGZdq{}a\PYGZdq{} es igual a uno.}
\PYG{n}{a} \PYG{o}{=} \PYG{l+m+mi}{1}
\PYG{n}{b} \PYG{o}{=} \PYG{l+m+mi}{2}

\PYG{c+c1}{\PYGZsh{} ¿Es \PYGZdq{}a\PYGZdq{} igual a 1?}
\PYG{n}{a} \PYG{o}{==} \PYG{l+m+mi}{1}
\PYG{c+c1}{\PYGZsh{} Si, es verdadero}
\PYG{k+kc}{True}

\PYG{c+c1}{\PYGZsh{} ¿Es \PYGZdq{}a\PYGZdq{} mayor que \PYGZdq{}b\PYGZdq{}?}
\PYG{n}{a} \PYG{o}{\PYGZgt{}} \PYG{n}{b}
\PYG{c+c1}{\PYGZsh{} No, es falso}
\PYG{k+kc}{False}

\PYG{c+c1}{\PYGZsh{} ¿Es \PYGZdq{}a\PYGZdq{} distinto de \PYGZdq{}b\PYGZdq{}?}
\PYG{n}{a} \PYG{o}{!=} \PYG{n}{b}
\PYG{c+c1}{\PYGZsh{} Si, es verdadero}
\PYG{k+kc}{True}
\end{sphinxVerbatim}


\section{Control de flujo \sphinxhyphen{}\textgreater{} \sphinxstyleliteralintitle{\sphinxupquote{if / elif / else}}}
\label{\detokenize{bool:control-de-flujo-if-elif-else}}
\sphinxAtStartPar
El flujo de un programa no tiene que ser siempre lineal.
Diferentes porciones de código pueden ejecutarse si alguna condicion esperada se cumple.
Para esto existe la sentencia \sphinxcode{\sphinxupquote{if}} (traducido, es el \sphinxstyleemphasis{si} condicional del español).

\sphinxAtStartPar
Veamos su funcionamiento con simples ejemplos:


\subsection{\sphinxstyleliteralintitle{\sphinxupquote{if}}}
\label{\detokenize{bool:if}}
\sphinxAtStartPar
Uso de \sphinxcode{\sphinxupquote{if}} solo:

\begin{sphinxVerbatim}[commandchars=\\\{\}]
\PYG{n}{a} \PYG{o}{==} \PYG{l+m+mi}{1}
\PYG{k}{if} \PYG{n}{a} \PYG{o}{\PYGZgt{}} \PYG{l+m+mi}{0}\PYG{p}{:}  \PYG{c+c1}{\PYGZsh{} nótese que la línea debe terminar en \PYGZdq{}:\PYGZdq{} como las funciones}
    \PYG{n+nb}{print}\PYG{p}{(}\PYG{l+s+s1}{\PYGZsq{}}\PYG{l+s+s1}{\PYGZdq{}}\PYG{l+s+s1}{a}\PYG{l+s+s1}{\PYGZdq{}}\PYG{l+s+s1}{ es mayor que cero}\PYG{l+s+s1}{\PYGZsq{}}\PYG{p}{)}
\end{sphinxVerbatim}

\sphinxAtStartPar
\sphinxstylestrong{Nota importante: la porción de código a ejecutaerse *si* se cumple la condición
debe estar tabulada (tal como lo hacemos en las funciones).}


\subsection{\sphinxstyleliteralintitle{\sphinxupquote{else}}}
\label{\detokenize{bool:else}}
\sphinxAtStartPar
Opcionalmente, podemos usar \sphinxcode{\sphinxupquote{else}} para capturar los casos que no cumplen la condición.

\begin{sphinxVerbatim}[commandchars=\\\{\}]
\PYG{n}{edad} \PYG{o}{=} \PYG{l+m+mi}{33}
\PYG{k}{if} \PYG{n}{edad} \PYG{o}{\PYGZgt{}}\PYG{o}{=} \PYG{l+m+mi}{18}\PYG{p}{:}
    \PYG{n+nb}{print}\PYG{p}{(}\PYG{l+s+s1}{\PYGZsq{}}\PYG{l+s+s1}{Es mayor de edad}\PYG{l+s+s1}{\PYGZsq{}}\PYG{p}{)}
    \PYG{n}{prohibir\PYGZus{}entrada} \PYG{o}{=} \PYG{k+kc}{False}
\PYG{k}{else}\PYG{p}{:}
    \PYG{n+nb}{print}\PYG{p}{(}\PYG{l+s+s1}{\PYGZsq{}}\PYG{l+s+s1}{No es mayor de edad}\PYG{l+s+s1}{\PYGZsq{}}\PYG{p}{)}
    \PYG{n}{prohibir\PYGZus{}entrada} \PYG{o}{=} \PYG{k+kc}{True}
\end{sphinxVerbatim}

\sphinxAtStartPar
\sphinxstylestrong{Nota importante: Las tabulaciones definen el inicio y el fin de las porciones
de código a ejecutar según la condición. Puede ser más de una línea.}


\subsection{\sphinxstyleliteralintitle{\sphinxupquote{elif}}}
\label{\detokenize{bool:elif}}
\sphinxAtStartPar
Entre \sphinxcode{\sphinxupquote{if}} y \sphinxcode{\sphinxupquote{else}} podemos usar uno o varios \sphinxcode{\sphinxupquote{elif}} para identificar mas casos buscados.
Nota: \sphinxcode{\sphinxupquote{elif}} es una forma abreviada para \sphinxcode{\sphinxupquote{else if}}.

\sphinxAtStartPar
Veamos un ejemplo un poco mas complejo de una función que usa \sphinxcode{\sphinxupquote{if/elif/else}}.

\begin{sphinxVerbatim}[commandchars=\\\{\}]
\PYG{k}{def} \PYG{n+nf}{identificar\PYGZus{}bicho}\PYG{p}{(}\PYG{n}{patas}\PYG{p}{)}\PYG{p}{:}
    \PYG{k}{if} \PYG{n}{patas} \PYG{o}{==} \PYG{l+m+mi}{6}\PYG{p}{:}
        \PYG{n+nb}{print}\PYG{p}{(}\PYG{l+s+s1}{\PYGZsq{}}\PYG{l+s+s1}{Es un insecto}\PYG{l+s+s1}{\PYGZsq{}}\PYG{p}{)}
    \PYG{k}{elif} \PYG{n}{patas} \PYG{o}{==} \PYG{l+m+mi}{8}\PYG{p}{:}
        \PYG{n+nb}{print}\PYG{p}{(}\PYG{l+s+s1}{\PYGZsq{}}\PYG{l+s+s1}{Es un arácnico}\PYG{l+s+s1}{\PYGZsq{}}\PYG{p}{)}
    \PYG{k}{else}\PYG{p}{:}
        \PYG{n+nb}{print}\PYG{p}{(}\PYG{l+s+s1}{\PYGZsq{}}\PYG{l+s+s1}{Bicho no identificado}\PYG{l+s+s1}{\PYGZsq{}}\PYG{p}{)}

\PYG{n}{identificar\PYGZus{}bicho}\PYG{p}{(}\PYG{l+m+mi}{6}\PYG{p}{)}
\PYG{l+s+s1}{\PYGZsq{}}\PYG{l+s+s1}{Es un insecto}\PYG{l+s+s1}{\PYGZsq{}}
\PYG{n}{identificar\PYGZus{}bicho}\PYG{p}{(}\PYG{l+m+mi}{8}\PYG{p}{)}
\PYG{l+s+s1}{\PYGZsq{}}\PYG{l+s+s1}{Es un arácnico}\PYG{l+s+s1}{\PYGZsq{}}
\PYG{n}{identificar\PYGZus{}bicho}\PYG{p}{(}\PYG{l+m+mi}{5}\PYG{p}{)}
\PYG{l+s+s1}{\PYGZsq{}}\PYG{l+s+s1}{Bicho no identificado}\PYG{l+s+s1}{\PYGZsq{}}
\end{sphinxVerbatim}


\subsection{\sphinxstyleliteralintitle{\sphinxupquote{and}} y \sphinxstyleliteralintitle{\sphinxupquote{or}}}
\label{\detokenize{bool:and-y-or}}
\sphinxAtStartPar
Si queremos consultar mas de una condicón simultanemante podemos
usar \sphinxcode{\sphinxupquote{and}} u \sphinxcode{\sphinxupquote{or}} (\sphinxstyleemphasis{y} u \sphinxstyleemphasis{o}).

\sphinxAtStartPar
Por ejemplo:

\begin{sphinxVerbatim}[commandchars=\\\{\}]
\PYG{n}{a} \PYG{o}{=} \PYG{l+m+mi}{1}
\PYG{n}{b} \PYG{o}{=} \PYG{l+m+mi}{2}

\PYG{k}{if} \PYG{n}{a} \PYG{o}{==} \PYG{l+m+mi}{2} \PYG{o+ow}{or} \PYG{n}{b} \PYG{o}{==} \PYG{l+m+mi}{2}\PYG{p}{:}
    \PYG{n+nb}{print}\PYG{p}{(}\PYG{l+s+s1}{\PYGZsq{}}\PYG{l+s+s1}{\PYGZdq{}}\PYG{l+s+s1}{a}\PYG{l+s+s1}{\PYGZdq{}}\PYG{l+s+s1}{ o }\PYG{l+s+s1}{\PYGZdq{}}\PYG{l+s+s1}{b}\PYG{l+s+s1}{\PYGZdq{}}\PYG{l+s+s1}{ valen dos (pueden ser ambos o cualquiera de los dos}\PYG{l+s+s1}{\PYGZsq{}}\PYG{p}{)}

\PYG{k}{if} \PYG{n}{a} \PYG{o}{==} \PYG{l+m+mi}{2} \PYG{o+ow}{and} \PYG{n}{b} \PYG{o}{==} \PYG{l+m+mi}{2}\PYG{p}{:}
    \PYG{n+nb}{print}\PYG{p}{(}\PYG{l+s+s1}{\PYGZsq{}}\PYG{l+s+s1}{\PYGZdq{}}\PYG{l+s+s1}{a}\PYG{l+s+s1}{\PYGZdq{}}\PYG{l+s+s1}{ Y }\PYG{l+s+s1}{\PYGZdq{}}\PYG{l+s+s1}{b}\PYG{l+s+s1}{\PYGZdq{}}\PYG{l+s+s1}{ valen dos}\PYG{l+s+s1}{\PYGZsq{}}\PYG{p}{)}
\end{sphinxVerbatim}


\subsection{Tareas}
\label{\detokenize{bool:tareas}}\begin{itemize}
\item {} 
\sphinxAtStartPar
Escribir una función que se llame \sphinxcode{\sphinxupquote{es\_par}} y que dado un número devuelva
\sphinxcode{\sphinxupquote{True}} o \sphinxcode{\sphinxupquote{False}} según corresponda. Tip: Los numeros pares tienen resto (operador \sphinxcode{\sphinxupquote{\%}})
cero al dividirlos por 2.

\item {} 
\sphinxAtStartPar
Escribir una función que reciba 3 parametros: \sphinxcode{\sphinxupquote{nombre}}, \sphinxcode{\sphinxupquote{edad}} y \sphinxcode{\sphinxupquote{termino\_secundario}}.
Si la edad es mayor que 18, \sphinxcode{\sphinxupquote{termino\_secundario}} es \sphinxcode{\sphinxupquote{True}} y el nombre termina (tip:
función \sphinxcode{\sphinxupquote{endswith}} de los \sphinxstyleemphasis{strings}) con «s» devuelve \sphinxcode{\sphinxupquote{True}}. En cualquier otro caso,
devuelve \sphinxcode{\sphinxupquote{False}}.

\item {} 
\sphinxAtStartPar
Escribir un programa que le pida al usuario que ingrese los datos
necesarios para calcular el
\sphinxhref{https://es.wikipedia.org/wiki/\%C3\%8Dndice\_de\_masa\_corporal}{índice de masa corporal}%
\begin{footnote}[5]\sphinxAtStartFootnote
\sphinxnolinkurl{https://es.wikipedia.org/wiki/\%C3\%8Dndice\_de\_masa\_corporal}
%
\end{footnote}
y finalmente informe (imprima algún mensaje) si el usuario tiene \sphinxstyleemphasis{Peso bajo}, \sphinxstyleemphasis{Normal},
\sphinxstyleemphasis{Sobrepeso} u \sphinxstyleemphasis{Obesidad}.

\end{itemize}

\sphinxstepscope


\chapter{Listas en Python \textless{}list\textgreater{}}
\label{\detokenize{list:listas-en-python-list}}\label{\detokenize{list::doc}}
\sphinxAtStartPar
Las listas en Python se usan para representar conjuntos ordenados de elementos.
Los objetos contenidos pueden ser de cualquier tipo (incluidas otras listas).

\sphinxAtStartPar
Nota: \sphinxstyleemphasis{Ordenado} quiere decir que la lista preserva el orden en que los elementos se
definieron, no que sus elementos se van a ordenar automaticamente.

\sphinxAtStartPar
Veamos un ejemplo simple de como definir una lista:

\begin{sphinxVerbatim}[commandchars=\\\{\}]
\PYG{n}{mi\PYGZus{}lista} \PYG{o}{=} \PYG{p}{[}\PYG{l+m+mi}{1}\PYG{p}{,} \PYG{l+m+mi}{2}\PYG{p}{,} \PYG{l+m+mi}{4}\PYG{p}{]}
\PYG{n}{otra\PYGZus{}lista} \PYG{o}{=} \PYG{p}{[}\PYG{l+s+s2}{\PYGZdq{}}\PYG{l+s+s2}{Jose}\PYG{l+s+s2}{\PYGZdq{}}\PYG{p}{,} \PYG{l+s+s2}{\PYGZdq{}}\PYG{l+s+s2}{Luisa}\PYG{l+s+s2}{\PYGZdq{}}\PYG{p}{,} \PYG{l+s+s2}{\PYGZdq{}}\PYG{l+s+s2}{Rafaela}\PYG{l+s+s2}{\PYGZdq{}}\PYG{p}{]}

\PYG{c+c1}{\PYGZsh{} tambien puede contener tipos diferentes}
\PYG{n}{variado} \PYG{o}{=} \PYG{p}{[}\PYG{l+m+mi}{1}\PYG{p}{,} \PYG{l+s+s2}{\PYGZdq{}}\PYG{l+s+s2}{Pedro}\PYG{l+s+s2}{\PYGZdq{}}\PYG{p}{,} \PYG{l+m+mi}{0}\PYG{p}{,} \PYG{l+s+s2}{\PYGZdq{}}\PYG{l+s+s2}{Nombre}\PYG{l+s+s2}{\PYGZdq{}}\PYG{p}{]}

\PYG{c+c1}{\PYGZsh{} Puede contener variables}
\PYG{n}{listas} \PYG{o}{=} \PYG{p}{[}\PYG{n}{mi\PYGZus{}lista}\PYG{p}{,} \PYG{n}{otra\PYGZus{}lista}\PYG{p}{,} \PYG{n}{variado}\PYG{p}{]}

\PYG{n+nb}{print}\PYG{p}{(}\PYG{n}{listas}\PYG{p}{)}
\PYG{p}{[}\PYG{p}{[}\PYG{l+m+mi}{1}\PYG{p}{,} \PYG{l+m+mi}{2}\PYG{p}{,} \PYG{l+m+mi}{4}\PYG{p}{]}\PYG{p}{,} \PYG{p}{[}\PYG{l+s+s1}{\PYGZsq{}}\PYG{l+s+s1}{Jose}\PYG{l+s+s1}{\PYGZsq{}}\PYG{p}{,} \PYG{l+s+s1}{\PYGZsq{}}\PYG{l+s+s1}{Luisa}\PYG{l+s+s1}{\PYGZsq{}}\PYG{p}{,} \PYG{l+s+s1}{\PYGZsq{}}\PYG{l+s+s1}{Rafaela}\PYG{l+s+s1}{\PYGZsq{}}\PYG{p}{]}\PYG{p}{,} \PYG{p}{[}\PYG{l+m+mi}{1}\PYG{p}{,} \PYG{l+s+s1}{\PYGZsq{}}\PYG{l+s+s1}{Pedro}\PYG{l+s+s1}{\PYGZsq{}}\PYG{p}{,} \PYG{l+m+mi}{0}\PYG{p}{,} \PYG{l+s+s1}{\PYGZsq{}}\PYG{l+s+s1}{Nombre}\PYG{l+s+s1}{\PYGZsq{}}\PYG{p}{]}\PYG{p}{]}
\end{sphinxVerbatim}

\sphinxAtStartPar
Python usa los corchetes \sphinxcode{\sphinxupquote{{[}{]}}} para delimitar donde empieza y donde
termina una lista. Para separar los elementos se usa la coma.

\sphinxAtStartPar
Veamos algunas funciones de las listas:

\begin{sphinxVerbatim}[commandchars=\\\{\}]
\PYG{n}{lista} \PYG{o}{=} \PYG{p}{[}\PYG{l+m+mi}{8}\PYG{p}{,} \PYG{l+m+mi}{9}\PYG{p}{]}
\PYG{c+c1}{\PYGZsh{} agregar un elemento al final}
\PYG{n}{lista}\PYG{o}{.}\PYG{n}{append}\PYG{p}{(}\PYG{l+m+mi}{3}\PYG{p}{)}
\PYG{n+nb}{print}\PYG{p}{(}\PYG{n}{lista}\PYG{p}{)}
\PYG{c+c1}{\PYGZsh{} [8, 9, 3]}

\PYG{c+c1}{\PYGZsh{} elimiar un elemento específico}
\PYG{n}{lista}\PYG{o}{.}\PYG{n}{remove}\PYG{p}{(}\PYG{l+m+mi}{9}\PYG{p}{)}
\PYG{n+nb}{print}\PYG{p}{(}\PYG{n}{lista}\PYG{p}{)}
\PYG{c+c1}{\PYGZsh{} [8, 3]}
\PYG{c+c1}{\PYGZsh{} Cuidado! si se intenta eliminar un elemento que no existe Python lanzar un error}

\PYG{c+c1}{\PYGZsh{} Tambien es posible eliminar por numero de indice y devolver lo eliminado}
\PYG{n}{primer\PYGZus{}elemento} \PYG{o}{=} \PYG{n}{lista}\PYG{o}{.}\PYG{n}{pop}\PYG{p}{(}\PYG{l+m+mi}{0}\PYG{p}{)}
\PYG{n+nb}{print}\PYG{p}{(}\PYG{l+s+sa}{f}\PYG{l+s+s1}{\PYGZsq{}}\PYG{l+s+s1}{El primero era: }\PYG{l+s+si}{\PYGZob{}}\PYG{n}{primer\PYGZus{}elemento}\PYG{l+s+si}{\PYGZcb{}}\PYG{l+s+s1}{\PYGZsq{}}\PYG{p}{)}

\PYG{c+c1}{\PYGZsh{} Para obtener el largo de una lista, se usa la funcion len()}
\PYG{n}{apellidos} \PYG{o}{=} \PYG{p}{[}\PYG{l+s+s2}{\PYGZdq{}}\PYG{l+s+s2}{gonzalez}\PYG{l+s+s2}{\PYGZdq{}}\PYG{p}{,} \PYG{l+s+s2}{\PYGZdq{}}\PYG{l+s+s2}{gomez}\PYG{l+s+s2}{\PYGZdq{}}\PYG{p}{,} \PYG{l+s+s2}{\PYGZdq{}}\PYG{l+s+s2}{rodriguez}\PYG{l+s+s2}{\PYGZdq{}}\PYG{p}{,} \PYG{l+s+s2}{\PYGZdq{}}\PYG{l+s+s2}{lopez}\PYG{l+s+s2}{\PYGZdq{}}\PYG{p}{,} \PYG{l+s+s2}{\PYGZdq{}}\PYG{l+s+s2}{garcia}\PYG{l+s+s2}{\PYGZdq{}}\PYG{p}{]}
\PYG{n}{total\PYGZus{}apellidos} \PYG{o}{=} \PYG{n+nb}{len}\PYG{p}{(}\PYG{n}{apellidos}\PYG{p}{)}
\PYG{n+nb}{print}\PYG{p}{(}\PYG{l+s+sa}{f}\PYG{l+s+s1}{\PYGZsq{}}\PYG{l+s+s1}{Total de apellidos: }\PYG{l+s+si}{\PYGZob{}}\PYG{n}{total\PYGZus{}apellidos}\PYG{l+s+si}{\PYGZcb{}}\PYG{l+s+s1}{\PYGZsq{}}\PYG{p}{)}
\PYG{c+c1}{\PYGZsh{} Total de apellidos: 5}

\PYG{c+c1}{\PYGZsh{} Ordenar los elementos de una lista}
\PYG{n}{lista}\PYG{o}{.}\PYG{n}{sort}\PYG{p}{(}\PYG{p}{)}
\PYG{c+c1}{\PYGZsh{} Si hay tipos que no son comparables, Python lanzara un error.}
\PYG{c+c1}{\PYGZsh{} Si todos son numeros se ordenaran de mayor a menor}
\PYG{c+c1}{\PYGZsh{} Si todos son strings, se ordenaran alfabéticamente}

\PYG{c+c1}{\PYGZsh{} Es posible modificar directamente algun elemento de la lista}
\PYG{c+c1}{\PYGZsh{} usando su numero de índice.}

\PYG{n}{lista} \PYG{o}{=} \PYG{p}{[}\PYG{l+m+mi}{3}\PYG{p}{,} \PYG{l+m+mi}{4}\PYG{p}{,} \PYG{l+m+mi}{5}\PYG{p}{,} \PYG{l+m+mi}{7}\PYG{p}{]}
\PYG{n}{lista}\PYG{p}{[}\PYG{l+m+mi}{1}\PYG{p}{]} \PYG{o}{=} \PYG{l+m+mi}{9}
\PYG{n+nb}{print}\PYG{p}{(}\PYG{n}{lista}\PYG{p}{)}
\PYG{p}{[}\PYG{l+m+mi}{3}\PYG{p}{,} \PYG{l+m+mi}{9}\PYG{p}{,} \PYG{l+m+mi}{5}\PYG{p}{,} \PYG{l+m+mi}{7}\PYG{p}{]}
\end{sphinxVerbatim}

\sphinxAtStartPar
Las listas tambien permiten obtener rápido cada uno de sus elementos u
obtener sub listas incluidas.
Usar los corchetes junto a la variable permite acceder a cualquier elemento
de la lista por su número de orden (comenzando por cero).
Por ejemplo
\begin{itemize}
\item {} 
\sphinxAtStartPar
\sphinxcode{\sphinxupquote{nombre\_de\_mi\_lista{[}0{]}}} devuelve el \sphinxstylestrong{primer} elemento de una lista.

\item {} 
\sphinxAtStartPar
\sphinxcode{\sphinxupquote{nombre\_de\_mi\_lista{[}1{]}}} devuelve el \sphinxstylestrong{segundo} elemento de una lista.

\end{itemize}

\sphinxAtStartPar
\sphinxstylestrong{Nota: si se pide un numero de elemento (se les llama índices) mayor a los
disponibles, Python lanzara un error.}

\sphinxAtStartPar
Comprensión de Diccionario en Python: Explicado con ejemplos

\sphinxAtStartPar
Es posible tambien acceder a los elementos en orden inverso:
\begin{itemize}
\item {} 
\sphinxAtStartPar
\sphinxcode{\sphinxupquote{nombre\_de\_mi\_lista{[}\sphinxhyphen{}1{]}}} devuelve el \sphinxstylestrong{último} elemento de una lista.

\item {} 
\sphinxAtStartPar
\sphinxcode{\sphinxupquote{nombre\_de\_mi\_lista{[}\sphinxhyphen{}2{]}}} devuelve el \sphinxstylestrong{anteúltimo} elemento de una lista.

\end{itemize}

\sphinxAtStartPar
Es posible tambien obtener una sublista:
\sphinxstylestrong{Nota: esto puede parecer anti\sphinxhyphen{}intuitivo y confuso. Requiere práctica habituarse}
\begin{itemize}
\item {} 
\sphinxAtStartPar
\sphinxcode{\sphinxupquote{nombre\_de\_mi\_lista{[}0:3{]}}} devuelve una nueva lista cuyo primer elemento es
\sphinxcode{\sphinxupquote{nombre\_de\_mi\_lista{[}0{]}}} y el últimos es \sphinxcode{\sphinxupquote{nombre\_de\_mi\_lista{[}2{]}}} (si, hasta
el 2). El segundo \sphinxstyleemphasis{parámetro} no esta incluido, el primero si. ¯\textbackslash{}\_(\sphinxhyphen{}\_\sphinxhyphen{})\_/¯

\item {} 
\sphinxAtStartPar
\sphinxcode{\sphinxupquote{nombre\_de\_mi\_lista{[}2:\sphinxhyphen{}2{]}}} devuelve una nueva lista cuyo primer elemento es
\sphinxcode{\sphinxupquote{nombre\_de\_mi\_lista{[}2{]}}} (el tercer elemento) y el últimos es
\sphinxcode{\sphinxupquote{nombre\_de\_mi\_lista{[}\sphinxhyphen{}3{]}}} ¯\textbackslash{}\_(\sphinxhyphen{}\_\sphinxhyphen{})\_/¯

\item {} 
\sphinxAtStartPar
\sphinxcode{\sphinxupquote{nombre\_de\_mi\_lista{[}\sphinxhyphen{}2:{]}}} devuelve los ultimos dos elementos, desde
\sphinxcode{\sphinxupquote{nombre\_de\_mi\_lista{[}\sphinxhyphen{}2{]}}} hasta el final (no especificar nada significa hasta
el final).

\end{itemize}

\sphinxAtStartPar
Veamos algunos ejemplos:

\begin{sphinxVerbatim}[commandchars=\\\{\}]
\PYG{n}{apellidos} \PYG{o}{=} \PYG{p}{[}\PYG{l+s+s2}{\PYGZdq{}}\PYG{l+s+s2}{gonzalez}\PYG{l+s+s2}{\PYGZdq{}}\PYG{p}{,} \PYG{l+s+s2}{\PYGZdq{}}\PYG{l+s+s2}{gomez}\PYG{l+s+s2}{\PYGZdq{}}\PYG{p}{,} \PYG{l+s+s2}{\PYGZdq{}}\PYG{l+s+s2}{rodriguez}\PYG{l+s+s2}{\PYGZdq{}}\PYG{p}{,} \PYG{l+s+s2}{\PYGZdq{}}\PYG{l+s+s2}{lopez}\PYG{l+s+s2}{\PYGZdq{}}\PYG{p}{,} \PYG{l+s+s2}{\PYGZdq{}}\PYG{l+s+s2}{garcia}\PYG{l+s+s2}{\PYGZdq{}}\PYG{p}{]}
\PYG{n}{primer\PYGZus{}apellido} \PYG{o}{=} \PYG{n}{apellidos}\PYG{p}{[}\PYG{l+m+mi}{0}\PYG{p}{]}
\PYG{n+nb}{print}\PYG{p}{(}\PYG{l+s+sa}{f}\PYG{l+s+s1}{\PYGZsq{}}\PYG{l+s+s1}{El primer apellido es: }\PYG{l+s+si}{\PYGZob{}}\PYG{n}{primer\PYGZus{}apellido}\PYG{l+s+si}{\PYGZcb{}}\PYG{l+s+s1}{\PYGZsq{}}\PYG{p}{)}
\PYG{c+c1}{\PYGZsh{} El primer apellido es: gonzalez}

\PYG{n}{ultimo\PYGZus{}apellido} \PYG{o}{=} \PYG{n}{apellidos}\PYG{p}{[}\PYG{o}{\PYGZhy{}}\PYG{l+m+mi}{1}\PYG{p}{]}
\PYG{n+nb}{print}\PYG{p}{(}\PYG{l+s+sa}{f}\PYG{l+s+s1}{\PYGZsq{}}\PYG{l+s+s1}{El último apellido es: }\PYG{l+s+si}{\PYGZob{}}\PYG{n}{ultimo\PYGZus{}apellido}\PYG{l+s+si}{\PYGZcb{}}\PYG{l+s+s1}{\PYGZsq{}}\PYG{p}{)}
\PYG{c+c1}{\PYGZsh{} El último apellido es: garcia}

\PYG{n}{primeros\PYGZus{}2} \PYG{o}{=} \PYG{n}{apellidos}\PYG{p}{[}\PYG{l+m+mi}{0}\PYG{p}{:}\PYG{l+m+mi}{2}\PYG{p}{]}
\PYG{n+nb}{print}\PYG{p}{(}\PYG{l+s+sa}{f}\PYG{l+s+s1}{\PYGZsq{}}\PYG{l+s+s1}{Los primeros dos: }\PYG{l+s+si}{\PYGZob{}}\PYG{n}{primeros\PYGZus{}2}\PYG{l+s+si}{\PYGZcb{}}\PYG{l+s+s1}{\PYGZsq{}}\PYG{p}{)}
\PYG{c+c1}{\PYGZsh{} Los primeros dos: [\PYGZsq{}gonzalez\PYGZsq{}, \PYGZsq{}gomez\PYGZsq{}]}

\PYG{n}{ultimos\PYGZus{}2} \PYG{o}{=} \PYG{n}{apellidos}\PYG{p}{[}\PYG{o}{\PYGZhy{}}\PYG{l+m+mi}{2}\PYG{p}{:}\PYG{p}{]}
\PYG{n+nb}{print}\PYG{p}{(}\PYG{l+s+sa}{f}\PYG{l+s+s1}{\PYGZsq{}}\PYG{l+s+s1}{Los últimos dos son: }\PYG{l+s+si}{\PYGZob{}}\PYG{n}{ultimos\PYGZus{}2}\PYG{l+s+si}{\PYGZcb{}}\PYG{l+s+s1}{\PYGZsq{}}\PYG{p}{)}
\PYG{c+c1}{\PYGZsh{} Los últimos dos son: [\PYGZsq{}lopez\PYGZsq{}, \PYGZsq{}garcia\PYGZsq{}]}

\PYG{c+c1}{\PYGZsh{} ordenar}
\PYG{n}{apellidos}\PYG{o}{.}\PYG{n}{sort}\PYG{p}{(}\PYG{p}{)}
\PYG{n+nb}{print}\PYG{p}{(}\PYG{l+s+sa}{f}\PYG{l+s+s1}{\PYGZsq{}}\PYG{l+s+s1}{Lista ordenada: }\PYG{l+s+si}{\PYGZob{}}\PYG{n}{apellidos}\PYG{l+s+si}{\PYGZcb{}}\PYG{l+s+s1}{\PYGZsq{}}\PYG{p}{)}
\PYG{c+c1}{\PYGZsh{} Lista ordenada: [\PYGZsq{}garcia\PYGZsq{}, \PYGZsq{}gomez\PYGZsq{}, \PYGZsq{}gonzalez\PYGZsq{}, \PYGZsq{}lopez\PYGZsq{}, \PYGZsq{}rodriguez\PYGZsq{}]}

\PYG{c+c1}{\PYGZsh{} invertir orden}
\PYG{n}{apellidos}\PYG{o}{.}\PYG{n}{reverse}\PYG{p}{(}\PYG{p}{)}
\PYG{n+nb}{print}\PYG{p}{(}\PYG{l+s+sa}{f}\PYG{l+s+s1}{\PYGZsq{}}\PYG{l+s+s1}{Lista invertida: }\PYG{l+s+si}{\PYGZob{}}\PYG{n}{apellidos}\PYG{l+s+si}{\PYGZcb{}}\PYG{l+s+s1}{\PYGZsq{}}\PYG{p}{)}
\PYG{c+c1}{\PYGZsh{} Lista invertida: [\PYGZsq{}rodriguez\PYGZsq{}, \PYGZsq{}lopez\PYGZsq{}, \PYGZsq{}gonzalez\PYGZsq{}, \PYGZsq{}gomez\PYGZsq{}, \PYGZsq{}garcia\PYGZsq{}]}
\end{sphinxVerbatim}

\begin{sphinxadmonition}{note}{funciones que transforman y funciones que devuelven}

\sphinxAtStartPar
Vale la pena notar que \sphinxcode{\sphinxupquote{reverse}} y \sphinxcode{\sphinxupquote{sort}} en las listas transforman al objeto que lo llama
y no devuelven ningún (\sphinxcode{\sphinxupquote{None}}) resultado; en cambio en los \sphinxstyleemphasis{strings}, las funciones \sphinxcode{\sphinxupquote{replace}},
\sphinxcode{\sphinxupquote{title}} y otras devuleven un resultado pero no cambian al objeto que los llamó.
\end{sphinxadmonition}

\sphinxAtStartPar
\sphinxstylestrong{¿Puede ser más complicado?}
Si, un poco más. Podemos usar un tercer parámetro. Este indica los saltos que
damos para seleccionar elementos. Predeterminado es 1 (vamos de un elemento al otro).

\sphinxAtStartPar
De esta forma \sphinxcode{\sphinxupquote{nombre\_de\_mi\_lista{[}1:6:2{]}}} significa \sphinxstyleemphasis{los elementos desde el segundo
al sexto de dos en dos} (desde \sphinxcode{\sphinxupquote{nombre\_de\_mi\_lista{[}1{]}}} a \sphinxcode{\sphinxupquote{nombre\_de\_mi\_lista{[}5{]}}}).
Y \sphinxcode{\sphinxupquote{nombre\_de\_mi\_lista{[}::\sphinxhyphen{}1{]}}} significa toda la lista completa
en sentido inverso (tambien podemos usar \sphinxcode{\sphinxupquote{nombre\_de\_mi\_lista.reverse()}} o \sphinxcode{\sphinxupquote{nombre\_de\_mi\_lista.sort(reverse=True)}}).


\section{Los \sphinxstyleemphasis{strings} tambien son listas}
\label{\detokenize{list:los-strings-tambien-son-listas}}
\sphinxAtStartPar
Python permite tratar a los \sphinxstyleemphasis{strings} como listas.
Podemos pensar que una palabra es una lista de letras.

\sphinxAtStartPar
Veamos algunos ejemplos:

\begin{sphinxVerbatim}[commandchars=\\\{\}]
\PYG{n}{nombre} \PYG{o}{=} \PYG{l+s+s2}{\PYGZdq{}}\PYG{l+s+s2}{Pedro}\PYG{l+s+s2}{\PYGZdq{}}
\PYG{n+nb}{print}\PYG{p}{(}\PYG{l+s+sa}{f}\PYG{l+s+s2}{\PYGZdq{}}\PYG{l+s+s2}{La primera letra de mi nombre es }\PYG{l+s+si}{\PYGZob{}}\PYG{n}{nombre}\PYG{p}{[}\PYG{l+m+mi}{0}\PYG{p}{]}\PYG{l+s+si}{\PYGZcb{}}\PYG{l+s+s2}{\PYGZdq{}}\PYG{p}{)}
\PYG{n+nb}{print}\PYG{p}{(}\PYG{l+s+sa}{f}\PYG{l+s+s2}{\PYGZdq{}}\PYG{l+s+s2}{La última letra de mi nombre es }\PYG{l+s+si}{\PYGZob{}}\PYG{n}{nombre}\PYG{p}{[}\PYG{o}{\PYGZhy{}}\PYG{l+m+mi}{1}\PYG{p}{]}\PYG{l+s+si}{\PYGZcb{}}\PYG{l+s+s2}{\PYGZdq{}}\PYG{p}{)}
\end{sphinxVerbatim}


\section{Función \sphinxstyleliteralintitle{\sphinxupquote{split}} de los \sphinxstyleemphasis{strings}}
\label{\detokenize{list:funcion-split-de-los-strings}}
\sphinxAtStartPar
Si quiero separar una frase en palabras Python ya incluye la funcion \sphinxcode{\sphinxupquote{split}} en
los \sphinxstyleemphasis{strings}. Esta función devuelve un objeto de tipo lista.

\sphinxAtStartPar
Veamos un ejemplo:

\begin{sphinxVerbatim}[commandchars=\\\{\}]
\PYG{n}{frase} \PYG{o}{=} \PYG{l+s+s2}{\PYGZdq{}}\PYG{l+s+s2}{Era el mejor de los tiempos y era el peor de los tiempos}\PYG{l+s+s2}{\PYGZdq{}}
\PYG{n}{palabras} \PYG{o}{=} \PYG{n}{frase}\PYG{o}{.}\PYG{n}{split}\PYG{p}{(}\PYG{p}{)}
\PYG{n+nb}{print}\PYG{p}{(}\PYG{n}{palabras}\PYG{p}{)}
\PYG{p}{[}\PYG{l+s+s1}{\PYGZsq{}}\PYG{l+s+s1}{Era}\PYG{l+s+s1}{\PYGZsq{}}\PYG{p}{,} \PYG{l+s+s1}{\PYGZsq{}}\PYG{l+s+s1}{el}\PYG{l+s+s1}{\PYGZsq{}}\PYG{p}{,} \PYG{l+s+s1}{\PYGZsq{}}\PYG{l+s+s1}{mejor}\PYG{l+s+s1}{\PYGZsq{}}\PYG{p}{,} \PYG{l+s+s1}{\PYGZsq{}}\PYG{l+s+s1}{de}\PYG{l+s+s1}{\PYGZsq{}}\PYG{p}{,} \PYG{l+s+s1}{\PYGZsq{}}\PYG{l+s+s1}{los}\PYG{l+s+s1}{\PYGZsq{}}\PYG{p}{,} \PYG{l+s+s1}{\PYGZsq{}}\PYG{l+s+s1}{tiempos}\PYG{l+s+s1}{\PYGZsq{}}\PYG{p}{,} \PYG{l+s+s1}{\PYGZsq{}}\PYG{l+s+s1}{y}\PYG{l+s+s1}{\PYGZsq{}}\PYG{p}{,} \PYG{l+s+s1}{\PYGZsq{}}\PYG{l+s+s1}{era}\PYG{l+s+s1}{\PYGZsq{}}\PYG{p}{,} \PYG{l+s+s1}{\PYGZsq{}}\PYG{l+s+s1}{el}\PYG{l+s+s1}{\PYGZsq{}}\PYG{p}{,} \PYG{l+s+s1}{\PYGZsq{}}\PYG{l+s+s1}{peor}\PYG{l+s+s1}{\PYGZsq{}}\PYG{p}{,} \PYG{l+s+s1}{\PYGZsq{}}\PYG{l+s+s1}{de}\PYG{l+s+s1}{\PYGZsq{}}\PYG{p}{,} \PYG{l+s+s1}{\PYGZsq{}}\PYG{l+s+s1}{los}\PYG{l+s+s1}{\PYGZsq{}}\PYG{p}{,} \PYG{l+s+s1}{\PYGZsq{}}\PYG{l+s+s1}{tiempos}\PYG{l+s+s1}{\PYGZsq{}}\PYG{p}{]}
\end{sphinxVerbatim}

\sphinxAtStartPar
La función \sphinxcode{\sphinxupquote{split}} tiene un parámetro llamado \sphinxcode{\sphinxupquote{separator}} que tiene como valor prederminado
\sphinxcode{\sphinxupquote{" "}} (un espacio, su uso más común). Esté parámetro indica que \sphinxstyleemphasis{caracter} se va a usar para
separar los elementos de la lista resultande.

\sphinxAtStartPar
Existen casos en que necesitamos separar por otros carcateres.

\sphinxAtStartPar
Veamos un ejemplo:

\begin{sphinxVerbatim}[commandchars=\\\{\}]
\PYG{n}{raw\PYGZus{}data} \PYG{o}{=} \PYG{l+s+s2}{\PYGZdq{}}\PYG{l+s+s2}{juana,pedro,fabiana,victor,jose,laura}\PYG{l+s+s2}{\PYGZdq{}}
\PYG{n}{nombres} \PYG{o}{=} \PYG{n}{raw\PYGZus{}data}\PYG{o}{.}\PYG{n}{split}\PYG{p}{(}\PYG{l+s+s1}{\PYGZsq{}}\PYG{l+s+s1}{,}\PYG{l+s+s1}{\PYGZsq{}}\PYG{p}{)}
\PYG{n+nb}{print}\PYG{p}{(}\PYG{n}{nombres}\PYG{p}{)}
\PYG{p}{[}\PYG{l+s+s1}{\PYGZsq{}}\PYG{l+s+s1}{juana}\PYG{l+s+s1}{\PYGZsq{}}\PYG{p}{,} \PYG{l+s+s1}{\PYGZsq{}}\PYG{l+s+s1}{pedro}\PYG{l+s+s1}{\PYGZsq{}}\PYG{p}{,} \PYG{l+s+s1}{\PYGZsq{}}\PYG{l+s+s1}{fabiana}\PYG{l+s+s1}{\PYGZsq{}}\PYG{p}{,} \PYG{l+s+s1}{\PYGZsq{}}\PYG{l+s+s1}{victor}\PYG{l+s+s1}{\PYGZsq{}}\PYG{p}{,} \PYG{l+s+s1}{\PYGZsq{}}\PYG{l+s+s1}{jose}\PYG{l+s+s1}{\PYGZsq{}}\PYG{p}{,} \PYG{l+s+s1}{\PYGZsq{}}\PYG{l+s+s1}{laura}\PYG{l+s+s1}{\PYGZsq{}}\PYG{p}{]}
\end{sphinxVerbatim}


\section{Tareas}
\label{\detokenize{list:tareas}}\begin{itemize}
\item {} 
\sphinxAtStartPar
Escribir una funcion que dada una palabra devuelva su tercera letra.

\item {} 
\sphinxAtStartPar
Escribir una funcion que reciba cuatro parametros obligatorios.
\begin{itemize}
\item {} 
\sphinxAtStartPar
Devuelva una lista

\item {} 
\sphinxAtStartPar
Esta lista debe contener tres de los cuatro elementos (hay que quitar el que sea más pequeño)

\item {} 
\sphinxAtStartPar
La lista devuelta debe estar ordenada de mayor a menor.

\item {} 
\sphinxAtStartPar
Ejemplo: \sphinxcode{\sphinxupquote{ordenar\_y\_quitar(4, 8, 9, 12)}} debe devolver \sphinxcode{\sphinxupquote{{[}12, 9, 8{]}}}.

\end{itemize}

\item {} 
\sphinxAtStartPar
Escribir una funcion que reciba un parametro llamado \sphinxcode{\sphinxupquote{nombre\_completo}} y devuelva una
lista de tres elementos (siempre con tres elementos).
\begin{itemize}
\item {} 
\sphinxAtStartPar
El primero de los elementos de la lista devuelta debe ser la primera
palabra de \sphinxcode{\sphinxupquote{nombre\_completo}} (separada con la función \sphinxcode{\sphinxupquote{split}})

\item {} 
\sphinxAtStartPar
Si \sphinxcode{\sphinxupquote{nombre\_completo}} separado con \sphinxcode{\sphinxupquote{split}} tiene solo un elemento,
agregar dos \sphinxstyleemphasis{strings} vacios para cumplir el requisito de devolver una
lista con tres elementos.

\item {} 
\sphinxAtStartPar
Si el \sphinxcode{\sphinxupquote{nombre\_completo}} tiene solo dos palabras, devolver una lista de la
forma \sphinxcode{\sphinxupquote{{[}\textquotesingle{}palabra1\textquotesingle{}, \textquotesingle{}\textquotesingle{}, \textquotesingle{}palabra2\textquotesingle{}{]}}}

\item {} 
\sphinxAtStartPar
Si el \sphinxcode{\sphinxupquote{nombre\_completo}} tiene tres palabras, devolver una lista de la
forma \sphinxcode{\sphinxupquote{{[}\textquotesingle{}palabra1\textquotesingle{}, \textquotesingle{}palabra2\textquotesingle{}, \textquotesingle{}palabra3\textquotesingle{}{]}}}

\item {} 
\sphinxAtStartPar
Si el \sphinxcode{\sphinxupquote{nombre\_completo}} tiene \sphinxstylestrong{más} de tres palabras, devolver una lista de la
forma \sphinxcode{\sphinxupquote{{[}\textquotesingle{}palabra1\textquotesingle{}, \textquotesingle{}palabra2\textquotesingle{}, \textquotesingle{}palabra3\textquotesingle{}{]}}}

\item {} 
\sphinxAtStartPar
Algunos ejemplos (suponiendo que la funcion se llame \sphinxcode{\sphinxupquote{separar\_nombre}}, puede ser otro).
Probarlos todos para asegurarse que funciona como es esperado.
\begin{itemize}
\item {} 
\sphinxAtStartPar
\sphinxcode{\sphinxupquote{separar\_nombre(\textquotesingle{}Juan\textquotesingle{})}} debe devolver \sphinxcode{\sphinxupquote{{[}\textquotesingle{}Juan\textquotesingle{}, \textquotesingle{}\textquotesingle{}, \textquotesingle{}\textquotesingle{}{]}}}

\item {} 
\sphinxAtStartPar
\sphinxcode{\sphinxupquote{separar\_nombre(\textquotesingle{}Juan Perez\textquotesingle{})}} debe devolver \sphinxcode{\sphinxupquote{{[}\textquotesingle{}Juan\textquotesingle{}, \textquotesingle{}\textquotesingle{}, \textquotesingle{}Perez\textquotesingle{}{]}}}

\item {} 
\sphinxAtStartPar
\sphinxcode{\sphinxupquote{separar\_nombre(\textquotesingle{}Juan Carlos Perez\textquotesingle{})}} debe devolver \sphinxcode{\sphinxupquote{{[}\textquotesingle{}Juan\textquotesingle{}, \textquotesingle{}Carlos\textquotesingle{}, \textquotesingle{}Perez\textquotesingle{}{]}}}

\item {} 
\sphinxAtStartPar
\sphinxcode{\sphinxupquote{separar\_nombre(\textquotesingle{}Juan Carlos Perez Valdez\textquotesingle{})}} debe devolver \sphinxcode{\sphinxupquote{{[}\textquotesingle{}Juan\textquotesingle{}, \textquotesingle{}Carlos\textquotesingle{}, \textquotesingle{}Perez\textquotesingle{}{]}}}

\end{itemize}

\end{itemize}

\end{itemize}


\section{Algunos ejemplos de uso}
\label{\detokenize{list:algunos-ejemplos-de-uso}}
\begin{sphinxVerbatim}[commandchars=\\\{\}]
\PYG{n}{lista} \PYG{o}{=} \PYG{p}{[}\PYG{l+m+mi}{1}\PYG{p}{,} \PYG{l+m+mi}{2}\PYG{p}{]}
\PYG{n}{lista}\PYG{o}{.}\PYG{n}{append}\PYG{p}{(}\PYG{l+m+mi}{3}\PYG{p}{)}

\PYG{n+nb}{print}\PYG{p}{(}\PYG{n}{lista}\PYG{p}{)}
\PYG{c+c1}{\PYGZsh{} [1, 2, 3]}

\PYG{c+c1}{\PYGZsh{} pueden contener cualquier tipo de dato}
\PYG{n}{lista} \PYG{o}{=} \PYG{p}{[}\PYG{l+m+mi}{1}\PYG{p}{,} \PYG{l+s+s2}{\PYGZdq{}}\PYG{l+s+s2}{a}\PYG{l+s+s2}{\PYGZdq{}}\PYG{p}{,} \PYG{p}{[}\PYG{p}{]}\PYG{p}{]}
\PYG{n}{lista}\PYG{o}{.}\PYG{n}{remove}\PYG{p}{(}\PYG{l+s+s2}{\PYGZdq{}}\PYG{l+s+s2}{a}\PYG{l+s+s2}{\PYGZdq{}}\PYG{p}{)}
\PYG{n+nb}{print}\PYG{p}{(}\PYG{n}{lista}\PYG{p}{)}
\PYG{c+c1}{\PYGZsh{} [1, []]}
\end{sphinxVerbatim}

\begin{sphinxVerbatim}[commandchars=\\\{\}]
\PYG{n}{lista} \PYG{o}{=} \PYG{p}{[}\PYG{l+s+s2}{\PYGZdq{}}\PYG{l+s+s2}{Juan}\PYG{l+s+s2}{\PYGZdq{}}\PYG{p}{,} \PYG{l+s+s2}{\PYGZdq{}}\PYG{l+s+s2}{Pedro}\PYG{l+s+s2}{\PYGZdq{}}\PYG{p}{,} \PYG{l+s+s2}{\PYGZdq{}}\PYG{l+s+s2}{Maria}\PYG{l+s+s2}{\PYGZdq{}}\PYG{p}{]}

\PYG{k}{for} \PYG{n}{persona} \PYG{o+ow}{in} \PYG{n}{lista}\PYG{p}{:}
    \PYG{n+nb}{print}\PYG{p}{(}\PYG{l+s+sa}{f}\PYG{l+s+s1}{\PYGZsq{}}\PYG{l+s+s1}{Hola }\PYG{l+s+si}{\PYGZob{}}\PYG{n}{persona}\PYG{l+s+si}{\PYGZcb{}}\PYG{l+s+s1}{\PYGZsq{}}\PYG{p}{)}

\PYG{l+s+sd}{\PYGZdq{}\PYGZdq{}\PYGZdq{} Muestra}
\PYG{l+s+sd}{Hola Juan}
\PYG{l+s+sd}{Hola Pedro}
\PYG{l+s+sd}{Hola Maria}
\PYG{l+s+sd}{\PYGZdq{}\PYGZdq{}\PYGZdq{}}
\end{sphinxVerbatim}

\begin{sphinxVerbatim}[commandchars=\\\{\}]
\PYG{c+c1}{\PYGZsh{} Listas por/de comprension}
\PYG{c+c1}{\PYGZsh{} mas ejemplos https://www.analyticslane.com/2019/09/23/listas\PYGZhy{}por\PYGZhy{}comprension\PYGZhy{}en\PYGZhy{}python/}

\PYG{n}{lista\PYGZus{}base} \PYG{o}{=} \PYG{p}{[}\PYG{l+m+mi}{1}\PYG{p}{,} \PYG{l+m+mi}{2}\PYG{p}{,} \PYG{l+m+mi}{3}\PYG{p}{,} \PYG{l+m+mi}{4}\PYG{p}{,} \PYG{l+m+mi}{5}\PYG{p}{,} \PYG{l+m+mi}{6}\PYG{p}{,} \PYG{l+m+mi}{7}\PYG{p}{,} \PYG{l+m+mi}{8}\PYG{p}{,} \PYG{l+m+mi}{9}\PYG{p}{,} \PYG{l+m+mi}{10}\PYG{p}{]}
\PYG{n}{lista} \PYG{o}{=} \PYG{p}{[}\PYG{n}{x} \PYG{k}{for} \PYG{n}{x} \PYG{o+ow}{in} \PYG{n}{lista\PYGZus{}base} \PYG{k}{if} \PYG{n}{x} \PYG{o}{\PYGZgt{}} \PYG{l+m+mi}{5}\PYG{p}{]}

\PYG{l+s+sd}{\PYGZdq{}\PYGZdq{}\PYGZdq{} También podría escribirse como }
\PYG{l+s+sd}{lista = [}
\PYG{l+s+sd}{    x}
\PYG{l+s+sd}{    for x in lista\PYGZus{}base}
\PYG{l+s+sd}{    if x \PYGZgt{} 5}
\PYG{l+s+sd}{]}
\PYG{l+s+sd}{\PYGZdq{}\PYGZdq{}\PYGZdq{}}

\PYG{c+c1}{\PYGZsh{} Equivalente a:}
\PYG{n}{lista2} \PYG{o}{=} \PYG{p}{[}\PYG{p}{]}
\PYG{k}{for} \PYG{n}{x} \PYG{o+ow}{in} \PYG{n}{lista\PYGZus{}base}\PYG{p}{:}
    \PYG{k}{if} \PYG{n}{x} \PYG{o}{\PYGZgt{}} \PYG{l+m+mi}{5}\PYG{p}{:}
        \PYG{n}{lista2}\PYG{o}{.}\PYG{n}{append}\PYG{p}{(}\PYG{n}{x}\PYG{p}{)}

\PYG{n+nb}{print}\PYG{p}{(}\PYG{l+s+sa}{f}\PYG{l+s+s2}{\PYGZdq{}}\PYG{l+s+s2}{Lista 1 }\PYG{l+s+si}{\PYGZob{}}\PYG{n}{lista}\PYG{l+s+si}{\PYGZcb{}}\PYG{l+s+s2}{\PYGZdq{}}\PYG{p}{)}
\PYG{n+nb}{print}\PYG{p}{(}\PYG{l+s+sa}{f}\PYG{l+s+s2}{\PYGZdq{}}\PYG{l+s+s2}{Lista 2 }\PYG{l+s+si}{\PYGZob{}}\PYG{n}{lista2}\PYG{l+s+si}{\PYGZcb{}}\PYG{l+s+s2}{\PYGZdq{}}\PYG{p}{)}

\PYG{n+nb}{print}\PYG{p}{(}\PYG{l+s+sa}{f}\PYG{l+s+s2}{\PYGZdq{}}\PYG{l+s+s2}{lista == lista2 }\PYG{l+s+si}{\PYGZob{}}\PYG{n}{lista}\PYG{o}{==}\PYG{n}{lista2}\PYG{l+s+si}{\PYGZcb{}}\PYG{l+s+s2}{\PYGZdq{}}\PYG{p}{)}

\PYG{l+s+sd}{\PYGZdq{}\PYGZdq{}\PYGZdq{} muestra}
\PYG{l+s+sd}{Lista 1 [6, 7, 8, 9, 10]}
\PYG{l+s+sd}{Lista 2 [6, 7, 8, 9, 10]}
\PYG{l+s+sd}{lista == lista2 True}
\PYG{l+s+sd}{\PYGZdq{}\PYGZdq{}\PYGZdq{}}
\end{sphinxVerbatim}

\sphinxstepscope


\chapter{Iterando ando: \sphinxstyleliteralintitle{\sphinxupquote{for}}}
\label{\detokenize{for:iterando-ando-for}}\label{\detokenize{for::doc}}
\sphinxAtStartPar
En Python algunos objetos se pueden iterar. Esto es: \sphinxstylestrong{recorrer su
elementos de uno a la vez}. Muchos objetos tiene definida una forma
de iterarse, por ejemplo las listas.

\sphinxAtStartPar
Veamos un ejemplo:

\begin{sphinxVerbatim}[commandchars=\\\{\}]
\PYG{n}{mi\PYGZus{}lista} \PYG{o}{=} \PYG{p}{[}\PYG{l+s+s2}{\PYGZdq{}}\PYG{l+s+s2}{Juan}\PYG{l+s+s2}{\PYGZdq{}}\PYG{p}{,} \PYG{l+s+s2}{\PYGZdq{}}\PYG{l+s+s2}{Pedro}\PYG{l+s+s2}{\PYGZdq{}}\PYG{p}{,} \PYG{l+s+s2}{\PYGZdq{}}\PYG{l+s+s2}{María}\PYG{l+s+s2}{\PYGZdq{}}\PYG{p}{]}

\PYG{k}{for} \PYG{n}{persona} \PYG{o+ow}{in} \PYG{n}{mi\PYGZus{}lista}\PYG{p}{:}
    \PYG{n}{saludo} \PYG{o}{=} \PYG{l+s+sa}{f}\PYG{l+s+s1}{\PYGZsq{}}\PYG{l+s+s1}{Hola }\PYG{l+s+si}{\PYGZob{}}\PYG{n}{persona}\PYG{l+s+si}{\PYGZcb{}}\PYG{l+s+s1}{!}\PYG{l+s+s1}{\PYGZsq{}}
    \PYG{n+nb}{print}\PYG{p}{(}\PYG{n}{saludo}\PYG{p}{)}
\PYG{n+nb}{print}\PYG{p}{(}\PYG{l+s+s1}{\PYGZsq{}}\PYG{l+s+s1}{Iteración terminada}\PYG{l+s+s1}{\PYGZsq{}}\PYG{p}{)}

\PYG{l+s+sd}{\PYGZdq{}\PYGZdq{}\PYGZdq{} Imprime}
\PYG{l+s+sd}{Hola Juan}
\PYG{l+s+sd}{Hola Pedro}
\PYG{l+s+sd}{Hola María}
\PYG{l+s+sd}{Iteración terminada}
\PYG{l+s+sd}{\PYGZdq{}\PYGZdq{}\PYGZdq{}}
\end{sphinxVerbatim}

\sphinxAtStartPar
Como podemos imaginar, la línea \sphinxcode{\sphinxupquote{print(saludo)}} se ejecuta tres veces.
Estos es, una por cada uno de los elementos de \sphinxcode{\sphinxupquote{mi\_lista}}.
La sentencia \sphinxcode{\sphinxupquote{for}} inicia un loop que \sphinxstylestrong{va a terminar cuando se quede
sin elementos que recorrer}.


\section{Anatomía de un \sphinxstyleliteralintitle{\sphinxupquote{for}}}
\label{\detokenize{for:anatomia-de-un-for}}
\sphinxAtStartPar
La línea \sphinxcode{\sphinxupquote{for persona in mi\_lista:}} podría traducirse como:
\sphinxstyleemphasis{por cada persona en mi\_lista hacer lo siguiente:}

\sphinxAtStartPar
Analicémosla:
\begin{itemize}
\item {} 
\sphinxAtStartPar
La sentencia \sphinxcode{\sphinxupquote{for}} indica que vamos a iniciar una iteración.

\item {} 
\sphinxAtStartPar
A continuación se requiere un nombre de variable (en nuestro
ejemplo \sphinxcode{\sphinxupquote{persona}}) para contener cada uno de los elementos de la iteración
dentro de la porción de código que se va a ejecutar en cada vuelta del bucle.

\item {} 
\sphinxAtStartPar
Luego sigue la palabra \sphinxcode{\sphinxupquote{in}} para que esta linea sea fácil de leer y como
indicador de que vamos a continuar con algún objeto iterable.

\item {} 
\sphinxAtStartPar
Finalmente se coloca el objeto sobre el que deseamos iterar.

\item {} 
\sphinxAtStartPar
Como todas las líneas que preceden a un bloque de código termina con \sphinxcode{\sphinxupquote{:}}.

\item {} 
\sphinxAtStartPar
Todo el código que necesitamos que se ejecute para cada uno de los elementos
debe estar tabulado. \sphinxstylestrong{La forma de saber donde termina esta porción de
codigo es que se termina la tabulación}.

\item {} 
\sphinxAtStartPar
Nota: la variable \sphinxcode{\sphinxupquote{persona}} solo está definida dentro del bloque del
código del \sphinxcode{\sphinxupquote{for}}. Fuera de el no está definida y dará un error si se
intenta usar.

\end{itemize}

\sphinxAtStartPar
¿Se puede anidar iteraciones?
Si, claro. Veamos un ejemplo:

\begin{sphinxVerbatim}[commandchars=\\\{\}]
\PYG{n}{lista\PYGZus{}con\PYGZus{}listas} \PYG{o}{=} \PYG{p}{[} \PYG{p}{[}\PYG{l+m+mi}{1}\PYG{p}{,} \PYG{l+m+mi}{12}\PYG{p}{]}\PYG{p}{,} \PYG{p}{[}\PYG{l+m+mi}{20}\PYG{p}{,} \PYG{l+m+mi}{22}\PYG{p}{]}\PYG{p}{,} \PYG{p}{[}\PYG{l+m+mi}{5}\PYG{p}{,} \PYG{l+m+mi}{8}\PYG{p}{]} \PYG{p}{]}

\PYG{k}{for} \PYG{n}{lista} \PYG{o+ow}{in} \PYG{n}{lista\PYGZus{}con\PYGZus{}listas}\PYG{p}{:}
    \PYG{k}{for} \PYG{n}{numero} \PYG{o+ow}{in} \PYG{n}{lista}\PYG{p}{:}
        \PYG{n+nb}{print}\PYG{p}{(}\PYG{n}{numero}\PYG{p}{)}

\PYG{c+c1}{\PYGZsh{} imprimirá:}
\PYG{l+m+mi}{1}
\PYG{l+m+mi}{12}
\PYG{l+m+mi}{20}
\PYG{l+m+mi}{22}
\PYG{l+m+mi}{5}
\PYG{l+m+mi}{8}
\end{sphinxVerbatim}


\section{\sphinxstyleliteralintitle{\sphinxupquote{continue}} y \sphinxstyleliteralintitle{\sphinxupquote{break}} en ciclos \sphinxstyleliteralintitle{\sphinxupquote{for}}}
\label{\detokenize{for:continue-y-break-en-ciclos-for}}
\sphinxAtStartPar
Dentro de una iteración, existen formas de salir de
ella (\sphinxcode{\sphinxupquote{break}}) y de saltearse un elemento (\sphinxcode{\sphinxupquote{continue}}).

\sphinxAtStartPar
Veamos un ejemplo:

\begin{sphinxVerbatim}[commandchars=\\\{\}]
\PYG{c+c1}{\PYGZsh{} for / continue / break}
\PYG{n}{paises} \PYG{o}{=} \PYG{p}{[}\PYG{l+s+s2}{\PYGZdq{}}\PYG{l+s+s2}{Argentina}\PYG{l+s+s2}{\PYGZdq{}}\PYG{p}{,} \PYG{l+s+s2}{\PYGZdq{}}\PYG{l+s+s2}{Brasil}\PYG{l+s+s2}{\PYGZdq{}}\PYG{p}{,} \PYG{l+s+s2}{\PYGZdq{}}\PYG{l+s+s2}{Chile}\PYG{l+s+s2}{\PYGZdq{}}\PYG{p}{,} \PYG{l+s+s2}{\PYGZdq{}}\PYG{l+s+s2}{Uruguay}\PYG{l+s+s2}{\PYGZdq{}}\PYG{p}{,} \PYG{l+s+s2}{\PYGZdq{}}\PYG{l+s+s2}{Venezuela}\PYG{l+s+s2}{\PYGZdq{}}\PYG{p}{,} \PYG{l+s+s2}{\PYGZdq{}}\PYG{l+s+s2}{Colombia}\PYG{l+s+s2}{\PYGZdq{}}\PYG{p}{,} \PYG{l+s+s2}{\PYGZdq{}}\PYG{l+s+s2}{Japón}\PYG{l+s+s2}{\PYGZdq{}}\PYG{p}{]}
\PYG{n+nb}{print}\PYG{p}{(}\PYG{l+s+s1}{\PYGZsq{}}\PYG{l+s+s1}{Paises}\PYG{l+s+s1}{\PYGZsq{}}\PYG{p}{)}
\PYG{k}{for} \PYG{n}{pais} \PYG{o+ow}{in} \PYG{n}{paises}\PYG{p}{:}
    \PYG{k}{if} \PYG{n}{pais} \PYG{o}{==} \PYG{l+s+s2}{\PYGZdq{}}\PYG{l+s+s2}{Chile}\PYG{l+s+s2}{\PYGZdq{}}\PYG{p}{:}
        \PYG{k}{continue}
    \PYG{n+nb}{print}\PYG{p}{(}\PYG{l+s+sa}{f}\PYG{l+s+s2}{\PYGZdq{}}\PYG{l+s+s2}{ \PYGZhy{} }\PYG{l+s+si}{\PYGZob{}}\PYG{n}{pais}\PYG{l+s+si}{\PYGZcb{}}\PYG{l+s+s2}{\PYGZdq{}}\PYG{p}{)}
    \PYG{k}{if} \PYG{n}{pais} \PYG{o}{==} \PYG{l+s+s2}{\PYGZdq{}}\PYG{l+s+s2}{Venezuela}\PYG{l+s+s2}{\PYGZdq{}}\PYG{p}{:}
        \PYG{k}{break}
\PYG{n+nb}{print}\PYG{p}{(}\PYG{l+s+s1}{\PYGZsq{}}\PYG{l+s+s1}{FIN}\PYG{l+s+s1}{\PYGZsq{}}\PYG{p}{)}
\PYG{l+s+sd}{\PYGZdq{}\PYGZdq{}\PYGZdq{} imprime}
\PYG{l+s+sd}{Paises}
\PYG{l+s+sd}{\PYGZhy{} Argentina}
\PYG{l+s+sd}{\PYGZhy{} Brasil}
\PYG{l+s+sd}{\PYGZhy{} Uruguay}
\PYG{l+s+sd}{\PYGZhy{} Venezuela}
\PYG{l+s+sd}{FIN}
\PYG{l+s+sd}{\PYGZdq{}\PYGZdq{}\PYGZdq{}}
\end{sphinxVerbatim}

\sphinxAtStartPar
Algunos elementos no pueden iterarse, por ejemplo los \sphinxcode{\sphinxupquote{int}}.
El error \sphinxcode{\sphinxupquote{\textquotesingle{}XXX\textquotesingle{} object is not iterable}} es la forma de informar esto.

\begin{sphinxVerbatim}[commandchars=\\\{\}]
\PYG{n}{a} \PYG{o}{=} \PYG{l+m+mi}{90}
\PYG{k}{for} \PYG{n}{x} \PYG{o+ow}{in} \PYG{n}{a}\PYG{p}{:}
    \PYG{n+nb}{print}\PYG{p}{(}\PYG{n}{x}\PYG{p}{)}

\PYG{l+s+sd}{\PYGZdq{}\PYGZdq{}\PYGZdq{} devolverá el error.}
\PYG{l+s+sd}{Traceback (most recent call last):}
\PYG{l+s+sd}{File \PYGZdq{}\PYGZlt{}stdin\PYGZgt{}\PYGZdq{}, line 1, in \PYGZlt{}module\PYGZgt{}}
\PYG{l+s+sd}{TypeError: \PYGZsq{}int\PYGZsq{} object is not iterable}
\PYG{l+s+sd}{\PYGZdq{}\PYGZdq{}\PYGZdq{}}
\end{sphinxVerbatim}


\section{\sphinxstyleliteralintitle{\sphinxupquote{if}} + \sphinxstyleliteralintitle{\sphinxupquote{in}}}
\label{\detokenize{for:if-in}}
\sphinxAtStartPar
Tambien podemos combinar \sphinxcode{\sphinxupquote{if}} con \sphinxcode{\sphinxupquote{in}} para saber si un objeto
esta contenido en una lista.

\begin{sphinxVerbatim}[commandchars=\\\{\}]
\PYG{n}{lista} \PYG{o}{=} \PYG{p}{[}\PYG{l+m+mi}{1}\PYG{p}{,} \PYG{l+m+mi}{2}\PYG{p}{,} \PYG{l+m+mi}{3}\PYG{p}{]}

\PYG{k}{if} \PYG{l+m+mi}{4} \PYG{o+ow}{in} \PYG{n}{lista}\PYG{p}{:}
    \PYG{n+nb}{print}\PYG{p}{(}\PYG{l+s+s2}{\PYGZdq{}}\PYG{l+s+s2}{4 está en la lista}\PYG{l+s+s2}{\PYGZdq{}}\PYG{p}{)}
\PYG{k}{else}\PYG{p}{:}
    \PYG{n+nb}{print}\PYG{p}{(}\PYG{l+s+s2}{\PYGZdq{}}\PYG{l+s+s2}{4 no está en la lista}\PYG{l+s+s2}{\PYGZdq{}}\PYG{p}{)}
\PYG{c+c1}{\PYGZsh{} imprime}
\PYG{c+c1}{\PYGZsh{} 4 no está en la lista}
\end{sphinxVerbatim}


\section{Iterando \sphinxstyleemphasis{strings}}
\label{\detokenize{for:iterando-strings}}
\sphinxAtStartPar
Como ya vimos antes, los strings tambien pueden usarse como
listas y por lo tanto se puede iterar.

\begin{sphinxVerbatim}[commandchars=\\\{\}]
\PYG{n}{nombre} \PYG{o}{=} \PYG{l+s+s2}{\PYGZdq{}}\PYG{l+s+s2}{Victor}\PYG{l+s+s2}{\PYGZdq{}}

\PYG{k}{for} \PYG{n}{letra} \PYG{o+ow}{in} \PYG{n}{nombre}\PYG{p}{:}
    \PYG{n+nb}{print}\PYG{p}{(}\PYG{n}{letra}\PYG{p}{)}

\PYG{l+s+sd}{\PYGZdq{}\PYGZdq{}\PYGZdq{} imprimirá}
\PYG{l+s+sd}{V}
\PYG{l+s+sd}{i}
\PYG{l+s+sd}{c}
\PYG{l+s+sd}{t}
\PYG{l+s+sd}{o}
\PYG{l+s+sd}{r}
\PYG{l+s+sd}{\PYGZdq{}\PYGZdq{}\PYGZdq{}}
\end{sphinxVerbatim}


\section{Agregado: \sphinxstyleliteralintitle{\sphinxupquote{range}}}
\label{\detokenize{for:agregado-range}}
\sphinxAtStartPar
Python tiene una funcion (técnicamente no es una función pero
por ahora podemos pensarla como tal) includa (\sphinxstyleemphasis{built\sphinxhyphen{}in}) que
permite iterar sobre una serie de números. Se llama \sphinxcode{\sphinxupquote{range}}
y podemos ver su funcionamiento con algunos ejemplos.
Cuando llamamos a \sphinxcode{\sphinxupquote{range}} con un solo un parámetro (siempre numeros)
este devuelve un objeto que se puede iterar. Incluye los números
desde cero hasta el valor pasado como parámetro menos uno
(comportamiento similar a los indices de las listas).

\begin{sphinxVerbatim}[commandchars=\\\{\}]
\PYG{k}{for} \PYG{n}{n} \PYG{o+ow}{in} \PYG{n+nb}{range}\PYG{p}{(}\PYG{l+m+mi}{3}\PYG{p}{)}\PYG{p}{:}
    \PYG{n+nb}{print}\PYG{p}{(}\PYG{n}{n}\PYG{p}{)}
\PYG{c+c1}{\PYGZsh{} imprimira (en lineas diferentes): 0 1 2}
\end{sphinxVerbatim}

\sphinxAtStartPar
Cuando llamamos a \sphinxcode{\sphinxupquote{range}} con dos parámetros (siempre numeros)
este devuelve un objeto que se puede iterar. Incluye los números
desde el primer parámetro hasta el valor pasado como segundo
parámetro menos uno.

\begin{sphinxVerbatim}[commandchars=\\\{\}]
\PYG{k}{for} \PYG{n}{n} \PYG{o+ow}{in} \PYG{n+nb}{range}\PYG{p}{(}\PYG{l+m+mi}{2}\PYG{p}{,} \PYG{l+m+mi}{8}\PYG{p}{)}\PYG{p}{:}
    \PYG{n+nb}{print}\PYG{p}{(}\PYG{n}{n}\PYG{p}{)}
\PYG{c+c1}{\PYGZsh{} imprimira (en lineas diferentes): 2 3 4 5 6 7}
\end{sphinxVerbatim}

\sphinxAtStartPar
Cuando llamamos a \sphinxcode{\sphinxupquote{range}} con tres parámetros (siempre numeros)
este devuelve un objeto que se puede iterar. Incluye los números
desde el primer parámetro hasta el valor pasado como segundo
parámetro menos uno. El último parámetro indica el tamaño del
salto entre un elemento y otro (por ejemplo 2 hará que el
resultado vaya de dos en dos).

\begin{sphinxVerbatim}[commandchars=\\\{\}]
\PYG{k}{for} \PYG{n}{n} \PYG{o+ow}{in} \PYG{n+nb}{range}\PYG{p}{(}\PYG{l+m+mi}{3}\PYG{p}{,} \PYG{l+m+mi}{10}\PYG{p}{,} \PYG{l+m+mi}{2}\PYG{p}{)}\PYG{p}{:}
    \PYG{n+nb}{print}\PYG{p}{(}\PYG{n}{n}\PYG{p}{)}
\PYG{c+c1}{\PYGZsh{} imprimira (en lineas diferentes): 3 5 7 9}
\end{sphinxVerbatim}

\sphinxAtStartPar
Tambien es posible convertir el resultado de \sphinxcode{\sphinxupquote{range}} a una lista
(que ya conocemos).

\begin{sphinxVerbatim}[commandchars=\\\{\}]
\PYG{n}{pares} \PYG{o}{=} \PYG{n+nb}{range}\PYG{p}{(}\PYG{l+m+mi}{0}\PYG{p}{,} \PYG{l+m+mi}{100}\PYG{p}{,} \PYG{l+m+mi}{2}\PYG{p}{)}
\PYG{n}{lista\PYGZus{}pares} \PYG{o}{=} \PYG{n+nb}{list}\PYG{p}{(}\PYG{n}{pares}\PYG{p}{)}
\PYG{n+nb}{print}\PYG{p}{(}\PYG{n}{lista\PYGZus{}pares}\PYG{p}{)}
\PYG{p}{[}\PYG{l+m+mi}{0}\PYG{p}{,} \PYG{l+m+mi}{2}\PYG{p}{,} \PYG{l+m+mi}{4}\PYG{p}{,} \PYG{l+m+mi}{6}\PYG{p}{,} \PYG{l+m+mi}{8}\PYG{p}{,} \PYG{l+m+mi}{10}\PYG{p}{,} \PYG{l+m+mi}{12}\PYG{p}{,} \PYG{l+m+mi}{14}\PYG{p}{,} \PYG{l+m+mi}{16}\PYG{p}{,} \PYG{l+m+mi}{18}\PYG{p}{,} \PYG{l+m+mi}{20}\PYG{p}{,} \PYG{l+m+mi}{22}\PYG{p}{,} \PYG{l+m+mi}{24}\PYG{p}{,} \PYG{l+m+mi}{26}\PYG{p}{,} \PYG{l+m+mi}{28}\PYG{p}{,} \PYG{l+m+mi}{30}\PYG{p}{,} \PYG{l+m+mi}{32}\PYG{p}{,}
\PYG{l+m+mi}{34}\PYG{p}{,} \PYG{l+m+mi}{36}\PYG{p}{,} \PYG{l+m+mi}{38}\PYG{p}{,} \PYG{l+m+mi}{40}\PYG{p}{,} \PYG{l+m+mi}{42}\PYG{p}{,} \PYG{l+m+mi}{44}\PYG{p}{,} \PYG{l+m+mi}{46}\PYG{p}{,} \PYG{l+m+mi}{48}\PYG{p}{,} \PYG{l+m+mi}{50}\PYG{p}{,} \PYG{l+m+mi}{52}\PYG{p}{,} \PYG{l+m+mi}{54}\PYG{p}{,} \PYG{l+m+mi}{56}\PYG{p}{,} \PYG{l+m+mi}{58}\PYG{p}{,} \PYG{l+m+mi}{60}\PYG{p}{,} \PYG{l+m+mi}{62}\PYG{p}{,} \PYG{l+m+mi}{64}\PYG{p}{,}
\PYG{l+m+mi}{66}\PYG{p}{,} \PYG{l+m+mi}{68}\PYG{p}{,} \PYG{l+m+mi}{70}\PYG{p}{,} \PYG{l+m+mi}{72}\PYG{p}{,} \PYG{l+m+mi}{74}\PYG{p}{,} \PYG{l+m+mi}{76}\PYG{p}{,} \PYG{l+m+mi}{78}\PYG{p}{,} \PYG{l+m+mi}{80}\PYG{p}{,} \PYG{l+m+mi}{82}\PYG{p}{,} \PYG{l+m+mi}{84}\PYG{p}{,} \PYG{l+m+mi}{86}\PYG{p}{,} \PYG{l+m+mi}{88}\PYG{p}{,} \PYG{l+m+mi}{90}\PYG{p}{,} \PYG{l+m+mi}{92}\PYG{p}{,} \PYG{l+m+mi}{94}\PYG{p}{,} \PYG{l+m+mi}{96}\PYG{p}{,} \PYG{l+m+mi}{98}\PYG{p}{]}
\end{sphinxVerbatim}


\section{Iterar \sphinxstyleemphasis{mientras} que algo suceda: \sphinxstyleliteralintitle{\sphinxupquote{while}}}
\label{\detokenize{for:iterar-mientras-que-algo-suceda-while}}
\sphinxAtStartPar
En algunas ocasiones necesitamos iterar hasta que algo cambie.
Por ejemplo, hasta que el usuario ingrese un número válido.

\sphinxAtStartPar
Para estos casos existe la sentencia \sphinxcode{\sphinxupquote{while}}.

\begin{sphinxVerbatim}[commandchars=\\\{\}]
\PYG{n}{numero\PYGZus{}final} \PYG{o}{=} \PYG{l+m+mi}{0}

\PYG{c+c1}{\PYGZsh{} No salimos hasta que el numero sea válido}
\PYG{k}{while} \PYG{n}{numero\PYGZus{}final} \PYG{o}{==} \PYG{l+m+mi}{0}\PYG{p}{:}
    \PYG{n}{numero} \PYG{o}{=} \PYG{n+nb}{input}\PYG{p}{(}\PYG{l+s+s2}{\PYGZdq{}}\PYG{l+s+s2}{Ingrese un número entre 1 y 10: }\PYG{l+s+s2}{\PYGZdq{}}\PYG{p}{)}
    \PYG{k}{if} \PYG{o+ow}{not} \PYG{n}{numero}\PYG{o}{.}\PYG{n}{isdigit}\PYG{p}{(}\PYG{p}{)}\PYG{p}{:}
        \PYG{n+nb}{print}\PYG{p}{(}\PYG{l+s+s2}{\PYGZdq{}}\PYG{l+s+s2}{No ingresaste un número válido}\PYG{l+s+s2}{\PYGZdq{}}\PYG{p}{)}
        \PYG{k}{continue}
    \PYG{k}{if} \PYG{n+nb}{int}\PYG{p}{(}\PYG{n}{numero}\PYG{p}{)} \PYG{o}{\PYGZlt{}} \PYG{l+m+mi}{1} \PYG{o+ow}{or} \PYG{n+nb}{int}\PYG{p}{(}\PYG{n}{numero}\PYG{p}{)} \PYG{o}{\PYGZgt{}} \PYG{l+m+mi}{10}\PYG{p}{:}
        \PYG{n+nb}{print}\PYG{p}{(}\PYG{l+s+s2}{\PYGZdq{}}\PYG{l+s+s2}{El número debe estar entre 1 y 10}\PYG{l+s+s2}{\PYGZdq{}}\PYG{p}{)}
        \PYG{k}{continue}
    \PYG{n}{numero\PYGZus{}final} \PYG{o}{=} \PYG{n+nb}{int}\PYG{p}{(}\PYG{n}{numero}\PYG{p}{)}

\PYG{n+nb}{print}\PYG{p}{(}\PYG{l+s+sa}{f}\PYG{l+s+s2}{\PYGZdq{}}\PYG{l+s+s2}{Ingresaste el número }\PYG{l+s+si}{\PYGZob{}}\PYG{n}{numero\PYGZus{}final}\PYG{l+s+si}{\PYGZcb{}}\PYG{l+s+s2}{\PYGZdq{}}\PYG{p}{)}

\PYG{l+s+sd}{\PYGZdq{}\PYGZdq{}\PYGZdq{} TEST}

\PYG{l+s+sd}{Ingrese un número entre 1 y 10: a}
\PYG{l+s+sd}{No ingresaste un número válido}
\PYG{l+s+sd}{Ingrese un número entre 1 y 10: 11}
\PYG{l+s+sd}{El número debe estar entre 1 y 10}
\PYG{l+s+sd}{Ingrese un número entre 1 y 10: 0}
\PYG{l+s+sd}{El número debe estar entre 1 y 10}
\PYG{l+s+sd}{Ingrese un número entre 1 y 10: 3}
\PYG{l+s+sd}{Ingresaste el número 3}

\PYG{l+s+sd}{\PYGZdq{}\PYGZdq{}\PYGZdq{}}
\end{sphinxVerbatim}


\section{Tareas}
\label{\detokenize{for:tareas}}\begin{itemize}
\item {} 
\sphinxAtStartPar
Escribir un programa que:
\begin{itemize}
\item {} 
\sphinxAtStartPar
Incluya al inicio una funcion llamada \sphinxcode{\sphinxupquote{es\_par}} y que devuelva
\sphinxstyleemphasis{verdadero} o \sphinxstyleemphasis{falso} según corresponda

\item {} 
\sphinxAtStartPar
Le solicite al usuario que ingrese una lista de números separados por coma.

\item {} 
\sphinxAtStartPar
Transformarme este texto ingresado a lista con \sphinxcode{\sphinxupquote{split(\textquotesingle{},\textquotesingle{})}}

\item {} 
\sphinxAtStartPar
Iterere todos ellos e imprima solo aquellos que son pares.

\end{itemize}

\item {} 
\sphinxAtStartPar
Escribir un programa que itere todos los múltiplos de 3 desde el cero
al 500000 e imprima solo aquellos que además son divisibles (el resto
de la division es cero) por 13.

\item {} 
\sphinxAtStartPar
Escribir un programa que responda estas preguntas:
\begin{itemize}
\item {} 
\sphinxAtStartPar
¿Cuantos números multiplos de siete menores a 10.000 terminan en \sphinxstyleemphasis{999}?

\item {} 
\sphinxAtStartPar
¿Cuales son?

\end{itemize}

\end{itemize}

\sphinxstepscope


\chapter{Cuidando nuestro código: \sphinxstyleliteralintitle{\sphinxupquote{git}}}
\label{\detokenize{git:cuidando-nuestro-codigo-git}}\label{\detokenize{git::doc}}
\sphinxAtStartPar
Git es un sistema de control de versiones para repositorios de código fuente
descentralizado basado en un modelo de ramificación.
Actualmente es el instrumento más usado en la industria del software para
administrar código fuente en equipos de trabajo.


\section{Control de versiones}
\label{\detokenize{git:control-de-versiones}}
\sphinxAtStartPar
Git permite hacer seguimiento del código que
escribimos y de todos los cambios que le hacemos. Es posible saber
en cada momento cuando y quien escribió cada línea de código.
También es posible ver y volver a versiones anteriores cuando sea
necesario.


\section{Descentralizado}
\label{\detokenize{git:descentralizado}}
\sphinxAtStartPar
No necesitamos un servidor central donde almacenar todo el codigo
junto a sus ramas y versiones.
Cada usuario puede tener copias completas e independientes del material.


\section{Modelo de ramificación}
\label{\detokenize{git:modelo-de-ramificacion}}
\sphinxAtStartPar
Git permite (y te alienta) a que uses múltiples \sphinxstyleemphasis{ramas} de trabajo
(usaremos indistintamente la palabra \sphinxstyleemphasis{branch} del ingles).
Cada cambio en el código (para corregir errores o agregar nuevas
funcionalidades) puede hacerse en una rama de trabajo independiente
que facilmente puede integrarse a otras ramas de trabajo.

\sphinxstepscope


\chapter{Git en la nube: GitHub}
\label{\detokenize{github:git-en-la-nube-github}}\label{\detokenize{github::doc}}
\sphinxAtStartPar
Luego de la aparición de Git surgieron muchos servicios que
permitían mantener una copia de tus repositorios en internet.
Todas las empresas de software usan alguno de estos servicios
por lo que trabajar con desarrollador requiere siempre conocer
y usar estos servicios.

\sphinxAtStartPar
El más popular (pero no el único) de ellos actualmente es
\sphinxhref{https://github.com/}{GitHub}%
\begin{footnote}[6]\sphinxAtStartFootnote
\sphinxnolinkurl{https://github.com/}
%
\end{footnote} y es por esto que lo usaremos
en este curso.
Es importante que accedas al sitio y crees una cuenta (son gratuitas
para la mayoría de sus funciones).

\sphinxAtStartPar
Un ejemplo de repositorio de código es este mismo curso.
Puedes ver el repositorio que incluye este manual y documentación
\sphinxhref{https://github.com/avdata99/programacion-para-no-programadores}{aquí}%
\begin{footnote}[7]\sphinxAtStartFootnote
\sphinxnolinkurl{https://github.com/avdata99/programacion-para-no-programadores}
%
\end{footnote}.

\noindent\sphinxincludegraphics{{repo-ppnp}.png}


\section{GitHub desde Visual Studio Code}
\label{\detokenize{github:github-desde-visual-studio-code}}
\sphinxAtStartPar
Visual Studio Code incluye la posibilidad de conectar tu cuenta de GitHub.
Estoy es muy útil para conectarte a repositorios de código y enviar tus
sugerencias o cambios.

\noindent\sphinxincludegraphics{{vscode-accounts}.png}

\sphinxAtStartPar
Despues de un proceso de conectar tu cuenta de GitHub, esta ya
(casi) quedará disponible para usar desde Visual Studio Code.

\noindent\sphinxincludegraphics{{vscode-accounts-logged-in}.png}

\sphinxAtStartPar
Finalmente, debes ir al menú \sphinxstyleemphasis{View} \sphinxhyphen{}\textgreater{} \sphinxstyleemphasis{Terminal} y colocar:

\begin{sphinxVerbatim}[commandchars=\\\{\}]
git config \PYGZhy{}\PYGZhy{}global user.email \PYG{l+s+s2}{\PYGZdq{}tu\PYGZhy{}email@xxx.com\PYGZdq{}}
git config \PYGZhy{}\PYGZhy{}global user.name \PYG{l+s+s2}{\PYGZdq{}tu\PYGZus{}nombre\PYGZus{}de\PYGZus{}usuario\PYGZdq{}}
\end{sphinxVerbatim}

\noindent\sphinxincludegraphics{{vscode-terminal}.png}

\sphinxAtStartPar
Para comenzar a trabajar sobre un repositorio de código debes usar el
buscador o colocar la URL del repositorio:
\sphinxurl{https://github.com/avdata99/programacion-para-no-programadores}

\noindent\sphinxincludegraphics{{vscode-open-clone}.png}

\sphinxAtStartPar
Luego de definir en que carpeta quedará tu copia de este repositorio
podrás comenzar a usarlo.

\noindent\sphinxincludegraphics{{vscode-cloned}.png}


\subsection{Tarea}
\label{\detokenize{github:tarea}}\begin{itemize}
\item {} 
\sphinxAtStartPar
Compartí con tus compañeros de curso tu nombre de usuario
en \sphinxhref{https://github.com/}{GitHub}%
\begin{footnote}[8]\sphinxAtStartFootnote
\sphinxnolinkurl{https://github.com/}
%
\end{footnote} para comenzar a trabajar en equipo.

\item {} 
\sphinxAtStartPar
Conectar tu cuenta de GitHub a Visual Studio Code.

\item {} 
\sphinxAtStartPar
Clonar el \sphinxhref{https://github.com/avdata99/programacion-para-no-programadores}{repositorio del curso}%
\begin{footnote}[9]\sphinxAtStartFootnote
\sphinxnolinkurl{https://github.com/avdata99/programacion-para-no-programadores}
%
\end{footnote}

\end{itemize}

\sphinxstepscope


\chapter{Mi primer \sphinxstyleemphasis{PR}}
\label{\detokenize{mi-primer-pr:mi-primer-pr}}\label{\detokenize{mi-primer-pr::doc}}
\sphinxAtStartPar
Como dijimos anteriormente, Git nos permite tener diferentes ramas de trabajo.
De esta forma, existe una rama principal (llamada \sphinxcode{\sphinxupquote{main}}, \sphinxcode{\sphinxupquote{master}} o como
cada equipo prefiera) de trabajo donde todos los cambios aceptados se integran
consolidando la versión oficial de tu producto de software.

\sphinxAtStartPar
La forma de enviar tus sugerencias de cambios al equipo es crear una nueva rama
de trabajo con tus cambios y solicitar la revisión a otros integrantes del equipo.

\sphinxAtStartPar
Esta solictud es conocida como \sphinxstyleemphasis{PR} y viene de \sphinxstyleemphasis{Pull request}. En otras plataformas
se le llama \sphinxstyleemphasis{MR} (por \sphinxstyleemphasis{Merge Request}).


\section{Pull request paso a paso}
\label{\detokenize{mi-primer-pr:pull-request-paso-a-paso}}
\sphinxAtStartPar
Ten en cuenta estos pasos cada vez que quieras proponer cambios en un proyecto de
software.


\subsection{Paso 1: Actualizar tu repositorio}
\label{\detokenize{mi-primer-pr:paso-1-actualizar-tu-repositorio}}
\sphinxAtStartPar
En primer lugar debes asegurarte que tu versión del código esta actualizada para
que tus cambios sean propuestos sobre la última versión del código.
Comenzar una rama a partir de una versión \sphinxstyleemphasis{vieja} de la rama principal es un error
común.

\sphinxAtStartPar
Antes de comenzar debes asegurarte de estar en la rama principal y de que esta este
actualziada. Esto se consigue con estos dos pasos (asumimos que la rama principal se
llama \sphinxcode{\sphinxupquote{master}} pero podría ser otra).

\sphinxAtStartPar
Desde tu terminal esto puede hacerse así:

\begin{sphinxVerbatim}[commandchars=\\\{\}]
\PYG{c+c1}{\PYGZsh{} posicionarme en la rama principal}
git checkout master
\PYG{c+c1}{\PYGZsh{} traer (pull) los ultimos cambios}
git pull
\end{sphinxVerbatim}

\sphinxAtStartPar
Desde Visual Studio Code puede hacerse gráficamente.

\sphinxAtStartPar
Checkout (posicionarse en una rama):

\noindent\sphinxincludegraphics{{vscode-checkout}.png}

\sphinxAtStartPar
Pull (traer los ultimos cambios):

\noindent\sphinxincludegraphics{{vscode-pull}.png}


\subsection{Paso 2: Crear una nueva rama}
\label{\detokenize{mi-primer-pr:paso-2-crear-una-nueva-rama}}
\sphinxAtStartPar
Crear nueva rama (\sphinxstyleemphasis{Create new branch}) con un nombre descriptivo
de los cambios que pensas implementar.

\noindent\sphinxincludegraphics{{vscode-checkout}.png}

\sphinxAtStartPar
Desde tu terminal esto puede hacerse con:

\begin{sphinxVerbatim}[commandchars=\\\{\}]
\PYG{c+c1}{\PYGZsh{} posicionarme en la rama principal}
git checkout \PYGZhy{}b nombre\PYGZus{}de\PYGZus{}mi\PYGZus{}rama
\end{sphinxVerbatim}

\sphinxAtStartPar
La rama de trabajo actual estará siempre visible en VSC
abajo a la izquierda en la pantalla.
Es importante saber en cada momento en que rama de trabajo estamos.


\subsection{Paso 3: Hacer los cambios en el código}
\label{\detokenize{mi-primer-pr:paso-3-hacer-los-cambios-en-el-codigo}}
\sphinxAtStartPar
Modificar o crear uno o más archivos con los cambios que esperamos proponer.
Es conveniente asegurarnos de probar los cambios localmente para estar
seguros que funcionan según esperamos.


\subsection{Paso 4: Agregar los archivos con cambios}
\label{\detokenize{mi-primer-pr:paso-4-agregar-los-archivos-con-cambios}}
\sphinxAtStartPar
Una vez hechos los cambios debemos agregar estos archivos a lo que será nuestro \sphinxstyleemphasis{commit}.
Un \sphinxstyleemphasis{commit} en Git es un conjunto de cambios en uno o más archivos.

\sphinxAtStartPar
Desde tu terminal esto puede hacerse con:

\begin{sphinxVerbatim}[commandchars=\\\{\}]
\PYG{c+c1}{\PYGZsh{} agregar los cambios en un archivo a lo que será nuestro commit}
git add archivo\PYGZus{}que\PYGZus{}cambio.py
\end{sphinxVerbatim}

\noindent\sphinxincludegraphics{{vscode-add}.png}

\sphinxAtStartPar
Todos los archivos agregados con \sphinxstyleemphasis{add} pasan a estar preparados para hacer
tu \sphinxstyleemphasis{commit} (se llama \sphinxstyleemphasis{staged} a esta \sphinxstyleemphasis{area de preparación}).
En la parte superior debes colocar un mensaje que describa el cambio que estas haciendo.

\noindent\sphinxincludegraphics{{vscode-staged}.png}


\subsection{Paso 5: Commit}
\label{\detokenize{mi-primer-pr:paso-5-commit}}
\sphinxAtStartPar
Finalmente con el boton \sphinxstyleemphasis{commit} se consolida este cambio en la rama actual.
Desde tu terminal esto puede hacerse con:

\begin{sphinxVerbatim}[commandchars=\\\{\}]
\PYG{c+c1}{\PYGZsh{} agregar los cambios en un archivo a lo que será nuestro commit}
git commit \PYGZhy{}m \PYG{l+s+s2}{\PYGZdq{}Descripcion de mi cambio\PYGZdq{}}
\end{sphinxVerbatim}

\sphinxAtStartPar
\sphinxstylestrong{Nota importante} Este \sphinxstyleemphasis{commit} solo existe en tu computadora. Al clonar el
repositorio de GitHub, creamos un nodo local e independiente de aquel y es
allí donde todos estos cambios quedan registrados hasta que finalmente hacemos
un \sphinxstyleemphasis{push}.


\subsection{Paso 6: Push}
\label{\detokenize{mi-primer-pr:paso-6-push}}
\sphinxAtStartPar
En este paso enviamos los cambios que tenemos localmente al repositorio en GitHub.
De esta forma otros usuarios podrían descargarse nuestra rama (por ejemplo para
colaborar en ella).

\noindent\sphinxincludegraphics{{vscode-push}.png}

\sphinxAtStartPar
Desde tu terminal esto puede hacerse con:

\begin{sphinxVerbatim}[commandchars=\\\{\}]
\PYG{c+c1}{\PYGZsh{} agregar los cambios en un archivo a lo que será nuestro commit}
git push
\end{sphinxVerbatim}

\sphinxAtStartPar
Este ciclo de hacer cambios (paso 3) hasta el push puede hacerse las veces que
sea necesario, incluso despues de creado el PR en el paso siguiente.


\subsection{Paso 7: \sphinxstyleemphasis{Pull Requets}}
\label{\detokenize{mi-primer-pr:paso-7-pull-requets}}
\sphinxAtStartPar
Finalmente desde VSC o desde la página web del repositorio podemos
iniciar un \sphinxstyleemphasis{PR} y asignar uno o más usuarios como revisores.

\noindent\sphinxincludegraphics{{vscode-pr}.png}

\sphinxAtStartPar
Es posible que los revisores soliciten nuevos cambios o correcciones.
En caso de ser necesario podemos voler a nuestra rama más tarde y
actualizar el PR haciendo nuevos \sphinxstyleemphasis{commits} hasta que finalmente nuestro
trabajo quede resuelto.


\subsection{Tarea}
\label{\detokenize{mi-primer-pr:tarea}}\begin{itemize}
\item {} 
\sphinxAtStartPar
Buscar en el \sphinxhref{https://github.com/avdata99/programacion-para-no-programadores}{repositorio de este curso}%
\begin{footnote}[10]\sphinxAtStartFootnote
\sphinxnolinkurl{https://github.com/avdata99/programacion-para-no-programadores}
%
\end{footnote}
(al que ya nos conectamos) cualquiera de los archivos \sphinxcode{\sphinxupquote{rst}} de la carpeta \sphinxcode{\sphinxupquote{source}} algún error
ortográfico o sugerencia sobre el texto de este manual y enviarlo como PR.

\item {} 
\sphinxAtStartPar
La carpeta \sphinxhref{https://github.com/avdata99/programacion-para-no-programadores/blob/master/ejercicios}{ejercicios}%
\begin{footnote}[11]\sphinxAtStartFootnote
\sphinxnolinkurl{https://github.com/avdata99/programacion-para-no-programadores/blob/master/ejercicios}
%
\end{footnote}
de este repositorio tiene ya problemas para resolver mediante \sphinxstyleemphasis{pull requests}.
Se espera un \sphinxstyleemphasis{PR} para cada ejercicio, por lo tanto cada uno requiere una
rama distinta \sphinxstylestrong{que siempre} salga desde master.
\sphinxstylestrong{Asegurarse de seguir todos los pasos para cada PR}.
\begin{itemize}
\item {} 
\sphinxAtStartPar
Hacer un PR con una propuesta de solución para el
\sphinxhref{https://github.com/avdata99/programacion-para-no-programadores/blob/master/ejercicios/ejercicio-001/ejercicio.py}{ejercicio 001}%
\begin{footnote}[12]\sphinxAtStartFootnote
\sphinxnolinkurl{https://github.com/avdata99/programacion-para-no-programadores/blob/master/ejercicios/ejercicio-001/ejercicio.py}
%
\end{footnote}
(contenido en este repositorio)

\end{itemize}

\begin{sphinxVerbatim}[commandchars=\\\{\}]
\PYG{l+s+sd}{\PYGZdq{}\PYGZdq{}\PYGZdq{}}
\PYG{l+s+sd}{Completar la funcion para que devuelva \PYGZdq{}Hola NOMBRE\PYGZdq{}}
\PYG{l+s+sd}{segun el \PYGZdq{}nombre\PYGZdq{} (string) que se pasa como parametro.}
\PYG{l+s+sd}{\PYGZdq{}\PYGZdq{}\PYGZdq{}}


\PYG{k}{def} \PYG{n+nf}{hola}\PYG{p}{(}\PYG{n}{nombre}\PYG{p}{)}\PYG{p}{:}
    \PYG{k}{pass}

\PYG{c+c1}{\PYGZsh{} \PYGZhy{}\PYGZhy{}\PYGZhy{}\PYGZhy{}\PYGZhy{}\PYGZhy{}\PYGZhy{}\PYGZhy{}\PYGZhy{}\PYGZhy{}\PYGZhy{}\PYGZhy{}\PYGZhy{}\PYGZhy{}\PYGZhy{}\PYGZhy{}\PYGZhy{}\PYGZhy{}\PYGZhy{}\PYGZhy{}\PYGZhy{}\PYGZhy{}\PYGZhy{}\PYGZhy{}\PYGZhy{}\PYGZhy{}\PYGZhy{}\PYGZhy{}\PYGZhy{}\PYGZhy{}\PYGZhy{}\PYGZhy{}\PYGZhy{}\PYGZhy{}\PYGZhy{}\PYGZhy{}\PYGZhy{}\PYGZhy{}\PYGZhy{}\PYGZhy{}\PYGZhy{}\PYGZhy{}\PYGZhy{}\PYGZhy{}\PYGZhy{}\PYGZhy{}\PYGZhy{}\PYGZhy{}\PYGZhy{}\PYGZhy{}\PYGZhy{}\PYGZhy{}\PYGZhy{}\PYGZhy{}\PYGZhy{}\PYGZhy{}\PYGZhy{}\PYGZhy{}\PYGZhy{}\PYGZhy{}\PYGZhy{}\PYGZhy{}\PYGZhy{}\PYGZhy{}\PYGZhy{}\PYGZhy{}\PYGZhy{}\PYGZhy{}\PYGZhy{}\PYGZhy{}\PYGZhy{}\PYGZhy{}}
\PYG{c+c1}{\PYGZsh{} NO BORRAR O MODIFICAR LAS LINEAS QUE SIGUEN}
\PYG{c+c1}{\PYGZsh{} \PYGZhy{}\PYGZhy{}\PYGZhy{}\PYGZhy{}\PYGZhy{}\PYGZhy{}\PYGZhy{}\PYGZhy{}\PYGZhy{}\PYGZhy{}\PYGZhy{}\PYGZhy{}\PYGZhy{}\PYGZhy{}\PYGZhy{}\PYGZhy{}\PYGZhy{}\PYGZhy{}\PYGZhy{}\PYGZhy{}\PYGZhy{}\PYGZhy{}\PYGZhy{}\PYGZhy{}\PYGZhy{}\PYGZhy{}\PYGZhy{}\PYGZhy{}\PYGZhy{}\PYGZhy{}\PYGZhy{}\PYGZhy{}\PYGZhy{}\PYGZhy{}\PYGZhy{}\PYGZhy{}\PYGZhy{}\PYGZhy{}\PYGZhy{}\PYGZhy{}\PYGZhy{}\PYGZhy{}\PYGZhy{}\PYGZhy{}\PYGZhy{}\PYGZhy{}\PYGZhy{}\PYGZhy{}\PYGZhy{}\PYGZhy{}\PYGZhy{}\PYGZhy{}\PYGZhy{}\PYGZhy{}\PYGZhy{}\PYGZhy{}\PYGZhy{}\PYGZhy{}\PYGZhy{}\PYGZhy{}\PYGZhy{}\PYGZhy{}\PYGZhy{}\PYGZhy{}\PYGZhy{}\PYGZhy{}\PYGZhy{}\PYGZhy{}\PYGZhy{}\PYGZhy{}\PYGZhy{}\PYGZhy{}}
\PYG{c+c1}{\PYGZsh{} Una vez terminada la tarea ejecutar este archivo.}
\PYG{c+c1}{\PYGZsh{} Si se ve la leyenda \PYGZsq{}Ejercicio terminado OK\PYGZsq{} el ejercicio se considera completado.}
\PYG{c+c1}{\PYGZsh{} La instruccion \PYGZdq{}assert\PYGZdq{} de Python lanzará un error si lo que se indica a}
\PYG{c+c1}{\PYGZsh{}   continuacion es falso.}
\PYG{c+c1}{\PYGZsh{} Si usas GitHub (o similares) podes hacer una nueva rama con esta solución,}
\PYG{c+c1}{\PYGZsh{}   crear un \PYGZdq{}pull request\PYGZdq{} y solicitar revision de un tercero.}

\PYG{k}{assert} \PYG{n}{hola}\PYG{p}{(}\PYG{l+s+s2}{\PYGZdq{}}\PYG{l+s+s2}{Juan}\PYG{l+s+s2}{\PYGZdq{}}\PYG{p}{)} \PYG{o}{==} \PYG{l+s+s2}{\PYGZdq{}}\PYG{l+s+s2}{Hola Juan}\PYG{l+s+s2}{\PYGZdq{}}
\PYG{k}{assert} \PYG{n}{hola}\PYG{p}{(}\PYG{l+s+s2}{\PYGZdq{}}\PYG{l+s+s2}{Pedro}\PYG{l+s+s2}{\PYGZdq{}}\PYG{p}{)} \PYG{o}{==} \PYG{l+s+s2}{\PYGZdq{}}\PYG{l+s+s2}{Hola Pedro}\PYG{l+s+s2}{\PYGZdq{}}

\PYG{n+nb}{print}\PYG{p}{(}\PYG{l+s+s1}{\PYGZsq{}}\PYG{l+s+s1}{Ejercicio terminado OK}\PYG{l+s+s1}{\PYGZsq{}}\PYG{p}{)}
\end{sphinxVerbatim}
\begin{itemize}
\item {} 
\sphinxAtStartPar
Hacer un PR con una propuesta de solución para el
\sphinxhref{https://github.com/avdata99/programacion-para-no-programadores/blob/master/ejercicios/ejercicio-002/ejercicio.py}{ejercicio 002}%
\begin{footnote}[13]\sphinxAtStartFootnote
\sphinxnolinkurl{https://github.com/avdata99/programacion-para-no-programadores/blob/master/ejercicios/ejercicio-002/ejercicio.py}
%
\end{footnote}
(contenido en este repositorio)

\end{itemize}

\begin{sphinxVerbatim}[commandchars=\\\{\}]
\PYG{l+s+sd}{\PYGZdq{}\PYGZdq{}\PYGZdq{}}
\PYG{l+s+sd}{Completar la funcion para que devuelva el resultado de la suma}
\PYG{l+s+sd}{de los parámetros pasados a y b.}
\PYG{l+s+sd}{\PYGZdq{}\PYGZdq{}\PYGZdq{}}


\PYG{k}{def} \PYG{n+nf}{suma}\PYG{p}{(}\PYG{n}{a}\PYG{p}{,} \PYG{n}{b}\PYG{p}{)}\PYG{p}{:}
    \PYG{k}{pass}


\PYG{c+c1}{\PYGZsh{} \PYGZhy{}\PYGZhy{}\PYGZhy{}\PYGZhy{}\PYGZhy{}\PYGZhy{}\PYGZhy{}\PYGZhy{}\PYGZhy{}\PYGZhy{}\PYGZhy{}\PYGZhy{}\PYGZhy{}\PYGZhy{}\PYGZhy{}\PYGZhy{}\PYGZhy{}\PYGZhy{}\PYGZhy{}\PYGZhy{}\PYGZhy{}\PYGZhy{}\PYGZhy{}\PYGZhy{}\PYGZhy{}\PYGZhy{}\PYGZhy{}\PYGZhy{}\PYGZhy{}\PYGZhy{}\PYGZhy{}\PYGZhy{}\PYGZhy{}\PYGZhy{}\PYGZhy{}\PYGZhy{}\PYGZhy{}\PYGZhy{}\PYGZhy{}\PYGZhy{}\PYGZhy{}\PYGZhy{}\PYGZhy{}\PYGZhy{}\PYGZhy{}\PYGZhy{}\PYGZhy{}\PYGZhy{}\PYGZhy{}\PYGZhy{}\PYGZhy{}\PYGZhy{}\PYGZhy{}\PYGZhy{}\PYGZhy{}\PYGZhy{}\PYGZhy{}\PYGZhy{}\PYGZhy{}\PYGZhy{}\PYGZhy{}\PYGZhy{}\PYGZhy{}\PYGZhy{}\PYGZhy{}\PYGZhy{}\PYGZhy{}\PYGZhy{}\PYGZhy{}\PYGZhy{}\PYGZhy{}\PYGZhy{}}
\PYG{c+c1}{\PYGZsh{} NO BORRAR O MODIFICAR LAS LINEAS QUE SIGUEN}
\PYG{c+c1}{\PYGZsh{} \PYGZhy{}\PYGZhy{}\PYGZhy{}\PYGZhy{}\PYGZhy{}\PYGZhy{}\PYGZhy{}\PYGZhy{}\PYGZhy{}\PYGZhy{}\PYGZhy{}\PYGZhy{}\PYGZhy{}\PYGZhy{}\PYGZhy{}\PYGZhy{}\PYGZhy{}\PYGZhy{}\PYGZhy{}\PYGZhy{}\PYGZhy{}\PYGZhy{}\PYGZhy{}\PYGZhy{}\PYGZhy{}\PYGZhy{}\PYGZhy{}\PYGZhy{}\PYGZhy{}\PYGZhy{}\PYGZhy{}\PYGZhy{}\PYGZhy{}\PYGZhy{}\PYGZhy{}\PYGZhy{}\PYGZhy{}\PYGZhy{}\PYGZhy{}\PYGZhy{}\PYGZhy{}\PYGZhy{}\PYGZhy{}\PYGZhy{}\PYGZhy{}\PYGZhy{}\PYGZhy{}\PYGZhy{}\PYGZhy{}\PYGZhy{}\PYGZhy{}\PYGZhy{}\PYGZhy{}\PYGZhy{}\PYGZhy{}\PYGZhy{}\PYGZhy{}\PYGZhy{}\PYGZhy{}\PYGZhy{}\PYGZhy{}\PYGZhy{}\PYGZhy{}\PYGZhy{}\PYGZhy{}\PYGZhy{}\PYGZhy{}\PYGZhy{}\PYGZhy{}\PYGZhy{}\PYGZhy{}\PYGZhy{}}
\PYG{c+c1}{\PYGZsh{} Una vez terminada la tarea ejecutar este archivo.}
\PYG{c+c1}{\PYGZsh{} Si se ve la leyenda \PYGZsq{}Ejercicio terminado OK\PYGZsq{} el ejercicio se considera completado.}
\PYG{c+c1}{\PYGZsh{} La instruccion \PYGZdq{}assert\PYGZdq{} de Python lanzará un error si lo que se indica a}
\PYG{c+c1}{\PYGZsh{}   continuacion es falso.}
\PYG{c+c1}{\PYGZsh{} Si usas GitHub (o similares) podes hacer una nueva rama con esta solución,}
\PYG{c+c1}{\PYGZsh{}   crear un \PYGZdq{}pull request\PYGZdq{} y solicitar revision de un tercero.}

\PYG{k}{assert} \PYG{n}{suma}\PYG{p}{(}\PYG{l+m+mi}{2}\PYG{p}{,} \PYG{l+m+mi}{6}\PYG{p}{)} \PYG{o}{==} \PYG{l+m+mi}{8}
\PYG{k}{assert} \PYG{n}{suma}\PYG{p}{(}\PYG{l+m+mi}{3}\PYG{p}{,} \PYG{l+m+mi}{3}\PYG{p}{)} \PYG{o}{==} \PYG{l+m+mi}{6}

\PYG{n+nb}{print}\PYG{p}{(}\PYG{l+s+s1}{\PYGZsq{}}\PYG{l+s+s1}{Ejercicio terminado OK}\PYG{l+s+s1}{\PYGZsq{}}\PYG{p}{)}
\end{sphinxVerbatim}

\end{itemize}

\sphinxstepscope


\chapter{Diccionarios: \sphinxstyleliteralintitle{\sphinxupquote{dict}}}
\label{\detokenize{dict:diccionarios-dict}}\label{\detokenize{dict::doc}}
\sphinxAtStartPar
Los diccionarios en Python son una estructura de datos preparadas para almacenar
colecciones de elementos (separados por coma) compuestos de una clave y un valor
(suele llamarlos en ingles \sphinxstyleemphasis{key}, \sphinxstyleemphasis{value}).
Los diccionarios se delimitan con llaves \sphinxcode{\sphinxupquote{\{\}}}.

\sphinxAtStartPar
Veamos un ejemplo:

\begin{sphinxVerbatim}[commandchars=\\\{\}]
\PYG{n}{persona} \PYG{o}{=} \PYG{p}{\PYGZob{}}
    \PYG{l+s+s2}{\PYGZdq{}}\PYG{l+s+s2}{nombre}\PYG{l+s+s2}{\PYGZdq{}}\PYG{p}{:} \PYG{l+s+s2}{\PYGZdq{}}\PYG{l+s+s2}{Luis}\PYG{l+s+s2}{\PYGZdq{}}\PYG{p}{,}
    \PYG{l+s+s2}{\PYGZdq{}}\PYG{l+s+s2}{apellido}\PYG{l+s+s2}{\PYGZdq{}}\PYG{p}{:} \PYG{l+s+s2}{\PYGZdq{}}\PYG{l+s+s2}{Colon}\PYG{l+s+s2}{\PYGZdq{}}\PYG{p}{,}
    \PYG{l+s+s2}{\PYGZdq{}}\PYG{l+s+s2}{edad}\PYG{l+s+s2}{\PYGZdq{}}\PYG{p}{:} \PYG{l+m+mi}{32}
\PYG{p}{\PYGZcb{}}
\PYG{n+nb}{type}\PYG{p}{(}\PYG{n}{persona}\PYG{p}{)}
\PYG{c+c1}{\PYGZsh{} devuelve \PYGZlt{}class \PYGZsq{}dict\PYGZsq{}\PYGZgt{}}
\end{sphinxVerbatim}

\sphinxAtStartPar
Los valores pueden accederse directamente desde sus claves.
Esto es así tanto para leer como para modificar cada elemento.
Las claves son únicas, no puede haber dos elementos con la misma clave.

\begin{sphinxVerbatim}[commandchars=\\\{\}]
\PYG{n}{persona} \PYG{o}{=} \PYG{p}{\PYGZob{}}
    \PYG{l+s+s2}{\PYGZdq{}}\PYG{l+s+s2}{nombre}\PYG{l+s+s2}{\PYGZdq{}}\PYG{p}{:} \PYG{l+s+s2}{\PYGZdq{}}\PYG{l+s+s2}{Luis}\PYG{l+s+s2}{\PYGZdq{}}\PYG{p}{,}
    \PYG{l+s+s2}{\PYGZdq{}}\PYG{l+s+s2}{apellido}\PYG{l+s+s2}{\PYGZdq{}}\PYG{p}{:} \PYG{l+s+s2}{\PYGZdq{}}\PYG{l+s+s2}{Colon}\PYG{l+s+s2}{\PYGZdq{}}\PYG{p}{,}
    \PYG{l+s+s2}{\PYGZdq{}}\PYG{l+s+s2}{edad}\PYG{l+s+s2}{\PYGZdq{}}\PYG{p}{:} \PYG{l+m+mi}{32}
\PYG{p}{\PYGZcb{}}

\PYG{n+nb}{print}\PYG{p}{(}\PYG{l+s+sa}{f}\PYG{l+s+s1}{\PYGZsq{}}\PYG{l+s+s1}{Nombre: }\PYG{l+s+si}{\PYGZob{}}\PYG{n}{persona}\PYG{p}{[}\PYG{l+s+s2}{\PYGZdq{}}\PYG{l+s+s2}{nombre}\PYG{l+s+s2}{\PYGZdq{}}\PYG{p}{]}\PYG{l+s+si}{\PYGZcb{}}\PYG{l+s+s1}{ }\PYG{l+s+si}{\PYGZob{}}\PYG{n}{persona}\PYG{p}{[}\PYG{l+s+s2}{\PYGZdq{}}\PYG{l+s+s2}{apellido}\PYG{l+s+s2}{\PYGZdq{}}\PYG{p}{]}\PYG{l+s+si}{\PYGZcb{}}\PYG{l+s+s1}{ tiene }\PYG{l+s+si}{\PYGZob{}}\PYG{n}{persona}\PYG{p}{[}\PYG{l+s+s2}{\PYGZdq{}}\PYG{l+s+s2}{edad}\PYG{l+s+s2}{\PYGZdq{}}\PYG{p}{]}\PYG{l+s+si}{\PYGZcb{}}\PYG{l+s+s1}{ años}\PYG{l+s+s1}{\PYGZsq{}}\PYG{p}{)}
\PYG{l+s+s1}{\PYGZsq{}}\PYG{l+s+s1}{Luis Colon tiene 32 años}\PYG{l+s+s1}{\PYGZsq{}}

\PYG{n+nb}{print}\PYG{p}{(}\PYG{n}{persona}\PYG{p}{[}\PYG{l+s+s2}{\PYGZdq{}}\PYG{l+s+s2}{nombre}\PYG{l+s+s2}{\PYGZdq{}}\PYG{p}{]}\PYG{p}{)}
\PYG{l+s+s1}{\PYGZsq{}}\PYG{l+s+s1}{Luis}\PYG{l+s+s1}{\PYGZsq{}}

\PYG{n}{edad} \PYG{o}{=} \PYG{n}{persona}\PYG{p}{[}\PYG{l+s+s2}{\PYGZdq{}}\PYG{l+s+s2}{edad}\PYG{l+s+s2}{\PYGZdq{}}\PYG{p}{]}
\PYG{n+nb}{print}\PYG{p}{(}\PYG{n}{edad}\PYG{p}{)}
\PYG{l+m+mi}{32}

\PYG{n}{persona}\PYG{p}{[}\PYG{l+s+s2}{\PYGZdq{}}\PYG{l+s+s2}{apellido}\PYG{l+s+s2}{\PYGZdq{}}\PYG{p}{]} \PYG{o}{=} \PYG{l+s+s2}{\PYGZdq{}}\PYG{l+s+s2}{Gonzalez}\PYG{l+s+s2}{\PYGZdq{}}
\PYG{n+nb}{print}\PYG{p}{(}\PYG{n}{persona}\PYG{p}{)}
\PYG{p}{\PYGZob{}}\PYG{l+s+s2}{\PYGZdq{}}\PYG{l+s+s2}{nombre}\PYG{l+s+s2}{\PYGZdq{}}\PYG{p}{:} \PYG{l+s+s2}{\PYGZdq{}}\PYG{l+s+s2}{Luis}\PYG{l+s+s2}{\PYGZdq{}}\PYG{p}{,} \PYG{l+s+s2}{\PYGZdq{}}\PYG{l+s+s2}{apellido}\PYG{l+s+s2}{\PYGZdq{}}\PYG{p}{:} \PYG{l+s+s2}{\PYGZdq{}}\PYG{l+s+s2}{Gonzalez}\PYG{l+s+s2}{\PYGZdq{}}\PYG{p}{,} \PYG{l+s+s2}{\PYGZdq{}}\PYG{l+s+s2}{edad}\PYG{l+s+s2}{\PYGZdq{}}\PYG{p}{:} \PYG{l+m+mi}{32}\PYG{p}{\PYGZcb{}}
\end{sphinxVerbatim}

\sphinxAtStartPar
En algunas ocasiones no vamos a tener certeza de que las claves a las
que queremos acceder existen.

\begin{sphinxVerbatim}[commandchars=\\\{\}]
\PYG{n}{persona} \PYG{o}{=} \PYG{p}{\PYGZob{}}
    \PYG{l+s+s2}{\PYGZdq{}}\PYG{l+s+s2}{nombre}\PYG{l+s+s2}{\PYGZdq{}}\PYG{p}{:} \PYG{l+s+s2}{\PYGZdq{}}\PYG{l+s+s2}{Luis}\PYG{l+s+s2}{\PYGZdq{}}\PYG{p}{,}
    \PYG{l+s+s2}{\PYGZdq{}}\PYG{l+s+s2}{apellido}\PYG{l+s+s2}{\PYGZdq{}}\PYG{p}{:} \PYG{l+s+s2}{\PYGZdq{}}\PYG{l+s+s2}{Colon}\PYG{l+s+s2}{\PYGZdq{}}\PYG{p}{,}
    \PYG{l+s+s2}{\PYGZdq{}}\PYG{l+s+s2}{edad}\PYG{l+s+s2}{\PYGZdq{}}\PYG{p}{:} \PYG{l+m+mi}{32}
\PYG{p}{\PYGZcb{}}

\PYG{c+c1}{\PYGZsh{} La siguiente línea lanzará un error (la clave no existe)}
\PYG{n}{persona}\PYG{p}{[}\PYG{l+s+s2}{\PYGZdq{}}\PYG{l+s+s2}{clave\PYGZus{}que\PYGZus{}no\PYGZus{}exite}\PYG{l+s+s2}{\PYGZdq{}}\PYG{p}{]}

\PYG{c+c1}{\PYGZsh{} Para acceder a elementos de los que no sabemos si su clave existe,}
\PYG{c+c1}{\PYGZsh{} usamos la función \PYGZdq{}get\PYGZdq{} de los diccionarios}
\PYG{n}{persona}\PYG{o}{.}\PYG{n}{get}\PYG{p}{(}\PYG{l+s+s2}{\PYGZdq{}}\PYG{l+s+s2}{clave\PYGZus{}que\PYGZus{}no\PYGZus{}exite}\PYG{l+s+s2}{\PYGZdq{}}\PYG{p}{)}
\PYG{c+c1}{\PYGZsh{} Si la clave (key) no existe la anterior línea no va a lanzar}
\PYG{c+c1}{\PYGZsh{} un error. Simplemente devolverá None (es el \PYGZdq{}nada\PYGZdq{} de Python)}

\PYG{c+c1}{\PYGZsh{} Si queremos un valor predeterminado para cuando la clave no}
\PYG{c+c1}{\PYGZsh{} esta definida podemos usar el segundo parámetro opcional de \PYGZdq{}get\PYGZdq{}}
\PYG{n}{persona}\PYG{o}{.}\PYG{n}{get}\PYG{p}{(}\PYG{l+s+s2}{\PYGZdq{}}\PYG{l+s+s2}{clave\PYGZus{}que\PYGZus{}no\PYGZus{}exite}\PYG{l+s+s2}{\PYGZdq{}}\PYG{p}{,} \PYG{l+s+s2}{\PYGZdq{}}\PYG{l+s+s2}{valor\PYGZus{}predeterminado}\PYG{l+s+s2}{\PYGZdq{}}\PYG{p}{)}

\PYG{c+c1}{\PYGZsh{} Tambien es posible \PYGZdq{}preguntar\PYGZdq{} si una clave existe}
\PYG{k}{if} \PYG{l+s+s2}{\PYGZdq{}}\PYG{l+s+s2}{clave\PYGZus{}buscada}\PYG{l+s+s2}{\PYGZdq{}} \PYG{o+ow}{in} \PYG{n}{persona}\PYG{p}{:}
    \PYG{n+nb}{print}\PYG{p}{(}\PYG{l+s+s2}{\PYGZdq{}}\PYG{l+s+s2}{clave\PYGZus{}buscada SI existe como clave en }\PYG{l+s+s2}{\PYGZsq{}}\PYG{l+s+s2}{persona}\PYG{l+s+s2}{\PYGZsq{}}\PYG{l+s+s2}{\PYGZdq{}}\PYG{p}{)}

\PYG{c+c1}{\PYGZsh{} Esto es posible porque los diccionarios definen un iterador con la}
\PYG{c+c1}{\PYGZsh{} lista de sus claves}

\PYG{k}{for} \PYG{n}{key} \PYG{o+ow}{in} \PYG{n}{persona}\PYG{p}{:}
    \PYG{n+nb}{print}\PYG{p}{(}\PYG{l+s+sa}{f}\PYG{l+s+s2}{\PYGZdq{}}\PYG{l+s+s2}{Clave en persona: }\PYG{l+s+si}{\PYGZob{}}\PYG{n}{key}\PYG{l+s+si}{\PYGZcb{}}\PYG{l+s+s2}{\PYGZdq{}}\PYG{p}{)}
\end{sphinxVerbatim}

\sphinxAtStartPar
Cada par de clave\sphinxhyphen{}valor es denominado como elemento (item).
La función \sphinxcode{\sphinxupquote{items}} de los diccionarios devuleve un objeto iterable
que podemos navegar con \sphinxcode{\sphinxupquote{for}} y que devuelve en cada paso una
clave y un valor.

\begin{sphinxVerbatim}[commandchars=\\\{\}]
\PYG{n}{persona} \PYG{o}{=} \PYG{p}{\PYGZob{}}
    \PYG{l+s+s2}{\PYGZdq{}}\PYG{l+s+s2}{nombre}\PYG{l+s+s2}{\PYGZdq{}}\PYG{p}{:} \PYG{l+s+s2}{\PYGZdq{}}\PYG{l+s+s2}{Juan}\PYG{l+s+s2}{\PYGZdq{}}\PYG{p}{,}
    \PYG{l+s+s2}{\PYGZdq{}}\PYG{l+s+s2}{apellido}\PYG{l+s+s2}{\PYGZdq{}}\PYG{p}{:} \PYG{l+s+s2}{\PYGZdq{}}\PYG{l+s+s2}{Colon}\PYG{l+s+s2}{\PYGZdq{}}\PYG{p}{,}
    \PYG{l+s+s2}{\PYGZdq{}}\PYG{l+s+s2}{edad}\PYG{l+s+s2}{\PYGZdq{}}\PYG{p}{:} \PYG{l+m+mi}{32}
\PYG{p}{\PYGZcb{}}

\PYG{k}{for} \PYG{n}{k}\PYG{p}{,} \PYG{n}{v} \PYG{o+ow}{in} \PYG{n}{persona}\PYG{o}{.}\PYG{n}{items}\PYG{p}{(}\PYG{p}{)}\PYG{p}{:}
    \PYG{n+nb}{print}\PYG{p}{(}\PYG{l+s+sa}{f}\PYG{l+s+s1}{\PYGZsq{}}\PYG{l+s+s1}{Item encontrado: Key:}\PYG{l+s+si}{\PYGZob{}}\PYG{n}{k}\PYG{l+s+si}{\PYGZcb{}}\PYG{l+s+s1}{, Value: }\PYG{l+s+si}{\PYGZob{}}\PYG{n}{v}\PYG{l+s+si}{\PYGZcb{}}\PYG{l+s+s1}{\PYGZsq{}}\PYG{p}{)}

\PYG{c+c1}{\PYGZsh{} Item encontrado: Key:nombre, Value: Juan}
\PYG{c+c1}{\PYGZsh{} Item encontrado: Key:apellido, Value: Colon}
\PYG{c+c1}{\PYGZsh{} Item encontrado: Key:edad, Value: 32}
\end{sphinxVerbatim}

\sphinxAtStartPar
Los valores puede ser de cualquier tipo e incluso conformar estructuras muy complejas.

\sphinxAtStartPar
Ejemplo:

\begin{sphinxVerbatim}[commandchars=\\\{\}]
\PYG{n}{persona} \PYG{o}{=} \PYG{p}{\PYGZob{}}
    \PYG{l+s+s2}{\PYGZdq{}}\PYG{l+s+s2}{nombre}\PYG{l+s+s2}{\PYGZdq{}}\PYG{p}{:} \PYG{l+s+s2}{\PYGZdq{}}\PYG{l+s+s2}{Luis}\PYG{l+s+s2}{\PYGZdq{}}\PYG{p}{,}
    \PYG{l+s+s2}{\PYGZdq{}}\PYG{l+s+s2}{apellido}\PYG{l+s+s2}{\PYGZdq{}}\PYG{p}{:} \PYG{l+s+s2}{\PYGZdq{}}\PYG{l+s+s2}{Colon}\PYG{l+s+s2}{\PYGZdq{}}\PYG{p}{,}
    \PYG{l+s+s2}{\PYGZdq{}}\PYG{l+s+s2}{edad}\PYG{l+s+s2}{\PYGZdq{}}\PYG{p}{:} \PYG{l+m+mi}{32}\PYG{p}{,}
    \PYG{l+s+s2}{\PYGZdq{}}\PYG{l+s+s2}{estudios}\PYG{l+s+s2}{\PYGZdq{}}\PYG{p}{:} \PYG{p}{\PYGZob{}}
        \PYG{l+s+s2}{\PYGZdq{}}\PYG{l+s+s2}{primario}\PYG{l+s+s2}{\PYGZdq{}}\PYG{p}{:} \PYG{k+kc}{True}\PYG{p}{,}
        \PYG{l+s+s2}{\PYGZdq{}}\PYG{l+s+s2}{secundario}\PYG{l+s+s2}{\PYGZdq{}}\PYG{p}{:} \PYG{k+kc}{True}\PYG{p}{,}
        \PYG{l+s+s2}{\PYGZdq{}}\PYG{l+s+s2}{terciario}\PYG{l+s+s2}{\PYGZdq{}}\PYG{p}{:} \PYG{k+kc}{False}\PYG{p}{,}
        \PYG{l+s+s2}{\PYGZdq{}}\PYG{l+s+s2}{universitario}\PYG{l+s+s2}{\PYGZdq{}}\PYG{p}{:} \PYG{k+kc}{False}
    \PYG{p}{\PYGZcb{}}\PYG{p}{,}
    \PYG{l+s+s2}{\PYGZdq{}}\PYG{l+s+s2}{experiencia\PYGZus{}laboral}\PYG{l+s+s2}{\PYGZdq{}}\PYG{p}{:} \PYG{p}{\PYGZob{}}
        \PYG{l+s+s2}{\PYGZdq{}}\PYG{l+s+s2}{2005\PYGZhy{}2008}\PYG{l+s+s2}{\PYGZdq{}}\PYG{p}{:} \PYG{p}{\PYGZob{}}
            \PYG{l+s+s2}{\PYGZdq{}}\PYG{l+s+s2}{empresa}\PYG{l+s+s2}{\PYGZdq{}}\PYG{p}{:} \PYG{l+s+s2}{\PYGZdq{}}\PYG{l+s+s2}{Ferreteria el cosito del coso}\PYG{l+s+s2}{\PYGZdq{}}\PYG{p}{,}
            \PYG{l+s+s2}{\PYGZdq{}}\PYG{l+s+s2}{cargo}\PYG{l+s+s2}{\PYGZdq{}}\PYG{p}{:} \PYG{l+s+s2}{\PYGZdq{}}\PYG{l+s+s2}{Vendedor}\PYG{l+s+s2}{\PYGZdq{}}\PYG{p}{,}
            \PYG{l+s+s2}{\PYGZdq{}}\PYG{l+s+s2}{sueldo}\PYG{l+s+s2}{\PYGZdq{}}\PYG{p}{:} \PYG{l+m+mi}{45000}\PYG{p}{,}
            \PYG{l+s+s2}{\PYGZdq{}}\PYG{l+s+s2}{tareas}\PYG{l+s+s2}{\PYGZdq{}}\PYG{p}{:} \PYG{p}{[}\PYG{l+s+s2}{\PYGZdq{}}\PYG{l+s+s2}{atender publico}\PYG{l+s+s2}{\PYGZdq{}}\PYG{p}{,} \PYG{l+s+s2}{\PYGZdq{}}\PYG{l+s+s2}{compras}\PYG{l+s+s2}{\PYGZdq{}}\PYG{p}{]}
            \PYG{p}{\PYGZcb{}}\PYG{p}{,}
        \PYG{l+s+s2}{\PYGZdq{}}\PYG{l+s+s2}{2009\PYGZhy{}2011}\PYG{l+s+s2}{\PYGZdq{}}\PYG{p}{:} \PYG{p}{\PYGZob{}}
            \PYG{l+s+s2}{\PYGZdq{}}\PYG{l+s+s2}{empresa}\PYG{l+s+s2}{\PYGZdq{}}\PYG{p}{:} \PYG{l+s+s2}{\PYGZdq{}}\PYG{l+s+s2}{Escuela Tecnica San Martin}\PYG{l+s+s2}{\PYGZdq{}}\PYG{p}{,}
            \PYG{l+s+s2}{\PYGZdq{}}\PYG{l+s+s2}{cargo}\PYG{l+s+s2}{\PYGZdq{}}\PYG{p}{:} \PYG{l+s+s2}{\PYGZdq{}}\PYG{l+s+s2}{profesor}\PYG{l+s+s2}{\PYGZdq{}}\PYG{p}{,}
            \PYG{l+s+s2}{\PYGZdq{}}\PYG{l+s+s2}{sueldo}\PYG{l+s+s2}{\PYGZdq{}}\PYG{p}{:} \PYG{l+m+mi}{75000}\PYG{p}{,}
            \PYG{l+s+s2}{\PYGZdq{}}\PYG{l+s+s2}{tareas}\PYG{l+s+s2}{\PYGZdq{}}\PYG{p}{:} \PYG{p}{[}\PYG{l+s+s2}{\PYGZdq{}}\PYG{l+s+s2}{dictar clases}\PYG{l+s+s2}{\PYGZdq{}}\PYG{p}{,} \PYG{l+s+s2}{\PYGZdq{}}\PYG{l+s+s2}{elaborar cursos}\PYG{l+s+s2}{\PYGZdq{}}\PYG{p}{]}
            \PYG{p}{\PYGZcb{}}
    \PYG{p}{\PYGZcb{}}
\PYG{p}{\PYGZcb{}}
\end{sphinxVerbatim}

\sphinxAtStartPar
En el ejemplo anterior \sphinxcode{\sphinxupquote{persona{[}"estudios"{]}}} es a su vez un diccionario por lo que
tiene a su ves las propiedades de un nuevo objeto de este tipo.
Las siguientes lineas son válidas para la estructura recién definida.

\begin{sphinxVerbatim}[commandchars=\\\{\}]
\PYG{n+nb}{print}\PYG{p}{(}\PYG{n}{persona}\PYG{p}{[}\PYG{l+s+s2}{\PYGZdq{}}\PYG{l+s+s2}{estudios}\PYG{l+s+s2}{\PYGZdq{}}\PYG{p}{]}\PYG{p}{[}\PYG{l+s+s2}{\PYGZdq{}}\PYG{l+s+s2}{primario}\PYG{l+s+s2}{\PYGZdq{}}\PYG{p}{]}\PYG{p}{)}
\PYG{k+kc}{True}

\PYG{k}{if} \PYG{n}{persona}\PYG{p}{[}\PYG{l+s+s2}{\PYGZdq{}}\PYG{l+s+s2}{estudios}\PYG{l+s+s2}{\PYGZdq{}}\PYG{p}{]}\PYG{p}{[}\PYG{l+s+s2}{\PYGZdq{}}\PYG{l+s+s2}{secundario}\PYG{l+s+s2}{\PYGZdq{}}\PYG{p}{]}\PYG{p}{:}
    \PYG{n+nb}{print}\PYG{p}{(}\PYG{l+s+s1}{\PYGZsq{}}\PYG{l+s+s1}{La persona termino el secundario}\PYG{l+s+s1}{\PYGZsq{}}\PYG{p}{)}
\end{sphinxVerbatim}

\sphinxAtStartPar
En el ejemplo anterior \sphinxcode{\sphinxupquote{persona{[}"experiencia\_laboral"{]}{[}"2005\sphinxhyphen{}2008"{]}{[}"tareas"{]}}} es a
su vez una lista por lo que tiene a su ves las propiedades de estas.
Las siguientes lineas son válidas para la estructura recién definida.

\begin{sphinxVerbatim}[commandchars=\\\{\}]
\PYG{k}{for} \PYG{n}{tareas} \PYG{o+ow}{in} \PYG{n}{persona}\PYG{p}{[}\PYG{l+s+s2}{\PYGZdq{}}\PYG{l+s+s2}{experiencia\PYGZus{}laboral}\PYG{l+s+s2}{\PYGZdq{}}\PYG{p}{]}\PYG{p}{[}\PYG{l+s+s2}{\PYGZdq{}}\PYG{l+s+s2}{2005\PYGZhy{}2008}\PYG{l+s+s2}{\PYGZdq{}}\PYG{p}{]}\PYG{p}{[}\PYG{l+s+s2}{\PYGZdq{}}\PYG{l+s+s2}{tareas}\PYG{l+s+s2}{\PYGZdq{}}\PYG{p}{]}\PYG{p}{:}
    \PYG{n+nb}{print}\PYG{p}{(}\PYG{l+s+sa}{f}\PYG{l+s+s2}{\PYGZdq{}}\PYG{l+s+s2}{Tarea: }\PYG{l+s+si}{\PYGZob{}}\PYG{n}{tarea}\PYG{l+s+si}{\PYGZcb{}}\PYG{l+s+s2}{\PYGZdq{}}\PYG{p}{)}
\end{sphinxVerbatim}

\sphinxAtStartPar
También es posible usar \sphinxcode{\sphinxupquote{dict()}} para crear un diccionario.

\begin{sphinxVerbatim}[commandchars=\\\{\}]
\PYG{n}{d3} \PYG{o}{=} \PYG{n+nb}{dict}\PYG{p}{(}
    \PYG{n}{nombre}\PYG{o}{=}\PYG{l+s+s1}{\PYGZsq{}}\PYG{l+s+s1}{Laura}\PYG{l+s+s1}{\PYGZsq{}}\PYG{p}{,}
    \PYG{n}{edad}\PYG{o}{=}\PYG{l+m+mi}{47}\PYG{p}{,}
    \PYG{n}{documento}\PYG{o}{=}\PYG{l+m+mi}{221029489}
\PYG{p}{)}
\PYG{n+nb}{type}\PYG{p}{(}\PYG{n}{d3}\PYG{p}{)}
\PYG{c+c1}{\PYGZsh{} muestra \PYGZlt{}class \PYGZsq{}dict\PYGZsq{}\PYGZgt{}}
\PYG{n+nb}{print}\PYG{p}{(}\PYG{n}{d3}\PYG{p}{)}
\PYG{p}{\PYGZob{}}\PYG{l+s+s1}{\PYGZsq{}}\PYG{l+s+s1}{nombre}\PYG{l+s+s1}{\PYGZsq{}}\PYG{p}{:} \PYG{l+s+s1}{\PYGZsq{}}\PYG{l+s+s1}{Laura}\PYG{l+s+s1}{\PYGZsq{}}\PYG{p}{,} \PYG{l+s+s1}{\PYGZsq{}}\PYG{l+s+s1}{edad}\PYG{l+s+s1}{\PYGZsq{}}\PYG{p}{:} \PYG{l+m+mi}{47}\PYG{p}{,} \PYG{l+s+s1}{\PYGZsq{}}\PYG{l+s+s1}{documento}\PYG{l+s+s1}{\PYGZsq{}}\PYG{p}{:} \PYG{l+m+mi}{221029489}\PYG{p}{\PYGZcb{}}
\end{sphinxVerbatim}

\begin{sphinxadmonition}{note}{No confundir diccionarios con \sphinxstyleliteralintitle{\sphinxupquote{set}}}

\sphinxAtStartPar
En Python tambien se usan las llaves \sphinxcode{\sphinxupquote{\{\}}} para definir conjuntos (\sphinxstyleemphasis{sets}).

\sphinxAtStartPar
Por ejemplo \sphinxcode{\sphinxupquote{s = \{1, 2, 3, 4, 5\}}} no es un dictionario, es un \sphinxstyleemphasis{set} y
\sphinxcode{\sphinxupquote{type(s)}} mostrará \sphinxcode{\sphinxupquote{\textless{}class \textquotesingle{}set\textquotesingle{}\textgreater{}}}

\sphinxAtStartPar
Los \sphinxstyleemphasis{sets} son listas no ordenadas de elementos únicos (los duplicados se
eliminan automáticamente).
\end{sphinxadmonition}


\section{Tareas}
\label{\detokenize{dict:tareas}}\begin{itemize}
\item {} 
\sphinxAtStartPar
Hacer un PR con una propuesta de solución para el
\sphinxhref{https://github.com/avdata99/programacion-para-no-programadores/blob/master/ejercicios/ejercicio-020/ejercicio.py}{ejercicio 020}%
\begin{footnote}[14]\sphinxAtStartFootnote
\sphinxnolinkurl{https://github.com/avdata99/programacion-para-no-programadores/blob/master/ejercicios/ejercicio-020/ejercicio.py}
%
\end{footnote}

\end{itemize}

\begin{sphinxVerbatim}[commandchars=\\\{\}]
\PYG{l+s+sd}{\PYGZdq{}\PYGZdq{}\PYGZdq{}}
\PYG{l+s+sd}{Completar la funcion para que devuelva la \PYGZdq{}frase\PYGZdq{} pasada como parámetro}
\PYG{l+s+sd}{reemplazadas todas sus vocales con la \PYGZdq{}a\PYGZdq{} (o cualquier otra \PYGZdq{}vocal\PYGZdq{} que se}
\PYG{l+s+sd}{pase como parámetro)}
\PYG{l+s+sd}{\PYGZdq{}\PYGZdq{}\PYGZdq{}}


\PYG{k}{def} \PYG{n+nf}{cambia\PYGZus{}vocales}\PYG{p}{(}\PYG{n}{frase}\PYG{p}{,} \PYG{n}{vocal}\PYG{o}{=}\PYG{l+s+s2}{\PYGZdq{}}\PYG{l+s+s2}{a}\PYG{l+s+s2}{\PYGZdq{}}\PYG{p}{)}\PYG{p}{:}
    \PYG{k}{pass}


\PYG{c+c1}{\PYGZsh{} \PYGZhy{}\PYGZhy{}\PYGZhy{}\PYGZhy{}\PYGZhy{}\PYGZhy{}\PYGZhy{}\PYGZhy{}\PYGZhy{}\PYGZhy{}\PYGZhy{}\PYGZhy{}\PYGZhy{}\PYGZhy{}\PYGZhy{}\PYGZhy{}\PYGZhy{}\PYGZhy{}\PYGZhy{}\PYGZhy{}\PYGZhy{}\PYGZhy{}\PYGZhy{}\PYGZhy{}\PYGZhy{}\PYGZhy{}\PYGZhy{}\PYGZhy{}\PYGZhy{}\PYGZhy{}\PYGZhy{}\PYGZhy{}\PYGZhy{}\PYGZhy{}\PYGZhy{}\PYGZhy{}\PYGZhy{}\PYGZhy{}\PYGZhy{}\PYGZhy{}\PYGZhy{}\PYGZhy{}\PYGZhy{}\PYGZhy{}\PYGZhy{}\PYGZhy{}\PYGZhy{}\PYGZhy{}\PYGZhy{}\PYGZhy{}\PYGZhy{}\PYGZhy{}\PYGZhy{}\PYGZhy{}\PYGZhy{}\PYGZhy{}\PYGZhy{}\PYGZhy{}\PYGZhy{}\PYGZhy{}\PYGZhy{}\PYGZhy{}\PYGZhy{}\PYGZhy{}\PYGZhy{}\PYGZhy{}\PYGZhy{}\PYGZhy{}\PYGZhy{}\PYGZhy{}\PYGZhy{}\PYGZhy{}}
\PYG{c+c1}{\PYGZsh{} NO BORRAR O MODIFICAR LAS LINEAS QUE SIGUEN}
\PYG{c+c1}{\PYGZsh{} \PYGZhy{}\PYGZhy{}\PYGZhy{}\PYGZhy{}\PYGZhy{}\PYGZhy{}\PYGZhy{}\PYGZhy{}\PYGZhy{}\PYGZhy{}\PYGZhy{}\PYGZhy{}\PYGZhy{}\PYGZhy{}\PYGZhy{}\PYGZhy{}\PYGZhy{}\PYGZhy{}\PYGZhy{}\PYGZhy{}\PYGZhy{}\PYGZhy{}\PYGZhy{}\PYGZhy{}\PYGZhy{}\PYGZhy{}\PYGZhy{}\PYGZhy{}\PYGZhy{}\PYGZhy{}\PYGZhy{}\PYGZhy{}\PYGZhy{}\PYGZhy{}\PYGZhy{}\PYGZhy{}\PYGZhy{}\PYGZhy{}\PYGZhy{}\PYGZhy{}\PYGZhy{}\PYGZhy{}\PYGZhy{}\PYGZhy{}\PYGZhy{}\PYGZhy{}\PYGZhy{}\PYGZhy{}\PYGZhy{}\PYGZhy{}\PYGZhy{}\PYGZhy{}\PYGZhy{}\PYGZhy{}\PYGZhy{}\PYGZhy{}\PYGZhy{}\PYGZhy{}\PYGZhy{}\PYGZhy{}\PYGZhy{}\PYGZhy{}\PYGZhy{}\PYGZhy{}\PYGZhy{}\PYGZhy{}\PYGZhy{}\PYGZhy{}\PYGZhy{}\PYGZhy{}\PYGZhy{}\PYGZhy{}}
\PYG{c+c1}{\PYGZsh{} Una vez terminada la tarea ejecutar este archivo.}
\PYG{c+c1}{\PYGZsh{} Si se ve la leyenda \PYGZsq{}Ejercicio terminado OK\PYGZsq{} el ejercicio se considera completado.}
\PYG{c+c1}{\PYGZsh{} La instruccion \PYGZdq{}assert\PYGZdq{} de Python lanzará un error si lo que se indica a}
\PYG{c+c1}{\PYGZsh{}   continuacion es falso.}
\PYG{c+c1}{\PYGZsh{} Si usas GitHub (o similares) podes hacer una nueva rama con esta solución,}
\PYG{c+c1}{\PYGZsh{}   crear un \PYGZdq{}pull request\PYGZdq{} y solicitar revision de un tercero.}

\PYG{k}{assert} \PYG{n}{cambia\PYGZus{}vocales}\PYG{p}{(}\PYG{l+s+s2}{\PYGZdq{}}\PYG{l+s+s2}{hola}\PYG{l+s+s2}{\PYGZdq{}}\PYG{p}{)} \PYG{o}{==} \PYG{l+s+s2}{\PYGZdq{}}\PYG{l+s+s2}{hala}\PYG{l+s+s2}{\PYGZdq{}}
\PYG{k}{assert} \PYG{n}{cambia\PYGZus{}vocales}\PYG{p}{(}\PYG{l+s+s2}{\PYGZdq{}}\PYG{l+s+s2}{Juan Carlos}\PYG{l+s+s2}{\PYGZdq{}}\PYG{p}{)} \PYG{o}{==} \PYG{l+s+s2}{\PYGZdq{}}\PYG{l+s+s2}{Jaan Carlas}\PYG{l+s+s2}{\PYGZdq{}}
\PYG{k}{assert} \PYG{n}{cambia\PYGZus{}vocales}\PYG{p}{(}\PYG{l+s+s2}{\PYGZdq{}}\PYG{l+s+s2}{Pepito}\PYG{l+s+s2}{\PYGZdq{}}\PYG{p}{,} \PYG{l+s+s2}{\PYGZdq{}}\PYG{l+s+s2}{e}\PYG{l+s+s2}{\PYGZdq{}}\PYG{p}{)} \PYG{o}{==} \PYG{l+s+s2}{\PYGZdq{}}\PYG{l+s+s2}{Pepete}\PYG{l+s+s2}{\PYGZdq{}}
\PYG{k}{assert} \PYG{n}{cambia\PYGZus{}vocales}\PYG{p}{(}\PYG{n}{vocal}\PYG{o}{=}\PYG{l+s+s2}{\PYGZdq{}}\PYG{l+s+s2}{i}\PYG{l+s+s2}{\PYGZdq{}}\PYG{p}{,} \PYG{n}{frase}\PYG{o}{=}\PYG{l+s+s2}{\PYGZdq{}}\PYG{l+s+s2}{me llamo juan}\PYG{l+s+s2}{\PYGZdq{}}\PYG{p}{)} \PYG{o}{==} \PYG{l+s+s2}{\PYGZdq{}}\PYG{l+s+s2}{mi llimi jiin}\PYG{l+s+s2}{\PYGZdq{}}

\PYG{c+c1}{\PYGZsh{} revisar mayúsculas y minúsculas}
\PYG{k}{assert} \PYG{n}{cambia\PYGZus{}vocales}\PYG{p}{(}\PYG{l+s+s2}{\PYGZdq{}}\PYG{l+s+s2}{HOli}\PYG{l+s+s2}{\PYGZdq{}}\PYG{p}{)} \PYG{o}{==} \PYG{l+s+s2}{\PYGZdq{}}\PYG{l+s+s2}{HAla}\PYG{l+s+s2}{\PYGZdq{}}

\PYG{n+nb}{print}\PYG{p}{(}\PYG{l+s+s1}{\PYGZsq{}}\PYG{l+s+s1}{Ejercicio terminado OK}\PYG{l+s+s1}{\PYGZsq{}}\PYG{p}{)}
\end{sphinxVerbatim}
\begin{itemize}
\item {} 
\sphinxAtStartPar
Hacer un PR con una propuesta de solución para el
\sphinxhref{https://github.com/avdata99/programacion-para-no-programadores/blob/master/ejercicios/ejercicio-030/ejercicio.py}{ejercicio 030}%
\begin{footnote}[15]\sphinxAtStartFootnote
\sphinxnolinkurl{https://github.com/avdata99/programacion-para-no-programadores/blob/master/ejercicios/ejercicio-030/ejercicio.py}
%
\end{footnote}

\end{itemize}

\begin{sphinxVerbatim}[commandchars=\\\{\}]
\PYG{l+s+sd}{\PYGZdq{}\PYGZdq{}\PYGZdq{}}
\PYG{l+s+sd}{La siguiente funcion busca contar cuantas veces se repite cada palabra.}
\PYG{l+s+sd}{Por ejemplo de \PYGZdq{}Hola Juan. Hola Pedro\PYGZdq{} devuelve \PYGZob{}\PYGZdq{}Hola\PYGZdq{}: 2, \PYGZdq{}Juan.\PYGZdq{}: 1, \PYGZdq{}Pedro\PYGZdq{}: 1\PYGZcb{}}
\PYG{l+s+sd}{Tarea: Corregir el error de la función para que devuelva el resultado correcto.}
\PYG{l+s+sd}{Se espera que los signos de puntuacion no afecten el resultado y que las mayusculas}
\PYG{l+s+sd}{y minusculas no cuenten como palabras diferentes.}
\PYG{l+s+sd}{\PYGZdq{}\PYGZdq{}\PYGZdq{}}


\PYG{k}{def} \PYG{n+nf}{contar\PYGZus{}palabras}\PYG{p}{(}\PYG{n}{frase}\PYG{p}{)}\PYG{p}{:}
    \PYG{l+s+sd}{\PYGZdq{}\PYGZdq{}\PYGZdq{} }
\PYG{l+s+sd}{    Esta funcion toma una frase y devuelve un diccionario}
\PYG{l+s+sd}{    con una llave por cada palabra y un valor igual a la}
\PYG{l+s+sd}{    cantidad de veces que aparece en la frase}
\PYG{l+s+sd}{    \PYGZdq{}\PYGZdq{}\PYGZdq{}}
    \PYG{n}{palabras} \PYG{o}{=} \PYG{n}{frase}\PYG{o}{.}\PYG{n}{split}\PYG{p}{(}\PYG{p}{)}
    \PYG{n}{resultados} \PYG{o}{=} \PYG{p}{\PYGZob{}}\PYG{p}{\PYGZcb{}}
    \PYG{k}{for} \PYG{n}{palabra} \PYG{o+ow}{in} \PYG{n}{palabras}\PYG{p}{:}
        \PYG{k}{if} \PYG{n}{palabra} \PYG{o+ow}{in} \PYG{n}{resultados}\PYG{o}{.}\PYG{n}{keys}\PYG{p}{(}\PYG{p}{)}\PYG{p}{:}
            \PYG{c+c1}{\PYGZsh{} si existe, sumarle 1}
            \PYG{n}{resultados}\PYG{p}{[}\PYG{n}{palabra}\PYG{p}{]} \PYG{o}{+}\PYG{o}{=} \PYG{l+m+mi}{1}
        \PYG{k}{else}\PYG{p}{:}
            \PYG{c+c1}{\PYGZsh{} si no existe, inicializarla}
            \PYG{n}{resultados}\PYG{p}{[}\PYG{n}{palabra}\PYG{p}{]} \PYG{o}{=} \PYG{l+m+mi}{0}
    \PYG{k}{return} \PYG{n}{resultados}


\PYG{c+c1}{\PYGZsh{} \PYGZhy{}\PYGZhy{}\PYGZhy{}\PYGZhy{}\PYGZhy{}\PYGZhy{}\PYGZhy{}\PYGZhy{}\PYGZhy{}\PYGZhy{}\PYGZhy{}\PYGZhy{}\PYGZhy{}\PYGZhy{}\PYGZhy{}\PYGZhy{}\PYGZhy{}\PYGZhy{}\PYGZhy{}\PYGZhy{}\PYGZhy{}\PYGZhy{}\PYGZhy{}\PYGZhy{}\PYGZhy{}\PYGZhy{}\PYGZhy{}\PYGZhy{}\PYGZhy{}\PYGZhy{}\PYGZhy{}\PYGZhy{}\PYGZhy{}\PYGZhy{}\PYGZhy{}\PYGZhy{}\PYGZhy{}\PYGZhy{}\PYGZhy{}\PYGZhy{}\PYGZhy{}\PYGZhy{}\PYGZhy{}\PYGZhy{}\PYGZhy{}\PYGZhy{}\PYGZhy{}\PYGZhy{}\PYGZhy{}\PYGZhy{}\PYGZhy{}\PYGZhy{}\PYGZhy{}\PYGZhy{}\PYGZhy{}\PYGZhy{}\PYGZhy{}\PYGZhy{}\PYGZhy{}\PYGZhy{}\PYGZhy{}\PYGZhy{}\PYGZhy{}\PYGZhy{}\PYGZhy{}\PYGZhy{}\PYGZhy{}\PYGZhy{}\PYGZhy{}\PYGZhy{}\PYGZhy{}\PYGZhy{}}
\PYG{c+c1}{\PYGZsh{} NO BORRAR O MODIFICAR LAS LINEAS QUE SIGUEN}
\PYG{c+c1}{\PYGZsh{} \PYGZhy{}\PYGZhy{}\PYGZhy{}\PYGZhy{}\PYGZhy{}\PYGZhy{}\PYGZhy{}\PYGZhy{}\PYGZhy{}\PYGZhy{}\PYGZhy{}\PYGZhy{}\PYGZhy{}\PYGZhy{}\PYGZhy{}\PYGZhy{}\PYGZhy{}\PYGZhy{}\PYGZhy{}\PYGZhy{}\PYGZhy{}\PYGZhy{}\PYGZhy{}\PYGZhy{}\PYGZhy{}\PYGZhy{}\PYGZhy{}\PYGZhy{}\PYGZhy{}\PYGZhy{}\PYGZhy{}\PYGZhy{}\PYGZhy{}\PYGZhy{}\PYGZhy{}\PYGZhy{}\PYGZhy{}\PYGZhy{}\PYGZhy{}\PYGZhy{}\PYGZhy{}\PYGZhy{}\PYGZhy{}\PYGZhy{}\PYGZhy{}\PYGZhy{}\PYGZhy{}\PYGZhy{}\PYGZhy{}\PYGZhy{}\PYGZhy{}\PYGZhy{}\PYGZhy{}\PYGZhy{}\PYGZhy{}\PYGZhy{}\PYGZhy{}\PYGZhy{}\PYGZhy{}\PYGZhy{}\PYGZhy{}\PYGZhy{}\PYGZhy{}\PYGZhy{}\PYGZhy{}\PYGZhy{}\PYGZhy{}\PYGZhy{}\PYGZhy{}\PYGZhy{}\PYGZhy{}\PYGZhy{}}
\PYG{c+c1}{\PYGZsh{} Una vez terminada la tarea ejecutar este archivo.}
\PYG{c+c1}{\PYGZsh{} Si se ve la leyenda \PYGZsq{}Ejercicio terminado OK\PYGZsq{} el ejercicio se considera completado.}
\PYG{c+c1}{\PYGZsh{} La instruccion \PYGZdq{}assert\PYGZdq{} de Python lanzará un error si lo que se indica a}
\PYG{c+c1}{\PYGZsh{}   continuacion es falso.}
\PYG{c+c1}{\PYGZsh{} Si usas GitHub (o similares) podes hacer una nueva rama con esta solución,}
\PYG{c+c1}{\PYGZsh{}   crear un \PYGZdq{}pull request\PYGZdq{} y solicitar revision de un tercero.}


\PYG{n}{f1} \PYG{o}{=} \PYG{n}{contar\PYGZus{}palabras}\PYG{p}{(}\PYG{l+s+s2}{\PYGZdq{}}\PYG{l+s+s2}{Hola dijo Juan. Hola dijo pedro}\PYG{l+s+s2}{\PYGZdq{}}\PYG{p}{)}
\PYG{k}{assert} \PYG{n}{f1}\PYG{p}{[}\PYG{l+s+s1}{\PYGZsq{}}\PYG{l+s+s1}{Hola}\PYG{l+s+s1}{\PYGZsq{}}\PYG{p}{]} \PYG{o}{==} \PYG{l+m+mi}{2}

\PYG{n}{himno} \PYG{o}{=} \PYG{l+s+s2}{\PYGZdq{}}\PYG{l+s+s2}{Oid mortales el grito sagrado. Libertad, libertad, libertad. Oid el ruido de rotas cadenas}\PYG{l+s+s2}{\PYGZdq{}}
\PYG{n}{palabras\PYGZus{}himno} \PYG{o}{=} \PYG{n}{contar\PYGZus{}palabras}\PYG{p}{(}\PYG{n}{himno}\PYG{p}{)}
\PYG{k}{assert} \PYG{n}{palabras\PYGZus{}himno}\PYG{p}{[}\PYG{l+s+s1}{\PYGZsq{}}\PYG{l+s+s1}{Oid}\PYG{l+s+s1}{\PYGZsq{}}\PYG{p}{]} \PYG{o}{==} \PYG{l+m+mi}{2}
\PYG{k}{assert} \PYG{n}{palabras\PYGZus{}himno}\PYG{p}{[}\PYG{l+s+s1}{\PYGZsq{}}\PYG{l+s+s1}{libertad}\PYG{l+s+s1}{\PYGZsq{}}\PYG{p}{]} \PYG{o}{==} \PYG{l+m+mi}{3}

\PYG{n+nb}{print}\PYG{p}{(}\PYG{l+s+s1}{\PYGZsq{}}\PYG{l+s+s1}{Ejercicio terminado OK}\PYG{l+s+s1}{\PYGZsq{}}\PYG{p}{)}
\end{sphinxVerbatim}
\begin{itemize}
\item {} 
\sphinxAtStartPar
Hacer un PR con una propuesta de solución para el
\sphinxhref{https://github.com/avdata99/programacion-para-no-programadores/blob/master/ejercicios/ejercicio-031/ejercicio.py}{ejercicio 031}%
\begin{footnote}[16]\sphinxAtStartFootnote
\sphinxnolinkurl{https://github.com/avdata99/programacion-para-no-programadores/blob/master/ejercicios/ejercicio-031/ejercicio.py}
%
\end{footnote}

\end{itemize}

\begin{sphinxVerbatim}[commandchars=\\\{\}]
\PYG{l+s+sd}{\PYGZdq{}\PYGZdq{}\PYGZdq{}}
\PYG{l+s+sd}{La siguiente funcion toma como parámetro una lista de diccionarios}
\PYG{l+s+sd}{y cuenta la cantidad de elementos que tiene un valor especifico en}
\PYG{l+s+sd}{una propiedad definida.}
\PYG{l+s+sd}{Por ejemplo de la lista}
\PYG{l+s+sd}{lista = [}
\PYG{l+s+sd}{    \PYGZob{}\PYGZdq{}genero\PYGZdq{}: \PYGZdq{}M\PYGZdq{}, \PYGZdq{}nombre\PYGZdq{}: \PYGZdq{}Juan\PYGZdq{}\PYGZcb{},}
\PYG{l+s+sd}{    \PYGZob{}\PYGZdq{}genero\PYGZdq{}: \PYGZdq{}F\PYGZdq{}, \PYGZdq{}nombre\PYGZdq{}: \PYGZdq{}Pablo\PYGZdq{}\PYGZcb{},}
\PYG{l+s+sd}{    \PYGZob{}\PYGZdq{}genero\PYGZdq{}: \PYGZdq{}F\PYGZdq{}, \PYGZdq{}nombre\PYGZdq{}: \PYGZdq{}Juana\PYGZdq{}, \PYGZdq{}apellido\PYGZdq{}: \PYGZdq{}Gomez\PYGZdq{}\PYGZcb{},}
\PYG{l+s+sd}{    \PYGZob{}\PYGZdq{}genero\PYGZdq{}: \PYGZdq{}M\PYGZdq{}, \PYGZdq{}nombre\PYGZdq{}: \PYGZdq{}Victor\PYGZdq{}\PYGZcb{},}
\PYG{l+s+sd}{    \PYGZob{}\PYGZdq{}genero\PYGZdq{}: \PYGZdq{}M\PYGZdq{}, \PYGZdq{}nombre\PYGZdq{}: \PYGZdq{}Juan Pablo\PYGZdq{}, \PYGZdq{}apellido\PYGZdq{}: \PYGZdq{}Velez\PYGZdq{}\PYGZcb{},}
\PYG{l+s+sd}{    \PYGZob{}\PYGZdq{}genero\PYGZdq{}: \PYGZdq{}F\PYGZdq{}, \PYGZdq{}nombre\PYGZdq{}: \PYGZdq{}Juana\PYGZdq{}\PYGZcb{},}
\PYG{l+s+sd}{    \PYGZob{}\PYGZdq{}genero\PYGZdq{}: \PYGZdq{}F\PYGZdq{}, \PYGZdq{}nombre\PYGZdq{}: \PYGZdq{}Victoria\PYGZdq{}\PYGZcb{}}
\PYG{l+s+sd}{]}
\PYG{l+s+sd}{Se esperan estos posibles resultados}

\PYG{l+s+sd}{contar\PYGZus{}si(lista, \PYGZdq{}genero\PYGZdq{}, \PYGZdq{}M\PYGZdq{}) == 3}
\PYG{l+s+sd}{contar\PYGZus{}si(lista, \PYGZdq{}genero\PYGZdq{}, \PYGZdq{}F\PYGZdq{}) == 4}
\PYG{l+s+sd}{contar\PYGZus{}si(lista, \PYGZdq{}nombre\PYGZdq{}, \PYGZdq{}Juana\PYGZdq{}) == 2}

\PYG{l+s+sd}{pero la funcion da error en algunos casos como}
\PYG{l+s+sd}{contar\PYGZus{}si(lista, \PYGZdq{}apellido\PYGZdq{}, \PYGZdq{}Gomez\PYGZdq{})}
\PYG{l+s+sd}{donde en realidad esperamos que devuelva 1}

\PYG{l+s+sd}{Tarea: Mejorar la función para que no de errores cuando una clave no existe}
\PYG{l+s+sd}{\PYGZdq{}\PYGZdq{}\PYGZdq{}}


\PYG{k}{def} \PYG{n+nf}{contar\PYGZus{}si}\PYG{p}{(}\PYG{n}{lista}\PYG{p}{,} \PYG{n}{propiedad}\PYG{p}{,} \PYG{n}{valor}\PYG{p}{)}\PYG{p}{:}
    \PYG{l+s+sd}{\PYGZdq{}\PYGZdq{}\PYGZdq{} }
\PYG{l+s+sd}{    Esta funcion cuenta la cantidad de elementos (diccionarios) en \PYGZdq{}lista\PYGZdq{}}
\PYG{l+s+sd}{    que tiene una \PYGZdq{}propiedad\PYGZdq{} con un \PYGZdq{}valor\PYGZdq{} específico.}
\PYG{l+s+sd}{    \PYGZdq{}\PYGZdq{}\PYGZdq{}}
    \PYG{n}{contador} \PYG{o}{=} \PYG{l+m+mi}{0}
    \PYG{k}{for} \PYG{n}{elemento} \PYG{o+ow}{in} \PYG{n}{lista}\PYG{p}{:}
        \PYG{k}{if} \PYG{n}{elemento}\PYG{p}{[}\PYG{n}{propiedad}\PYG{p}{]} \PYG{o}{==} \PYG{n}{valor}\PYG{p}{:}
            \PYG{n}{contador} \PYG{o}{+}\PYG{o}{=} \PYG{l+m+mi}{1}

    \PYG{k}{return} \PYG{n}{contador}

\PYG{c+c1}{\PYGZsh{} \PYGZhy{}\PYGZhy{}\PYGZhy{}\PYGZhy{}\PYGZhy{}\PYGZhy{}\PYGZhy{}\PYGZhy{}\PYGZhy{}\PYGZhy{}\PYGZhy{}\PYGZhy{}\PYGZhy{}\PYGZhy{}\PYGZhy{}\PYGZhy{}\PYGZhy{}\PYGZhy{}\PYGZhy{}\PYGZhy{}\PYGZhy{}\PYGZhy{}\PYGZhy{}\PYGZhy{}\PYGZhy{}\PYGZhy{}\PYGZhy{}\PYGZhy{}\PYGZhy{}\PYGZhy{}\PYGZhy{}\PYGZhy{}\PYGZhy{}\PYGZhy{}\PYGZhy{}\PYGZhy{}\PYGZhy{}\PYGZhy{}\PYGZhy{}\PYGZhy{}\PYGZhy{}\PYGZhy{}\PYGZhy{}\PYGZhy{}\PYGZhy{}\PYGZhy{}\PYGZhy{}\PYGZhy{}\PYGZhy{}\PYGZhy{}\PYGZhy{}\PYGZhy{}\PYGZhy{}\PYGZhy{}\PYGZhy{}\PYGZhy{}\PYGZhy{}\PYGZhy{}\PYGZhy{}\PYGZhy{}\PYGZhy{}\PYGZhy{}\PYGZhy{}\PYGZhy{}\PYGZhy{}\PYGZhy{}\PYGZhy{}\PYGZhy{}\PYGZhy{}\PYGZhy{}\PYGZhy{}\PYGZhy{}}
\PYG{c+c1}{\PYGZsh{} NO BORRAR O MODIFICAR LAS LINEAS QUE SIGUEN}
\PYG{c+c1}{\PYGZsh{} \PYGZhy{}\PYGZhy{}\PYGZhy{}\PYGZhy{}\PYGZhy{}\PYGZhy{}\PYGZhy{}\PYGZhy{}\PYGZhy{}\PYGZhy{}\PYGZhy{}\PYGZhy{}\PYGZhy{}\PYGZhy{}\PYGZhy{}\PYGZhy{}\PYGZhy{}\PYGZhy{}\PYGZhy{}\PYGZhy{}\PYGZhy{}\PYGZhy{}\PYGZhy{}\PYGZhy{}\PYGZhy{}\PYGZhy{}\PYGZhy{}\PYGZhy{}\PYGZhy{}\PYGZhy{}\PYGZhy{}\PYGZhy{}\PYGZhy{}\PYGZhy{}\PYGZhy{}\PYGZhy{}\PYGZhy{}\PYGZhy{}\PYGZhy{}\PYGZhy{}\PYGZhy{}\PYGZhy{}\PYGZhy{}\PYGZhy{}\PYGZhy{}\PYGZhy{}\PYGZhy{}\PYGZhy{}\PYGZhy{}\PYGZhy{}\PYGZhy{}\PYGZhy{}\PYGZhy{}\PYGZhy{}\PYGZhy{}\PYGZhy{}\PYGZhy{}\PYGZhy{}\PYGZhy{}\PYGZhy{}\PYGZhy{}\PYGZhy{}\PYGZhy{}\PYGZhy{}\PYGZhy{}\PYGZhy{}\PYGZhy{}\PYGZhy{}\PYGZhy{}\PYGZhy{}\PYGZhy{}\PYGZhy{}}
\PYG{c+c1}{\PYGZsh{} Una vez terminada la tarea ejecutar este archivo.}
\PYG{c+c1}{\PYGZsh{} Si se ve la leyenda \PYGZsq{}Ejercicio terminado OK\PYGZsq{} el ejercicio se considera completado.}
\PYG{c+c1}{\PYGZsh{} La instruccion \PYGZdq{}assert\PYGZdq{} de Python lanzará un error si lo que se indica a}
\PYG{c+c1}{\PYGZsh{}   continuacion es falso.}
\PYG{c+c1}{\PYGZsh{} Si usas GitHub (o similares) podes hacer una nueva rama con esta solución,}
\PYG{c+c1}{\PYGZsh{}   crear un \PYGZdq{}pull request\PYGZdq{} y solicitar revision de un tercero.}


\PYG{n}{lista} \PYG{o}{=} \PYG{p}{[}
    \PYG{p}{\PYGZob{}}\PYG{l+s+s2}{\PYGZdq{}}\PYG{l+s+s2}{genero}\PYG{l+s+s2}{\PYGZdq{}}\PYG{p}{:} \PYG{l+s+s2}{\PYGZdq{}}\PYG{l+s+s2}{M}\PYG{l+s+s2}{\PYGZdq{}}\PYG{p}{,} \PYG{l+s+s2}{\PYGZdq{}}\PYG{l+s+s2}{nombre}\PYG{l+s+s2}{\PYGZdq{}}\PYG{p}{:} \PYG{l+s+s2}{\PYGZdq{}}\PYG{l+s+s2}{Juan}\PYG{l+s+s2}{\PYGZdq{}}\PYG{p}{\PYGZcb{}}\PYG{p}{,}
    \PYG{p}{\PYGZob{}}\PYG{l+s+s2}{\PYGZdq{}}\PYG{l+s+s2}{genero}\PYG{l+s+s2}{\PYGZdq{}}\PYG{p}{:} \PYG{l+s+s2}{\PYGZdq{}}\PYG{l+s+s2}{F}\PYG{l+s+s2}{\PYGZdq{}}\PYG{p}{,} \PYG{l+s+s2}{\PYGZdq{}}\PYG{l+s+s2}{nombre}\PYG{l+s+s2}{\PYGZdq{}}\PYG{p}{:} \PYG{l+s+s2}{\PYGZdq{}}\PYG{l+s+s2}{Pablo}\PYG{l+s+s2}{\PYGZdq{}}\PYG{p}{\PYGZcb{}}\PYG{p}{,}
    \PYG{p}{\PYGZob{}}\PYG{l+s+s2}{\PYGZdq{}}\PYG{l+s+s2}{genero}\PYG{l+s+s2}{\PYGZdq{}}\PYG{p}{:} \PYG{l+s+s2}{\PYGZdq{}}\PYG{l+s+s2}{F}\PYG{l+s+s2}{\PYGZdq{}}\PYG{p}{,} \PYG{l+s+s2}{\PYGZdq{}}\PYG{l+s+s2}{nombre}\PYG{l+s+s2}{\PYGZdq{}}\PYG{p}{:} \PYG{l+s+s2}{\PYGZdq{}}\PYG{l+s+s2}{Juana}\PYG{l+s+s2}{\PYGZdq{}}\PYG{p}{,} \PYG{l+s+s2}{\PYGZdq{}}\PYG{l+s+s2}{apellido}\PYG{l+s+s2}{\PYGZdq{}}\PYG{p}{:} \PYG{l+s+s2}{\PYGZdq{}}\PYG{l+s+s2}{Gomez}\PYG{l+s+s2}{\PYGZdq{}}\PYG{p}{\PYGZcb{}}\PYG{p}{,}
    \PYG{p}{\PYGZob{}}\PYG{l+s+s2}{\PYGZdq{}}\PYG{l+s+s2}{genero}\PYG{l+s+s2}{\PYGZdq{}}\PYG{p}{:} \PYG{l+s+s2}{\PYGZdq{}}\PYG{l+s+s2}{M}\PYG{l+s+s2}{\PYGZdq{}}\PYG{p}{,} \PYG{l+s+s2}{\PYGZdq{}}\PYG{l+s+s2}{nombre}\PYG{l+s+s2}{\PYGZdq{}}\PYG{p}{:} \PYG{l+s+s2}{\PYGZdq{}}\PYG{l+s+s2}{Victor}\PYG{l+s+s2}{\PYGZdq{}}\PYG{p}{\PYGZcb{}}\PYG{p}{,}
    \PYG{p}{\PYGZob{}}\PYG{l+s+s2}{\PYGZdq{}}\PYG{l+s+s2}{genero}\PYG{l+s+s2}{\PYGZdq{}}\PYG{p}{:} \PYG{l+s+s2}{\PYGZdq{}}\PYG{l+s+s2}{M}\PYG{l+s+s2}{\PYGZdq{}}\PYG{p}{,} \PYG{l+s+s2}{\PYGZdq{}}\PYG{l+s+s2}{nombre}\PYG{l+s+s2}{\PYGZdq{}}\PYG{p}{:} \PYG{l+s+s2}{\PYGZdq{}}\PYG{l+s+s2}{Juan Pablo}\PYG{l+s+s2}{\PYGZdq{}}\PYG{p}{,} \PYG{l+s+s2}{\PYGZdq{}}\PYG{l+s+s2}{apellido}\PYG{l+s+s2}{\PYGZdq{}}\PYG{p}{:} \PYG{l+s+s2}{\PYGZdq{}}\PYG{l+s+s2}{Velez}\PYG{l+s+s2}{\PYGZdq{}}\PYG{p}{\PYGZcb{}}\PYG{p}{,}
    \PYG{p}{\PYGZob{}}\PYG{l+s+s2}{\PYGZdq{}}\PYG{l+s+s2}{genero}\PYG{l+s+s2}{\PYGZdq{}}\PYG{p}{:} \PYG{l+s+s2}{\PYGZdq{}}\PYG{l+s+s2}{F}\PYG{l+s+s2}{\PYGZdq{}}\PYG{p}{,} \PYG{l+s+s2}{\PYGZdq{}}\PYG{l+s+s2}{nombre}\PYG{l+s+s2}{\PYGZdq{}}\PYG{p}{:} \PYG{l+s+s2}{\PYGZdq{}}\PYG{l+s+s2}{Juana}\PYG{l+s+s2}{\PYGZdq{}}\PYG{p}{\PYGZcb{}}\PYG{p}{,}
    \PYG{p}{\PYGZob{}}\PYG{l+s+s2}{\PYGZdq{}}\PYG{l+s+s2}{genero}\PYG{l+s+s2}{\PYGZdq{}}\PYG{p}{:} \PYG{l+s+s2}{\PYGZdq{}}\PYG{l+s+s2}{F}\PYG{l+s+s2}{\PYGZdq{}}\PYG{p}{,} \PYG{l+s+s2}{\PYGZdq{}}\PYG{l+s+s2}{nombre}\PYG{l+s+s2}{\PYGZdq{}}\PYG{p}{:} \PYG{l+s+s2}{\PYGZdq{}}\PYG{l+s+s2}{Victoria}\PYG{l+s+s2}{\PYGZdq{}}\PYG{p}{\PYGZcb{}}
\PYG{p}{]}

\PYG{k}{assert} \PYG{n}{contar\PYGZus{}si}\PYG{p}{(}\PYG{n}{lista}\PYG{p}{,} \PYG{l+s+s2}{\PYGZdq{}}\PYG{l+s+s2}{genero}\PYG{l+s+s2}{\PYGZdq{}}\PYG{p}{,} \PYG{l+s+s2}{\PYGZdq{}}\PYG{l+s+s2}{M}\PYG{l+s+s2}{\PYGZdq{}}\PYG{p}{)} \PYG{o}{==} \PYG{l+m+mi}{3}
\PYG{k}{assert} \PYG{n}{contar\PYGZus{}si}\PYG{p}{(}\PYG{n}{lista}\PYG{p}{,} \PYG{l+s+s2}{\PYGZdq{}}\PYG{l+s+s2}{genero}\PYG{l+s+s2}{\PYGZdq{}}\PYG{p}{,} \PYG{l+s+s2}{\PYGZdq{}}\PYG{l+s+s2}{F}\PYG{l+s+s2}{\PYGZdq{}}\PYG{p}{)} \PYG{o}{==} \PYG{l+m+mi}{4}
\PYG{k}{assert} \PYG{n}{contar\PYGZus{}si}\PYG{p}{(}\PYG{n}{lista}\PYG{p}{,} \PYG{l+s+s2}{\PYGZdq{}}\PYG{l+s+s2}{nombre}\PYG{l+s+s2}{\PYGZdq{}}\PYG{p}{,} \PYG{l+s+s2}{\PYGZdq{}}\PYG{l+s+s2}{Juana}\PYG{l+s+s2}{\PYGZdq{}}\PYG{p}{)} \PYG{o}{==} \PYG{l+m+mi}{2}
\PYG{k}{assert} \PYG{n}{contar\PYGZus{}si}\PYG{p}{(}\PYG{n}{lista}\PYG{p}{,} \PYG{l+s+s2}{\PYGZdq{}}\PYG{l+s+s2}{apellido}\PYG{l+s+s2}{\PYGZdq{}}\PYG{p}{,} \PYG{l+s+s2}{\PYGZdq{}}\PYG{l+s+s2}{Gomez}\PYG{l+s+s2}{\PYGZdq{}}\PYG{p}{)} \PYG{o}{==} \PYG{l+m+mi}{1}
\PYG{k}{assert} \PYG{n}{contar\PYGZus{}si}\PYG{p}{(}\PYG{n}{lista}\PYG{p}{,} \PYG{l+s+s2}{\PYGZdq{}}\PYG{l+s+s2}{apellido}\PYG{l+s+s2}{\PYGZdq{}}\PYG{p}{,} \PYG{l+s+s2}{\PYGZdq{}}\PYG{l+s+s2}{Perez}\PYG{l+s+s2}{\PYGZdq{}}\PYG{p}{)} \PYG{o}{==} \PYG{l+m+mi}{0}

\PYG{n+nb}{print}\PYG{p}{(}\PYG{l+s+s1}{\PYGZsq{}}\PYG{l+s+s1}{Ejercicio terminado OK}\PYG{l+s+s1}{\PYGZsq{}}\PYG{p}{)}
\end{sphinxVerbatim}
\begin{itemize}
\item {} 
\sphinxAtStartPar
Hacer un PR con una propuesta de solución para el
\sphinxhref{https://github.com/avdata99/programacion-para-no-programadores/blob/master/ejercicios/ejercicio-032/ejercicio.py}{ejercicio 032}%
\begin{footnote}[17]\sphinxAtStartFootnote
\sphinxnolinkurl{https://github.com/avdata99/programacion-para-no-programadores/blob/master/ejercicios/ejercicio-032/ejercicio.py}
%
\end{footnote}

\end{itemize}

\begin{sphinxVerbatim}[commandchars=\\\{\}]
\PYG{l+s+sd}{\PYGZdq{}\PYGZdq{}\PYGZdq{}}
\PYG{l+s+sd}{El siguiente código permite crear facturas de ventas y agregar items.}
\PYG{l+s+sd}{El código no tiene fallas pero el cliente desea que la lista de items}
\PYG{l+s+sd}{no tenga productos duplicados. Si se intenta agregar un producto por}
\PYG{l+s+sd}{segunda vez, la función deberia darse cuenta y actualizar los valores}
\PYG{l+s+sd}{de ese item y no agregarlo como nuevo item.}
\PYG{l+s+sd}{\PYGZdq{}\PYGZdq{}\PYGZdq{}}


\PYG{k}{def} \PYG{n+nf}{agregar\PYGZus{}item}\PYG{p}{(}\PYG{n}{factura}\PYG{p}{,} \PYG{n}{producto}\PYG{p}{,} \PYG{n}{precio\PYGZus{}unitario}\PYG{p}{,} \PYG{n}{cantidad}\PYG{o}{=}\PYG{l+m+mi}{1}\PYG{p}{)}\PYG{p}{:}
    \PYG{l+s+sd}{\PYGZdq{}\PYGZdq{}\PYGZdq{}}
\PYG{l+s+sd}{    Agregar un item a una factura que se pasa como parámetro}
\PYG{l+s+sd}{    \PYGZdq{}\PYGZdq{}\PYGZdq{}}
    \PYG{n}{precio\PYGZus{}total} \PYG{o}{=} \PYG{n}{cantidad} \PYG{o}{*} \PYG{n}{precio\PYGZus{}unitario}
    \PYG{n}{item} \PYG{o}{=} \PYG{p}{\PYGZob{}}
        \PYG{l+s+s1}{\PYGZsq{}}\PYG{l+s+s1}{producto}\PYG{l+s+s1}{\PYGZsq{}}\PYG{p}{:} \PYG{n}{producto}\PYG{p}{,}
        \PYG{l+s+s1}{\PYGZsq{}}\PYG{l+s+s1}{precio\PYGZus{}unitario}\PYG{l+s+s1}{\PYGZsq{}}\PYG{p}{:} \PYG{n}{precio\PYGZus{}unitario}\PYG{p}{,}
        \PYG{l+s+s1}{\PYGZsq{}}\PYG{l+s+s1}{cantidad}\PYG{l+s+s1}{\PYGZsq{}}\PYG{p}{:} \PYG{n}{cantidad}\PYG{p}{,}
        \PYG{l+s+s1}{\PYGZsq{}}\PYG{l+s+s1}{precio\PYGZus{}total}\PYG{l+s+s1}{\PYGZsq{}}\PYG{p}{:} \PYG{n}{precio\PYGZus{}total}\PYG{p}{,}
    \PYG{p}{\PYGZcb{}}

    \PYG{n}{factura}\PYG{p}{[}\PYG{l+s+s1}{\PYGZsq{}}\PYG{l+s+s1}{items}\PYG{l+s+s1}{\PYGZsq{}}\PYG{p}{]}\PYG{o}{.}\PYG{n}{append}\PYG{p}{(}\PYG{n}{item}\PYG{p}{)}
    \PYG{n}{factura}\PYG{p}{[}\PYG{l+s+s1}{\PYGZsq{}}\PYG{l+s+s1}{total}\PYG{l+s+s1}{\PYGZsq{}}\PYG{p}{]} \PYG{o}{+}\PYG{o}{=} \PYG{n}{precio\PYGZus{}total}

\PYG{k}{def} \PYG{n+nf}{crear\PYGZus{}factura}\PYG{p}{(}\PYG{p}{)}\PYG{p}{:}
    \PYG{l+s+sd}{\PYGZdq{}\PYGZdq{}\PYGZdq{} Crear una factura \PYGZdq{}\PYGZdq{}\PYGZdq{}}
    \PYG{n}{factura} \PYG{o}{=} \PYG{p}{\PYGZob{}}
        \PYG{l+s+s1}{\PYGZsq{}}\PYG{l+s+s1}{total}\PYG{l+s+s1}{\PYGZsq{}}\PYG{p}{:} \PYG{l+m+mi}{0}\PYG{p}{,}
        \PYG{l+s+s1}{\PYGZsq{}}\PYG{l+s+s1}{items}\PYG{l+s+s1}{\PYGZsq{}}\PYG{p}{:} \PYG{p}{[}\PYG{p}{]}  \PYG{c+c1}{\PYGZsh{} lista de items}
    \PYG{p}{\PYGZcb{}}

    \PYG{k}{return} \PYG{n}{factura}

\PYG{n}{mi\PYGZus{}factura} \PYG{o}{=} \PYG{n}{crear\PYGZus{}factura}\PYG{p}{(}\PYG{p}{)}
\PYG{n}{agregar\PYGZus{}item}\PYG{p}{(}\PYG{n}{mi\PYGZus{}factura}\PYG{p}{,} \PYG{l+s+s1}{\PYGZsq{}}\PYG{l+s+s1}{Alfajor}\PYG{l+s+s1}{\PYGZsq{}}\PYG{p}{,} \PYG{l+m+mi}{150}\PYG{p}{,} \PYG{l+m+mi}{3}\PYG{p}{)}
\PYG{n}{agregar\PYGZus{}item}\PYG{p}{(}\PYG{n}{mi\PYGZus{}factura}\PYG{p}{,} \PYG{l+s+s1}{\PYGZsq{}}\PYG{l+s+s1}{Turron}\PYG{l+s+s1}{\PYGZsq{}}\PYG{p}{,} \PYG{l+m+mi}{53}\PYG{p}{)}
\PYG{n}{agregar\PYGZus{}item}\PYG{p}{(}\PYG{n}{mi\PYGZus{}factura}\PYG{p}{,} \PYG{l+s+s1}{\PYGZsq{}}\PYG{l+s+s1}{Turron}\PYG{l+s+s1}{\PYGZsq{}}\PYG{p}{,} \PYG{l+m+mi}{53}\PYG{p}{)}


\PYG{c+c1}{\PYGZsh{} \PYGZhy{}\PYGZhy{}\PYGZhy{}\PYGZhy{}\PYGZhy{}\PYGZhy{}\PYGZhy{}\PYGZhy{}\PYGZhy{}\PYGZhy{}\PYGZhy{}\PYGZhy{}\PYGZhy{}\PYGZhy{}\PYGZhy{}\PYGZhy{}\PYGZhy{}\PYGZhy{}\PYGZhy{}\PYGZhy{}\PYGZhy{}\PYGZhy{}\PYGZhy{}\PYGZhy{}\PYGZhy{}\PYGZhy{}\PYGZhy{}\PYGZhy{}\PYGZhy{}\PYGZhy{}\PYGZhy{}\PYGZhy{}\PYGZhy{}\PYGZhy{}\PYGZhy{}\PYGZhy{}\PYGZhy{}\PYGZhy{}\PYGZhy{}\PYGZhy{}\PYGZhy{}\PYGZhy{}\PYGZhy{}\PYGZhy{}\PYGZhy{}\PYGZhy{}\PYGZhy{}\PYGZhy{}\PYGZhy{}\PYGZhy{}\PYGZhy{}\PYGZhy{}\PYGZhy{}\PYGZhy{}\PYGZhy{}\PYGZhy{}\PYGZhy{}\PYGZhy{}\PYGZhy{}\PYGZhy{}\PYGZhy{}\PYGZhy{}\PYGZhy{}\PYGZhy{}\PYGZhy{}\PYGZhy{}\PYGZhy{}\PYGZhy{}\PYGZhy{}\PYGZhy{}\PYGZhy{}\PYGZhy{}}
\PYG{c+c1}{\PYGZsh{} NO BORRAR O MODIFICAR LAS LINEAS QUE SIGUEN}
\PYG{c+c1}{\PYGZsh{} \PYGZhy{}\PYGZhy{}\PYGZhy{}\PYGZhy{}\PYGZhy{}\PYGZhy{}\PYGZhy{}\PYGZhy{}\PYGZhy{}\PYGZhy{}\PYGZhy{}\PYGZhy{}\PYGZhy{}\PYGZhy{}\PYGZhy{}\PYGZhy{}\PYGZhy{}\PYGZhy{}\PYGZhy{}\PYGZhy{}\PYGZhy{}\PYGZhy{}\PYGZhy{}\PYGZhy{}\PYGZhy{}\PYGZhy{}\PYGZhy{}\PYGZhy{}\PYGZhy{}\PYGZhy{}\PYGZhy{}\PYGZhy{}\PYGZhy{}\PYGZhy{}\PYGZhy{}\PYGZhy{}\PYGZhy{}\PYGZhy{}\PYGZhy{}\PYGZhy{}\PYGZhy{}\PYGZhy{}\PYGZhy{}\PYGZhy{}\PYGZhy{}\PYGZhy{}\PYGZhy{}\PYGZhy{}\PYGZhy{}\PYGZhy{}\PYGZhy{}\PYGZhy{}\PYGZhy{}\PYGZhy{}\PYGZhy{}\PYGZhy{}\PYGZhy{}\PYGZhy{}\PYGZhy{}\PYGZhy{}\PYGZhy{}\PYGZhy{}\PYGZhy{}\PYGZhy{}\PYGZhy{}\PYGZhy{}\PYGZhy{}\PYGZhy{}\PYGZhy{}\PYGZhy{}\PYGZhy{}\PYGZhy{}}
\PYG{c+c1}{\PYGZsh{} Una vez terminada la tarea ejecutar este archivo.}
\PYG{c+c1}{\PYGZsh{} Si se ve la leyenda \PYGZsq{}Ejercicio terminado OK\PYGZsq{} el ejercicio se considera completado.}
\PYG{c+c1}{\PYGZsh{} La instruccion \PYGZdq{}assert\PYGZdq{} de Python lanzará un error si lo que se indica a}
\PYG{c+c1}{\PYGZsh{}   continuacion es falso.}
\PYG{c+c1}{\PYGZsh{} Si usas GitHub (o similares) podes hacer una nueva rama con esta solución,}
\PYG{c+c1}{\PYGZsh{}   crear un \PYGZdq{}pull request\PYGZdq{} y solicitar revision de un tercero.}

\PYG{k}{assert} \PYG{n}{mi\PYGZus{}factura}\PYG{p}{[}\PYG{l+s+s1}{\PYGZsq{}}\PYG{l+s+s1}{total}\PYG{l+s+s1}{\PYGZsq{}}\PYG{p}{]} \PYG{o}{==} \PYG{l+m+mi}{556}

\PYG{n}{items\PYGZus{}en\PYGZus{}factura} \PYG{o}{=} \PYG{n}{mi\PYGZus{}factura}\PYG{p}{[}\PYG{l+s+s1}{\PYGZsq{}}\PYG{l+s+s1}{items}\PYG{l+s+s1}{\PYGZsq{}}\PYG{p}{]}
\PYG{k}{assert} \PYG{n+nb}{len}\PYG{p}{(}\PYG{n}{items\PYGZus{}en\PYGZus{}factura}\PYG{p}{)} \PYG{o}{==} \PYG{l+m+mi}{2}

\PYG{n+nb}{print}\PYG{p}{(}\PYG{l+s+s1}{\PYGZsq{}}\PYG{l+s+s1}{Ejercicio terminado OK}\PYG{l+s+s1}{\PYGZsq{}}\PYG{p}{)}
\end{sphinxVerbatim}
\begin{itemize}
\item {} 
\sphinxAtStartPar
Hacer un PR con una propuesta de solución para el
\sphinxhref{https://github.com/avdata99/programacion-para-no-programadores/blob/master/ejercicios/ejercicio-041/ejercicio.py}{ejercicio 041}%
\begin{footnote}[18]\sphinxAtStartFootnote
\sphinxnolinkurl{https://github.com/avdata99/programacion-para-no-programadores/blob/master/ejercicios/ejercicio-041/ejercicio.py}
%
\end{footnote}

\end{itemize}

\begin{sphinxVerbatim}[commandchars=\\\{\}]
\PYG{l+s+sd}{\PYGZdq{}\PYGZdq{}\PYGZdq{}}
\PYG{l+s+sd}{La funcion \PYGZdq{}crear\PYGZus{}mazo\PYGZus{}cartas\PYGZus{}espaniolas\PYGZdq{} funciona casi bien.}
\PYG{l+s+sd}{Por algun motivo faltan algunas cartas.}
\PYG{l+s+sd}{La tarea de este ejercicio es reparar esta función para que el mazo este completo}
\PYG{l+s+sd}{\PYGZdq{}\PYGZdq{}\PYGZdq{}}


\PYG{k}{def} \PYG{n+nf}{crear\PYGZus{}mazo\PYGZus{}cartas\PYGZus{}espaniolas}\PYG{p}{(}\PYG{p}{)}\PYG{p}{:}
    \PYG{n}{palos} \PYG{o}{=} \PYG{p}{[}\PYG{l+s+s1}{\PYGZsq{}}\PYG{l+s+s1}{oro}\PYG{l+s+s1}{\PYGZsq{}}\PYG{p}{,} \PYG{l+s+s1}{\PYGZsq{}}\PYG{l+s+s1}{copa}\PYG{l+s+s1}{\PYGZsq{}}\PYG{p}{,} \PYG{l+s+s1}{\PYGZsq{}}\PYG{l+s+s1}{espada}\PYG{l+s+s1}{\PYGZsq{}}\PYG{p}{,} \PYG{l+s+s1}{\PYGZsq{}}\PYG{l+s+s1}{basto}\PYG{l+s+s1}{\PYGZsq{}}\PYG{p}{]}
    \PYG{n}{mazo} \PYG{o}{=} \PYG{p}{[}\PYG{p}{]}
    \PYG{k}{for} \PYG{n}{n} \PYG{o+ow}{in} \PYG{n+nb}{range}\PYG{p}{(}\PYG{l+m+mi}{1}\PYG{p}{,} \PYG{l+m+mi}{12}\PYG{p}{)}\PYG{p}{:}
        \PYG{k}{for} \PYG{n}{palo} \PYG{o+ow}{in} \PYG{n}{palos}\PYG{p}{:}
            \PYG{n}{carta} \PYG{o}{=} \PYG{p}{\PYGZob{}}\PYG{l+s+s1}{\PYGZsq{}}\PYG{l+s+s1}{numero}\PYG{l+s+s1}{\PYGZsq{}}\PYG{p}{:} \PYG{n}{n}\PYG{p}{,} \PYG{l+s+s1}{\PYGZsq{}}\PYG{l+s+s1}{palo}\PYG{l+s+s1}{\PYGZsq{}}\PYG{p}{:} \PYG{n}{palo}\PYG{p}{\PYGZcb{}}
            \PYG{n}{mazo}\PYG{o}{.}\PYG{n}{append}\PYG{p}{(}\PYG{n}{carta}\PYG{p}{)}
    \PYG{k}{return} \PYG{n}{mazo}

\PYG{n}{mazo\PYGZus{}esp} \PYG{o}{=} \PYG{n}{crear\PYGZus{}mazo\PYGZus{}cartas\PYGZus{}espaniolas}\PYG{p}{(}\PYG{p}{)}
\PYG{n+nb}{print}\PYG{p}{(}\PYG{n}{mazo\PYGZus{}esp}\PYG{p}{)}

\PYG{c+c1}{\PYGZsh{} \PYGZhy{}\PYGZhy{}\PYGZhy{}\PYGZhy{}\PYGZhy{}\PYGZhy{}\PYGZhy{}\PYGZhy{}\PYGZhy{}\PYGZhy{}\PYGZhy{}\PYGZhy{}\PYGZhy{}\PYGZhy{}\PYGZhy{}\PYGZhy{}\PYGZhy{}\PYGZhy{}\PYGZhy{}\PYGZhy{}\PYGZhy{}\PYGZhy{}\PYGZhy{}\PYGZhy{}\PYGZhy{}\PYGZhy{}\PYGZhy{}\PYGZhy{}\PYGZhy{}\PYGZhy{}\PYGZhy{}\PYGZhy{}\PYGZhy{}\PYGZhy{}\PYGZhy{}\PYGZhy{}\PYGZhy{}\PYGZhy{}\PYGZhy{}\PYGZhy{}\PYGZhy{}\PYGZhy{}\PYGZhy{}\PYGZhy{}\PYGZhy{}\PYGZhy{}\PYGZhy{}\PYGZhy{}\PYGZhy{}\PYGZhy{}\PYGZhy{}\PYGZhy{}\PYGZhy{}\PYGZhy{}\PYGZhy{}\PYGZhy{}\PYGZhy{}\PYGZhy{}\PYGZhy{}\PYGZhy{}\PYGZhy{}\PYGZhy{}\PYGZhy{}\PYGZhy{}\PYGZhy{}\PYGZhy{}\PYGZhy{}\PYGZhy{}\PYGZhy{}\PYGZhy{}\PYGZhy{}\PYGZhy{}}
\PYG{c+c1}{\PYGZsh{} NO BORRAR O MODIFICAR LAS LINEAS QUE SIGUEN}
\PYG{c+c1}{\PYGZsh{} \PYGZhy{}\PYGZhy{}\PYGZhy{}\PYGZhy{}\PYGZhy{}\PYGZhy{}\PYGZhy{}\PYGZhy{}\PYGZhy{}\PYGZhy{}\PYGZhy{}\PYGZhy{}\PYGZhy{}\PYGZhy{}\PYGZhy{}\PYGZhy{}\PYGZhy{}\PYGZhy{}\PYGZhy{}\PYGZhy{}\PYGZhy{}\PYGZhy{}\PYGZhy{}\PYGZhy{}\PYGZhy{}\PYGZhy{}\PYGZhy{}\PYGZhy{}\PYGZhy{}\PYGZhy{}\PYGZhy{}\PYGZhy{}\PYGZhy{}\PYGZhy{}\PYGZhy{}\PYGZhy{}\PYGZhy{}\PYGZhy{}\PYGZhy{}\PYGZhy{}\PYGZhy{}\PYGZhy{}\PYGZhy{}\PYGZhy{}\PYGZhy{}\PYGZhy{}\PYGZhy{}\PYGZhy{}\PYGZhy{}\PYGZhy{}\PYGZhy{}\PYGZhy{}\PYGZhy{}\PYGZhy{}\PYGZhy{}\PYGZhy{}\PYGZhy{}\PYGZhy{}\PYGZhy{}\PYGZhy{}\PYGZhy{}\PYGZhy{}\PYGZhy{}\PYGZhy{}\PYGZhy{}\PYGZhy{}\PYGZhy{}\PYGZhy{}\PYGZhy{}\PYGZhy{}\PYGZhy{}\PYGZhy{}}
\PYG{c+c1}{\PYGZsh{} Una vez terminada la tarea ejecutar este archivo.}
\PYG{c+c1}{\PYGZsh{} Si se ve la leyenda \PYGZsq{}Ejercicio terminado OK\PYGZsq{} el ejercicio se considera completado.}
\PYG{c+c1}{\PYGZsh{} La instruccion \PYGZdq{}assert\PYGZdq{} de Python lanzará un error si lo que se indica a}
\PYG{c+c1}{\PYGZsh{}   continuacion es falso.}
\PYG{c+c1}{\PYGZsh{} Si usas GitHub (o similares) podes hacer una nueva rama con esta solución,}
\PYG{c+c1}{\PYGZsh{}   crear un \PYGZdq{}pull request\PYGZdq{} y solicitar revision de un tercero.}


\PYG{k}{assert} \PYG{p}{\PYGZob{}}\PYG{l+s+s1}{\PYGZsq{}}\PYG{l+s+s1}{numero}\PYG{l+s+s1}{\PYGZsq{}}\PYG{p}{:} \PYG{l+m+mi}{10}\PYG{p}{,} \PYG{l+s+s1}{\PYGZsq{}}\PYG{l+s+s1}{palo}\PYG{l+s+s1}{\PYGZsq{}}\PYG{p}{:} \PYG{l+s+s1}{\PYGZsq{}}\PYG{l+s+s1}{oro}\PYG{l+s+s1}{\PYGZsq{}}\PYG{p}{\PYGZcb{}} \PYG{o+ow}{in} \PYG{n}{mazo\PYGZus{}esp}
\PYG{k}{assert} \PYG{p}{\PYGZob{}}\PYG{l+s+s1}{\PYGZsq{}}\PYG{l+s+s1}{numero}\PYG{l+s+s1}{\PYGZsq{}}\PYG{p}{:} \PYG{l+m+mi}{11}\PYG{p}{,} \PYG{l+s+s1}{\PYGZsq{}}\PYG{l+s+s1}{palo}\PYG{l+s+s1}{\PYGZsq{}}\PYG{p}{:} \PYG{l+s+s1}{\PYGZsq{}}\PYG{l+s+s1}{oro}\PYG{l+s+s1}{\PYGZsq{}}\PYG{p}{\PYGZcb{}} \PYG{o+ow}{in} \PYG{n}{mazo\PYGZus{}esp}
\PYG{k}{assert} \PYG{p}{\PYGZob{}}\PYG{l+s+s1}{\PYGZsq{}}\PYG{l+s+s1}{numero}\PYG{l+s+s1}{\PYGZsq{}}\PYG{p}{:} \PYG{l+m+mi}{12}\PYG{p}{,} \PYG{l+s+s1}{\PYGZsq{}}\PYG{l+s+s1}{palo}\PYG{l+s+s1}{\PYGZsq{}}\PYG{p}{:} \PYG{l+s+s1}{\PYGZsq{}}\PYG{l+s+s1}{oro}\PYG{l+s+s1}{\PYGZsq{}}\PYG{p}{\PYGZcb{}} \PYG{o+ow}{in} \PYG{n}{mazo\PYGZus{}esp}

\PYG{n+nb}{print}\PYG{p}{(}\PYG{l+s+s1}{\PYGZsq{}}\PYG{l+s+s1}{Ejercicio terminado OK}\PYG{l+s+s1}{\PYGZsq{}}\PYG{p}{)}
\end{sphinxVerbatim}
\begin{itemize}
\item {} 
\sphinxAtStartPar
Hacer un PR con una propuesta de solución para el
\sphinxhref{https://github.com/avdata99/programacion-para-no-programadores/blob/master/ejercicios/ejercicio-042/ejercicio.py}{ejercicio 042}%
\begin{footnote}[19]\sphinxAtStartFootnote
\sphinxnolinkurl{https://github.com/avdata99/programacion-para-no-programadores/blob/master/ejercicios/ejercicio-042/ejercicio.py}
%
\end{footnote}

\end{itemize}

\begin{sphinxVerbatim}[commandchars=\\\{\}]
\PYG{l+s+sd}{\PYGZdq{}\PYGZdq{}\PYGZdq{}}

\PYG{l+s+sd}{La funcion \PYGZdq{}crear\PYGZus{}mazo\PYGZus{}cartas\PYGZus{}poker\PYGZdq{} esta incompleta y necesita ser completada para}
\PYG{l+s+sd}{devolver una lista de diccionarios con todas las cartas disponibles en un mazo de poker.}

\PYG{l+s+sd}{Nota: El ejercicio 041* ya muestra una función similar que puede usarse como ayuda}

\PYG{l+s+sd}{* https://github.com/avdata99/programacion\PYGZhy{}para\PYGZhy{}no\PYGZhy{}programadores/blob/master/ejercicios/ejercicio\PYGZhy{}041/ejercicio.py}

\PYG{l+s+sd}{\PYGZdq{}\PYGZdq{}\PYGZdq{}}


\PYG{k}{def} \PYG{n+nf}{crear\PYGZus{}mazo\PYGZus{}cartas\PYGZus{}poker}\PYG{p}{(}\PYG{p}{)}\PYG{p}{:}
    \PYG{n}{palos} \PYG{o}{=} \PYG{p}{[}\PYG{l+s+s1}{\PYGZsq{}}\PYG{l+s+s1}{pica}\PYG{l+s+s1}{\PYGZsq{}}\PYG{p}{,} \PYG{l+s+s1}{\PYGZsq{}}\PYG{l+s+s1}{trebol}\PYG{l+s+s1}{\PYGZsq{}}\PYG{p}{,} \PYG{l+s+s1}{\PYGZsq{}}\PYG{l+s+s1}{corazon}\PYG{l+s+s1}{\PYGZsq{}}\PYG{p}{,} \PYG{l+s+s1}{\PYGZsq{}}\PYG{l+s+s1}{diamante}\PYG{l+s+s1}{\PYGZsq{}}\PYG{p}{]}
    \PYG{n}{mazo} \PYG{o}{=} \PYG{p}{[}\PYG{p}{]}

    \PYG{c+c1}{\PYGZsh{} COMPLETAR la lista con un diccionario por cada carta de}
    \PYG{c+c1}{\PYGZsh{} la forma \PYGZob{}\PYGZsq{}numero\PYGZsq{}: X, \PYGZsq{}palo\PYGZsq{}: Y\PYGZcb{}}
    \PYG{c+c1}{\PYGZsh{} El test de la parte inferior de este archivo ayuda a validar}
    \PYG{c+c1}{\PYGZsh{} el resultado esperado}

    \PYG{k}{return} \PYG{n}{mazo}

\PYG{c+c1}{\PYGZsh{} \PYGZhy{}\PYGZhy{}\PYGZhy{}\PYGZhy{}\PYGZhy{}\PYGZhy{}\PYGZhy{}\PYGZhy{}\PYGZhy{}\PYGZhy{}\PYGZhy{}\PYGZhy{}\PYGZhy{}\PYGZhy{}\PYGZhy{}\PYGZhy{}\PYGZhy{}\PYGZhy{}\PYGZhy{}\PYGZhy{}\PYGZhy{}\PYGZhy{}\PYGZhy{}\PYGZhy{}\PYGZhy{}\PYGZhy{}\PYGZhy{}\PYGZhy{}\PYGZhy{}\PYGZhy{}\PYGZhy{}\PYGZhy{}\PYGZhy{}\PYGZhy{}\PYGZhy{}\PYGZhy{}\PYGZhy{}\PYGZhy{}\PYGZhy{}\PYGZhy{}\PYGZhy{}\PYGZhy{}\PYGZhy{}\PYGZhy{}\PYGZhy{}\PYGZhy{}\PYGZhy{}\PYGZhy{}\PYGZhy{}\PYGZhy{}\PYGZhy{}\PYGZhy{}\PYGZhy{}\PYGZhy{}\PYGZhy{}\PYGZhy{}\PYGZhy{}\PYGZhy{}\PYGZhy{}\PYGZhy{}\PYGZhy{}\PYGZhy{}\PYGZhy{}\PYGZhy{}\PYGZhy{}\PYGZhy{}\PYGZhy{}\PYGZhy{}\PYGZhy{}\PYGZhy{}\PYGZhy{}\PYGZhy{}}
\PYG{c+c1}{\PYGZsh{} NO BORRAR O MODIFICAR LAS LINEAS QUE SIGUEN}
\PYG{c+c1}{\PYGZsh{} \PYGZhy{}\PYGZhy{}\PYGZhy{}\PYGZhy{}\PYGZhy{}\PYGZhy{}\PYGZhy{}\PYGZhy{}\PYGZhy{}\PYGZhy{}\PYGZhy{}\PYGZhy{}\PYGZhy{}\PYGZhy{}\PYGZhy{}\PYGZhy{}\PYGZhy{}\PYGZhy{}\PYGZhy{}\PYGZhy{}\PYGZhy{}\PYGZhy{}\PYGZhy{}\PYGZhy{}\PYGZhy{}\PYGZhy{}\PYGZhy{}\PYGZhy{}\PYGZhy{}\PYGZhy{}\PYGZhy{}\PYGZhy{}\PYGZhy{}\PYGZhy{}\PYGZhy{}\PYGZhy{}\PYGZhy{}\PYGZhy{}\PYGZhy{}\PYGZhy{}\PYGZhy{}\PYGZhy{}\PYGZhy{}\PYGZhy{}\PYGZhy{}\PYGZhy{}\PYGZhy{}\PYGZhy{}\PYGZhy{}\PYGZhy{}\PYGZhy{}\PYGZhy{}\PYGZhy{}\PYGZhy{}\PYGZhy{}\PYGZhy{}\PYGZhy{}\PYGZhy{}\PYGZhy{}\PYGZhy{}\PYGZhy{}\PYGZhy{}\PYGZhy{}\PYGZhy{}\PYGZhy{}\PYGZhy{}\PYGZhy{}\PYGZhy{}\PYGZhy{}\PYGZhy{}\PYGZhy{}\PYGZhy{}}
\PYG{c+c1}{\PYGZsh{} Una vez terminada la tarea ejecutar este archivo.}
\PYG{c+c1}{\PYGZsh{} Si se ve la leyenda \PYGZsq{}Ejercicio terminado OK\PYGZsq{} el ejercicio se considera completado.}
\PYG{c+c1}{\PYGZsh{} La instruccion \PYGZdq{}assert\PYGZdq{} de Python lanzará un error si lo que se indica a}
\PYG{c+c1}{\PYGZsh{}   continuacion es falso.}
\PYG{c+c1}{\PYGZsh{} Si usas GitHub (o similares) podes hacer una nueva rama con esta solución,}
\PYG{c+c1}{\PYGZsh{}   crear un \PYGZdq{}pull request\PYGZdq{} y solicitar revision de un tercero.}

\PYG{n}{mazo\PYGZus{}poker} \PYG{o}{=} \PYG{n}{crear\PYGZus{}mazo\PYGZus{}cartas\PYGZus{}poker}\PYG{p}{(}\PYG{p}{)}

\PYG{k}{assert} \PYG{p}{\PYGZob{}}\PYG{l+s+s1}{\PYGZsq{}}\PYG{l+s+s1}{numero}\PYG{l+s+s1}{\PYGZsq{}}\PYG{p}{:} \PYG{l+m+mi}{9}\PYG{p}{,} \PYG{l+s+s1}{\PYGZsq{}}\PYG{l+s+s1}{palo}\PYG{l+s+s1}{\PYGZsq{}}\PYG{p}{:} \PYG{l+s+s1}{\PYGZsq{}}\PYG{l+s+s1}{pica}\PYG{l+s+s1}{\PYGZsq{}}\PYG{p}{\PYGZcb{}} \PYG{o+ow}{in} \PYG{n}{mazo\PYGZus{}poker}
\PYG{k}{assert} \PYG{p}{\PYGZob{}}\PYG{l+s+s1}{\PYGZsq{}}\PYG{l+s+s1}{numero}\PYG{l+s+s1}{\PYGZsq{}}\PYG{p}{:} \PYG{l+m+mi}{10}\PYG{p}{,} \PYG{l+s+s1}{\PYGZsq{}}\PYG{l+s+s1}{palo}\PYG{l+s+s1}{\PYGZsq{}}\PYG{p}{:} \PYG{l+s+s1}{\PYGZsq{}}\PYG{l+s+s1}{pica}\PYG{l+s+s1}{\PYGZsq{}}\PYG{p}{\PYGZcb{}} \PYG{o+ow}{in} \PYG{n}{mazo\PYGZus{}poker}
\PYG{k}{assert} \PYG{p}{\PYGZob{}}\PYG{l+s+s1}{\PYGZsq{}}\PYG{l+s+s1}{numero}\PYG{l+s+s1}{\PYGZsq{}}\PYG{p}{:} \PYG{l+s+s1}{\PYGZsq{}}\PYG{l+s+s1}{J}\PYG{l+s+s1}{\PYGZsq{}}\PYG{p}{,} \PYG{l+s+s1}{\PYGZsq{}}\PYG{l+s+s1}{palo}\PYG{l+s+s1}{\PYGZsq{}}\PYG{p}{:} \PYG{l+s+s1}{\PYGZsq{}}\PYG{l+s+s1}{pica}\PYG{l+s+s1}{\PYGZsq{}}\PYG{p}{\PYGZcb{}} \PYG{o+ow}{in} \PYG{n}{mazo\PYGZus{}poker}
\PYG{k}{assert} \PYG{p}{\PYGZob{}}\PYG{l+s+s1}{\PYGZsq{}}\PYG{l+s+s1}{numero}\PYG{l+s+s1}{\PYGZsq{}}\PYG{p}{:} \PYG{l+s+s1}{\PYGZsq{}}\PYG{l+s+s1}{Q}\PYG{l+s+s1}{\PYGZsq{}}\PYG{p}{,} \PYG{l+s+s1}{\PYGZsq{}}\PYG{l+s+s1}{palo}\PYG{l+s+s1}{\PYGZsq{}}\PYG{p}{:} \PYG{l+s+s1}{\PYGZsq{}}\PYG{l+s+s1}{pica}\PYG{l+s+s1}{\PYGZsq{}}\PYG{p}{\PYGZcb{}} \PYG{o+ow}{in} \PYG{n}{mazo\PYGZus{}poker}
\PYG{k}{assert} \PYG{p}{\PYGZob{}}\PYG{l+s+s1}{\PYGZsq{}}\PYG{l+s+s1}{numero}\PYG{l+s+s1}{\PYGZsq{}}\PYG{p}{:} \PYG{l+s+s1}{\PYGZsq{}}\PYG{l+s+s1}{K}\PYG{l+s+s1}{\PYGZsq{}}\PYG{p}{,} \PYG{l+s+s1}{\PYGZsq{}}\PYG{l+s+s1}{palo}\PYG{l+s+s1}{\PYGZsq{}}\PYG{p}{:} \PYG{l+s+s1}{\PYGZsq{}}\PYG{l+s+s1}{pica}\PYG{l+s+s1}{\PYGZsq{}}\PYG{p}{\PYGZcb{}} \PYG{o+ow}{in} \PYG{n}{mazo\PYGZus{}poker}

\PYG{k}{assert} \PYG{p}{\PYGZob{}}\PYG{l+s+s1}{\PYGZsq{}}\PYG{l+s+s1}{numero}\PYG{l+s+s1}{\PYGZsq{}}\PYG{p}{:} \PYG{l+m+mi}{9}\PYG{p}{,} \PYG{l+s+s1}{\PYGZsq{}}\PYG{l+s+s1}{palo}\PYG{l+s+s1}{\PYGZsq{}}\PYG{p}{:} \PYG{l+s+s1}{\PYGZsq{}}\PYG{l+s+s1}{diamante}\PYG{l+s+s1}{\PYGZsq{}}\PYG{p}{\PYGZcb{}} \PYG{o+ow}{in} \PYG{n}{mazo\PYGZus{}poker}
\PYG{k}{assert} \PYG{p}{\PYGZob{}}\PYG{l+s+s1}{\PYGZsq{}}\PYG{l+s+s1}{numero}\PYG{l+s+s1}{\PYGZsq{}}\PYG{p}{:} \PYG{l+m+mi}{10}\PYG{p}{,} \PYG{l+s+s1}{\PYGZsq{}}\PYG{l+s+s1}{palo}\PYG{l+s+s1}{\PYGZsq{}}\PYG{p}{:} \PYG{l+s+s1}{\PYGZsq{}}\PYG{l+s+s1}{diamante}\PYG{l+s+s1}{\PYGZsq{}}\PYG{p}{\PYGZcb{}} \PYG{o+ow}{in} \PYG{n}{mazo\PYGZus{}poker}
\PYG{k}{assert} \PYG{p}{\PYGZob{}}\PYG{l+s+s1}{\PYGZsq{}}\PYG{l+s+s1}{numero}\PYG{l+s+s1}{\PYGZsq{}}\PYG{p}{:} \PYG{l+s+s1}{\PYGZsq{}}\PYG{l+s+s1}{J}\PYG{l+s+s1}{\PYGZsq{}}\PYG{p}{,} \PYG{l+s+s1}{\PYGZsq{}}\PYG{l+s+s1}{palo}\PYG{l+s+s1}{\PYGZsq{}}\PYG{p}{:} \PYG{l+s+s1}{\PYGZsq{}}\PYG{l+s+s1}{diamante}\PYG{l+s+s1}{\PYGZsq{}}\PYG{p}{\PYGZcb{}} \PYG{o+ow}{in} \PYG{n}{mazo\PYGZus{}poker}
\PYG{k}{assert} \PYG{p}{\PYGZob{}}\PYG{l+s+s1}{\PYGZsq{}}\PYG{l+s+s1}{numero}\PYG{l+s+s1}{\PYGZsq{}}\PYG{p}{:} \PYG{l+s+s1}{\PYGZsq{}}\PYG{l+s+s1}{Q}\PYG{l+s+s1}{\PYGZsq{}}\PYG{p}{,} \PYG{l+s+s1}{\PYGZsq{}}\PYG{l+s+s1}{palo}\PYG{l+s+s1}{\PYGZsq{}}\PYG{p}{:} \PYG{l+s+s1}{\PYGZsq{}}\PYG{l+s+s1}{diamante}\PYG{l+s+s1}{\PYGZsq{}}\PYG{p}{\PYGZcb{}} \PYG{o+ow}{in} \PYG{n}{mazo\PYGZus{}poker}
\PYG{k}{assert} \PYG{p}{\PYGZob{}}\PYG{l+s+s1}{\PYGZsq{}}\PYG{l+s+s1}{numero}\PYG{l+s+s1}{\PYGZsq{}}\PYG{p}{:} \PYG{l+s+s1}{\PYGZsq{}}\PYG{l+s+s1}{K}\PYG{l+s+s1}{\PYGZsq{}}\PYG{p}{,} \PYG{l+s+s1}{\PYGZsq{}}\PYG{l+s+s1}{palo}\PYG{l+s+s1}{\PYGZsq{}}\PYG{p}{:} \PYG{l+s+s1}{\PYGZsq{}}\PYG{l+s+s1}{diamante}\PYG{l+s+s1}{\PYGZsq{}}\PYG{p}{\PYGZcb{}} \PYG{o+ow}{in} \PYG{n}{mazo\PYGZus{}poker}

\PYG{n+nb}{print}\PYG{p}{(}\PYG{l+s+s1}{\PYGZsq{}}\PYG{l+s+s1}{Ejercicio terminado OK}\PYG{l+s+s1}{\PYGZsq{}}\PYG{p}{)}
\end{sphinxVerbatim}


\section{Algunos ejemplos de uso}
\label{\detokenize{dict:algunos-ejemplos-de-uso}}
\begin{sphinxVerbatim}[commandchars=\\\{\}]
\PYG{n}{data} \PYG{o}{=} \PYG{p}{\PYGZob{}}
    \PYG{l+s+s1}{\PYGZsq{}}\PYG{l+s+s1}{edad}\PYG{l+s+s1}{\PYGZsq{}}\PYG{p}{:} \PYG{l+m+mi}{32}\PYG{p}{,}
    \PYG{l+s+s1}{\PYGZsq{}}\PYG{l+s+s1}{nombre}\PYG{l+s+s1}{\PYGZsq{}}\PYG{p}{:} \PYG{l+s+s1}{\PYGZsq{}}\PYG{l+s+s1}{Juan}\PYG{l+s+s1}{\PYGZsq{}}
\PYG{p}{\PYGZcb{}}

\PYG{n}{nombre} \PYG{o}{=} \PYG{n}{data}\PYG{p}{[}\PYG{l+s+s1}{\PYGZsq{}}\PYG{l+s+s1}{nombre}\PYG{l+s+s1}{\PYGZsq{}}\PYG{p}{]}  \PYG{c+c1}{\PYGZsh{} equivalente a data.get(\PYGZsq{}nombre)}
\PYG{n}{edad} \PYG{o}{=} \PYG{n}{data}\PYG{p}{[}\PYG{l+s+s1}{\PYGZsq{}}\PYG{l+s+s1}{edad}\PYG{l+s+s1}{\PYGZsq{}}\PYG{p}{]}
\PYG{n+nb}{print}\PYG{p}{(}\PYG{l+s+sa}{f}\PYG{l+s+s1}{\PYGZsq{}}\PYG{l+s+s1}{Nombre: }\PYG{l+s+si}{\PYGZob{}}\PYG{n}{nombre}\PYG{l+s+si}{\PYGZcb{}}\PYG{l+s+s1}{, edad }\PYG{l+s+si}{\PYGZob{}}\PYG{n}{edad}\PYG{l+s+si}{\PYGZcb{}}\PYG{l+s+s1}{ años}\PYG{l+s+s1}{\PYGZsq{}}\PYG{p}{)}

\PYG{l+s+sd}{\PYGZdq{}\PYGZdq{}\PYGZdq{}}
\PYG{l+s+sd}{data[\PYGZsq{}algo\PYGZhy{}que\PYGZhy{}no\PYGZhy{}existe\PYGZsq{}]}
\PYG{l+s+sd}{\PYGZsh{} genera un error }
\PYG{l+s+sd}{\PYGZsh{} KeyError: \PYGZsq{}algo\PYGZhy{}que\PYGZhy{}no\PYGZhy{}existe\PYGZsq{}}
\PYG{l+s+sd}{\PYGZsh{} mientras que }
\PYG{l+s+sd}{data.get(\PYGZsq{}algo\PYGZhy{}que\PYGZhy{}no\PYGZhy{}existe\PYGZsq{})}
\PYG{l+s+sd}{devuelve None}
\PYG{l+s+sd}{\PYGZdq{}\PYGZdq{}\PYGZdq{}}

\PYG{c+c1}{\PYGZsh{} Nombre: Juan, edad 32 años}

\PYG{k}{for} \PYG{n}{k}\PYG{p}{,} \PYG{n}{v} \PYG{o+ow}{in} \PYG{n}{data}\PYG{o}{.}\PYG{n}{items}\PYG{p}{(}\PYG{p}{)}\PYG{p}{:}
    \PYG{n+nb}{print}\PYG{p}{(}\PYG{l+s+sa}{f}\PYG{l+s+s1}{\PYGZsq{}}\PYG{l+s+s1}{Item encontrado: Key:}\PYG{l+s+si}{\PYGZob{}}\PYG{n}{k}\PYG{l+s+si}{\PYGZcb{}}\PYG{l+s+s1}{, Value: }\PYG{l+s+si}{\PYGZob{}}\PYG{n}{v}\PYG{l+s+si}{\PYGZcb{}}\PYG{l+s+s1}{\PYGZsq{}}\PYG{p}{)}

\PYG{c+c1}{\PYGZsh{} Item encontrado: Key:edad, Value: 32}
\PYG{c+c1}{\PYGZsh{} Item encontrado: Key:nombre, Value: Juan}
\end{sphinxVerbatim}

\begin{sphinxVerbatim}[commandchars=\\\{\}]

\PYG{n}{data} \PYG{o}{=} \PYG{p}{\PYGZob{}}
    \PYG{l+s+s1}{\PYGZsq{}}\PYG{l+s+s1}{edad}\PYG{l+s+s1}{\PYGZsq{}}\PYG{p}{:} \PYG{l+m+mi}{32}\PYG{p}{,}
    \PYG{l+s+s1}{\PYGZsq{}}\PYG{l+s+s1}{nombre}\PYG{l+s+s1}{\PYGZsq{}}\PYG{p}{:} \PYG{l+s+s1}{\PYGZsq{}}\PYG{l+s+s1}{Juan}\PYG{l+s+s1}{\PYGZsq{}}\PYG{p}{,}
    \PYG{l+s+s1}{\PYGZsq{}}\PYG{l+s+s1}{educacion}\PYG{l+s+s1}{\PYGZsq{}}\PYG{p}{:} \PYG{p}{\PYGZob{}}
        \PYG{l+s+s1}{\PYGZsq{}}\PYG{l+s+s1}{secundario}\PYG{l+s+s1}{\PYGZsq{}}\PYG{p}{:} \PYG{l+s+s1}{\PYGZsq{}}\PYG{l+s+s1}{Monserrat}\PYG{l+s+s1}{\PYGZsq{}}\PYG{p}{,}
        \PYG{l+s+s1}{\PYGZsq{}}\PYG{l+s+s1}{universidad}\PYG{l+s+s1}{\PYGZsq{}}\PYG{p}{:} \PYG{l+s+s1}{\PYGZsq{}}\PYG{l+s+s1}{UNC}\PYG{l+s+s1}{\PYGZsq{}}\PYG{p}{,}
    \PYG{p}{\PYGZcb{}}
\PYG{p}{\PYGZcb{}}

\PYG{n}{nombre} \PYG{o}{=} \PYG{n}{data}\PYG{p}{[}\PYG{l+s+s1}{\PYGZsq{}}\PYG{l+s+s1}{nombre}\PYG{l+s+s1}{\PYGZsq{}}\PYG{p}{]}
\PYG{n}{edad} \PYG{o}{=} \PYG{n}{data}\PYG{p}{[}\PYG{l+s+s1}{\PYGZsq{}}\PYG{l+s+s1}{edad}\PYG{l+s+s1}{\PYGZsq{}}\PYG{p}{]}
\PYG{n}{universidad} \PYG{o}{=} \PYG{n}{data}\PYG{p}{[}\PYG{l+s+s1}{\PYGZsq{}}\PYG{l+s+s1}{educacion}\PYG{l+s+s1}{\PYGZsq{}}\PYG{p}{]}\PYG{p}{[}\PYG{l+s+s1}{\PYGZsq{}}\PYG{l+s+s1}{universidad}\PYG{l+s+s1}{\PYGZsq{}}\PYG{p}{]}

\PYG{n+nb}{print}\PYG{p}{(}\PYG{l+s+sa}{f}\PYG{l+s+s1}{\PYGZsq{}}\PYG{l+s+s1}{Nombre: }\PYG{l+s+si}{\PYGZob{}}\PYG{n}{nombre}\PYG{l+s+si}{\PYGZcb{}}\PYG{l+s+s1}{, edad }\PYG{l+s+si}{\PYGZob{}}\PYG{n}{edad}\PYG{l+s+si}{\PYGZcb{}}\PYG{l+s+s1}{ años. Universidad: }\PYG{l+s+si}{\PYGZob{}}\PYG{n}{universidad}\PYG{l+s+si}{\PYGZcb{}}\PYG{l+s+s1}{\PYGZsq{}}\PYG{p}{)}
\PYG{c+c1}{\PYGZsh{} Nombre: Juan, edad 32 años. Universidad: UNC}

\PYG{c+c1}{\PYGZsh{} Agregarle datos}
\PYG{n}{data}\PYG{p}{[}\PYG{l+s+s1}{\PYGZsq{}}\PYG{l+s+s1}{ocupacion}\PYG{l+s+s1}{\PYGZsq{}}\PYG{p}{]} \PYG{o}{=} \PYG{l+s+s1}{\PYGZsq{}}\PYG{l+s+s1}{Desarrollador}\PYG{l+s+s1}{\PYGZsq{}}
\PYG{n}{data}\PYG{p}{[}\PYG{l+s+s1}{\PYGZsq{}}\PYG{l+s+s1}{educacion}\PYG{l+s+s1}{\PYGZsq{}}\PYG{p}{]}\PYG{p}{[}\PYG{l+s+s1}{\PYGZsq{}}\PYG{l+s+s1}{primario}\PYG{l+s+s1}{\PYGZsq{}}\PYG{p}{]} \PYG{o}{=} \PYG{l+s+s1}{\PYGZsq{}}\PYG{l+s+s1}{San Juan}\PYG{l+s+s1}{\PYGZsq{}}

\PYG{n+nb}{print}\PYG{p}{(}\PYG{n}{data}\PYG{p}{)}
\PYG{c+c1}{\PYGZsh{} \PYGZob{}\PYGZsq{}edad\PYGZsq{}: 32, \PYGZsq{}nombre\PYGZsq{}: \PYGZsq{}Juan\PYGZsq{}, \PYGZsq{}educacion\PYGZsq{}: \PYGZob{}\PYGZsq{}secundario\PYGZsq{}: \PYGZsq{}Monserrat\PYGZsq{}, \PYGZsq{}universidad\PYGZsq{}: \PYGZsq{}UNC\PYGZsq{}, \PYGZsq{}primario\PYGZsq{}: \PYGZsq{}San Juan\PYGZsq{}\PYGZcb{}, \PYGZsq{}ocupacion\PYGZsq{}: \PYGZsq{}Desarrollador\PYGZsq{}\PYGZcb{}}

\PYG{n+nb}{print}\PYG{p}{(}\PYG{n}{data}\PYG{o}{.}\PYG{n}{get}\PYG{p}{(}\PYG{l+s+s1}{\PYGZsq{}}\PYG{l+s+s1}{ocupacion}\PYG{l+s+s1}{\PYGZsq{}}\PYG{p}{)}\PYG{p}{)}
\PYG{c+c1}{\PYGZsh{} \PYGZsq{}Desarrollador\PYGZsq{}}

\PYG{n+nb}{print}\PYG{p}{(}\PYG{n}{data}\PYG{o}{.}\PYG{n}{get}\PYG{p}{(}\PYG{l+s+s1}{\PYGZsq{}}\PYG{l+s+s1}{NO EXISTE}\PYG{l+s+s1}{\PYGZsq{}}\PYG{p}{)}\PYG{p}{)}
\PYG{c+c1}{\PYGZsh{} None}

\PYG{n+nb}{print}\PYG{p}{(}\PYG{n}{data}\PYG{o}{.}\PYG{n}{get}\PYG{p}{(}\PYG{l+s+s1}{\PYGZsq{}}\PYG{l+s+s1}{NO EXISTE}\PYG{l+s+s1}{\PYGZsq{}}\PYG{p}{,} \PYG{l+s+s1}{\PYGZsq{}}\PYG{l+s+s1}{valor predeterminado}\PYG{l+s+s1}{\PYGZsq{}}\PYG{p}{)}\PYG{p}{)}
\PYG{c+c1}{\PYGZsh{} valor predeterminado}
\end{sphinxVerbatim}

\begin{sphinxVerbatim}[commandchars=\\\{\}]
\PYG{c+c1}{\PYGZsh{} lista de diccionarios}
\PYG{n}{data} \PYG{o}{=} \PYG{p}{\PYGZob{}}
    \PYG{l+s+s1}{\PYGZsq{}}\PYG{l+s+s1}{personas}\PYG{l+s+s1}{\PYGZsq{}}\PYG{p}{:} \PYG{p}{[}
        \PYG{p}{\PYGZob{}}\PYG{l+s+s1}{\PYGZsq{}}\PYG{l+s+s1}{nombre}\PYG{l+s+s1}{\PYGZsq{}}\PYG{p}{:} \PYG{l+s+s1}{\PYGZsq{}}\PYG{l+s+s1}{Juan}\PYG{l+s+s1}{\PYGZsq{}}\PYG{p}{,} \PYG{l+s+s1}{\PYGZsq{}}\PYG{l+s+s1}{edad}\PYG{l+s+s1}{\PYGZsq{}}\PYG{p}{:} \PYG{l+m+mi}{20}\PYG{p}{\PYGZcb{}}\PYG{p}{,}
        \PYG{p}{\PYGZob{}}\PYG{l+s+s1}{\PYGZsq{}}\PYG{l+s+s1}{nombre}\PYG{l+s+s1}{\PYGZsq{}}\PYG{p}{:} \PYG{l+s+s1}{\PYGZsq{}}\PYG{l+s+s1}{Pedro}\PYG{l+s+s1}{\PYGZsq{}}\PYG{p}{,} \PYG{l+s+s1}{\PYGZsq{}}\PYG{l+s+s1}{edad}\PYG{l+s+s1}{\PYGZsq{}}\PYG{p}{:} \PYG{l+m+mi}{30}\PYG{p}{\PYGZcb{}}\PYG{p}{,}
        \PYG{p}{\PYGZob{}}\PYG{l+s+s1}{\PYGZsq{}}\PYG{l+s+s1}{nombre}\PYG{l+s+s1}{\PYGZsq{}}\PYG{p}{:} \PYG{l+s+s1}{\PYGZsq{}}\PYG{l+s+s1}{María}\PYG{l+s+s1}{\PYGZsq{}}\PYG{p}{,} \PYG{l+s+s1}{\PYGZsq{}}\PYG{l+s+s1}{edad}\PYG{l+s+s1}{\PYGZsq{}}\PYG{p}{:} \PYG{l+m+mi}{40}\PYG{p}{\PYGZcb{}}\PYG{p}{,}
    \PYG{p}{]}
\PYG{p}{\PYGZcb{}}

\PYG{n+nb}{print}\PYG{p}{(}\PYG{l+s+s1}{\PYGZsq{}}\PYG{l+s+s1}{PERSONAS:}\PYG{l+s+s1}{\PYGZsq{}}\PYG{p}{)}
\PYG{k}{for} \PYG{n}{persona} \PYG{o+ow}{in} \PYG{n}{data}\PYG{p}{[}\PYG{l+s+s1}{\PYGZsq{}}\PYG{l+s+s1}{personas}\PYG{l+s+s1}{\PYGZsq{}}\PYG{p}{]}\PYG{p}{:}
    \PYG{n}{nombre} \PYG{o}{=} \PYG{n}{persona}\PYG{p}{[}\PYG{l+s+s1}{\PYGZsq{}}\PYG{l+s+s1}{nombre}\PYG{l+s+s1}{\PYGZsq{}}\PYG{p}{]}
    \PYG{n}{edad} \PYG{o}{=} \PYG{n}{persona}\PYG{p}{[}\PYG{l+s+s1}{\PYGZsq{}}\PYG{l+s+s1}{edad}\PYG{l+s+s1}{\PYGZsq{}}\PYG{p}{]}
    \PYG{n+nb}{print}\PYG{p}{(}\PYG{l+s+sa}{f}\PYG{l+s+s1}{\PYGZsq{}}\PYG{l+s+s1}{ \PYGZhy{} Nombre: }\PYG{l+s+si}{\PYGZob{}}\PYG{n}{nombre}\PYG{l+s+si}{\PYGZcb{}}\PYG{l+s+s1}{, edad }\PYG{l+s+si}{\PYGZob{}}\PYG{n}{edad}\PYG{l+s+si}{\PYGZcb{}}\PYG{l+s+s1}{ años}\PYG{l+s+s1}{\PYGZsq{}}\PYG{p}{)}

\PYG{l+s+sd}{\PYGZdq{}\PYGZdq{}\PYGZdq{} resultados}

\PYG{l+s+sd}{PERSONAS:}
\PYG{l+s+sd}{ \PYGZhy{} Nombre: Juan, edad 20 años}
\PYG{l+s+sd}{ \PYGZhy{} Nombre: Pedro, edad 30 años}
\PYG{l+s+sd}{ \PYGZhy{} Nombre: María, edad 40 años}

\PYG{l+s+sd}{\PYGZdq{}\PYGZdq{}\PYGZdq{}}
\end{sphinxVerbatim}

\sphinxstepscope


\chapter{Librerías incluida: \sphinxstyleliteralintitle{\sphinxupquote{random}}}
\label{\detokenize{random:librerias-incluida-random}}\label{\detokenize{random::doc}}
\sphinxAtStartPar
Asi como hay funciones incluidas (\sphinxstyleemphasis{built\sphinxhyphen{}ins}) que se pueden usar sin
estar declaradas tambien hay más herramientas de Python disponibles pero
que requieren ser \sphinxstyleemphasis{importadas}.

\sphinxAtStartPar
Importar (con el comando \sphinxcode{\sphinxupquote{import}} o de la forma \sphinxcode{\sphinxupquote{from X import Y}})
en Python es disponibilizar nueva herramientas (fuenciones y otras) en
nuestro código.


\section{Python al azar: \sphinxstyleliteralintitle{\sphinxupquote{random}}}
\label{\detokenize{random:python-al-azar-random}}
\sphinxAtStartPar
El módulo \sphinxcode{\sphinxupquote{random}} incluye una serie de funciones que permiten
darle aleatoriedad a nuestro código.

\sphinxAtStartPar
Veamos un ejemplo. La función \sphinxcode{\sphinxupquote{randint}} genera numeros al alzar
entre dos valores pasados como parámetros.

\begin{sphinxVerbatim}[commandchars=\\\{\}]
\PYG{k+kn}{from} \PYG{n+nn}{random} \PYG{k+kn}{import} \PYG{n}{randint}

\PYG{n}{al\PYGZus{}azar} \PYG{o}{=} \PYG{n}{randint}\PYG{p}{(}\PYG{l+m+mi}{1}\PYG{p}{,} \PYG{l+m+mi}{100}\PYG{p}{)}
\PYG{n+nb}{print}\PYG{p}{(}\PYG{n}{al\PYGZus{}azar}\PYG{p}{)}

\PYG{c+c1}{\PYGZsh{} otra forma de usarlo podría ser}
\PYG{k+kn}{import} \PYG{n+nn}{random}
\PYG{n}{al\PYGZus{}azar} \PYG{o}{=} \PYG{n}{random}\PYG{o}{.}\PYG{n}{randint}\PYG{p}{(}\PYG{l+m+mi}{1}\PYG{p}{,} \PYG{l+m+mi}{100}\PYG{p}{)}
\PYG{n+nb}{print}\PYG{p}{(}\PYG{n}{al\PYGZus{}azar}\PYG{p}{)}
\end{sphinxVerbatim}

\sphinxAtStartPar
Asi como el \sphinxcode{\sphinxupquote{.}} se usa en \sphinxcode{\sphinxupquote{objeto.propiedad}} tambien se usa
en \sphinxcode{\sphinxupquote{modulo.objeto\_en\_modulo}}.


\section{\sphinxstyleliteralintitle{\sphinxupquote{choice}}: Elegir un opción al azar de una lista.}
\label{\detokenize{random:choice-elegir-un-opcion-al-azar-de-una-lista}}
\sphinxAtStartPar
La función \sphinxcode{\sphinxupquote{choice}} recibe como parámetro una lista de la cual
devolverá un elemento al azar.

\begin{sphinxVerbatim}[commandchars=\\\{\}]
\PYG{k+kn}{from} \PYG{n+nn}{random} \PYG{k+kn}{import} \PYG{n}{choice}

\PYG{n}{opciones} \PYG{o}{=} \PYG{p}{[}\PYG{l+s+s2}{\PYGZdq{}}\PYG{l+s+s2}{piedra}\PYG{l+s+s2}{\PYGZdq{}}\PYG{p}{,} \PYG{l+s+s2}{\PYGZdq{}}\PYG{l+s+s2}{papel}\PYG{l+s+s2}{\PYGZdq{}}\PYG{p}{,} \PYG{l+s+s2}{\PYGZdq{}}\PYG{l+s+s2}{tijeras}\PYG{l+s+s2}{\PYGZdq{}}\PYG{p}{]}
\PYG{n}{opcion\PYGZus{}elegida} \PYG{o}{=} \PYG{n}{choice}\PYG{p}{(}\PYG{n}{opciones}\PYG{p}{)}
\PYG{n+nb}{print}\PYG{p}{(}\PYG{n}{opcion\PYGZus{}elegida}\PYG{p}{)}
\PYG{c+c1}{\PYGZsh{} tijeras}
\end{sphinxVerbatim}


\section{\sphinxstyleliteralintitle{\sphinxupquote{shuffle}}: Mezclar una lista}
\label{\detokenize{random:shuffle-mezclar-una-lista}}
\sphinxAtStartPar
Otra función del módulo \sphinxcode{\sphinxupquote{random}} es \sphinxcode{\sphinxupquote{shuffle}} que recibe una lista
como parámetro y la desordena aleatoriamente.

\sphinxAtStartPar
Ejemplo:

\begin{sphinxVerbatim}[commandchars=\\\{\}]
\PYG{k+kn}{from} \PYG{n+nn}{random} \PYG{k+kn}{import} \PYG{n}{shuffle}

\PYG{n}{mazo} \PYG{o}{=} \PYG{p}{[}
    \PYG{p}{\PYGZob{}}\PYG{l+s+s2}{\PYGZdq{}}\PYG{l+s+s2}{numero}\PYG{l+s+s2}{\PYGZdq{}}\PYG{p}{:} \PYG{l+m+mi}{1}\PYG{p}{,} \PYG{l+s+s2}{\PYGZdq{}}\PYG{l+s+s2}{palo}\PYG{l+s+s2}{\PYGZdq{}}\PYG{p}{:} \PYG{l+s+s2}{\PYGZdq{}}\PYG{l+s+s2}{espada}\PYG{l+s+s2}{\PYGZdq{}}\PYG{p}{\PYGZcb{}}\PYG{p}{,}
    \PYG{p}{\PYGZob{}}\PYG{l+s+s2}{\PYGZdq{}}\PYG{l+s+s2}{numero}\PYG{l+s+s2}{\PYGZdq{}}\PYG{p}{:} \PYG{l+m+mi}{2}\PYG{p}{,} \PYG{l+s+s2}{\PYGZdq{}}\PYG{l+s+s2}{palo}\PYG{l+s+s2}{\PYGZdq{}}\PYG{p}{:} \PYG{l+s+s2}{\PYGZdq{}}\PYG{l+s+s2}{espada}\PYG{l+s+s2}{\PYGZdq{}}\PYG{p}{\PYGZcb{}}\PYG{p}{,}
    \PYG{c+c1}{\PYGZsh{} ...}
    \PYG{p}{\PYGZob{}}\PYG{l+s+s2}{\PYGZdq{}}\PYG{l+s+s2}{numero}\PYG{l+s+s2}{\PYGZdq{}}\PYG{p}{:} \PYG{l+m+mi}{11}\PYG{p}{,} \PYG{l+s+s2}{\PYGZdq{}}\PYG{l+s+s2}{palo}\PYG{l+s+s2}{\PYGZdq{}}\PYG{p}{:} \PYG{l+s+s2}{\PYGZdq{}}\PYG{l+s+s2}{oro}\PYG{l+s+s2}{\PYGZdq{}}\PYG{p}{\PYGZcb{}}\PYG{p}{,}
    \PYG{p}{\PYGZob{}}\PYG{l+s+s2}{\PYGZdq{}}\PYG{l+s+s2}{numero}\PYG{l+s+s2}{\PYGZdq{}}\PYG{p}{:} \PYG{l+m+mi}{12}\PYG{p}{,} \PYG{l+s+s2}{\PYGZdq{}}\PYG{l+s+s2}{palo}\PYG{l+s+s2}{\PYGZdq{}}\PYG{p}{:} \PYG{l+s+s2}{\PYGZdq{}}\PYG{l+s+s2}{oro}\PYG{l+s+s2}{\PYGZdq{}}\PYG{p}{\PYGZcb{}}\PYG{p}{,}
\PYG{p}{]}
\PYG{n}{suffle}\PYG{p}{(}\PYG{n}{mazo}\PYG{p}{)}
\PYG{n+nb}{print}\PYG{p}{(}\PYG{n}{mazo}\PYG{p}{[}\PYG{l+m+mi}{0}\PYG{p}{]}\PYG{p}{)}
\PYG{c+c1}{\PYGZsh{} \PYGZob{}\PYGZdq{}numero\PYGZdq{}: 11, \PYGZdq{}palo\PYGZdq{}: \PYGZdq{}oro\PYGZdq{}\PYGZcb{}}
\end{sphinxVerbatim}


\subsection{Tareas}
\label{\detokenize{random:tareas}}\begin{itemize}
\item {} 
\sphinxAtStartPar
Escribir un programa que elija un numero al azar entre 1 y 100 y
le pida al usuario que lo adivine. El programa debe decirle al usuario si el numero que ingresó es mayor o menor
al numero que se eligió al azar. El programa debe terminar cuando el usuario adivine el numero elegido al azar.

\item {} 
\sphinxAtStartPar
Hacer un PR con una propuesta de solución para el
\sphinxhref{https://github.com/avdata99/programacion-para-no-programadores/blob/master/ejercicios/ejercicio-045/ejercicio.py}{ejercicio 045}%
\begin{footnote}[20]\sphinxAtStartFootnote
\sphinxnolinkurl{https://github.com/avdata99/programacion-para-no-programadores/blob/master/ejercicios/ejercicio-045/ejercicio.py}
%
\end{footnote}
(contenido en este repositorio)

\end{itemize}

\begin{sphinxVerbatim}[commandchars=\\\{\}]
\PYG{l+s+sd}{\PYGZdq{}\PYGZdq{}\PYGZdq{}}
\PYG{l+s+sd}{La funcion \PYGZdq{}carta\PYGZus{}poker\PYGZus{}al\PYGZus{}azar\PYGZdq{} ya funciona como es esperado}
\PYG{l+s+sd}{La tarea aquí es completar la funcion \PYGZdq{}carta\PYGZus{}espaniola\PYGZus{}al\PYGZus{}azar\PYGZdq{}}
\PYG{l+s+sd}{para que devuleva resultados válidos.}
\PYG{l+s+sd}{Nota: se debe importar y usar la funcion \PYGZdq{}randint\PYGZdq{} de \PYGZdq{}random\PYGZdq{}}
\PYG{l+s+sd}{\PYGZdq{}\PYGZdq{}\PYGZdq{}}

\PYG{k+kn}{from} \PYG{n+nn}{random} \PYG{k+kn}{import} \PYG{n}{choice}

\PYG{k}{def} \PYG{n+nf}{carta\PYGZus{}poker\PYGZus{}al\PYGZus{}azar}\PYG{p}{(}\PYG{p}{)}\PYG{p}{:}
    \PYG{n}{palos} \PYG{o}{=} \PYG{p}{[}\PYG{l+s+s1}{\PYGZsq{}}\PYG{l+s+s1}{pica}\PYG{l+s+s1}{\PYGZsq{}}\PYG{p}{,} \PYG{l+s+s1}{\PYGZsq{}}\PYG{l+s+s1}{trebol}\PYG{l+s+s1}{\PYGZsq{}}\PYG{p}{,} \PYG{l+s+s1}{\PYGZsq{}}\PYG{l+s+s1}{corazon}\PYG{l+s+s1}{\PYGZsq{}}\PYG{p}{,} \PYG{l+s+s1}{\PYGZsq{}}\PYG{l+s+s1}{diamante}\PYG{l+s+s1}{\PYGZsq{}}\PYG{p}{]}
    \PYG{n}{numeros} \PYG{o}{=} \PYG{n+nb}{list}\PYG{p}{(}\PYG{n+nb}{range}\PYG{p}{(}\PYG{l+m+mi}{1}\PYG{p}{,} \PYG{l+m+mi}{11}\PYG{p}{)}\PYG{p}{)} \PYG{o}{+} \PYG{p}{[}\PYG{l+s+s1}{\PYGZsq{}}\PYG{l+s+s1}{J}\PYG{l+s+s1}{\PYGZsq{}}\PYG{p}{,} \PYG{l+s+s1}{\PYGZsq{}}\PYG{l+s+s1}{Q}\PYG{l+s+s1}{\PYGZsq{}}\PYG{p}{,} \PYG{l+s+s1}{\PYGZsq{}}\PYG{l+s+s1}{K}\PYG{l+s+s1}{\PYGZsq{}}\PYG{p}{]}
    \PYG{n}{nro} \PYG{o}{=} \PYG{n}{choice}\PYG{p}{(}\PYG{n}{numeros}\PYG{p}{)}
    \PYG{n}{palo} \PYG{o}{=} \PYG{n}{choice}\PYG{p}{(}\PYG{n}{palos}\PYG{p}{)}

    \PYG{n}{carta} \PYG{o}{=} \PYG{p}{\PYGZob{}}\PYG{l+s+s2}{\PYGZdq{}}\PYG{l+s+s2}{numero}\PYG{l+s+s2}{\PYGZdq{}}\PYG{p}{:} \PYG{n}{nro}\PYG{p}{,} \PYG{l+s+s2}{\PYGZdq{}}\PYG{l+s+s2}{palo}\PYG{l+s+s2}{\PYGZdq{}}\PYG{p}{:} \PYG{n}{palo}\PYG{p}{\PYGZcb{}}
    \PYG{k}{return} \PYG{n}{carta}

\PYG{k}{def} \PYG{n+nf}{carta\PYGZus{}espaniola\PYGZus{}al\PYGZus{}azar}\PYG{p}{(}\PYG{p}{)}\PYG{p}{:}
    \PYG{n}{palos} \PYG{o}{=} \PYG{p}{[}\PYG{l+s+s1}{\PYGZsq{}}\PYG{l+s+s1}{oro}\PYG{l+s+s1}{\PYGZsq{}}\PYG{p}{,} \PYG{l+s+s1}{\PYGZsq{}}\PYG{l+s+s1}{basto}\PYG{l+s+s1}{\PYGZsq{}}\PYG{p}{,} \PYG{l+s+s1}{\PYGZsq{}}\PYG{l+s+s1}{espada}\PYG{l+s+s1}{\PYGZsq{}}\PYG{p}{,} \PYG{l+s+s1}{\PYGZsq{}}\PYG{l+s+s1}{copa}\PYG{l+s+s1}{\PYGZsq{}}\PYG{p}{]}
    
    \PYG{c+c1}{\PYGZsh{} corregir estas dos lineas para devolver valores válidos}
    \PYG{n}{nro} \PYG{o}{=} \PYG{l+m+mi}{0}
    \PYG{n}{palo} \PYG{o}{=} \PYG{l+s+s2}{\PYGZdq{}}\PYG{l+s+s2}{\PYGZdq{}}

    \PYG{n}{carta} \PYG{o}{=} \PYG{p}{\PYGZob{}}\PYG{l+s+s2}{\PYGZdq{}}\PYG{l+s+s2}{numero}\PYG{l+s+s2}{\PYGZdq{}}\PYG{p}{:} \PYG{n}{nro}\PYG{p}{,} \PYG{l+s+s2}{\PYGZdq{}}\PYG{l+s+s2}{palo}\PYG{l+s+s2}{\PYGZdq{}}\PYG{p}{:} \PYG{n}{palo}\PYG{p}{\PYGZcb{}}
    \PYG{k}{return} \PYG{n}{carta}

\PYG{c+c1}{\PYGZsh{} mirar ejemplos de ambas funciones}
\PYG{k}{for} \PYG{n}{n} \PYG{o+ow}{in} \PYG{n+nb}{range}\PYG{p}{(}\PYG{l+m+mi}{20}\PYG{p}{)}\PYG{p}{:}
    \PYG{n}{carta1} \PYG{o}{=} \PYG{n}{carta\PYGZus{}poker\PYGZus{}al\PYGZus{}azar}\PYG{p}{(}\PYG{p}{)}
    \PYG{n}{carta2} \PYG{o}{=} \PYG{n}{carta\PYGZus{}espaniola\PYGZus{}al\PYGZus{}azar}\PYG{p}{(}\PYG{p}{)}
    \PYG{n+nb}{print}\PYG{p}{(}\PYG{n}{carta1}\PYG{p}{,} \PYG{n}{carta2}\PYG{p}{)}


\PYG{c+c1}{\PYGZsh{} \PYGZhy{}\PYGZhy{}\PYGZhy{}\PYGZhy{}\PYGZhy{}\PYGZhy{}\PYGZhy{}\PYGZhy{}\PYGZhy{}\PYGZhy{}\PYGZhy{}\PYGZhy{}\PYGZhy{}\PYGZhy{}\PYGZhy{}\PYGZhy{}\PYGZhy{}\PYGZhy{}\PYGZhy{}\PYGZhy{}\PYGZhy{}\PYGZhy{}\PYGZhy{}\PYGZhy{}\PYGZhy{}\PYGZhy{}\PYGZhy{}\PYGZhy{}\PYGZhy{}\PYGZhy{}\PYGZhy{}\PYGZhy{}\PYGZhy{}\PYGZhy{}\PYGZhy{}\PYGZhy{}\PYGZhy{}\PYGZhy{}\PYGZhy{}\PYGZhy{}\PYGZhy{}\PYGZhy{}\PYGZhy{}\PYGZhy{}\PYGZhy{}\PYGZhy{}\PYGZhy{}\PYGZhy{}\PYGZhy{}\PYGZhy{}\PYGZhy{}\PYGZhy{}\PYGZhy{}\PYGZhy{}\PYGZhy{}\PYGZhy{}\PYGZhy{}\PYGZhy{}\PYGZhy{}\PYGZhy{}\PYGZhy{}\PYGZhy{}\PYGZhy{}\PYGZhy{}\PYGZhy{}\PYGZhy{}\PYGZhy{}\PYGZhy{}\PYGZhy{}\PYGZhy{}\PYGZhy{}\PYGZhy{}}
\PYG{c+c1}{\PYGZsh{} NO BORRAR O MODIFICAR LAS LINEAS QUE SIGUEN}
\PYG{c+c1}{\PYGZsh{} \PYGZhy{}\PYGZhy{}\PYGZhy{}\PYGZhy{}\PYGZhy{}\PYGZhy{}\PYGZhy{}\PYGZhy{}\PYGZhy{}\PYGZhy{}\PYGZhy{}\PYGZhy{}\PYGZhy{}\PYGZhy{}\PYGZhy{}\PYGZhy{}\PYGZhy{}\PYGZhy{}\PYGZhy{}\PYGZhy{}\PYGZhy{}\PYGZhy{}\PYGZhy{}\PYGZhy{}\PYGZhy{}\PYGZhy{}\PYGZhy{}\PYGZhy{}\PYGZhy{}\PYGZhy{}\PYGZhy{}\PYGZhy{}\PYGZhy{}\PYGZhy{}\PYGZhy{}\PYGZhy{}\PYGZhy{}\PYGZhy{}\PYGZhy{}\PYGZhy{}\PYGZhy{}\PYGZhy{}\PYGZhy{}\PYGZhy{}\PYGZhy{}\PYGZhy{}\PYGZhy{}\PYGZhy{}\PYGZhy{}\PYGZhy{}\PYGZhy{}\PYGZhy{}\PYGZhy{}\PYGZhy{}\PYGZhy{}\PYGZhy{}\PYGZhy{}\PYGZhy{}\PYGZhy{}\PYGZhy{}\PYGZhy{}\PYGZhy{}\PYGZhy{}\PYGZhy{}\PYGZhy{}\PYGZhy{}\PYGZhy{}\PYGZhy{}\PYGZhy{}\PYGZhy{}\PYGZhy{}\PYGZhy{}}
\PYG{c+c1}{\PYGZsh{} Una vez terminada la tarea ejecutar este archivo.}
\PYG{c+c1}{\PYGZsh{} Si se ve la leyenda \PYGZsq{}Ejercicio terminado OK\PYGZsq{} el ejercicio se considera completado.}
\PYG{c+c1}{\PYGZsh{} La instruccion \PYGZdq{}assert\PYGZdq{} de Python lanzará un error si lo que se indica a}
\PYG{c+c1}{\PYGZsh{}   continuacion es falso.}
\PYG{c+c1}{\PYGZsh{} Si usas GitHub (o similares) podes hacer una nueva rama con esta solución,}
\PYG{c+c1}{\PYGZsh{}   crear un \PYGZdq{}pull request\PYGZdq{} y solicitar revision de un tercero.}

\PYG{n}{carta} \PYG{o}{=} \PYG{n}{carta\PYGZus{}espaniola\PYGZus{}al\PYGZus{}azar}\PYG{p}{(}\PYG{p}{)}
\PYG{k}{assert} \PYG{n+nb}{type}\PYG{p}{(}\PYG{n}{carta}\PYG{p}{[}\PYG{l+s+s1}{\PYGZsq{}}\PYG{l+s+s1}{numero}\PYG{l+s+s1}{\PYGZsq{}}\PYG{p}{]}\PYG{p}{)} \PYG{o}{==} \PYG{n+nb}{int}
\PYG{k}{assert} \PYG{n}{carta}\PYG{p}{[}\PYG{l+s+s1}{\PYGZsq{}}\PYG{l+s+s1}{numero}\PYG{l+s+s1}{\PYGZsq{}}\PYG{p}{]} \PYG{o}{\PYGZgt{}} \PYG{l+m+mi}{0}
\PYG{k}{assert} \PYG{n}{carta}\PYG{p}{[}\PYG{l+s+s1}{\PYGZsq{}}\PYG{l+s+s1}{numero}\PYG{l+s+s1}{\PYGZsq{}}\PYG{p}{]} \PYG{o}{\PYGZlt{}}\PYG{o}{=} \PYG{l+m+mi}{12}
\PYG{k}{assert} \PYG{n}{carta}\PYG{p}{[}\PYG{l+s+s1}{\PYGZsq{}}\PYG{l+s+s1}{palo}\PYG{l+s+s1}{\PYGZsq{}}\PYG{p}{]} \PYG{o+ow}{in} \PYG{p}{[}\PYG{l+s+s1}{\PYGZsq{}}\PYG{l+s+s1}{oro}\PYG{l+s+s1}{\PYGZsq{}}\PYG{p}{,} \PYG{l+s+s1}{\PYGZsq{}}\PYG{l+s+s1}{basto}\PYG{l+s+s1}{\PYGZsq{}}\PYG{p}{,} \PYG{l+s+s1}{\PYGZsq{}}\PYG{l+s+s1}{espada}\PYG{l+s+s1}{\PYGZsq{}}\PYG{p}{,} \PYG{l+s+s1}{\PYGZsq{}}\PYG{l+s+s1}{copa}\PYG{l+s+s1}{\PYGZsq{}}\PYG{p}{]}

\PYG{n+nb}{print}\PYG{p}{(}\PYG{l+s+s1}{\PYGZsq{}}\PYG{l+s+s1}{Ejercicio terminado OK}\PYG{l+s+s1}{\PYGZsq{}}\PYG{p}{)}
\end{sphinxVerbatim}
\begin{itemize}
\item {} 
\sphinxAtStartPar
Hacer un PR con una propuesta de solución para el
\sphinxhref{https://github.com/avdata99/programacion-para-no-programadores/blob/master/ejercicios/ejercicio-046/ejercicio.py}{ejercicio 046}%
\begin{footnote}[21]\sphinxAtStartFootnote
\sphinxnolinkurl{https://github.com/avdata99/programacion-para-no-programadores/blob/master/ejercicios/ejercicio-046/ejercicio.py}
%
\end{footnote}

\end{itemize}

\begin{sphinxVerbatim}[commandchars=\\\{\}]
\PYG{l+s+sd}{\PYGZdq{}\PYGZdq{}\PYGZdq{}}
\PYG{l+s+sd}{La siguiente funcion pretende ser usada para generar}
\PYG{l+s+sd}{10 numeros como resultados del sorteo de quiniela.}
\PYG{l+s+sd}{Como podrán notar es bastante mala.}
\PYG{l+s+sd}{Tarea:}
\PYG{l+s+sd}{ \PYGZhy{} Asegurarse que los resultados sean aleatorios}
\PYG{l+s+sd}{ \PYGZhy{} Asegurarse que la función respete los tres parametros}

\PYG{l+s+sd}{\PYGZdq{}\PYGZdq{}\PYGZdq{}}


\PYG{k}{def} \PYG{n+nf}{generar\PYGZus{}quiniela}\PYG{p}{(}\PYG{n}{minimo}\PYG{o}{=}\PYG{l+m+mi}{0}\PYG{p}{,} \PYG{n}{maximo}\PYG{o}{=}\PYG{l+m+mi}{9999}\PYG{p}{,} \PYG{n}{total\PYGZus{}numeros}\PYG{o}{=}\PYG{l+m+mi}{10}\PYG{p}{)}\PYG{p}{:}
    \PYG{l+s+sd}{\PYGZdq{}\PYGZdq{}\PYGZdq{} Generar varios numeros al azar definidos}
\PYG{l+s+sd}{        entre máximo y minimo (sin numeros duplicados) \PYGZdq{}\PYGZdq{}\PYGZdq{}}
    \PYG{k}{return} \PYG{p}{[}\PYG{l+m+mi}{0}\PYG{p}{,} \PYG{l+m+mi}{1}\PYG{p}{,} \PYG{l+m+mi}{2}\PYG{p}{,} \PYG{l+m+mi}{3}\PYG{p}{,} \PYG{l+m+mi}{4}\PYG{p}{,} \PYG{l+m+mi}{5}\PYG{p}{,} \PYG{l+m+mi}{6}\PYG{p}{,} \PYG{l+m+mi}{7}\PYG{p}{,} \PYG{l+m+mi}{8}\PYG{p}{,} \PYG{l+m+mi}{9}\PYG{p}{]}

\PYG{n}{numeros\PYGZus{}quiniela} \PYG{o}{=} \PYG{n}{generar\PYGZus{}quiniela}\PYG{p}{(}\PYG{p}{)}
\PYG{n+nb}{print}\PYG{p}{(}\PYG{l+s+sa}{f}\PYG{l+s+s1}{\PYGZsq{}}\PYG{l+s+s1}{Numeros resultantes }\PYG{l+s+si}{\PYGZob{}}\PYG{n}{numeros\PYGZus{}quiniela}\PYG{l+s+si}{\PYGZcb{}}\PYG{l+s+s1}{\PYGZsq{}}\PYG{p}{)}

\PYG{c+c1}{\PYGZsh{} \PYGZhy{}\PYGZhy{}\PYGZhy{}\PYGZhy{}\PYGZhy{}\PYGZhy{}\PYGZhy{}\PYGZhy{}\PYGZhy{}\PYGZhy{}\PYGZhy{}\PYGZhy{}\PYGZhy{}\PYGZhy{}\PYGZhy{}\PYGZhy{}\PYGZhy{}\PYGZhy{}\PYGZhy{}\PYGZhy{}\PYGZhy{}\PYGZhy{}\PYGZhy{}\PYGZhy{}\PYGZhy{}\PYGZhy{}\PYGZhy{}\PYGZhy{}\PYGZhy{}\PYGZhy{}\PYGZhy{}\PYGZhy{}\PYGZhy{}\PYGZhy{}\PYGZhy{}\PYGZhy{}\PYGZhy{}\PYGZhy{}\PYGZhy{}\PYGZhy{}\PYGZhy{}\PYGZhy{}\PYGZhy{}\PYGZhy{}\PYGZhy{}\PYGZhy{}\PYGZhy{}\PYGZhy{}\PYGZhy{}\PYGZhy{}\PYGZhy{}\PYGZhy{}\PYGZhy{}\PYGZhy{}\PYGZhy{}\PYGZhy{}\PYGZhy{}\PYGZhy{}\PYGZhy{}\PYGZhy{}\PYGZhy{}\PYGZhy{}\PYGZhy{}\PYGZhy{}\PYGZhy{}\PYGZhy{}\PYGZhy{}\PYGZhy{}\PYGZhy{}\PYGZhy{}\PYGZhy{}\PYGZhy{}}
\PYG{c+c1}{\PYGZsh{} NO BORRAR O MODIFICAR LAS LINEAS QUE SIGUEN}
\PYG{c+c1}{\PYGZsh{} \PYGZhy{}\PYGZhy{}\PYGZhy{}\PYGZhy{}\PYGZhy{}\PYGZhy{}\PYGZhy{}\PYGZhy{}\PYGZhy{}\PYGZhy{}\PYGZhy{}\PYGZhy{}\PYGZhy{}\PYGZhy{}\PYGZhy{}\PYGZhy{}\PYGZhy{}\PYGZhy{}\PYGZhy{}\PYGZhy{}\PYGZhy{}\PYGZhy{}\PYGZhy{}\PYGZhy{}\PYGZhy{}\PYGZhy{}\PYGZhy{}\PYGZhy{}\PYGZhy{}\PYGZhy{}\PYGZhy{}\PYGZhy{}\PYGZhy{}\PYGZhy{}\PYGZhy{}\PYGZhy{}\PYGZhy{}\PYGZhy{}\PYGZhy{}\PYGZhy{}\PYGZhy{}\PYGZhy{}\PYGZhy{}\PYGZhy{}\PYGZhy{}\PYGZhy{}\PYGZhy{}\PYGZhy{}\PYGZhy{}\PYGZhy{}\PYGZhy{}\PYGZhy{}\PYGZhy{}\PYGZhy{}\PYGZhy{}\PYGZhy{}\PYGZhy{}\PYGZhy{}\PYGZhy{}\PYGZhy{}\PYGZhy{}\PYGZhy{}\PYGZhy{}\PYGZhy{}\PYGZhy{}\PYGZhy{}\PYGZhy{}\PYGZhy{}\PYGZhy{}\PYGZhy{}\PYGZhy{}\PYGZhy{}}
\PYG{c+c1}{\PYGZsh{} Una vez terminada la tarea ejecutar este archivo.}
\PYG{c+c1}{\PYGZsh{} Si se ve la leyenda \PYGZsq{}Ejercicio terminado OK\PYGZsq{} el ejercicio se considera completado.}
\PYG{c+c1}{\PYGZsh{} La instruccion \PYGZdq{}assert\PYGZdq{} de Python lanzará un error si lo que se indica a}
\PYG{c+c1}{\PYGZsh{}   continuacion es falso.}
\PYG{c+c1}{\PYGZsh{} Si usas GitHub (o similares) podes hacer una nueva rama con esta solución,}
\PYG{c+c1}{\PYGZsh{}   crear un \PYGZdq{}pull request\PYGZdq{} y solicitar revision de un tercero.}

\PYG{c+c1}{\PYGZsh{} Prueba de resultados basicos}
\PYG{k}{assert} \PYG{n+nb}{type}\PYG{p}{(}\PYG{n}{numeros\PYGZus{}quiniela}\PYG{p}{)} \PYG{o}{==} \PYG{n+nb}{list}
\PYG{k}{assert} \PYG{n+nb}{len}\PYG{p}{(}\PYG{n}{numeros\PYGZus{}quiniela}\PYG{p}{)} \PYG{o}{==} \PYG{l+m+mi}{10}
\PYG{k}{for} \PYG{n}{n} \PYG{o+ow}{in} \PYG{n}{numeros\PYGZus{}quiniela}\PYG{p}{:}
    \PYG{k}{assert} \PYG{n+nb}{type}\PYG{p}{(}\PYG{n}{n}\PYG{p}{)} \PYG{o}{==} \PYG{n+nb}{int}
    \PYG{k}{assert} \PYG{n}{n} \PYG{o}{\PYGZgt{}}\PYG{o}{=} \PYG{l+m+mi}{0}
    \PYG{k}{assert} \PYG{n}{n} \PYG{o}{\PYGZlt{}}\PYG{o}{=} \PYG{l+m+mi}{9999}

\PYG{c+c1}{\PYGZsh{} Pruebas con otros parametros}
\PYG{n}{minimo} \PYG{o}{=} \PYG{l+m+mi}{90}
\PYG{n}{maximo} \PYG{o}{=} \PYG{l+m+mi}{200}
\PYG{n}{total\PYGZus{}numeros} \PYG{o}{=} \PYG{l+m+mi}{7}
\PYG{n}{numeros\PYGZus{}quiniela} \PYG{o}{=} \PYG{n}{generar\PYGZus{}quiniela}\PYG{p}{(}\PYG{n}{minimo}\PYG{o}{=}\PYG{n}{minimo}\PYG{p}{,} \PYG{n}{maximo}\PYG{o}{=}\PYG{n}{maximo}\PYG{p}{,} \PYG{n}{total\PYGZus{}numeros}\PYG{o}{=}\PYG{n}{total\PYGZus{}numeros}\PYG{p}{)}
\PYG{k}{assert} \PYG{n+nb}{type}\PYG{p}{(}\PYG{n}{numeros\PYGZus{}quiniela}\PYG{p}{)} \PYG{o}{==} \PYG{n+nb}{list}
\PYG{k}{assert} \PYG{n+nb}{len}\PYG{p}{(}\PYG{n}{numeros\PYGZus{}quiniela}\PYG{p}{)} \PYG{o}{==} \PYG{n}{total\PYGZus{}numeros}
\PYG{k}{for} \PYG{n}{n} \PYG{o+ow}{in} \PYG{n}{numeros\PYGZus{}quiniela}\PYG{p}{:}
    \PYG{k}{assert} \PYG{n+nb}{type}\PYG{p}{(}\PYG{n}{n}\PYG{p}{)} \PYG{o}{==} \PYG{n+nb}{int}
    \PYG{k}{assert} \PYG{n}{n} \PYG{o}{\PYGZgt{}}\PYG{o}{=} \PYG{n}{minimo}
    \PYG{k}{assert} \PYG{n}{n} \PYG{o}{\PYGZlt{}}\PYG{o}{=} \PYG{n}{maximo}

\PYG{n}{minimo} \PYG{o}{=} \PYG{l+m+mi}{0}
\PYG{n}{maximo} \PYG{o}{=} \PYG{l+m+mi}{100}
\PYG{n}{total\PYGZus{}numeros} \PYG{o}{=} \PYG{l+m+mi}{3}
\PYG{n}{numeros\PYGZus{}quiniela} \PYG{o}{=} \PYG{n}{generar\PYGZus{}quiniela}\PYG{p}{(}\PYG{n}{minimo}\PYG{o}{=}\PYG{n}{minimo}\PYG{p}{,} \PYG{n}{maximo}\PYG{o}{=}\PYG{n}{maximo}\PYG{p}{,} \PYG{n}{total\PYGZus{}numeros}\PYG{o}{=}\PYG{n}{total\PYGZus{}numeros}\PYG{p}{)}
\PYG{k}{assert} \PYG{n+nb}{type}\PYG{p}{(}\PYG{n}{numeros\PYGZus{}quiniela}\PYG{p}{)} \PYG{o}{==} \PYG{n+nb}{list}
\PYG{k}{assert} \PYG{n+nb}{len}\PYG{p}{(}\PYG{n}{numeros\PYGZus{}quiniela}\PYG{p}{)} \PYG{o}{==} \PYG{n}{total\PYGZus{}numeros}
\PYG{k}{for} \PYG{n}{n} \PYG{o+ow}{in} \PYG{n}{numeros\PYGZus{}quiniela}\PYG{p}{:}
    \PYG{k}{assert} \PYG{n+nb}{type}\PYG{p}{(}\PYG{n}{n}\PYG{p}{)} \PYG{o}{==} \PYG{n+nb}{int}
    \PYG{k}{assert} \PYG{n}{n} \PYG{o}{\PYGZgt{}}\PYG{o}{=} \PYG{n}{minimo}
    \PYG{k}{assert} \PYG{n}{n} \PYG{o}{\PYGZlt{}}\PYG{o}{=} \PYG{n}{maximo}

\PYG{n+nb}{print}\PYG{p}{(}\PYG{l+s+s1}{\PYGZsq{}}\PYG{l+s+s1}{Ejercicio terminado OK}\PYG{l+s+s1}{\PYGZsq{}}\PYG{p}{)}
\end{sphinxVerbatim}
\begin{itemize}
\item {} 
\sphinxAtStartPar
Hacer un PR con una propuesta de solución para el
\sphinxhref{https://github.com/avdata99/programacion-para-no-programadores/blob/master/ejercicios/ejercicio-047/ejercicio.py}{ejercicio 047}%
\begin{footnote}[22]\sphinxAtStartFootnote
\sphinxnolinkurl{https://github.com/avdata99/programacion-para-no-programadores/blob/master/ejercicios/ejercicio-047/ejercicio.py}
%
\end{footnote}

\end{itemize}

\begin{sphinxVerbatim}[commandchars=\\\{\}]
\PYG{l+s+sd}{\PYGZdq{}\PYGZdq{}\PYGZdq{}}
\PYG{l+s+sd}{La funcion \PYGZdq{}detectar\PYGZus{}escaleras\PYGZdq{} funciona bastante bien.}
\PYG{l+s+sd}{Toma una lista de numeros y busca escaleras ascendentes}
\PYG{l+s+sd}{dentro de la secuencia de numeros.}
\PYG{l+s+sd}{Esta funcion detecta casi todas las secuencias pero no todas}
\PYG{l+s+sd}{Tarea:}
\PYG{l+s+sd}{ \PYGZhy{} Arreglar la funcion para que pase los tests y encuentre}
\PYG{l+s+sd}{   todas las escaleras existentes}
\PYG{l+s+sd}{Nota: probablemente vas a necesitar depurar el código para}
\PYG{l+s+sd}{entender como funciona y por que falla.}
\PYG{l+s+sd}{\PYGZdq{}\PYGZdq{}\PYGZdq{}}

\PYG{k+kn}{from} \PYG{n+nn}{random} \PYG{k+kn}{import} \PYG{n}{randint}

\PYG{k}{def} \PYG{n+nf}{tirar\PYGZus{}dados}\PYG{p}{(}\PYG{n}{lados}\PYG{o}{=}\PYG{l+m+mi}{6}\PYG{p}{,} \PYG{n}{veces}\PYG{o}{=}\PYG{l+m+mi}{100}\PYG{p}{)}\PYG{p}{:}
    \PYG{l+s+sd}{\PYGZdq{}\PYGZdq{}\PYGZdq{} Generar aleatoriamente los tiros de un lado de cantidad}
\PYG{l+s+sd}{        de lados variable.}
\PYG{l+s+sd}{    \PYGZdq{}\PYGZdq{}\PYGZdq{}}
    
    \PYG{n}{tiros} \PYG{o}{=} \PYG{p}{[}\PYG{p}{]}
    \PYG{k}{for} \PYG{n}{\PYGZus{}} \PYG{o+ow}{in} \PYG{n+nb}{range}\PYG{p}{(}\PYG{n}{veces}\PYG{p}{)}\PYG{p}{:}
        \PYG{n}{numero} \PYG{o}{=} \PYG{n}{randint}\PYG{p}{(}\PYG{l+m+mi}{1}\PYG{p}{,} \PYG{n}{lados}\PYG{p}{)}
        \PYG{n}{tiros}\PYG{o}{.}\PYG{n}{append}\PYG{p}{(}\PYG{n}{numero}\PYG{p}{)}
    
    \PYG{k}{return} \PYG{n}{tiros}

\PYG{k}{def} \PYG{n+nf}{detectar\PYGZus{}escaleras}\PYG{p}{(}\PYG{n}{lista\PYGZus{}tiros}\PYG{p}{)}\PYG{p}{:}
    \PYG{l+s+sd}{\PYGZdq{}\PYGZdq{}\PYGZdq{} Detectar las escaleras ascendentes encontradas}
\PYG{l+s+sd}{        en los tiros de al menos 3 elementos \PYGZdq{}\PYGZdq{}\PYGZdq{}}
    
    \PYG{c+c1}{\PYGZsh{} aqui guardamos las escaleras que se consiguieron con 3 o más elementos}
    \PYG{n}{escaleras\PYGZus{}buenas} \PYG{o}{=} \PYG{p}{[}\PYG{p}{]}
    \PYG{c+c1}{\PYGZsh{} lista temporal para ir testeando todos los tiros}
    \PYG{n}{escalera} \PYG{o}{=} \PYG{p}{[}\PYG{p}{]}
    \PYG{k}{for} \PYG{n}{tiro} \PYG{o+ow}{in} \PYG{n}{lista\PYGZus{}tiros}\PYG{p}{:}
        \PYG{k}{if} \PYG{n+nb}{len}\PYG{p}{(}\PYG{n}{escalera}\PYG{p}{)} \PYG{o}{==} \PYG{l+m+mi}{0}\PYG{p}{:}
            \PYG{c+c1}{\PYGZsh{} recien empiezo a buscar}
            \PYG{n}{escalera}\PYG{o}{.}\PYG{n}{append}\PYG{p}{(}\PYG{n}{tiro}\PYG{p}{)}
        \PYG{k}{elif} \PYG{n}{escalera}\PYG{p}{[}\PYG{o}{\PYGZhy{}}\PYG{l+m+mi}{1}\PYG{p}{]} \PYG{o}{==} \PYG{n}{tiro} \PYG{o}{\PYGZhy{}} \PYG{l+m+mi}{1}\PYG{p}{:}
            \PYG{c+c1}{\PYGZsh{} La escalera de prueba ya empezo y el numero actual}
            \PYG{c+c1}{\PYGZsh{} esta en escalera con el ultimo encontrado}
            \PYG{n}{escalera}\PYG{o}{.}\PYG{n}{append}\PYG{p}{(}\PYG{n}{tiro}\PYG{p}{)}
        \PYG{k}{else}\PYG{p}{:}
            \PYG{c+c1}{\PYGZsh{} este tiro no esta en escalera con el anterior}
            \PYG{c+c1}{\PYGZsh{} ver si la escalera conseguida tiene 3 o mas elementos}
            \PYG{c+c1}{\PYGZsh{} para guardarle}
            \PYG{k}{if} \PYG{n+nb}{len}\PYG{p}{(}\PYG{n}{escalera}\PYG{p}{)} \PYG{o}{\PYGZgt{}}\PYG{o}{=} \PYG{l+m+mi}{3}\PYG{p}{:}
                \PYG{n}{escaleras\PYGZus{}buenas}\PYG{o}{.}\PYG{n}{append}\PYG{p}{(}\PYG{n}{escalera}\PYG{p}{)}
            \PYG{c+c1}{\PYGZsh{} como este número no sirve para la escalera anterior}
            \PYG{c+c1}{\PYGZsh{} es el que inicializa la deteccion de la nueva}
            \PYG{n}{escalera} \PYG{o}{=} \PYG{p}{[}\PYG{n}{tiro}\PYG{p}{]}

    \PYG{k}{return} \PYG{n}{escaleras\PYGZus{}buenas}


\PYG{n}{tiros} \PYG{o}{=} \PYG{n}{tirar\PYGZus{}dados}\PYG{p}{(}\PYG{n}{veces}\PYG{o}{=}\PYG{l+m+mi}{10000}\PYG{p}{)}
\PYG{n}{escaleras} \PYG{o}{=} \PYG{n}{detectar\PYGZus{}escaleras}\PYG{p}{(}\PYG{n}{tiros}\PYG{p}{)}
\PYG{n+nb}{print}\PYG{p}{(}\PYG{l+s+sa}{f}\PYG{l+s+s1}{\PYGZsq{}}\PYG{l+s+s1}{Escaleras encontradas }\PYG{l+s+si}{\PYGZob{}}\PYG{n+nb}{len}\PYG{p}{(}\PYG{n}{escaleras}\PYG{p}{)}\PYG{l+s+si}{\PYGZcb{}}\PYG{l+s+s1}{\PYGZsq{}}\PYG{p}{)}


\PYG{c+c1}{\PYGZsh{} \PYGZhy{}\PYGZhy{}\PYGZhy{}\PYGZhy{}\PYGZhy{}\PYGZhy{}\PYGZhy{}\PYGZhy{}\PYGZhy{}\PYGZhy{}\PYGZhy{}\PYGZhy{}\PYGZhy{}\PYGZhy{}\PYGZhy{}\PYGZhy{}\PYGZhy{}\PYGZhy{}\PYGZhy{}\PYGZhy{}\PYGZhy{}\PYGZhy{}\PYGZhy{}\PYGZhy{}\PYGZhy{}\PYGZhy{}\PYGZhy{}\PYGZhy{}\PYGZhy{}\PYGZhy{}\PYGZhy{}\PYGZhy{}\PYGZhy{}\PYGZhy{}\PYGZhy{}\PYGZhy{}\PYGZhy{}\PYGZhy{}\PYGZhy{}\PYGZhy{}\PYGZhy{}\PYGZhy{}\PYGZhy{}\PYGZhy{}\PYGZhy{}\PYGZhy{}\PYGZhy{}\PYGZhy{}\PYGZhy{}\PYGZhy{}\PYGZhy{}\PYGZhy{}\PYGZhy{}\PYGZhy{}\PYGZhy{}\PYGZhy{}\PYGZhy{}\PYGZhy{}\PYGZhy{}\PYGZhy{}\PYGZhy{}\PYGZhy{}\PYGZhy{}\PYGZhy{}\PYGZhy{}\PYGZhy{}\PYGZhy{}\PYGZhy{}\PYGZhy{}\PYGZhy{}\PYGZhy{}\PYGZhy{}}
\PYG{c+c1}{\PYGZsh{} NO BORRAR O MODIFICAR LAS LINEAS QUE SIGUEN}
\PYG{c+c1}{\PYGZsh{} \PYGZhy{}\PYGZhy{}\PYGZhy{}\PYGZhy{}\PYGZhy{}\PYGZhy{}\PYGZhy{}\PYGZhy{}\PYGZhy{}\PYGZhy{}\PYGZhy{}\PYGZhy{}\PYGZhy{}\PYGZhy{}\PYGZhy{}\PYGZhy{}\PYGZhy{}\PYGZhy{}\PYGZhy{}\PYGZhy{}\PYGZhy{}\PYGZhy{}\PYGZhy{}\PYGZhy{}\PYGZhy{}\PYGZhy{}\PYGZhy{}\PYGZhy{}\PYGZhy{}\PYGZhy{}\PYGZhy{}\PYGZhy{}\PYGZhy{}\PYGZhy{}\PYGZhy{}\PYGZhy{}\PYGZhy{}\PYGZhy{}\PYGZhy{}\PYGZhy{}\PYGZhy{}\PYGZhy{}\PYGZhy{}\PYGZhy{}\PYGZhy{}\PYGZhy{}\PYGZhy{}\PYGZhy{}\PYGZhy{}\PYGZhy{}\PYGZhy{}\PYGZhy{}\PYGZhy{}\PYGZhy{}\PYGZhy{}\PYGZhy{}\PYGZhy{}\PYGZhy{}\PYGZhy{}\PYGZhy{}\PYGZhy{}\PYGZhy{}\PYGZhy{}\PYGZhy{}\PYGZhy{}\PYGZhy{}\PYGZhy{}\PYGZhy{}\PYGZhy{}\PYGZhy{}\PYGZhy{}\PYGZhy{}}
\PYG{c+c1}{\PYGZsh{} Una vez terminada la tarea ejecutar este archivo.}
\PYG{c+c1}{\PYGZsh{} Si se ve la leyenda \PYGZsq{}Ejercicio terminado OK\PYGZsq{} el ejercicio se considera completado.}
\PYG{c+c1}{\PYGZsh{} La instruccion \PYGZdq{}assert\PYGZdq{} de Python lanzará un error si lo que se indica a}
\PYG{c+c1}{\PYGZsh{}   continuacion es falso.}
\PYG{c+c1}{\PYGZsh{} Si usas GitHub (o similares) podes hacer una nueva rama con esta solución,}
\PYG{c+c1}{\PYGZsh{}   crear un \PYGZdq{}pull request\PYGZdq{} y solicitar revision de un tercero.}

\PYG{c+c1}{\PYGZsh{} Probar tiros donde yo ya conozco los resultados}
\PYG{n}{tiros\PYGZus{}test} \PYG{o}{=} \PYG{p}{[}\PYG{l+m+mi}{1}\PYG{p}{,} \PYG{l+m+mi}{5}\PYG{p}{,} \PYG{l+m+mi}{6}\PYG{p}{,}
              \PYG{l+m+mi}{3}\PYG{p}{,} \PYG{l+m+mi}{4}\PYG{p}{,} \PYG{l+m+mi}{5}\PYG{p}{,}
              \PYG{l+m+mi}{1}\PYG{p}{,} \PYG{l+m+mi}{5}\PYG{p}{,} \PYG{l+m+mi}{6}\PYG{p}{,}
              \PYG{l+m+mi}{1}\PYG{p}{,} \PYG{l+m+mi}{2}\PYG{p}{,} \PYG{l+m+mi}{3}\PYG{p}{,}
              \PYG{l+m+mi}{2}\PYG{p}{,} \PYG{l+m+mi}{2}\PYG{p}{,} \PYG{l+m+mi}{2}\PYG{p}{]}

\PYG{n}{escaleras} \PYG{o}{=} \PYG{n}{detectar\PYGZus{}escaleras}\PYG{p}{(}\PYG{n}{tiros\PYGZus{}test}\PYG{p}{)}
\PYG{k}{assert} \PYG{n}{escaleras} \PYG{o}{==} \PYG{p}{[}
    \PYG{p}{[}\PYG{l+m+mi}{3}\PYG{p}{,} \PYG{l+m+mi}{4}\PYG{p}{,} \PYG{l+m+mi}{5}\PYG{p}{]}\PYG{p}{,}
    \PYG{p}{[}\PYG{l+m+mi}{1}\PYG{p}{,} \PYG{l+m+mi}{2}\PYG{p}{,} \PYG{l+m+mi}{3}\PYG{p}{]}\PYG{p}{,}
\PYG{p}{]}

\PYG{n}{tiros\PYGZus{}test} \PYG{o}{=} \PYG{p}{[}\PYG{l+m+mi}{1}\PYG{p}{,} \PYG{l+m+mi}{5}\PYG{p}{,} \PYG{l+m+mi}{2}\PYG{p}{,} \PYG{l+m+mi}{3}\PYG{p}{,} \PYG{l+m+mi}{2}\PYG{p}{,} \PYG{l+m+mi}{1}\PYG{p}{,} \PYG{l+m+mi}{4}\PYG{p}{,} \PYG{l+m+mi}{5}\PYG{p}{,} \PYG{l+m+mi}{5}\PYG{p}{,} \PYG{l+m+mi}{1}\PYG{p}{,} \PYG{l+m+mi}{2}\PYG{p}{,} \PYG{l+m+mi}{6}\PYG{p}{,} \PYG{l+m+mi}{3}\PYG{p}{,} \PYG{l+m+mi}{3}\PYG{p}{,} \PYG{l+m+mi}{3}\PYG{p}{,} \PYG{l+m+mi}{1}\PYG{p}{,} \PYG{l+m+mi}{2}\PYG{p}{,} \PYG{l+m+mi}{5}\PYG{p}{,} \PYG{l+m+mi}{5}\PYG{p}{,} \PYG{l+m+mi}{2}\PYG{p}{,} \PYG{l+m+mi}{4}\PYG{p}{,}
              \PYG{l+m+mi}{3}\PYG{p}{,} \PYG{l+m+mi}{4}\PYG{p}{,} \PYG{l+m+mi}{5}\PYG{p}{,}
              \PYG{l+m+mi}{2}\PYG{p}{,} \PYG{l+m+mi}{3}\PYG{p}{,} \PYG{l+m+mi}{4}\PYG{p}{,} \PYG{l+m+mi}{5}\PYG{p}{,}
              \PYG{l+m+mi}{1}\PYG{p}{,} \PYG{l+m+mi}{5}\PYG{p}{,} \PYG{l+m+mi}{2}\PYG{p}{,} \PYG{l+m+mi}{4}\PYG{p}{,} \PYG{l+m+mi}{2}\PYG{p}{,} \PYG{l+m+mi}{5}\PYG{p}{,} \PYG{l+m+mi}{6}\PYG{p}{,} \PYG{l+m+mi}{6}\PYG{p}{,} \PYG{l+m+mi}{5}\PYG{p}{,} \PYG{l+m+mi}{5}\PYG{p}{,} \PYG{l+m+mi}{4}\PYG{p}{,} \PYG{l+m+mi}{2}\PYG{p}{,} \PYG{l+m+mi}{6}\PYG{p}{,} \PYG{l+m+mi}{1}\PYG{p}{,} \PYG{l+m+mi}{4}\PYG{p}{,} \PYG{l+m+mi}{5}\PYG{p}{,} \PYG{l+m+mi}{3}\PYG{p}{,} \PYG{l+m+mi}{1}\PYG{p}{,} \PYG{l+m+mi}{6}\PYG{p}{,} \PYG{l+m+mi}{2}\PYG{p}{,} \PYG{l+m+mi}{2}\PYG{p}{,}
              \PYG{l+m+mi}{1}\PYG{p}{,} \PYG{l+m+mi}{2}\PYG{p}{,} \PYG{l+m+mi}{3}\PYG{p}{,} \PYG{l+m+mi}{4}\PYG{p}{,} \PYG{l+m+mi}{5}\PYG{p}{,}
              \PYG{l+m+mi}{3}\PYG{p}{,} \PYG{l+m+mi}{1}\PYG{p}{,} \PYG{l+m+mi}{2}\PYG{p}{,} \PYG{l+m+mi}{6}\PYG{p}{,} \PYG{l+m+mi}{5}\PYG{p}{,} \PYG{l+m+mi}{6}\PYG{p}{,} \PYG{l+m+mi}{2}\PYG{p}{,} \PYG{l+m+mi}{5}\PYG{p}{,} \PYG{l+m+mi}{1}\PYG{p}{,} \PYG{l+m+mi}{4}\PYG{p}{,} \PYG{l+m+mi}{4}\PYG{p}{,} \PYG{l+m+mi}{5}\PYG{p}{,} \PYG{l+m+mi}{4}\PYG{p}{,} \PYG{l+m+mi}{3}\PYG{p}{,} \PYG{l+m+mi}{2}\PYG{p}{,} \PYG{l+m+mi}{1}\PYG{p}{,} \PYG{l+m+mi}{1}\PYG{p}{,} \PYG{l+m+mi}{3}\PYG{p}{,} \PYG{l+m+mi}{5}\PYG{p}{,} \PYG{l+m+mi}{6}\PYG{p}{,} \PYG{l+m+mi}{1}\PYG{p}{,}
              \PYG{l+m+mi}{1}\PYG{p}{,} \PYG{l+m+mi}{2}\PYG{p}{,} \PYG{l+m+mi}{3}\PYG{p}{]}

\PYG{n}{escaleras} \PYG{o}{=} \PYG{n}{detectar\PYGZus{}escaleras}\PYG{p}{(}\PYG{n}{tiros\PYGZus{}test}\PYG{p}{)}
\PYG{k}{assert} \PYG{n}{escaleras} \PYG{o}{==} \PYG{p}{[}
    \PYG{p}{[}\PYG{l+m+mi}{3}\PYG{p}{,} \PYG{l+m+mi}{4}\PYG{p}{,} \PYG{l+m+mi}{5}\PYG{p}{]}\PYG{p}{,}
    \PYG{p}{[}\PYG{l+m+mi}{2}\PYG{p}{,} \PYG{l+m+mi}{3}\PYG{p}{,} \PYG{l+m+mi}{4}\PYG{p}{,} \PYG{l+m+mi}{5}\PYG{p}{]}\PYG{p}{,}
    \PYG{p}{[}\PYG{l+m+mi}{1}\PYG{p}{,} \PYG{l+m+mi}{2}\PYG{p}{,} \PYG{l+m+mi}{3}\PYG{p}{,} \PYG{l+m+mi}{4}\PYG{p}{,} \PYG{l+m+mi}{5}\PYG{p}{]}\PYG{p}{,}
    \PYG{p}{[}\PYG{l+m+mi}{1}\PYG{p}{,} \PYG{l+m+mi}{2}\PYG{p}{,} \PYG{l+m+mi}{3}\PYG{p}{]}
\PYG{p}{]}

\PYG{n}{tiros\PYGZus{}test} \PYG{o}{=} \PYG{p}{[}\PYG{l+m+mi}{1}\PYG{p}{,} \PYG{l+m+mi}{2}\PYG{p}{,} \PYG{l+m+mi}{3}\PYG{p}{]}
\PYG{n}{escaleras} \PYG{o}{=} \PYG{n}{detectar\PYGZus{}escaleras}\PYG{p}{(}\PYG{n}{tiros\PYGZus{}test}\PYG{p}{)}
\PYG{k}{assert} \PYG{n}{escaleras} \PYG{o}{==} \PYG{p}{[}\PYG{p}{[}\PYG{l+m+mi}{1}\PYG{p}{,} \PYG{l+m+mi}{2}\PYG{p}{,} \PYG{l+m+mi}{3}\PYG{p}{]}\PYG{p}{]}

\PYG{n+nb}{print}\PYG{p}{(}\PYG{l+s+s1}{\PYGZsq{}}\PYG{l+s+s1}{Ejercicio terminado OK}\PYG{l+s+s1}{\PYGZsq{}}\PYG{p}{)}
\end{sphinxVerbatim}

\sphinxstepscope


\chapter{Librerías incluida: \sphinxstyleliteralintitle{\sphinxupquote{datetime}}}
\label{\detokenize{datetime:librerias-incluida-datetime}}\label{\detokenize{datetime::doc}}
\sphinxAtStartPar
Asi como hay funciones incluidas (\sphinxstyleemphasis{built\sphinxhyphen{}ins}) que se pueden usar sin
estar declaradas tambien hay más herramientas de Python disponibles pero
que requieren ser \sphinxstyleemphasis{importadas}.

\sphinxAtStartPar
Importar (con el comando \sphinxcode{\sphinxupquote{import}} o de la forma \sphinxcode{\sphinxupquote{from X import Y}})
en Python es disponibilizar nueva herramientas (fuenciones y otras) en
nuestro código.


\section{Fecha y hora: \sphinxstyleliteralintitle{\sphinxupquote{datetime}}}
\label{\detokenize{datetime:fecha-y-hora-datetime}}
\sphinxAtStartPar
El manejo de fechas y horas básico en Python se hace con la librería \sphinxcode{\sphinxupquote{datetime}}.


\section{Fechas simples con \sphinxstyleliteralintitle{\sphinxupquote{date}}}
\label{\detokenize{datetime:fechas-simples-con-date}}
\sphinxAtStartPar
Si solo necesitamos una fecha general solo con día + mes + año podemos usar
objetos de tipo \sphinxcode{\sphinxupquote{date}}.

\begin{sphinxVerbatim}[commandchars=\\\{\}]
\PYG{k+kn}{from} \PYG{n+nn}{datetime} \PYG{k+kn}{import} \PYG{n}{date}

\PYG{n}{hoy} \PYG{o}{=} \PYG{n}{date}\PYG{o}{.}\PYG{n}{today}\PYG{p}{(}\PYG{p}{)}
\PYG{n+nb}{print}\PYG{p}{(}\PYG{n}{hoy}\PYG{p}{)}
\PYG{c+c1}{\PYGZsh{} 2022\PYGZhy{}02\PYGZhy{}12}
\PYG{n+nb}{print}\PYG{p}{(}\PYG{n+nb}{type}\PYG{p}{(}\PYG{n}{hoy}\PYG{p}{)}\PYG{p}{)}
\PYG{c+c1}{\PYGZsh{} \PYGZlt{}class \PYGZsq{}datetime.date\PYGZsq{}\PYGZgt{}}

\PYG{c+c1}{\PYGZsh{} date (año, mes, dia)}
\PYG{n}{agosto\PYGZus{}27\PYGZus{}2022} \PYG{o}{=} \PYG{n}{date}\PYG{p}{(}\PYG{l+m+mi}{2022}\PYG{p}{,} \PYG{l+m+mi}{8}\PYG{p}{,} \PYG{l+m+mi}{27}\PYG{p}{)}
\end{sphinxVerbatim}


\section{Variación de tiempo con \sphinxstyleliteralintitle{\sphinxupquote{timedelta}}}
\label{\detokenize{datetime:variacion-de-tiempo-con-timedelta}}
\sphinxAtStartPar
Para sumar o restar persiodos de tiempo a una fecha existe \sphinxcode{\sphinxupquote{timedelta}}:

\begin{sphinxVerbatim}[commandchars=\\\{\}]
\PYG{k+kn}{from} \PYG{n+nn}{datetime} \PYG{k+kn}{import} \PYG{n}{date}\PYG{p}{,} \PYG{n}{timedelta}
\PYG{n}{hoy} \PYG{o}{=} \PYG{n}{date}\PYG{o}{.}\PYG{n}{today}\PYG{p}{(}\PYG{p}{)}
\PYG{n}{manana} \PYG{o}{=} \PYG{n}{hoy} \PYG{o}{+} \PYG{n}{timedelta}\PYG{p}{(}\PYG{n}{days}\PYG{o}{=}\PYG{l+m+mi}{1}\PYG{p}{)}
\PYG{n+nb}{print}\PYG{p}{(}\PYG{n}{manana}\PYG{p}{)}
\PYG{c+c1}{\PYGZsh{} 2022\PYGZhy{}02\PYGZhy{}13}
\end{sphinxVerbatim}


\section{Fecha + hora = momento exacto con \sphinxstyleliteralintitle{\sphinxupquote{datetime}}}
\label{\detokenize{datetime:fecha-hora-momento-exacto-con-datetime}}
\sphinxAtStartPar
Si necesitamos más precision que solo dia + mes + año debemos usar \sphinxcode{\sphinxupquote{datetime.datetime}}

\sphinxAtStartPar
\sphinxstylestrong{Nota: datetime como libreria incluye un objeto con el mismo nombre pero
son cosas distintas. Uno es la libreria y otro es la clase para construir
fecha/horas o datetimes.}

\begin{sphinxVerbatim}[commandchars=\\\{\}]
\PYG{k+kn}{from} \PYG{n+nn}{datetime} \PYG{k+kn}{import} \PYG{n}{datetime}\PYG{p}{,} \PYG{n}{timedelta}

\PYG{n}{ahora} \PYG{o}{=} \PYG{n}{datetime}\PYG{o}{.}\PYG{n}{now}\PYG{p}{(}\PYG{p}{)}
\PYG{n+nb}{print}\PYG{p}{(}\PYG{n}{ahora}\PYG{p}{)}
\PYG{c+c1}{\PYGZsh{} 2022\PYGZhy{}02\PYGZhy{}12 14:35:16.687589}

\PYG{n+nb}{print}\PYG{p}{(}\PYG{n+nb}{type}\PYG{p}{(}\PYG{n}{ahora}\PYG{p}{)}\PYG{p}{)}
\PYG{c+c1}{\PYGZsh{} \PYGZlt{}class \PYGZsq{}datetime.datetime\PYGZsq{}\PYGZgt{}}

\PYG{n}{en\PYGZus{}15} \PYG{o}{=} \PYG{n}{ahora} \PYG{o}{+} \PYG{n}{timedelta}\PYG{p}{(}\PYG{n}{minutes}\PYG{o}{=}\PYG{l+m+mi}{15}\PYG{p}{)}
\PYG{n+nb}{print}\PYG{p}{(}\PYG{n}{en\PYGZus{}15}\PYG{p}{)}
\PYG{c+c1}{\PYGZsh{} 2022\PYGZhy{}02\PYGZhy{}12 14:50:16.687589}

\PYG{c+c1}{\PYGZsh{} la diferencia entre dos fechas, es un timedelta}
\PYG{n}{a} \PYG{o}{=} \PYG{n}{datetime}\PYG{p}{(}\PYG{l+m+mi}{2022}\PYG{p}{,}\PYG{l+m+mi}{3}\PYG{p}{,}\PYG{l+m+mi}{4}\PYG{p}{)}
\PYG{n}{b} \PYG{o}{=} \PYG{n}{datetime}\PYG{p}{(}\PYG{l+m+mi}{1988}\PYG{p}{,}\PYG{l+m+mi}{2}\PYG{p}{,}\PYG{l+m+mi}{11}\PYG{p}{,}\PYG{l+m+mi}{6}\PYG{p}{,}\PYG{l+m+mi}{7}\PYG{p}{,}\PYG{l+m+mi}{8}\PYG{p}{)}
\PYG{n}{c} \PYG{o}{=} \PYG{n}{a} \PYG{o}{\PYGZhy{}} \PYG{n}{b}
\PYG{n+nb}{print}\PYG{p}{(}\PYG{n}{c}\PYG{p}{)}
\PYG{c+c1}{\PYGZsh{} imprime 12439 days, 17:52:52}
\PYG{n+nb}{type}\PYG{p}{(}\PYG{n}{c}\PYG{p}{)}
\PYG{c+c1}{\PYGZsh{} imprime \PYGZlt{}class \PYGZsq{}datetime.timedelta\PYGZsq{}\PYGZgt{}}
\end{sphinxVerbatim}


\section{fecha \textless{}—\textgreater{} string}
\label{\detokenize{datetime:fecha-string}}
\sphinxAtStartPar
En muchos casos es requirido pasar de fecha a strings y viceversa.
Para estos casos se usan las funciones \sphinxcode{\sphinxupquote{strftime}} (fecha a
string) y \sphinxcode{\sphinxupquote{strptime}} (string a fecha).

\begin{sphinxVerbatim}[commandchars=\\\{\}]
\PYG{k+kn}{from} \PYG{n+nn}{datetime} \PYG{k+kn}{import} \PYG{n}{date}\PYG{p}{,} \PYG{n}{datetime}

\PYG{n}{hoy} \PYG{o}{=} \PYG{n}{date}\PYG{o}{.}\PYG{n}{today}\PYG{p}{(}\PYG{p}{)}
\PYG{n+nb}{print}\PYG{p}{(}\PYG{n}{hoy}\PYG{o}{.}\PYG{n}{strftime}\PYG{p}{(}\PYG{l+s+s2}{\PYGZdq{}}\PYG{l+s+si}{\PYGZpc{}d}\PYG{l+s+s2}{/}\PYG{l+s+s2}{\PYGZpc{}}\PYG{l+s+s2}{m/}\PYG{l+s+s2}{\PYGZpc{}}\PYG{l+s+s2}{Y}\PYG{l+s+s2}{\PYGZdq{}}\PYG{p}{)}\PYG{p}{)}
\PYG{c+c1}{\PYGZsh{} 12/02/2022}
\PYG{n+nb}{print}\PYG{p}{(}\PYG{n}{hoy}\PYG{o}{.}\PYG{n}{strftime}\PYG{p}{(}\PYG{l+s+s2}{\PYGZdq{}}\PYG{l+s+s2}{\PYGZpc{}}\PYG{l+s+s2}{Y\PYGZhy{}}\PYG{l+s+s2}{\PYGZpc{}}\PYG{l+s+s2}{m\PYGZhy{}}\PYG{l+s+si}{\PYGZpc{}d}\PYG{l+s+s2}{\PYGZdq{}}\PYG{p}{)}\PYG{p}{)}
\PYG{c+c1}{\PYGZsh{} 2022\PYGZhy{}02\PYGZhy{}12}
\PYG{n+nb}{print}\PYG{p}{(}\PYG{n}{hoy}\PYG{o}{.}\PYG{n}{strftime}\PYG{p}{(}\PYG{l+s+s2}{\PYGZdq{}}\PYG{l+s+si}{\PYGZpc{}d}\PYG{l+s+s2}{ de }\PYG{l+s+s2}{\PYGZpc{}}\PYG{l+s+s2}{B de }\PYG{l+s+s2}{\PYGZpc{}}\PYG{l+s+s2}{Y}\PYG{l+s+s2}{\PYGZdq{}}\PYG{p}{)}\PYG{p}{)}
\PYG{c+c1}{\PYGZsh{} 12 de February de 2022}

\PYG{c+c1}{\PYGZsh{} pasar de string a fecha}
\PYG{n}{fecha\PYGZus{}str} \PYG{o}{=} \PYG{l+s+s1}{\PYGZsq{}}\PYG{l+s+s1}{2022\PYGZhy{}09\PYGZhy{}21}\PYG{l+s+s1}{\PYGZsq{}}
\PYG{n}{fecha} \PYG{o}{=} \PYG{n}{datetime}\PYG{o}{.}\PYG{n}{strptime}\PYG{p}{(}\PYG{n}{fecha\PYGZus{}str}\PYG{p}{,} \PYG{l+s+s2}{\PYGZdq{}}\PYG{l+s+s2}{\PYGZpc{}}\PYG{l+s+s2}{Y\PYGZhy{}}\PYG{l+s+s2}{\PYGZpc{}}\PYG{l+s+s2}{m\PYGZhy{}}\PYG{l+s+si}{\PYGZpc{}d}\PYG{l+s+s2}{\PYGZdq{}}\PYG{p}{)}
\PYG{n+nb}{print}\PYG{p}{(}\PYG{n}{fecha}\PYG{p}{)}
\PYG{c+c1}{\PYGZsh{} 2022\PYGZhy{}09\PYGZhy{}21 00:00:00}
\end{sphinxVerbatim}


\subsection{Tareas}
\label{\detokenize{datetime:tareas}}\begin{itemize}
\item {} 
\sphinxAtStartPar
Hacer una función que le pida al usuario que inserte su fecha
de nacimiento y se le devuelva hace cuantos días nació.

\end{itemize}

\sphinxstepscope


\chapter{Clases y objetos}
\label{\detokenize{class:clases-y-objetos}}\label{\detokenize{class::doc}}
\sphinxAtStartPar
En Python todo es un objeto. Por ejemplo, todas las variables para almacenar
textos son \sphinxstyleemphasis{objetos} del tipo \sphinxcode{\sphinxupquote{str}} (les decimos \sphinxstyleemphasis{strings}).
Hemos visto ya que estos objetos tienen propiedades y funciones (\sphinxcode{\sphinxupquote{upper}},
\sphinxcode{\sphinxupquote{lower}}, \sphinxcode{\sphinxupquote{strip}}, etc).
¿Te preguntaste dónde estan definidas esas funciones?
¿Quien decide que puede y que no puede hacerse con un objeto?

\sphinxAtStartPar
La respuestas a todo son las \sphinxstylestrong{clases}. En Python (y en cualquier lenguaje que use
\sphinxstyleemphasis{Programación Orientada a Objetos}) podemos definir nuestros propios tipos de objetos.
Cuando los definimos hacemos la elección de que como funcionará y que propiedades y
funciones incluirá.


\section{Mi primera clase}
\label{\detokenize{class:mi-primera-clase}}
\sphinxAtStartPar
Veamos un ejemplo basico. Queremos definir un tipo de dato nuevo: \sphinxcode{\sphinxupquote{Persona}}.

\fvset{hllines={, 1,}}%
\begin{sphinxVerbatim}[commandchars=\\\{\},numbers=left,firstnumber=1,stepnumber=1]
\PYG{k}{class} \PYG{n+nc}{Persona}\PYG{p}{:}
    \PYG{k}{def} \PYG{n+nf+fm}{\PYGZus{}\PYGZus{}init\PYGZus{}\PYGZus{}}\PYG{p}{(}\PYG{n+nb+bp}{self}\PYG{p}{,} \PYG{n}{nombre}\PYG{p}{,} \PYG{n}{apellido}\PYG{p}{)}\PYG{p}{:}
        \PYG{n+nb+bp}{self}\PYG{o}{.}\PYG{n}{nombre} \PYG{o}{=} \PYG{n}{nombre}
        \PYG{n+nb+bp}{self}\PYG{o}{.}\PYG{n}{apellido} \PYG{o}{=} \PYG{n}{apellido}

\PYG{c+c1}{\PYGZsh{} \PYGZhy{}\PYGZhy{}\PYGZhy{}\PYGZhy{}\PYGZhy{}\PYGZhy{}\PYGZhy{}\PYGZhy{}\PYGZhy{}\PYGZhy{}\PYGZhy{}\PYGZhy{}\PYGZhy{}\PYGZhy{}\PYGZhy{}\PYGZhy{}\PYGZhy{}\PYGZhy{}\PYGZhy{} FIN DE LA CLASE \PYGZhy{}\PYGZhy{}\PYGZhy{}\PYGZhy{}\PYGZhy{}\PYGZhy{}\PYGZhy{}\PYGZhy{}\PYGZhy{}\PYGZhy{}\PYGZhy{}\PYGZhy{}\PYGZhy{}\PYGZhy{}\PYGZhy{}\PYGZhy{}\PYGZhy{}\PYGZhy{}\PYGZhy{}\PYGZhy{}\PYGZhy{}\PYGZhy{}\PYGZhy{}\PYGZhy{}\PYGZhy{}\PYGZhy{}\PYGZhy{}\PYGZhy{}\PYGZhy{}\PYGZhy{}\PYGZhy{}\PYGZhy{}\PYGZhy{}\PYGZhy{}\PYGZhy{}\PYGZhy{}}

\PYG{c+c1}{\PYGZsh{} Crear nuestros objetos de tipo Persona}
\PYG{n}{juan} \PYG{o}{=} \PYG{n}{Persona}\PYG{p}{(}\PYG{n}{nombre}\PYG{o}{=}\PYG{l+s+s1}{\PYGZsq{}}\PYG{l+s+s1}{juan carlos}\PYG{l+s+s1}{\PYGZsq{}}\PYG{p}{,} \PYG{n}{apellido}\PYG{o}{=}\PYG{l+s+s1}{\PYGZsq{}}\PYG{l+s+s1}{perez}\PYG{l+s+s1}{\PYGZsq{}}\PYG{p}{)}
\PYG{n}{pedro} \PYG{o}{=} \PYG{n}{Persona}\PYG{p}{(}\PYG{n}{nombre}\PYG{o}{=}\PYG{l+s+s1}{\PYGZsq{}}\PYG{l+s+s1}{pedro }\PYG{l+s+s1}{\PYGZsq{}}\PYG{p}{,} \PYG{n}{apellido}\PYG{o}{=}\PYG{l+s+s1}{\PYGZsq{}}\PYG{l+s+s1}{gomez}\PYG{l+s+s1}{\PYGZsq{}}\PYG{p}{)}
\PYG{n}{luis} \PYG{o}{=} \PYG{n}{Persona}\PYG{p}{(}\PYG{l+s+s1}{\PYGZsq{}}\PYG{l+s+s1}{Luis}\PYG{l+s+s1}{\PYGZsq{}}\PYG{p}{,} \PYG{l+s+s1}{\PYGZsq{}}\PYG{l+s+s1}{Velez }\PYG{l+s+s1}{\PYGZsq{}}\PYG{p}{)}

\PYG{c+c1}{\PYGZsh{} Ver el tipo}
\PYG{n+nb}{type}\PYG{p}{(}\PYG{n}{juan}\PYG{p}{)}
\PYG{c+c1}{\PYGZsh{} imprime \PYGZlt{}class \PYGZsq{}\PYGZus{}\PYGZus{}main\PYGZus{}\PYGZus{}.Persona\PYGZsq{}\PYGZgt{}}

\PYG{c+c1}{\PYGZsh{} imprimir propiedades de nuestro objeto}
\PYG{n+nb}{print}\PYG{p}{(}\PYG{l+s+sa}{f}\PYG{l+s+s1}{\PYGZsq{}}\PYG{l+s+si}{\PYGZob{}}\PYG{n}{luis}\PYG{o}{.}\PYG{n}{nombre}\PYG{l+s+si}{\PYGZcb{}}\PYG{l+s+s1}{ }\PYG{l+s+si}{\PYGZob{}}\PYG{n}{luis}\PYG{o}{.}\PYG{n}{apellido}\PYG{l+s+si}{\PYGZcb{}}\PYG{l+s+s1}{\PYGZsq{}}\PYG{p}{)}
\PYG{l+s+s1}{\PYGZsq{}}\PYG{l+s+s1}{Luis Velez}\PYG{l+s+s1}{\PYGZsq{}}
\end{sphinxVerbatim}
\sphinxresetverbatimhllines


\subsection{Anatomia de una clase}
\label{\detokenize{class:anatomia-de-una-clase}}\begin{itemize}
\item {} 
\sphinxAtStartPar
\sphinxcode{\sphinxupquote{class}} es una palabra reservada de Python (no la podemos usar como nombre de variable)
que usamos para indicar que estamos definiendo una clase.

\item {} 
\sphinxAtStartPar
Despues de \sphinxcode{\sphinxupquote{class}} agregamos el nombre de nuestra clase. Se deben cumplir las mismas
reglas que para los nombres de variables (no pueden empezar con números, no pueden tener
espacios, etc). Se suele usar aqui el formato \sphinxstyleemphasis{CamelCase} y siempre con mayúscula inicial.
No es obligatorio.

\item {} 
\sphinxAtStartPar
Despues del nombre de la clase podemos colocar parentesis (no es obligatorio). Se usarán
en caso de querer aprovechar algo llamado \sphinxstyleemphasis{Herencia} que veremos más adelante.

\item {} 
\sphinxAtStartPar
Finalmente agregamos \sphinxstyleemphasis{:} (dos puntos) para indicar que terminamos de definir el encabezado
de la clase y vamos a comenzar con el bloque de código de ella.

\item {} 
\sphinxAtStartPar
El código de la clase debe estar tabulado hacia la derecha. \sphinxstylestrong{Esta es una de las grandes
diferencias que Python tiene con los demás lenguajes de programación}. El código propio de
la clase comienza tabulado y termina cuando el código vuelve a la izquierda. Muchos otros
lenguajes de programación usan las llaves \sphinxcode{\sphinxupquote{\{\}}} para delimitar donde empiezan y terminan
los bloques de código.

\item {} 
\sphinxAtStartPar
Dentro del código de la clase debemos definir las propiedades y funciones que queremos que
tengan nuestros objetos. Podemos agregar tantas propiedades y funciones como sean necesarias.
Veremos más detalles a continuación.

\item {} 
\sphinxAtStartPar
\sphinxcode{\sphinxupquote{\_\_init\_\_}} es hasta aquí nuestra única función (nos damos cuenta porque usa el \sphinxcode{\sphinxupquote{def}} que
ya conocemos junto a un grupo de parámetros). Cuando veamos estos guiones bajos dobles
debemos interpretar que es una herramienta interna de Python. En este caso, todas las clases
tienen una función de inicialización (y siempre se llama \sphinxcode{\sphinxupquote{\_\_init\_\_}}).
Ser la función inicializadora quiere decir que se va a ejecutar cuando
los usuarios de la clase la usen para construir un objeto de este tipo.

\item {} 
\sphinxAtStartPar
\sphinxcode{\sphinxupquote{self}} es el primer parámetro de la función \sphinxcode{\sphinxupquote{\_\_init\_\_}}. Hablaremos en particular de este
parámetro más adelante. Por ahora podemos decir que \sphinxstylestrong{es obligatorio que todas las funciones
de las clases tengan un parametro que represente a cada objeto creado con esta clase}. Es
por esto que veremos a \sphinxcode{\sphinxupquote{self}} en (\sphinxstyleemphasis{casi}, hay situaciones especiales) todas las funciones
de una clase.

\item {} 
\sphinxAtStartPar
Una vez definida la clase (y terminada anulando la tabulación y volviendo a la izquierda
el código), esta se puede llamar simplemente con su nombre y entre paréntesis todos los
parámetros que se hayan definido en \sphinxcode{\sphinxupquote{\_\_init\_\_}} (despues de \sphinxcode{\sphinxupquote{self}}).

\end{itemize}


\subsection{Clases y objetos}
\label{\detokenize{class:id1}}
\sphinxAtStartPar
Las clases se usan para definir el comportamiento de los objetos que podremos crear a partir
de ellas. Al proceso de crear un nuevo objeto se le dice \sphinxstyleemphasis{instanciar} y a cada objeto se le
puede llamar \sphinxstyleemphasis{instancia}.
En las siguientes líneas …

\begin{sphinxVerbatim}[commandchars=\\\{\}]
\PYG{n}{juan} \PYG{o}{=} \PYG{n}{Persona}\PYG{p}{(}\PYG{n}{nombre}\PYG{o}{=}\PYG{l+s+s1}{\PYGZsq{}}\PYG{l+s+s1}{juan carlos}\PYG{l+s+s1}{\PYGZsq{}}\PYG{p}{,} \PYG{n}{apellido}\PYG{o}{=}\PYG{l+s+s1}{\PYGZsq{}}\PYG{l+s+s1}{perez}\PYG{l+s+s1}{\PYGZsq{}}\PYG{p}{)}
\PYG{n}{pedro} \PYG{o}{=} \PYG{n}{Persona}\PYG{p}{(}\PYG{n}{nombre}\PYG{o}{=}\PYG{l+s+s1}{\PYGZsq{}}\PYG{l+s+s1}{pedro}\PYG{l+s+s1}{\PYGZsq{}}\PYG{p}{,} \PYG{n}{apellido}\PYG{o}{=}\PYG{l+s+s1}{\PYGZsq{}}\PYG{l+s+s1}{gomez}\PYG{l+s+s1}{\PYGZsq{}}\PYG{p}{)}
\PYG{c+c1}{\PYGZsh{} como en las funciones, no es obligatorio nombrar los parámetros}
\PYG{n}{luis} \PYG{o}{=} \PYG{n}{Persona}\PYG{p}{(}\PYG{l+s+s1}{\PYGZsq{}}\PYG{l+s+s1}{Luis}\PYG{l+s+s1}{\PYGZsq{}}\PYG{p}{,} \PYG{l+s+s1}{\PYGZsq{}}\PYG{l+s+s1}{Velez }\PYG{l+s+s1}{\PYGZsq{}}\PYG{p}{)}
\end{sphinxVerbatim}

\sphinxAtStartPar
se crean tres \sphinxstyleemphasis{instancias} del tipo \sphinxcode{\sphinxupquote{Persona}}. Cada una de ellas es un objeto distinto y por
lo tanto las propiedades que contienen son independientes y se procesan de manera aislada.

\begin{sphinxVerbatim}[commandchars=\\\{\}]
\PYG{n+nb}{print}\PYG{p}{(}\PYG{n}{juan}\PYG{o}{.}\PYG{n}{nombre}\PYG{p}{)}
\PYG{l+s+s1}{\PYGZsq{}}\PYG{l+s+s1}{juan carlos}\PYG{l+s+s1}{\PYGZsq{}}
\PYG{n+nb}{print}\PYG{p}{(}\PYG{n}{pedro}\PYG{o}{.}\PYG{n}{nombre}\PYG{p}{)}
\PYG{l+s+s1}{\PYGZsq{}}\PYG{l+s+s1}{pedro}\PYG{l+s+s1}{\PYGZsq{}}
\end{sphinxVerbatim}

\sphinxAtStartPar
Podemos pensar a los objetos o \sphinxstyleemphasis{instancias} como una versión concreta de una clase.


\subsection{¿\sphinxstyleliteralintitle{\sphinxupquote{self}}?}
\label{\detokenize{class:self}}
\sphinxAtStartPar
Salvo algunas excepciones todas las funciones de las clases deben tener como primer
parámetro a \sphinxcode{\sphinxupquote{self}}. De esta forma, todo el objeto estará disponible dentro de cada
función de la clase. Esto es obligatorio y olvidar colocarla generará errores difíciles
de detectar en nuestras primeras experiencias con clases.

\sphinxAtStartPar
Cuando llamamos a funciones de la clase que usan \sphinxcode{\sphinxupquote{self}}, no debemos pasar nada en
lugar de este paràmetro. Debemos ignorarlo cuando estamos usando nuestro objeto.
Esto es visible en todos los ejemplos usados en este manual.


\subsection{Contenido de un clase}
\label{\detokenize{class:contenido-de-un-clase}}
\sphinxAtStartPar
Dentro de la clase podemos definir las propiedades y funciones que nuestros objetos tendrán
cuando sean instanciados.

\sphinxAtStartPar
Cuando escribimos \sphinxcode{\sphinxupquote{self.PROPIEDAD = VALOR}} estamos indicando que los usuarios podrán usar
estas propiedades en los objetos definidos.

\sphinxAtStartPar
Estas líneas …

\fvset{hllines={, 3, 4,}}%
\begin{sphinxVerbatim}[commandchars=\\\{\}]
\PYG{k}{class} \PYG{n+nc}{Persona}\PYG{p}{:}
    \PYG{k}{def} \PYG{n+nf+fm}{\PYGZus{}\PYGZus{}init\PYGZus{}\PYGZus{}}\PYG{p}{(}\PYG{n+nb+bp}{self}\PYG{p}{,} \PYG{n}{nombre}\PYG{p}{,} \PYG{n}{apellido}\PYG{p}{)}\PYG{p}{:}
        \PYG{n+nb+bp}{self}\PYG{o}{.}\PYG{n}{nombre} \PYG{o}{=} \PYG{n}{nombre}
        \PYG{n+nb+bp}{self}\PYG{o}{.}\PYG{n}{apellido} \PYG{o}{=} \PYG{n}{apellido}
\end{sphinxVerbatim}
\sphinxresetverbatimhllines

\sphinxAtStartPar
indican que todos los objetos de tipo \sphinxcode{\sphinxupquote{Persona}} tendrán disponibles las propiedades
\sphinxcode{\sphinxupquote{nombre}} y \sphinxcode{\sphinxupquote{apellido}}. Además el valor de ellas será inicializado con los parámetros
que usuario pase al construir estos objetos.
Al no tener valores predeterminados (funciona igual que las funciones) ambas propiedades
son \sphinxstylestrong{obligatorias}. Si intentamos crear una \sphinxcode{\sphinxupquote{Persona}} con algo como:
\sphinxcode{\sphinxupquote{juan = Persona(nombre=\textquotesingle{}juan carlos\textquotesingle{})}} obtendremos un error, el \sphinxcode{\sphinxupquote{apellido}} es
obligatorio.

\sphinxAtStartPar
Asi como estas propiedades están definidas además de poder ser leidas como hemos visto,
tambien pueden modificarse libremente.

\begin{sphinxVerbatim}[commandchars=\\\{\}]
\PYG{n}{diego} \PYG{o}{=} \PYG{n}{Persona}\PYG{p}{(}\PYG{l+s+s1}{\PYGZsq{}}\PYG{l+s+s1}{Diego}\PYG{l+s+s1}{\PYGZsq{}}\PYG{p}{,} \PYG{l+s+s1}{\PYGZsq{}}\PYG{l+s+s1}{Algun apellido}\PYG{l+s+s1}{\PYGZsq{}}\PYG{p}{)}
\PYG{n}{diego}\PYG{o}{.}\PYG{n}{apellido} \PYG{o}{=} \PYG{l+s+s1}{\PYGZsq{}}\PYG{l+s+s1}{Solis}\PYG{l+s+s1}{\PYGZsq{}}
\PYG{n+nb}{print}\PYG{p}{(}\PYG{n}{diego}\PYG{o}{.}\PYG{n}{apellido}\PYG{p}{)}
\PYG{l+s+s1}{\PYGZsq{}}\PYG{l+s+s1}{Solis}\PYG{l+s+s1}{\PYGZsq{}}
\end{sphinxVerbatim}

\sphinxAtStartPar
Esta forma de definir y usar las propiedades en nuestros objetos es poco segura.
En nuestra clase, el siguiente comportamiento si esta permitido:

\fvset{hllines={, 2,}}%
\begin{sphinxVerbatim}[commandchars=\\\{\}]
\PYG{n}{diego} \PYG{o}{=} \PYG{n}{Persona}\PYG{p}{(}\PYG{l+s+s1}{\PYGZsq{}}\PYG{l+s+s1}{Diego}\PYG{l+s+s1}{\PYGZsq{}}\PYG{p}{,} \PYG{l+s+s1}{\PYGZsq{}}\PYG{l+s+s1}{Algun apellido}\PYG{l+s+s1}{\PYGZsq{}}\PYG{p}{)}
\PYG{n}{diego}\PYG{o}{.}\PYG{n}{apellido} \PYG{o}{=} \PYG{l+m+mi}{17}  \PYG{c+c1}{\PYGZsh{} sería bueno evitar esto}
\PYG{n+nb}{print}\PYG{p}{(}\PYG{n}{diego}\PYG{o}{.}\PYG{n}{apellido}\PYG{p}{)}
\PYG{l+m+mi}{17}
\end{sphinxVerbatim}
\sphinxresetverbatimhllines


\subsection{Tareas}
\label{\detokenize{class:tareas}}\begin{itemize}
\item {} 
\sphinxAtStartPar
Crear una clase llamada \sphinxcode{\sphinxupquote{Auto}} que se inicialice con algunas propiedades obligatorias y otras opcionales

\end{itemize}

\sphinxstepscope


\chapter{Propiedades controladas}
\label{\detokenize{class-props:propiedades-controladas}}\label{\detokenize{class-props::doc}}
\sphinxAtStartPar
Es posible tomar los valores que el usuario nos pasa al inicializar los objetos
(en \sphinxcode{\sphinxupquote{\_\_init\_\_}}) y guardarlos en variables privadas. Usamos el guión bajo inicial
en la propiedades para indicar que estas son internas de la clase y que no deberián
ser usadas. Esto es una convención pero no es obligatorio. Python no lanzará un error
si los usuarios de nuestra clase usan estas propiedades privadas.

\fvset{hllines={, 3, 4, 8, 12,}}%
\begin{sphinxVerbatim}[commandchars=\\\{\},numbers=left,firstnumber=1,stepnumber=1]
\PYG{k}{class} \PYG{n+nc}{Persona}\PYG{p}{:}
    \PYG{k}{def} \PYG{n+nf+fm}{\PYGZus{}\PYGZus{}init\PYGZus{}\PYGZus{}}\PYG{p}{(}\PYG{n+nb+bp}{self}\PYG{p}{,} \PYG{n}{nombre}\PYG{p}{,} \PYG{n}{apellido}\PYG{p}{)}\PYG{p}{:}
        \PYG{n+nb+bp}{self}\PYG{o}{.}\PYG{n}{\PYGZus{}nombre} \PYG{o}{=} \PYG{n}{nombre}
        \PYG{n+nb+bp}{self}\PYG{o}{.}\PYG{n}{\PYGZus{}apellido} \PYG{o}{=} \PYG{n}{apellido}

    \PYG{n+nd}{@property}
    \PYG{k}{def} \PYG{n+nf}{nombre}\PYG{p}{(}\PYG{n+nb+bp}{self}\PYG{p}{)}\PYG{p}{:}
        \PYG{k}{return} \PYG{n+nb+bp}{self}\PYG{o}{.}\PYG{n}{\PYGZus{}nombre}

    \PYG{n+nd}{@property}
    \PYG{k}{def} \PYG{n+nf}{apellido}\PYG{p}{(}\PYG{n+nb+bp}{self}\PYG{p}{)}\PYG{p}{:}
        \PYG{k}{return} \PYG{n+nb+bp}{self}\PYG{o}{.}\PYG{n}{\PYGZus{}apellido}
\end{sphinxVerbatim}
\sphinxresetverbatimhllines


\section{¿Que es \sphinxstyleliteralintitle{\sphinxupquote{@property}}?}
\label{\detokenize{class-props:que-es-property}}
\sphinxAtStartPar
Antes de la definición de una función (esto puede hacerse dentro y fuera de las clases)
es posible agregar lo que llamamos un decorador (o función decoradora). La sintaxis para
hacerlo es simplemente agregar esta línea sobre la definición de una función y comenzarla
con un \sphinxcode{\sphinxupquote{@}}. Estos decoradores se usan para modificar a la función de
formas que por el momento exceden lo que necesitamos conocer.
En particular, \sphinxcode{\sphinxupquote{@property}} es usado por las clases en Python para marcar que una
propiedad \sphinxstylestrong{existe} y que la funcion \sphinxstyleemphasis{decorada}
sera la encargada de atender las llamadas de \sphinxstylestrong{lectura} de esta propiedad.
La escritura/modificación de cada propiedad se hace de otra forma.

\sphinxAtStartPar
Hasta aquí estas propiedades son \sphinxstyleemphasis{solo lectura}

\fvset{hllines={, 7,}}%
\begin{sphinxVerbatim}[commandchars=\\\{\},numbers=left,firstnumber=1,stepnumber=1]
\PYG{n}{victor} \PYG{o}{=} \PYG{n}{Persona}\PYG{p}{(}\PYG{l+s+s1}{\PYGZsq{}}\PYG{l+s+s1}{Victor}\PYG{l+s+s1}{\PYGZsq{}}\PYG{p}{,} \PYG{l+s+s1}{\PYGZsq{}}\PYG{l+s+s1}{Fernandez}\PYG{l+s+s1}{\PYGZsq{}}\PYG{p}{)}
\PYG{n+nb}{print}\PYG{p}{(}\PYG{n}{victor}\PYG{o}{.}\PYG{n}{apellido}\PYG{p}{)}
\PYG{c+c1}{\PYGZsh{} funciona y devule}
\PYG{l+s+s1}{\PYGZsq{}}\PYG{l+s+s1}{Fernandez}\PYG{l+s+s1}{\PYGZsq{}}
\PYG{c+c1}{\PYGZsh{} La siguiente línea fallará porque no esta todavía definido como funcionará}
\PYG{c+c1}{\PYGZsh{} la asignación de esta propiedad}
\PYG{n}{victor}\PYG{o}{.}\PYG{n}{apellido} \PYG{o}{=} \PYG{l+s+s1}{\PYGZsq{}}\PYG{l+s+s1}{Gonzalez}\PYG{l+s+s1}{\PYGZsq{}}
\end{sphinxVerbatim}
\sphinxresetverbatimhllines

\sphinxAtStartPar
A las funciones de una clase para \sphinxstylestrong{leer} una propiedad se les llama \sphinxcode{\sphinxupquote{getter}}
y las las funciones para escribir un nuevo valor a una propiedad se las llama
\sphinxcode{\sphinxupquote{setter}} (por \sphinxstyleemphasis{get} y \sphinxstyleemphasis{set} del ingles: \sphinxstyleemphasis{obtener} y \sphinxstyleemphasis{definir}).

\sphinxAtStartPar
Formalmente ahora podemos decir que nuestras propiedades \sphinxcode{\sphinxupquote{nombre}} y \sphinxcode{\sphinxupquote{apellido}}
tienen \sphinxcode{\sphinxupquote{getter}} pero no \sphinxcode{\sphinxupquote{setter}}.

\sphinxAtStartPar
Veamos un ejemplo de \sphinxcode{\sphinxupquote{setter}} para la propiedad \sphinxcode{\sphinxupquote{nombre}} de la clase \sphinxcode{\sphinxupquote{Persona}}:

\fvset{hllines={, 3, 4, 8,}}%
\begin{sphinxVerbatim}[commandchars=\\\{\},numbers=left,firstnumber=1,stepnumber=1]
\PYG{n+nd}{@nombre}\PYG{o}{.}\PYG{n}{setter}
\PYG{k}{def} \PYG{n+nf}{nombre}\PYG{p}{(}\PYG{n+nb+bp}{self}\PYG{p}{,} \PYG{n}{value}\PYG{p}{)}\PYG{p}{:}
    \PYG{c+c1}{\PYGZsh{} Antes de escribir mi variable privada \PYGZus{}nombre, revisar que}
    \PYG{c+c1}{\PYGZsh{} cuampla con los requisitos definidos}
    \PYG{k}{if} \PYG{n+nb}{type}\PYG{p}{(}\PYG{n}{value}\PYG{p}{)} \PYG{o}{!=} \PYG{n+nb}{str}\PYG{p}{:}
        \PYG{c+c1}{\PYGZsh{} Si no es del tipo *string* lanzaremos (raise) un error}
        \PYG{c+c1}{\PYGZsh{} (excepción) del Tipo Exception (hay otros tipos).}
        \PYG{k}{raise} \PYG{n+ne}{Exception}\PYG{p}{(}\PYG{l+s+s1}{\PYGZsq{}}\PYG{l+s+s1}{Nombre inválido. Solo string permitido}\PYG{l+s+s1}{\PYGZsq{}}\PYG{p}{)}

    \PYG{c+c1}{\PYGZsh{} solo si pasa las validaciones (podrían ser varias)}
    \PYG{c+c1}{\PYGZsh{} sobreescribimos nuestra variable privada con el nuevo valor.}
    \PYG{n+nb+bp}{self}\PYG{o}{.}\PYG{n}{\PYGZus{}nombre} \PYG{o}{=} \PYG{n}{value}
\end{sphinxVerbatim}
\sphinxresetverbatimhllines

\sphinxAtStartPar
La función definida para ser \sphinxcode{\sphinxupquote{setter}} debe cumplir las siguientes condiciones:
\begin{itemize}
\item {} 
\sphinxAtStartPar
Tener un decorador de la forma \sphinxcode{\sphinxupquote{@NOMBRE\_DE\_LA\_PROPIEDAD.setter}}.

\item {} 
\sphinxAtStartPar
Tener el mismo nombre que la función \sphinxcode{\sphinxupquote{getter}}.

\item {} 
\sphinxAtStartPar
Incluir un parámetro para recibir el valor que el usuario quiere definir
(usualmente lo llamaremos \sphinxcode{\sphinxupquote{value}}).

\end{itemize}

\sphinxAtStartPar
Es posible tambien definir propiedades personalizadas a gusto.

\fvset{hllines={, 2, 7,}}%
\begin{sphinxVerbatim}[commandchars=\\\{\},numbers=left,firstnumber=1,stepnumber=1]
\PYG{n+nd}{@property}
\PYG{k}{def} \PYG{n+nf}{nombre\PYGZus{}completo}\PYG{p}{(}\PYG{n+nb+bp}{self}\PYG{p}{)}\PYG{p}{:}
    \PYG{l+s+sd}{\PYGZdq{}\PYGZdq{}\PYGZdq{} devuelve el nombre completo \PYGZdq{}\PYGZdq{}\PYGZdq{}}
    \PYG{k}{return} \PYG{l+s+sa}{f}\PYG{l+s+s1}{\PYGZsq{}}\PYG{l+s+si}{\PYGZob{}}\PYG{n+nb+bp}{self}\PYG{o}{.}\PYG{n}{\PYGZus{}nombre}\PYG{l+s+si}{\PYGZcb{}}\PYG{l+s+s1}{ }\PYG{l+s+si}{\PYGZob{}}\PYG{n+nb+bp}{self}\PYG{o}{.}\PYG{n}{\PYGZus{}apellido}\PYG{l+s+si}{\PYGZcb{}}\PYG{l+s+s1}{\PYGZsq{}}

\PYG{n+nd}{@property}
\PYG{k}{def} \PYG{n+nf}{nombre\PYGZus{}formal}\PYG{p}{(}\PYG{n+nb+bp}{self}\PYG{p}{)}\PYG{p}{:}
    \PYG{l+s+sd}{\PYGZdq{}\PYGZdq{}\PYGZdq{} devuelve el nombre completo \PYGZdq{}\PYGZdq{}\PYGZdq{}}
    \PYG{k}{return} \PYG{l+s+sa}{f}\PYG{l+s+s1}{\PYGZsq{}}\PYG{l+s+si}{\PYGZob{}}\PYG{n+nb+bp}{self}\PYG{o}{.}\PYG{n}{\PYGZus{}apellido}\PYG{l+s+si}{\PYGZcb{}}\PYG{l+s+s1}{, }\PYG{l+s+si}{\PYGZob{}}\PYG{n+nb+bp}{self}\PYG{o}{.}\PYG{n}{\PYGZus{}nombre}\PYG{l+s+si}{\PYGZcb{}}\PYG{l+s+s1}{\PYGZsq{}}
\end{sphinxVerbatim}
\sphinxresetverbatimhllines

\sphinxAtStartPar
Estas propiedades solo tienen sentido para ser leidas. Es por esto que no tienen una funcion \sphinxcode{\sphinxupquote{setter}}.

\sphinxAtStartPar
Ejemplos de uso:

\begin{sphinxVerbatim}[commandchars=\\\{\}]
\PYG{n}{juan} \PYG{o}{=} \PYG{n}{Persona}\PYG{p}{(}\PYG{l+s+s1}{\PYGZsq{}}\PYG{l+s+s1}{juan carlos}\PYG{l+s+s1}{\PYGZsq{}}\PYG{p}{,} \PYG{l+s+s1}{\PYGZsq{}}\PYG{l+s+s1}{perez}\PYG{l+s+s1}{\PYGZsq{}}\PYG{p}{)}
\PYG{n+nb}{print}\PYG{p}{(}\PYG{n}{juan}\PYG{o}{.}\PYG{n}{nombre\PYGZus{}completo}\PYG{p}{)}
\PYG{l+s+s1}{\PYGZsq{}}\PYG{l+s+s1}{juan carlos perez}\PYG{l+s+s1}{\PYGZsq{}}
\PYG{n+nb}{print}\PYG{p}{(}\PYG{n}{juan}\PYG{o}{.}\PYG{n}{nombre\PYGZus{}formal}\PYG{p}{)}
\PYG{l+s+s1}{\PYGZsq{}}\PYG{l+s+s1}{perez, juan carlos}\PYG{l+s+s1}{\PYGZsq{}}
\PYG{c+c1}{\PYGZsh{} Si intento asignar una propiedad que es solo lectura (no tienen una funcion setter)}
\PYG{c+c1}{\PYGZsh{} dará un error \PYGZdq{}can\PYGZsq{}t set attribute\PYGZdq{} (no se puede asignar esta propiedad)}
\PYG{n}{juan}\PYG{o}{.}\PYG{n}{nombre\PYGZus{}completo} \PYG{o}{=} \PYG{l+s+s1}{\PYGZsq{}}\PYG{l+s+s1}{Nuevo nombre completo}\PYG{l+s+s1}{\PYGZsq{}}
\end{sphinxVerbatim}

\sphinxAtStartPar
Las propiedades \sphinxcode{\sphinxupquote{nombre}} y \sphinxcode{\sphinxupquote{apellido}} se pueden leer y escribir.
Las propiedades \sphinxcode{\sphinxupquote{nombre\_completo}} y \sphinxcode{\sphinxupquote{nombre\_formal}} son simplemente combinaciones útiles
de otras propiedades básica. Solo se puede leer.


\chapter{Funciones de mi clase}
\label{\detokenize{class-props:funciones-de-mi-clase}}
\sphinxAtStartPar
Es también posible definir funciones

\begin{sphinxVerbatim}[commandchars=\\\{\}]
\PYG{k}{def} \PYG{n+nf}{limpiar}\PYG{p}{(}\PYG{n+nb+bp}{self}\PYG{p}{)}\PYG{p}{:}
    \PYG{l+s+sd}{\PYGZdq{}\PYGZdq{}\PYGZdq{} Mejorar el nombre y el apellido \PYGZdq{}\PYGZdq{}\PYGZdq{}}
    \PYG{n+nb+bp}{self}\PYG{o}{.}\PYG{n}{\PYGZus{}nombre} \PYG{o}{=} \PYG{n+nb+bp}{self}\PYG{o}{.}\PYG{n}{\PYGZus{}nombre}\PYG{o}{.}\PYG{n}{strip}\PYG{p}{(}\PYG{p}{)}\PYG{o}{.}\PYG{n}{title}\PYG{p}{(}\PYG{p}{)}
    \PYG{n+nb+bp}{self}\PYG{o}{.}\PYG{n}{\PYGZus{}apellido} \PYG{o}{=} \PYG{n+nb+bp}{self}\PYG{o}{.}\PYG{n}{\PYGZus{}apellido}\PYG{o}{.}\PYG{n}{strip}\PYG{p}{(}\PYG{p}{)}\PYG{o}{.}\PYG{n}{title}\PYG{p}{(}\PYG{p}{)}

\PYG{k}{def} \PYG{n+nf}{encabezado}\PYG{p}{(}\PYG{n+nb+bp}{self}\PYG{p}{,} \PYG{n}{titulo}\PYG{p}{,} \PYG{n}{limpiar}\PYG{o}{=}\PYG{k+kc}{True}\PYG{p}{)}\PYG{p}{:}
    \PYG{l+s+sd}{\PYGZdq{}\PYGZdq{}\PYGZdq{} Genera y devuelve el nombre completo con}
\PYG{l+s+sd}{        \PYGZdq{}Sr.\PYGZdq{} \PYGZdq{}Sra.\PYGZdq{} o algun otro titulo.}
\PYG{l+s+sd}{        Opcionalmente se puede limpiar el nombre \PYGZdq{}\PYGZdq{}\PYGZdq{}}
    \PYG{c+c1}{\PYGZsh{} limpiar el nombre si se solicita}
    \PYG{k}{if} \PYG{n}{limpiar}\PYG{p}{:}
        \PYG{n+nb+bp}{self}\PYG{o}{.}\PYG{n}{limpiar}\PYG{p}{(}\PYG{p}{)}
    \PYG{k}{return} \PYG{l+s+sa}{f}\PYG{l+s+s1}{\PYGZsq{}}\PYG{l+s+si}{\PYGZob{}}\PYG{n}{titulo}\PYG{l+s+si}{\PYGZcb{}}\PYG{l+s+s1}{ }\PYG{l+s+si}{\PYGZob{}}\PYG{n+nb+bp}{self}\PYG{o}{.}\PYG{n}{nombre\PYGZus{}completo}\PYG{l+s+si}{\PYGZcb{}}\PYG{l+s+s1}{\PYGZsq{}}

\PYG{c+c1}{\PYGZsh{} podemos tambien emular el comportamiento de los strings}
\PYG{c+c1}{\PYGZsh{} e incluso copiar nombres de funciones de ellos}
\PYG{k}{def} \PYG{n+nf}{lower}\PYG{p}{(}\PYG{n+nb+bp}{self}\PYG{p}{)}\PYG{p}{:}
    \PYG{l+s+sd}{\PYGZdq{}\PYGZdq{}\PYGZdq{} devuelve el nombre completo en minusculas \PYGZdq{}\PYGZdq{}\PYGZdq{}}
    \PYG{k}{return} \PYG{n+nb+bp}{self}\PYG{o}{.}\PYG{n}{nombre\PYGZus{}completo}\PYG{o}{.}\PYG{n}{lower}\PYG{p}{(}\PYG{p}{)}

\PYG{k}{def} \PYG{n+nf}{upper}\PYG{p}{(}\PYG{n+nb+bp}{self}\PYG{p}{)}\PYG{p}{:}
    \PYG{l+s+sd}{\PYGZdq{}\PYGZdq{}\PYGZdq{} devuelve el nombre completo en minusculas \PYGZdq{}\PYGZdq{}\PYGZdq{}}
    \PYG{k}{return} \PYG{n+nb+bp}{self}\PYG{o}{.}\PYG{n}{nombre\PYGZus{}completo}\PYG{o}{.}\PYG{n}{upper}\PYG{p}{(}\PYG{p}{)}
\end{sphinxVerbatim}

\sphinxAtStartPar
Algunos ejemplos de uso con estas nuevas funciones:

\begin{sphinxVerbatim}[commandchars=\\\{\}]
\PYG{n}{juan} \PYG{o}{=} \PYG{n}{Persona}\PYG{p}{(}\PYG{l+s+s1}{\PYGZsq{}}\PYG{l+s+s1}{juan carlos}\PYG{l+s+s1}{\PYGZsq{}}\PYG{p}{,} \PYG{l+s+s1}{\PYGZsq{}}\PYG{l+s+s1}{perez}\PYG{l+s+s1}{\PYGZsq{}}\PYG{p}{)}
\PYG{n+nb}{print}\PYG{p}{(}\PYG{n}{juan}\PYG{o}{.}\PYG{n}{nombre\PYGZus{}completo}\PYG{p}{)}
\PYG{c+c1}{\PYGZsh{} \PYGZsq{}juan carlos perez\PYGZsq{}}
\PYG{n+nb}{print}\PYG{p}{(}\PYG{n}{juan}\PYG{o}{.}\PYG{n}{nombre\PYGZus{}formal}\PYG{p}{)}
\PYG{c+c1}{\PYGZsh{} \PYGZsq{}perez, juan carlos\PYGZsq{}}

\PYG{n}{enc} \PYG{o}{=} \PYG{n}{juan}\PYG{o}{.}\PYG{n}{encabezado}\PYG{p}{(}\PYG{l+s+s1}{\PYGZsq{}}\PYG{l+s+s1}{Sr.}\PYG{l+s+s1}{\PYGZsq{}}\PYG{p}{,} \PYG{n}{limpiar}\PYG{o}{=}\PYG{k+kc}{False}\PYG{p}{)}
\PYG{n+nb}{print}\PYG{p}{(}\PYG{n}{enc}\PYG{p}{)}
\PYG{c+c1}{\PYGZsh{} \PYGZsq{}Sr. juan carlos perez\PYGZsq{}}

\PYG{n}{enc} \PYG{o}{=} \PYG{n}{juan}\PYG{o}{.}\PYG{n}{encabezado}\PYG{p}{(}\PYG{l+s+s1}{\PYGZsq{}}\PYG{l+s+s1}{Sr.}\PYG{l+s+s1}{\PYGZsq{}}\PYG{p}{)}
\PYG{n+nb}{print}\PYG{p}{(}\PYG{n}{enc}\PYG{p}{)}
\PYG{c+c1}{\PYGZsh{} \PYGZsq{}Sr. Juan Carlos Perez\PYGZsq{}}

\PYG{n+nb}{print}\PYG{p}{(}\PYG{n}{juan}\PYG{o}{.}\PYG{n}{nombre}\PYG{p}{)}
\PYG{c+c1}{\PYGZsh{} El nombre fue limpiado}
\PYG{c+c1}{\PYGZsh{} \PYGZsq{}Juan Carlos\PYGZsq{}}

\PYG{n+nb}{print}\PYG{p}{(}\PYG{n}{juan}\PYG{o}{.}\PYG{n}{upper}\PYG{p}{(}\PYG{p}{)}\PYG{p}{)}
\PYG{c+c1}{\PYGZsh{} \PYGZsq{}JUAN CARLOS PEREZ\PYGZsq{}}
\end{sphinxVerbatim}

\sphinxAtStartPar
\sphinxstylestrong{Nota importante: Las funciones se llaman con los parentesis (y parámetros si se requieren) y las
propiedades se llaman sin ellos (y no puede requerir parámetros).}

\sphinxAtStartPar
Código de la clase final
\sphinxhref{https://github.com/avdata99/programacion-para-no-programadores/blob/master/code/01-basics/class-00.py}{aquí}%
\begin{footnote}[23]\sphinxAtStartFootnote
\sphinxnolinkurl{https://github.com/avdata99/programacion-para-no-programadores/blob/master/code/01-basics/class-00.py}
%
\end{footnote}.

\begin{sphinxVerbatim}[commandchars=\\\{\}]

\PYG{k}{class} \PYG{n+nc}{Persona}\PYG{p}{:}
    \PYG{k}{def} \PYG{n+nf+fm}{\PYGZus{}\PYGZus{}init\PYGZus{}\PYGZus{}}\PYG{p}{(}\PYG{n+nb+bp}{self}\PYG{p}{,} \PYG{n}{nombre}\PYG{p}{,} \PYG{n}{apellido}\PYG{p}{)}\PYG{p}{:}
        \PYG{n+nb+bp}{self}\PYG{o}{.}\PYG{n}{\PYGZus{}nombre} \PYG{o}{=} \PYG{n}{nombre}
        \PYG{n+nb+bp}{self}\PYG{o}{.}\PYG{n}{\PYGZus{}apellido} \PYG{o}{=} \PYG{n}{apellido}

    \PYG{n+nd}{@property}
    \PYG{k}{def} \PYG{n+nf}{nombre}\PYG{p}{(}\PYG{n+nb+bp}{self}\PYG{p}{)}\PYG{p}{:}
        \PYG{k}{return} \PYG{n+nb+bp}{self}\PYG{o}{.}\PYG{n}{\PYGZus{}nombre}

    \PYG{n+nd}{@nombre}\PYG{o}{.}\PYG{n}{setter}
    \PYG{k}{def} \PYG{n+nf}{nombre}\PYG{p}{(}\PYG{n+nb+bp}{self}\PYG{p}{,} \PYG{n}{value}\PYG{p}{)}\PYG{p}{:}
        \PYG{k}{if} \PYG{n+nb}{type}\PYG{p}{(}\PYG{n}{value}\PYG{p}{)} \PYG{o}{!=} \PYG{n+nb}{str}\PYG{p}{:}
            \PYG{k}{raise} \PYG{n+ne}{Exception}\PYG{p}{(}\PYG{l+s+s1}{\PYGZsq{}}\PYG{l+s+s1}{Nombre inválido. Solo string permitido}\PYG{l+s+s1}{\PYGZsq{}}\PYG{p}{)}
        \PYG{n+nb+bp}{self}\PYG{o}{.}\PYG{n}{\PYGZus{}nombre} \PYG{o}{=} \PYG{n}{value}

    \PYG{n+nd}{@property}
    \PYG{k}{def} \PYG{n+nf}{apellido}\PYG{p}{(}\PYG{n+nb+bp}{self}\PYG{p}{)}\PYG{p}{:}
        \PYG{k}{return} \PYG{n+nb+bp}{self}\PYG{o}{.}\PYG{n}{\PYGZus{}nombre}

    \PYG{n+nd}{@nombre}\PYG{o}{.}\PYG{n}{setter}
    \PYG{k}{def} \PYG{n+nf}{apellido}\PYG{p}{(}\PYG{n+nb+bp}{self}\PYG{p}{,} \PYG{n}{value}\PYG{p}{)}\PYG{p}{:}
        \PYG{k}{if} \PYG{n+nb}{type}\PYG{p}{(}\PYG{n}{value}\PYG{p}{)} \PYG{o}{!=} \PYG{n+nb}{str}\PYG{p}{:}
            \PYG{k}{raise} \PYG{n+ne}{Exception}\PYG{p}{(}\PYG{l+s+s1}{\PYGZsq{}}\PYG{l+s+s1}{Apellido inválido. Solo string permitido}\PYG{l+s+s1}{\PYGZsq{}}\PYG{p}{)}
        \PYG{n+nb+bp}{self}\PYG{o}{.}\PYG{n}{\PYGZus{}nombre} \PYG{o}{=} \PYG{n}{value}

    \PYG{n+nd}{@property}
    \PYG{k}{def} \PYG{n+nf}{nombre\PYGZus{}completo}\PYG{p}{(}\PYG{n+nb+bp}{self}\PYG{p}{)}\PYG{p}{:}
        \PYG{l+s+sd}{\PYGZdq{}\PYGZdq{}\PYGZdq{} devuelve el nombre completo \PYGZdq{}\PYGZdq{}\PYGZdq{}}
        \PYG{k}{return} \PYG{l+s+sa}{f}\PYG{l+s+s1}{\PYGZsq{}}\PYG{l+s+si}{\PYGZob{}}\PYG{n+nb+bp}{self}\PYG{o}{.}\PYG{n}{\PYGZus{}nombre}\PYG{l+s+si}{\PYGZcb{}}\PYG{l+s+s1}{ }\PYG{l+s+si}{\PYGZob{}}\PYG{n+nb+bp}{self}\PYG{o}{.}\PYG{n}{\PYGZus{}apellido}\PYG{l+s+si}{\PYGZcb{}}\PYG{l+s+s1}{\PYGZsq{}}

    \PYG{n+nd}{@property}
    \PYG{k}{def} \PYG{n+nf}{nombre\PYGZus{}formal}\PYG{p}{(}\PYG{n+nb+bp}{self}\PYG{p}{)}\PYG{p}{:}
        \PYG{l+s+sd}{\PYGZdq{}\PYGZdq{}\PYGZdq{} devuelve el nombre completo en modo formal \PYGZdq{}\PYGZdq{}\PYGZdq{}}
        \PYG{k}{return} \PYG{l+s+sa}{f}\PYG{l+s+s1}{\PYGZsq{}}\PYG{l+s+si}{\PYGZob{}}\PYG{n+nb+bp}{self}\PYG{o}{.}\PYG{n}{\PYGZus{}apellido}\PYG{l+s+si}{\PYGZcb{}}\PYG{l+s+s1}{, }\PYG{l+s+si}{\PYGZob{}}\PYG{n+nb+bp}{self}\PYG{o}{.}\PYG{n}{\PYGZus{}nombre}\PYG{l+s+si}{\PYGZcb{}}\PYG{l+s+s1}{\PYGZsq{}}

    \PYG{k}{def} \PYG{n+nf}{limpiar}\PYG{p}{(}\PYG{n+nb+bp}{self}\PYG{p}{)}\PYG{p}{:}
        \PYG{l+s+sd}{\PYGZdq{}\PYGZdq{}\PYGZdq{} Mejorar el nombre y el apellido \PYGZdq{}\PYGZdq{}\PYGZdq{}}
        \PYG{n+nb+bp}{self}\PYG{o}{.}\PYG{n}{\PYGZus{}nombre} \PYG{o}{=} \PYG{n+nb+bp}{self}\PYG{o}{.}\PYG{n}{\PYGZus{}nombre}\PYG{o}{.}\PYG{n}{strip}\PYG{p}{(}\PYG{p}{)}\PYG{o}{.}\PYG{n}{title}\PYG{p}{(}\PYG{p}{)}
        \PYG{n+nb+bp}{self}\PYG{o}{.}\PYG{n}{\PYGZus{}apellido} \PYG{o}{=} \PYG{n+nb+bp}{self}\PYG{o}{.}\PYG{n}{\PYGZus{}apellido}\PYG{o}{.}\PYG{n}{strip}\PYG{p}{(}\PYG{p}{)}\PYG{o}{.}\PYG{n}{title}\PYG{p}{(}\PYG{p}{)}

    \PYG{k}{def} \PYG{n+nf}{encabezado}\PYG{p}{(}\PYG{n+nb+bp}{self}\PYG{p}{,} \PYG{n}{titulo}\PYG{p}{,} \PYG{n}{limpiar}\PYG{o}{=}\PYG{k+kc}{True}\PYG{p}{)}\PYG{p}{:}
        \PYG{l+s+sd}{\PYGZdq{}\PYGZdq{}\PYGZdq{} Genera y devuelve el nombre completo con}
\PYG{l+s+sd}{            \PYGZdq{}Sr.\PYGZdq{} \PYGZdq{}Sra.\PYGZdq{} o algun otro titulo.}
\PYG{l+s+sd}{            Opcionalmente se puede limpiar el nombre \PYGZdq{}\PYGZdq{}\PYGZdq{}}
        \PYG{c+c1}{\PYGZsh{} limpiar el nombre si se solicita}
        \PYG{k}{if} \PYG{n}{limpiar}\PYG{p}{:}
            \PYG{n+nb+bp}{self}\PYG{o}{.}\PYG{n}{limpiar}\PYG{p}{(}\PYG{p}{)}
        \PYG{k}{return} \PYG{l+s+sa}{f}\PYG{l+s+s1}{\PYGZsq{}}\PYG{l+s+si}{\PYGZob{}}\PYG{n}{titulo}\PYG{l+s+si}{\PYGZcb{}}\PYG{l+s+s1}{ }\PYG{l+s+si}{\PYGZob{}}\PYG{n+nb+bp}{self}\PYG{o}{.}\PYG{n}{nombre\PYGZus{}completo}\PYG{l+s+si}{\PYGZcb{}}\PYG{l+s+s1}{\PYGZsq{}}

    \PYG{c+c1}{\PYGZsh{} podemos tambien emular el comportamiento de los strings}
    \PYG{c+c1}{\PYGZsh{} e incluso copiar nombres de funciones de ellos}
    \PYG{k}{def} \PYG{n+nf}{lower}\PYG{p}{(}\PYG{n+nb+bp}{self}\PYG{p}{)}\PYG{p}{:}
        \PYG{l+s+sd}{\PYGZdq{}\PYGZdq{}\PYGZdq{} devuelve el nombre completo en minusculas \PYGZdq{}\PYGZdq{}\PYGZdq{}}
        \PYG{k}{return} \PYG{n+nb+bp}{self}\PYG{o}{.}\PYG{n}{nombre\PYGZus{}completo}\PYG{o}{.}\PYG{n}{lower}\PYG{p}{(}\PYG{p}{)}

    \PYG{k}{def} \PYG{n+nf}{upper}\PYG{p}{(}\PYG{n+nb+bp}{self}\PYG{p}{)}\PYG{p}{:}
        \PYG{l+s+sd}{\PYGZdq{}\PYGZdq{}\PYGZdq{} devuelve el nombre completo en minusculas \PYGZdq{}\PYGZdq{}\PYGZdq{}}
        \PYG{k}{return} \PYG{n+nb+bp}{self}\PYG{o}{.}\PYG{n}{nombre\PYGZus{}completo}\PYG{o}{.}\PYG{n}{upper}\PYG{p}{(}\PYG{p}{)}
\end{sphinxVerbatim}


\section{Tareas}
\label{\detokenize{class-props:tareas}}\begin{itemize}
\item {} 
\sphinxAtStartPar
Hacer un PR a la
\sphinxhref{https://github.com/avdata99/programacion-para-no-programadores/blob/master/code/01-basics/class-00.py}{clase Persona}%
\begin{footnote}[24]\sphinxAtStartFootnote
\sphinxnolinkurl{https://github.com/avdata99/programacion-para-no-programadores/blob/master/code/01-basics/class-00.py}
%
\end{footnote}
para agregar la propiedad edad.

\item {} 
\sphinxAtStartPar
Hacer un PR a la
\sphinxhref{https://github.com/avdata99/programacion-para-no-programadores/blob/master/code/01-basics/class-04.py}{clase Carta}%
\begin{footnote}[25]\sphinxAtStartFootnote
\sphinxnolinkurl{https://github.com/avdata99/programacion-para-no-programadores/blob/master/code/01-basics/class-04.py}
%
\end{footnote}
para validar que el \sphinxcode{\sphinxupquote{palo}} es string en su función \sphinxcode{\sphinxupquote{setter}}.

\item {} 
\sphinxAtStartPar
Hacer un PR a la
\sphinxhref{https://github.com/avdata99/programacion-para-no-programadores/blob/master/code/01-basics/class-04.py}{clase Carta}%
\begin{footnote}[26]\sphinxAtStartFootnote
\sphinxnolinkurl{https://github.com/avdata99/programacion-para-no-programadores/blob/master/code/01-basics/class-04.py}
%
\end{footnote}
para validar que el \sphinxcode{\sphinxupquote{numero}} es mayor que cero y menor o igual que 12 en su función \sphinxcode{\sphinxupquote{setter}}.

\end{itemize}


\section{Algunos ejemplos de uso}
\label{\detokenize{class-props:algunos-ejemplos-de-uso}}
\begin{sphinxVerbatim}[commandchars=\\\{\}]
\PYG{l+s+sd}{\PYGZdq{}\PYGZdq{}\PYGZdq{}}
\PYG{l+s+sd}{Clase Carta para juegos de cartas}
\PYG{l+s+sd}{\PYGZdq{}\PYGZdq{}\PYGZdq{}}

\PYG{k}{class} \PYG{n+nc}{Carta}\PYG{p}{:}
    \PYG{k}{def} \PYG{n+nf+fm}{\PYGZus{}\PYGZus{}init\PYGZus{}\PYGZus{}}\PYG{p}{(}\PYG{n+nb+bp}{self}\PYG{p}{,} \PYG{n}{numero}\PYG{p}{,} \PYG{n}{palo}\PYG{p}{)}\PYG{p}{:}
        \PYG{n+nb+bp}{self}\PYG{o}{.}\PYG{n}{\PYGZus{}numero} \PYG{o}{=} \PYG{n}{numero}
        \PYG{n+nb+bp}{self}\PYG{o}{.}\PYG{n}{\PYGZus{}palo} \PYG{o}{=} \PYG{n}{palo}

    \PYG{n+nd}{@property}
    \PYG{k}{def} \PYG{n+nf}{numero}\PYG{p}{(}\PYG{n+nb+bp}{self}\PYG{p}{)}\PYG{p}{:}
        \PYG{k}{return} \PYG{n+nb+bp}{self}\PYG{o}{.}\PYG{n}{\PYGZus{}numero}

    \PYG{n+nd}{@numero}\PYG{o}{.}\PYG{n}{setter}
    \PYG{k}{def} \PYG{n+nf}{numero}\PYG{p}{(}\PYG{n+nb+bp}{self}\PYG{p}{,} \PYG{n}{value}\PYG{p}{)}\PYG{p}{:}
        \PYG{k}{if} \PYG{n+nb}{type}\PYG{p}{(}\PYG{n}{value}\PYG{p}{)} \PYG{o}{!=} \PYG{n+nb}{int}\PYG{p}{:}
            \PYG{k}{raise} \PYG{n+ne}{Exception}\PYG{p}{(}\PYG{l+s+s1}{\PYGZsq{}}\PYG{l+s+s1}{Solo están permitidos numeros}\PYG{l+s+s1}{\PYGZsq{}}\PYG{p}{)}
        \PYG{n+nb+bp}{self}\PYG{o}{.}\PYG{n}{\PYGZus{}numero} \PYG{o}{=} \PYG{n}{value}

    \PYG{n+nd}{@property}
    \PYG{k}{def} \PYG{n+nf}{palo}\PYG{p}{(}\PYG{n+nb+bp}{self}\PYG{p}{)}\PYG{p}{:}
        \PYG{k}{return} \PYG{n+nb+bp}{self}\PYG{o}{.}\PYG{n}{\PYGZus{}palo}

    \PYG{n+nd}{@palo}\PYG{o}{.}\PYG{n}{setter}
    \PYG{k}{def} \PYG{n+nf}{palo}\PYG{p}{(}\PYG{n+nb+bp}{self}\PYG{p}{,} \PYG{n}{value}\PYG{p}{)}\PYG{p}{:}
        \PYG{n+nb+bp}{self}\PYG{o}{.}\PYG{n}{\PYGZus{}palo} \PYG{o}{=} \PYG{n}{value}

    \PYG{k}{def} \PYG{n+nf+fm}{\PYGZus{}\PYGZus{}str\PYGZus{}\PYGZus{}}\PYG{p}{(}\PYG{n+nb+bp}{self}\PYG{p}{)}\PYG{p}{:}
        \PYG{k}{return} \PYG{l+s+sa}{f}\PYG{l+s+s1}{\PYGZsq{}}\PYG{l+s+si}{\PYGZob{}}\PYG{n+nb+bp}{self}\PYG{o}{.}\PYG{n}{numero}\PYG{l+s+si}{\PYGZcb{}}\PYG{l+s+s1}{ de }\PYG{l+s+si}{\PYGZob{}}\PYG{n+nb+bp}{self}\PYG{o}{.}\PYG{n}{palo}\PYG{l+s+si}{\PYGZcb{}}\PYG{l+s+s1}{\PYGZsq{}}

\PYG{c+c1}{\PYGZsh{} Pruebas}

\PYG{n}{carta1} \PYG{o}{=} \PYG{n}{Carta}\PYG{p}{(}\PYG{l+m+mi}{3}\PYG{p}{,} \PYG{l+s+s1}{\PYGZsq{}}\PYG{l+s+s1}{espada}\PYG{l+s+s1}{\PYGZsq{}}\PYG{p}{)}
\PYG{n+nb}{print}\PYG{p}{(}\PYG{n+nb}{str}\PYG{p}{(}\PYG{n}{carta1}\PYG{p}{)}\PYG{p}{)}
\PYG{l+s+s1}{\PYGZsq{}}\PYG{l+s+s1}{3 de espada}\PYG{l+s+s1}{\PYGZsq{}}
\end{sphinxVerbatim}

\sphinxstepscope


\chapter{Funciones especiales de las clases}
\label{\detokenize{class-extras:funciones-especiales-de-las-clases}}\label{\detokenize{class-extras::doc}}
\sphinxAtStartPar
Tambien conocidos como \sphinxstyleemphasis{métodos mágicos} (podemos pensar a la palabra método como
sinónimo de \sphinxstyleemphasis{función}) estas funciones se aplican a situaciones habituales de otros
objetos de Python. Estas situación son la suma, resta, division, comparación, etc.


\section{\_\_add\_\_}
\label{\detokenize{class-extras:add}}
\sphinxAtStartPar
Que pasara si quisiéramos sumar dos \sphinxcode{\sphinxupquote{Personas}} tal como las vimos en la clase anterior:

\begin{sphinxVerbatim}[commandchars=\\\{\}]
\PYG{n}{juan} \PYG{o}{=} \PYG{n}{Persona}\PYG{p}{(}\PYG{l+s+s1}{\PYGZsq{}}\PYG{l+s+s1}{Juan}\PYG{l+s+s1}{\PYGZsq{}}\PYG{p}{,} \PYG{l+s+s1}{\PYGZsq{}}\PYG{l+s+s1}{Perez}\PYG{l+s+s1}{\PYGZsq{}}\PYG{p}{)}
\PYG{n}{victor} \PYG{o}{=} \PYG{n}{Persona}\PYG{p}{(}\PYG{l+s+s1}{\PYGZsq{}}\PYG{l+s+s1}{Victor}\PYG{l+s+s1}{\PYGZsq{}}\PYG{p}{,} \PYG{l+s+s1}{\PYGZsq{}}\PYG{l+s+s1}{Gutierrez}\PYG{l+s+s1}{\PYGZsq{}}\PYG{p}{)}

\PYG{n}{a} \PYG{o}{=} \PYG{n}{juan} \PYG{o}{+} \PYG{n}{victor}
\end{sphinxVerbatim}

\sphinxAtStartPar
Este código daría un error porque no esta definida la función especial (o mágica) \sphinxcode{\sphinxupquote{\_\_add\_\_}}.
No esta definida porque esta suma no tendría sentido en estos objetos.
Veamos un ejemplo donde si pudiera ser útil.

\begin{sphinxVerbatim}[commandchars=\\\{\}]
\PYG{k}{class} \PYG{n+nc}{FacturaServicio}\PYG{p}{:}
    \PYG{l+s+sd}{\PYGZdq{}\PYGZdq{}\PYGZdq{} Cada factura para el pago de servicios hogareños \PYGZdq{}\PYGZdq{}\PYGZdq{}}
    \PYG{k}{def} \PYG{n+nf+fm}{\PYGZus{}\PYGZus{}init\PYGZus{}\PYGZus{}}\PYG{p}{(}\PYG{n+nb+bp}{self}\PYG{p}{,} \PYG{n}{monto}\PYG{p}{,} \PYG{n}{servicio}\PYG{p}{)}\PYG{p}{:}
        \PYG{n+nb+bp}{self}\PYG{o}{.}\PYG{n}{monto} \PYG{o}{=} \PYG{n}{monto}
        \PYG{n+nb+bp}{self}\PYG{o}{.}\PYG{n}{servicio} \PYG{o}{=} \PYG{n}{servicio}

    \PYG{k}{def} \PYG{n+nf+fm}{\PYGZus{}\PYGZus{}add\PYGZus{}\PYGZus{}}\PYG{p}{(}\PYG{n+nb+bp}{self}\PYG{p}{,} \PYG{n}{otro}\PYG{p}{)}\PYG{p}{:}
        \PYG{l+s+sd}{\PYGZdq{}\PYGZdq{}\PYGZdq{} Sumar esta factura a otra factura}
\PYG{l+s+sd}{            Notar que el resultado no es otro objeto de este tipo,}
\PYG{l+s+sd}{            es solo un numero.\PYGZdq{}\PYGZdq{}\PYGZdq{}}
        \PYG{k}{if} \PYG{n+nb}{type}\PYG{p}{(}\PYG{n}{otro}\PYG{p}{)} \PYG{o}{!=} \PYG{n}{FacturaServicio}\PYG{p}{:}
            \PYG{k}{raise} \PYG{n+ne}{Exception}\PYG{p}{(}\PYG{l+s+s1}{\PYGZsq{}}\PYG{l+s+s1}{La suma solo está permitida para objetos del mismo tipo}\PYG{l+s+s1}{\PYGZsq{}}\PYG{p}{)}

        \PYG{k}{return} \PYG{n+nb+bp}{self}\PYG{o}{.}\PYG{n}{monto} \PYG{o}{+} \PYG{n}{otro}\PYG{o}{.}\PYG{n}{monto}
\end{sphinxVerbatim}

\sphinxAtStartPar
La funcion especial \sphinxcode{\sphinxupquote{\_\_add\_\_}} debe incluir un parámetro despues de \sphinxcode{\sphinxupquote{self}} en el que
recibiremos cualquiera sea el objeto al que debemos sumarnos.

\sphinxAtStartPar
Veamos este código en acción:

\begin{sphinxVerbatim}[commandchars=\\\{\}]
\PYG{n}{f1} \PYG{o}{=} \PYG{n}{FacturaServicio}\PYG{p}{(}\PYG{l+m+mf}{3500.90}\PYG{p}{,} \PYG{l+s+s1}{\PYGZsq{}}\PYG{l+s+s1}{Internet}\PYG{l+s+s1}{\PYGZsq{}}\PYG{p}{)}
\PYG{n}{f2} \PYG{o}{=} \PYG{n}{FacturaServicio}\PYG{p}{(}\PYG{l+m+mf}{1806.06}\PYG{p}{,} \PYG{l+s+s1}{\PYGZsq{}}\PYG{l+s+s1}{Telefono}\PYG{l+s+s1}{\PYGZsq{}}\PYG{p}{)}

\PYG{n+nb}{print}\PYG{p}{(}\PYG{n}{f1} \PYG{o}{+} \PYG{n}{f2}\PYG{p}{)}
\PYG{l+m+mf}{5306.96}
\end{sphinxVerbatim}

\sphinxAtStartPar
La función \sphinxcode{\sphinxupquote{\_\_add\_\_}} devuelve un numero pero podría haber casos donde se devuelvan otros tipos de
datos. Muchas veces podemos esperar que dos objetos del mismo tipo al sumarse devuelvan un nuevo
objeto de ese tipo pero no es siempre el caso. Esto puede definirse a gusto.
De la misma forma, podríamos sumar nuestro objetos con lo de otra clase. Nosotros lo hemos bloqueado
(lanzando un error) pero podríamos hacerlo si fuera necesario. Incluso podríamos devolver resultados
distintos por cada tipo de objeto al que sumamos nuestro objeto.


\section{\_\_str\_\_}
\label{\detokenize{class-extras:str}}
\sphinxAtStartPar
Es probablemente la función especial más usada. Se usa para definir que texto se va a devolver
cuando el usuario necesite una representación \sphinxstyleemphasis{string} de este objeto.

\begin{sphinxVerbatim}[commandchars=\\\{\}]
\PYG{k}{def} \PYG{n+nf+fm}{\PYGZus{}\PYGZus{}str\PYGZus{}\PYGZus{}}\PYG{p}{(}\PYG{n+nb+bp}{self}\PYG{p}{)}\PYG{p}{:}
    \PYG{k}{return} \PYG{l+s+sa}{f}\PYG{l+s+s1}{\PYGZsq{}}\PYG{l+s+s1}{\PYGZdl{} }\PYG{l+s+si}{\PYGZob{}}\PYG{n+nb+bp}{self}\PYG{o}{.}\PYG{n}{monto}\PYG{l+s+si}{\PYGZcb{}}\PYG{l+s+s1}{ a pagar por el servicio de }\PYG{l+s+si}{\PYGZob{}}\PYG{n+nb+bp}{self}\PYG{o}{.}\PYG{n}{servicio}\PYG{l+s+si}{\PYGZcb{}}\PYG{l+s+s1}{\PYGZsq{}}
\end{sphinxVerbatim}

\sphinxAtStartPar
Ejemplo de uso:

\begin{sphinxVerbatim}[commandchars=\\\{\}]
\PYG{n}{f1} \PYG{o}{=} \PYG{n}{FacturaServicio}\PYG{p}{(}\PYG{l+m+mf}{3500.90}\PYG{p}{,} \PYG{l+s+s1}{\PYGZsq{}}\PYG{l+s+s1}{Internet}\PYG{l+s+s1}{\PYGZsq{}}\PYG{p}{)}
\PYG{n}{f1\PYGZus{}str} \PYG{o}{=} \PYG{n+nb}{str}\PYG{p}{(}\PYG{n}{f1}\PYG{p}{)}
\PYG{n+nb}{print}\PYG{p}{(}\PYG{n}{f1\PYGZus{}str}\PYG{p}{)}

\PYG{c+c1}{\PYGZsh{} o directamente cuando se quiere imprimir nuestro objetio}
\PYG{n}{f1} \PYG{o}{=} \PYG{n}{FacturaServicio}\PYG{p}{(}\PYG{l+m+mf}{3500.90}\PYG{p}{,} \PYG{l+s+s1}{\PYGZsq{}}\PYG{l+s+s1}{Internet}\PYG{l+s+s1}{\PYGZsq{}}\PYG{p}{)}
\PYG{n+nb}{print}\PYG{p}{(}\PYG{n}{f1}\PYG{p}{)}
\end{sphinxVerbatim}


\section{\_\_eq\_\_}
\label{\detokenize{class-extras:eq}}
\sphinxAtStartPar
Si queremos permitir la comparación de objetos de nuestra clase se puede definir esta función.
Esta función será llamada cuando nuestro objeto sea comparado con otro mediante el operador \sphinxcode{\sphinxupquote{==}}.
Al igual que \sphinxcode{\sphinxupquote{\_\_add\_\_}}, podríamos comparar nuestro objetos con lo de otra clase si fuera necesario
(este no es el caso).

\begin{sphinxVerbatim}[commandchars=\\\{\}]
\PYG{k}{def} \PYG{n+nf+fm}{\PYGZus{}\PYGZus{}eq\PYGZus{}\PYGZus{}}\PYG{p}{(}\PYG{n+nb+bp}{self}\PYG{p}{,} \PYG{n}{otro}\PYG{p}{)}\PYG{p}{:}
    \PYG{l+s+sd}{\PYGZdq{}\PYGZdq{}\PYGZdq{} Revisar si son iguales a otra factura \PYGZdq{}\PYGZdq{}\PYGZdq{}}

    \PYG{k}{if} \PYG{n+nb}{type}\PYG{p}{(}\PYG{n}{otro}\PYG{p}{)} \PYG{o}{!=} \PYG{n}{FacturaServicio}\PYG{p}{:}
        \PYG{k}{raise} \PYG{n+ne}{Exception}\PYG{p}{(}\PYG{l+s+s1}{\PYGZsq{}}\PYG{l+s+s1}{La comparacion solo está permitida para objetos del mismo tipo}\PYG{l+s+s1}{\PYGZsq{}}\PYG{p}{)}

    \PYG{n}{montos\PYGZus{}iguales} \PYG{o}{=} \PYG{n+nb+bp}{self}\PYG{o}{.}\PYG{n}{monto} \PYG{o}{==} \PYG{n}{otro}\PYG{o}{.}\PYG{n}{monto}
    \PYG{n}{servicios\PYGZus{}iguales} \PYG{o}{=} \PYG{n+nb+bp}{self}\PYG{o}{.}\PYG{n}{servicio} \PYG{o}{==} \PYG{n}{otro}\PYG{o}{.}\PYG{n}{servicio}

    \PYG{k}{return} \PYG{n}{montos\PYGZus{}iguales} \PYG{o+ow}{and} \PYG{n}{servicios\PYGZus{}iguales}
\end{sphinxVerbatim}

\sphinxAtStartPar
Veamoslo en acción:

\begin{sphinxVerbatim}[commandchars=\\\{\}]
\PYG{n}{f1} \PYG{o}{=} \PYG{n}{FacturaServicio}\PYG{p}{(}\PYG{l+m+mf}{1500.90}\PYG{p}{,} \PYG{l+s+s1}{\PYGZsq{}}\PYG{l+s+s1}{Internet}\PYG{l+s+s1}{\PYGZsq{}}\PYG{p}{)}
\PYG{n}{f2} \PYG{o}{=} \PYG{n}{FacturaServicio}\PYG{p}{(}\PYG{l+m+mf}{1500.90}\PYG{p}{,} \PYG{l+s+s1}{\PYGZsq{}}\PYG{l+s+s1}{Internet}\PYG{l+s+s1}{\PYGZsq{}}\PYG{p}{)}
\PYG{n}{f3} \PYG{o}{=} \PYG{n}{FacturaServicio}\PYG{p}{(}\PYG{l+m+mf}{3500.90}\PYG{p}{,} \PYG{l+s+s1}{\PYGZsq{}}\PYG{l+s+s1}{Internet}\PYG{l+s+s1}{\PYGZsq{}}\PYG{p}{)}

\PYG{k}{if} \PYG{n}{f1} \PYG{o}{==} \PYG{n}{f2}\PYG{p}{:}
    \PYG{n+nb}{print}\PYG{p}{(}\PYG{l+s+s1}{\PYGZsq{}}\PYG{l+s+s1}{f1 y f2 SI son iguales}\PYG{l+s+s1}{\PYGZsq{}}\PYG{p}{)}
\PYG{k}{else}\PYG{p}{:}
    \PYG{n+nb}{print}\PYG{p}{(}\PYG{l+s+s1}{\PYGZsq{}}\PYG{l+s+s1}{f1 y f2 NO son iguales}\PYG{l+s+s1}{\PYGZsq{}}\PYG{p}{)}

\PYG{k}{if} \PYG{n}{f2} \PYG{o}{==} \PYG{n}{f3}\PYG{p}{:}
    \PYG{n+nb}{print}\PYG{p}{(}\PYG{l+s+s1}{\PYGZsq{}}\PYG{l+s+s1}{f2 y f3 SI son iguales}\PYG{l+s+s1}{\PYGZsq{}}\PYG{p}{)}
\PYG{k}{else}\PYG{p}{:}
    \PYG{n+nb}{print}\PYG{p}{(}\PYG{l+s+s1}{\PYGZsq{}}\PYG{l+s+s1}{f2 y f3 NO son iguales}\PYG{l+s+s1}{\PYGZsq{}}\PYG{p}{)}

\PYG{l+s+sd}{\PYGZdq{}\PYGZdq{}\PYGZdq{}}
\PYG{l+s+sd}{f1 y f2 SI son iguales}
\PYG{l+s+sd}{f2 y f3 NO son iguales}
\PYG{l+s+sd}{\PYGZdq{}\PYGZdq{}\PYGZdq{}}
\end{sphinxVerbatim}


\chapter{Código final de nuestra clase}
\label{\detokenize{class-extras:codigo-final-de-nuestra-clase}}
\sphinxAtStartPar
Disponible \sphinxhref{https://github.com/avdata99/programacion-para-no-programadores/blob/master/code/01-basics/class-01.py}{aquí}%
\begin{footnote}[27]\sphinxAtStartFootnote
\sphinxnolinkurl{https://github.com/avdata99/programacion-para-no-programadores/blob/master/code/01-basics/class-01.py}
%
\end{footnote}.

\begin{sphinxVerbatim}[commandchars=\\\{\}]
\PYG{k}{class} \PYG{n+nc}{FacturaServicio}\PYG{p}{:}
    \PYG{l+s+sd}{\PYGZdq{}\PYGZdq{}\PYGZdq{} Cada factura para el pago de servicios hogareños \PYGZdq{}\PYGZdq{}\PYGZdq{}}
    \PYG{k}{def} \PYG{n+nf+fm}{\PYGZus{}\PYGZus{}init\PYGZus{}\PYGZus{}}\PYG{p}{(}\PYG{n+nb+bp}{self}\PYG{p}{,} \PYG{n}{monto}\PYG{p}{,} \PYG{n}{servicio}\PYG{p}{)}\PYG{p}{:}
        \PYG{n+nb+bp}{self}\PYG{o}{.}\PYG{n}{monto} \PYG{o}{=} \PYG{n}{monto}
        \PYG{n+nb+bp}{self}\PYG{o}{.}\PYG{n}{servicio} \PYG{o}{=} \PYG{n}{servicio}

    \PYG{k}{def} \PYG{n+nf+fm}{\PYGZus{}\PYGZus{}add\PYGZus{}\PYGZus{}}\PYG{p}{(}\PYG{n+nb+bp}{self}\PYG{p}{,} \PYG{n}{otro}\PYG{p}{)}\PYG{p}{:}
        \PYG{l+s+sd}{\PYGZdq{}\PYGZdq{}\PYGZdq{} Sumar esta factura a otra factura }
\PYG{l+s+sd}{            Notar que el resultado no es otro objeto de este tipo,}
\PYG{l+s+sd}{            es solo un numero.\PYGZdq{}\PYGZdq{}\PYGZdq{}}
        \PYG{k}{if} \PYG{n+nb}{type}\PYG{p}{(}\PYG{n}{otro}\PYG{p}{)} \PYG{o}{!=} \PYG{n}{FacturaServicio}\PYG{p}{:}
            \PYG{k}{raise} \PYG{n+ne}{Exception}\PYG{p}{(}\PYG{l+s+s1}{\PYGZsq{}}\PYG{l+s+s1}{La suma solo está permitida para objetos del mismo tipo}\PYG{l+s+s1}{\PYGZsq{}}\PYG{p}{)}

        \PYG{k}{return} \PYG{n+nb+bp}{self}\PYG{o}{.}\PYG{n}{monto} \PYG{o}{+} \PYG{n}{otro}\PYG{o}{.}\PYG{n}{monto}

    \PYG{k}{def} \PYG{n+nf+fm}{\PYGZus{}\PYGZus{}str\PYGZus{}\PYGZus{}}\PYG{p}{(}\PYG{n+nb+bp}{self}\PYG{p}{)}\PYG{p}{:}
        \PYG{k}{return} \PYG{l+s+sa}{f}\PYG{l+s+s1}{\PYGZsq{}}\PYG{l+s+s1}{\PYGZdl{} }\PYG{l+s+si}{\PYGZob{}}\PYG{n+nb+bp}{self}\PYG{o}{.}\PYG{n}{monto}\PYG{l+s+si}{\PYGZcb{}}\PYG{l+s+s1}{ a pagar por el servicio de }\PYG{l+s+si}{\PYGZob{}}\PYG{n+nb+bp}{self}\PYG{o}{.}\PYG{n}{servicio}\PYG{l+s+si}{\PYGZcb{}}\PYG{l+s+s1}{\PYGZsq{}}

    \PYG{k}{def} \PYG{n+nf+fm}{\PYGZus{}\PYGZus{}eq\PYGZus{}\PYGZus{}}\PYG{p}{(}\PYG{n+nb+bp}{self}\PYG{p}{,} \PYG{n}{otro}\PYG{p}{)}\PYG{p}{:}
        \PYG{l+s+sd}{\PYGZdq{}\PYGZdq{}\PYGZdq{} Revisar si son iguales a otra factura \PYGZdq{}\PYGZdq{}\PYGZdq{}}

        \PYG{k}{if} \PYG{n+nb}{type}\PYG{p}{(}\PYG{n}{otro}\PYG{p}{)} \PYG{o}{!=} \PYG{n}{FacturaServicio}\PYG{p}{:}
            \PYG{k}{raise} \PYG{n+ne}{Exception}\PYG{p}{(}\PYG{l+s+s1}{\PYGZsq{}}\PYG{l+s+s1}{La comparacion solo está permitida para objetos del mismo tipo}\PYG{l+s+s1}{\PYGZsq{}}\PYG{p}{)}
        
        \PYG{n}{montos\PYGZus{}iguales} \PYG{o}{=} \PYG{n+nb+bp}{self}\PYG{o}{.}\PYG{n}{monto} \PYG{o}{==} \PYG{n}{otro}\PYG{o}{.}\PYG{n}{monto}
        \PYG{n}{servicios\PYGZus{}iguales} \PYG{o}{=} \PYG{n+nb+bp}{self}\PYG{o}{.}\PYG{n}{servicio} \PYG{o}{==} \PYG{n}{otro}\PYG{o}{.}\PYG{n}{servicio}

        \PYG{k}{return} \PYG{n}{montos\PYGZus{}iguales} \PYG{o+ow}{and} \PYG{n}{servicios\PYGZus{}iguales}
\end{sphinxVerbatim}


\chapter{Otras funciones especiales}
\label{\detokenize{class-extras:otras-funciones-especiales}}
\sphinxAtStartPar
Estas algunas otras de las funciones especiales.
\begin{itemize}
\item {} 
\sphinxAtStartPar
\sphinxcode{\sphinxupquote{\_\_mul\_\_}}: Multiplipicación

\item {} 
\sphinxAtStartPar
\sphinxcode{\sphinxupquote{\_\_sub\_\_}}: Resta (\sphinxstyleemphasis{Substraction})

\item {} 
\sphinxAtStartPar
Para que nuestros objetos se comporten como diccionarios

\end{itemize}
\begin{itemize}
\item {} 
\sphinxAtStartPar
\sphinxcode{\sphinxupquote{\_\_getitem\_\_}}: Obtener un item con la clave que se pasa como parámetro

\item {} 
\sphinxAtStartPar
\sphinxcode{\sphinxupquote{\_\_setitem\_\_}}: Definir un item con la clave y el valor que se pasan como parámetro

\item {} 
\sphinxAtStartPar
\sphinxcode{\sphinxupquote{\_\_delitem\_\_}}: Eliminar el item que tiene la clave que se pasa como parámetro

\end{itemize}
\begin{itemize}
\item {} 
\sphinxAtStartPar
\sphinxcode{\sphinxupquote{\_\_ne\_\_}}: No igual (\sphinxstyleemphasis{Not equal}) \sphinxcode{\sphinxupquote{!=}}

\item {} 
\sphinxAtStartPar
\sphinxcode{\sphinxupquote{\_\_lt\_\_}}: Menor que (\sphinxstyleemphasis{less than}) \sphinxcode{\sphinxupquote{\textless{}}}

\item {} 
\sphinxAtStartPar
\sphinxcode{\sphinxupquote{\_\_gt\_\_}}: Mayor que (\sphinxstyleemphasis{grater than}) \sphinxcode{\sphinxupquote{\textgreater{}}}

\item {} 
\sphinxAtStartPar
\sphinxcode{\sphinxupquote{\_\_neg\_\_}}: Negativo (para cuando usan \sphinxcode{\sphinxupquote{\sphinxhyphen{}MY\sphinxhyphen{}OBJETO}})

\end{itemize}

\sphinxAtStartPar
\sphinxstylestrong{Y hay muchas más.}


\section{Tareas}
\label{\detokenize{class-extras:tareas}}\begin{itemize}
\item {} 
\sphinxAtStartPar
Hacer un PR a la
\sphinxhref{https://github.com/avdata99/programacion-para-no-programadores/blob/master/code/01-basics/class-04.py}{clase Carta}%
\begin{footnote}[28]\sphinxAtStartFootnote
\sphinxnolinkurl{https://github.com/avdata99/programacion-para-no-programadores/blob/master/code/01-basics/class-04.py}
%
\end{footnote}
para agregar la función \sphinxcode{\sphinxupquote{\_\_add\_\_}} para que devuelva un \sphinxcode{\sphinxupquote{int}} calculando el envido \sphinxstylestrong{solo} de esas dos cartas. \sphinxstylestrong{Incluir
multiples asserts al final que pruebe al menos tres sumas (diferentes y variadas) y sus resultados (envidos) esperados}.

\item {} 
\sphinxAtStartPar
Hacer un PR a la
\sphinxhref{https://github.com/avdata99/programacion-para-no-programadores/blob/master/code/01-basics/class-04.py}{clase Carta}%
\begin{footnote}[29]\sphinxAtStartFootnote
\sphinxnolinkurl{https://github.com/avdata99/programacion-para-no-programadores/blob/master/code/01-basics/class-04.py}
%
\end{footnote}
para agregar la función \sphinxcode{\sphinxupquote{\_\_eq\_\_}} para que devuelva \sphinxcode{\sphinxupquote{True}} solo cuando el numero y el palo sean iguales.

\end{itemize}


\section{Algunos ejemplos de uso}
\label{\detokenize{class-extras:algunos-ejemplos-de-uso}}
\begin{sphinxVerbatim}[commandchars=\\\{\}]
\PYG{l+s+sd}{\PYGZdq{}\PYGZdq{}\PYGZdq{}}
\PYG{l+s+sd}{Ejemplo de una clase para manejar fracciones}
\PYG{l+s+sd}{(de numerador y denominador entero y positivo)}
\PYG{l+s+sd}{\PYGZdq{}\PYGZdq{}\PYGZdq{}}


\PYG{k}{class} \PYG{n+nc}{Fraccion}\PYG{p}{:}
    \PYG{l+s+sd}{\PYGZdq{}\PYGZdq{}\PYGZdq{} Clase para manejar fracciones de numeros enteros positivos \PYGZdq{}\PYGZdq{}\PYGZdq{}}

    \PYG{k}{def} \PYG{n+nf+fm}{\PYGZus{}\PYGZus{}init\PYGZus{}\PYGZus{}}\PYG{p}{(}\PYG{n+nb+bp}{self}\PYG{p}{,} \PYG{n}{numerador}\PYG{p}{,} \PYG{n}{denominador}\PYG{p}{)}\PYG{p}{:}
        \PYG{k}{if} \PYG{n+nb}{type}\PYG{p}{(}\PYG{n}{numerador}\PYG{p}{)} \PYG{o}{!=} \PYG{n+nb}{int} \PYG{o+ow}{or} \PYG{n+nb}{type}\PYG{p}{(}\PYG{n}{denominador}\PYG{p}{)} \PYG{o}{!=} \PYG{n+nb}{int}\PYG{p}{:}
            \PYG{k}{raise} \PYG{n+ne}{ValueError}\PYG{p}{(}\PYG{l+s+s1}{\PYGZsq{}}\PYG{l+s+s1}{Solo numeros enteros aceptados}\PYG{l+s+s1}{\PYGZsq{}}\PYG{p}{)}
        \PYG{k}{if} \PYG{n}{numerador} \PYG{o}{\PYGZlt{}} \PYG{l+m+mi}{0} \PYG{o+ow}{or} \PYG{n}{denominador} \PYG{o}{\PYGZlt{}} \PYG{l+m+mi}{1}\PYG{p}{:}
            \PYG{k}{raise} \PYG{n+ne}{ValueError}\PYG{p}{(}\PYG{l+s+s1}{\PYGZsq{}}\PYG{l+s+s1}{Solo numeros positivos aceptados}\PYG{l+s+s1}{\PYGZsq{}}\PYG{p}{)}
    
        \PYG{n+nb+bp}{self}\PYG{o}{.}\PYG{n}{\PYGZus{}numerador} \PYG{o}{=} \PYG{n}{numerador}
        \PYG{n+nb+bp}{self}\PYG{o}{.}\PYG{n}{\PYGZus{}denominador} \PYG{o}{=} \PYG{n}{denominador}
        \PYG{n+nb+bp}{self}\PYG{o}{.}\PYG{n}{\PYGZus{}simplificar}\PYG{p}{(}\PYG{p}{)}

    \PYG{n+nd}{@property}
    \PYG{k}{def} \PYG{n+nf}{numerador}\PYG{p}{(}\PYG{n+nb+bp}{self}\PYG{p}{)}\PYG{p}{:}
        \PYG{k}{return} \PYG{n+nb+bp}{self}\PYG{o}{.}\PYG{n}{\PYGZus{}numerador}
    
    \PYG{n+nd}{@numerador}\PYG{o}{.}\PYG{n}{setter}
    \PYG{k}{def} \PYG{n+nf}{numerador}\PYG{p}{(}\PYG{n+nb+bp}{self}\PYG{p}{,} \PYG{n}{num}\PYG{p}{)}\PYG{p}{:}
        \PYG{k}{if} \PYG{n+nb}{type}\PYG{p}{(}\PYG{n}{num}\PYG{p}{)} \PYG{o}{!=} \PYG{n+nb}{int}\PYG{p}{:}
            \PYG{k}{raise} \PYG{n+ne}{ValueError}\PYG{p}{(}\PYG{l+s+s1}{\PYGZsq{}}\PYG{l+s+s1}{Tipo de dato no admitido para numerador}\PYG{l+s+s1}{\PYGZsq{}}\PYG{p}{)}
        \PYG{k}{if} \PYG{n}{num} \PYG{o}{\PYGZlt{}} \PYG{l+m+mi}{1}\PYG{p}{:}
            \PYG{k}{raise} \PYG{n+ne}{ValueError}\PYG{p}{(}\PYG{l+s+s1}{\PYGZsq{}}\PYG{l+s+s1}{Valor de numerador no admitido}\PYG{l+s+s1}{\PYGZsq{}}\PYG{p}{)}
        \PYG{n+nb+bp}{self}\PYG{o}{.}\PYG{n}{\PYGZus{}numerador} \PYG{o}{=} \PYG{n}{num}
        \PYG{n+nb+bp}{self}\PYG{o}{.}\PYG{n}{\PYGZus{}simplificar}\PYG{p}{(}\PYG{p}{)}

    \PYG{n+nd}{@property}
    \PYG{k}{def} \PYG{n+nf}{denominador}\PYG{p}{(}\PYG{n+nb+bp}{self}\PYG{p}{)}\PYG{p}{:}
        \PYG{k}{return} \PYG{n+nb+bp}{self}\PYG{o}{.}\PYG{n}{\PYGZus{}denominador}
    
    \PYG{n+nd}{@denominador}\PYG{o}{.}\PYG{n}{setter}
    \PYG{k}{def} \PYG{n+nf}{denominador}\PYG{p}{(}\PYG{n+nb+bp}{self}\PYG{p}{,} \PYG{n}{den}\PYG{p}{)}\PYG{p}{:}
        \PYG{k}{if} \PYG{n+nb}{type}\PYG{p}{(}\PYG{n}{den}\PYG{p}{)} \PYG{o}{!=} \PYG{n+nb}{int}\PYG{p}{:}
            \PYG{k}{raise} \PYG{n+ne}{ValueError}\PYG{p}{(}\PYG{l+s+s1}{\PYGZsq{}}\PYG{l+s+s1}{Tipo de dato no admitido para denominador}\PYG{l+s+s1}{\PYGZsq{}}\PYG{p}{)}
        \PYG{k}{if} \PYG{n}{den} \PYG{o}{\PYGZlt{}}\PYG{o}{=} \PYG{l+m+mi}{0}\PYG{p}{:}
            \PYG{k}{raise} \PYG{n+ne}{ValueError}\PYG{p}{(}\PYG{l+s+s1}{\PYGZsq{}}\PYG{l+s+s1}{Valor de denominador no admitido}\PYG{l+s+s1}{\PYGZsq{}}\PYG{p}{)}
        \PYG{n+nb+bp}{self}\PYG{o}{.}\PYG{n}{\PYGZus{}denominador} \PYG{o}{=} \PYG{n}{den}
        \PYG{n+nb+bp}{self}\PYG{o}{.}\PYG{n}{\PYGZus{}simplificar}\PYG{p}{(}\PYG{p}{)}

    \PYG{k}{def} \PYG{n+nf}{\PYGZus{}simplificar}\PYG{p}{(}\PYG{n+nb+bp}{self}\PYG{p}{)}\PYG{p}{:}
        \PYG{l+s+sd}{\PYGZdq{}\PYGZdq{}\PYGZdq{} Simplificar la fraccion a los numeros mas bajos posibles \PYGZdq{}\PYGZdq{}\PYGZdq{}}
        \PYG{n}{men} \PYG{o}{=} \PYG{n+nb}{min}\PYG{p}{(}\PYG{n+nb+bp}{self}\PYG{o}{.}\PYG{n}{\PYGZus{}numerador}\PYG{p}{,} \PYG{n+nb+bp}{self}\PYG{o}{.}\PYG{n}{\PYGZus{}denominador}\PYG{p}{)}
        \PYG{k}{for} \PYG{n}{n} \PYG{o+ow}{in} \PYG{n+nb}{range}\PYG{p}{(}\PYG{n}{men}\PYG{p}{,} \PYG{l+m+mi}{1}\PYG{p}{,} \PYG{o}{\PYGZhy{}}\PYG{l+m+mi}{1}\PYG{p}{)}\PYG{p}{:}
            \PYG{c+c1}{\PYGZsh{} Si este numero divide a los dos, entonces los divido}
            \PYG{k}{if} \PYG{n+nb+bp}{self}\PYG{o}{.}\PYG{n}{\PYGZus{}numerador} \PYG{o}{\PYGZpc{}} \PYG{n}{n} \PYG{o}{==} \PYG{l+m+mi}{0} \PYG{o+ow}{and} \PYG{n+nb+bp}{self}\PYG{o}{.}\PYG{n}{\PYGZus{}denominador} \PYG{o}{\PYGZpc{}} \PYG{n}{n} \PYG{o}{==} \PYG{l+m+mi}{0}\PYG{p}{:}
                \PYG{n+nb+bp}{self}\PYG{o}{.}\PYG{n}{\PYGZus{}numerador} \PYG{o}{=} \PYG{n+nb}{int}\PYG{p}{(}\PYG{n+nb+bp}{self}\PYG{o}{.}\PYG{n}{\PYGZus{}numerador} \PYG{o}{/} \PYG{n}{n}\PYG{p}{)}
                \PYG{n+nb+bp}{self}\PYG{o}{.}\PYG{n}{\PYGZus{}denominador} \PYG{o}{=} \PYG{n+nb}{int}\PYG{p}{(}\PYG{n+nb+bp}{self}\PYG{o}{.}\PYG{n}{\PYGZus{}denominador} \PYG{o}{/} \PYG{n}{n}\PYG{p}{)}
                \PYG{k}{break}

    \PYG{k}{def} \PYG{n+nf+fm}{\PYGZus{}\PYGZus{}add\PYGZus{}\PYGZus{}}\PYG{p}{(}\PYG{n+nb+bp}{self}\PYG{p}{,} \PYG{n}{otro}\PYG{p}{)}\PYG{p}{:}
        \PYG{l+s+sd}{\PYGZdq{}\PYGZdq{}\PYGZdq{} Sumar fracciones\PYGZdq{}\PYGZdq{}\PYGZdq{}}
        \PYG{k}{if} \PYG{n+nb}{type}\PYG{p}{(}\PYG{n}{otro}\PYG{p}{)} \PYG{o}{!=} \PYG{n}{Fraccion}\PYG{p}{:}
            \PYG{k}{raise} \PYG{n+ne}{ValueError}\PYG{p}{(}\PYG{l+s+s1}{\PYGZsq{}}\PYG{l+s+s1}{Solo suma aceptada entre fracciones}\PYG{l+s+s1}{\PYGZsq{}}\PYG{p}{)}
    
        \PYG{n}{nuevo\PYGZus{}denominador} \PYG{o}{=} \PYG{n+nb+bp}{self}\PYG{o}{.}\PYG{n}{\PYGZus{}denominador} \PYG{o}{*} \PYG{n}{otro}\PYG{o}{.}\PYG{n}{denominador}
        \PYG{n}{nuevo\PYGZus{}numerador} \PYG{o}{=} \PYG{n+nb+bp}{self}\PYG{o}{.}\PYG{n}{\PYGZus{}numerador} \PYG{o}{*} \PYG{n}{otro}\PYG{o}{.}\PYG{n}{denominador} \PYG{o}{+} \PYG{n}{otro}\PYG{o}{.}\PYG{n}{numerador} \PYG{o}{*} \PYG{n+nb+bp}{self}\PYG{o}{.}\PYG{n}{\PYGZus{}denominador}

        \PYG{k}{return} \PYG{n}{Fraccion}\PYG{p}{(}\PYG{n}{nuevo\PYGZus{}numerador}\PYG{p}{,} \PYG{n}{nuevo\PYGZus{}denominador}\PYG{p}{)}

    \PYG{k}{def} \PYG{n+nf+fm}{\PYGZus{}\PYGZus{}eq\PYGZus{}\PYGZus{}}\PYG{p}{(}\PYG{n+nb+bp}{self}\PYG{p}{,} \PYG{n}{otro}\PYG{p}{)}\PYG{p}{:}
        \PYG{k}{if} \PYG{n+nb}{type}\PYG{p}{(}\PYG{n}{otro}\PYG{p}{)} \PYG{o}{!=} \PYG{n}{Fraccion}\PYG{p}{:}
            \PYG{k}{raise} \PYG{n+ne}{ValueError}\PYG{p}{(}\PYG{l+s+s1}{\PYGZsq{}}\PYG{l+s+s1}{No son objetos iguales}\PYG{l+s+s1}{\PYGZsq{}}\PYG{p}{)}

        \PYG{k}{return} \PYG{n+nb+bp}{self}\PYG{o}{.}\PYG{n}{\PYGZus{}numerador} \PYG{o}{==} \PYG{n}{otro}\PYG{o}{.}\PYG{n}{numerador} \PYG{o+ow}{and} \PYG{n+nb+bp}{self}\PYG{o}{.}\PYG{n}{\PYGZus{}denominador} \PYG{o}{==} \PYG{n}{otro}\PYG{o}{.}\PYG{n}{denominador}

    \PYG{k}{def} \PYG{n+nf+fm}{\PYGZus{}\PYGZus{}str\PYGZus{}\PYGZus{}}\PYG{p}{(}\PYG{n+nb+bp}{self}\PYG{p}{)}\PYG{p}{:}
        \PYG{k}{return} \PYG{l+s+sa}{f}\PYG{l+s+s1}{\PYGZsq{}}\PYG{l+s+s1}{(}\PYG{l+s+si}{\PYGZob{}}\PYG{n+nb+bp}{self}\PYG{o}{.}\PYG{n}{\PYGZus{}numerador}\PYG{l+s+si}{\PYGZcb{}}\PYG{l+s+s1}{ / }\PYG{l+s+si}{\PYGZob{}}\PYG{n+nb+bp}{self}\PYG{o}{.}\PYG{n}{\PYGZus{}denominador}\PYG{l+s+si}{\PYGZcb{}}\PYG{l+s+s1}{)}\PYG{l+s+s1}{\PYGZsq{}}

    \PYG{k}{def} \PYG{n+nf+fm}{\PYGZus{}\PYGZus{}repr\PYGZus{}\PYGZus{}}\PYG{p}{(}\PYG{n+nb+bp}{self}\PYG{p}{)}\PYG{p}{:}
        \PYG{k}{return} \PYG{l+s+sa}{f}\PYG{l+s+s1}{\PYGZsq{}}\PYG{l+s+s1}{\PYGZlt{}Fraccion }\PYG{l+s+si}{\PYGZob{}}\PYG{n+nb+bp}{self}\PYG{o}{.}\PYG{n}{\PYGZus{}numerador}\PYG{l+s+si}{\PYGZcb{}}\PYG{l+s+s1}{ / }\PYG{l+s+si}{\PYGZob{}}\PYG{n+nb+bp}{self}\PYG{o}{.}\PYG{n}{\PYGZus{}denominador}\PYG{l+s+si}{\PYGZcb{}}\PYG{l+s+s1}{\PYGZgt{}}\PYG{l+s+s1}{\PYGZsq{}}


\PYG{c+c1}{\PYGZsh{} Pruebas de funcionamiento}

\PYG{k}{assert} \PYG{n}{Fraccion}\PYG{p}{(}\PYG{l+m+mi}{2}\PYG{p}{,} \PYG{l+m+mi}{3}\PYG{p}{)} \PYG{o}{+} \PYG{n}{Fraccion}\PYG{p}{(}\PYG{l+m+mi}{1}\PYG{p}{,} \PYG{l+m+mi}{3}\PYG{p}{)} \PYG{o}{==} \PYG{n}{Fraccion}\PYG{p}{(}\PYG{l+m+mi}{1}\PYG{p}{,} \PYG{l+m+mi}{1}\PYG{p}{)}
\PYG{k}{assert} \PYG{n}{Fraccion}\PYG{p}{(}\PYG{l+m+mi}{4}\PYG{p}{,} \PYG{l+m+mi}{5}\PYG{p}{)} \PYG{o}{+} \PYG{n}{Fraccion}\PYG{p}{(}\PYG{l+m+mi}{3}\PYG{p}{,} \PYG{l+m+mi}{5}\PYG{p}{)} \PYG{o}{==} \PYG{n}{Fraccion}\PYG{p}{(}\PYG{l+m+mi}{7}\PYG{p}{,} \PYG{l+m+mi}{5}\PYG{p}{)}
\PYG{k}{assert} \PYG{n}{Fraccion}\PYG{p}{(}\PYG{l+m+mi}{4}\PYG{p}{,} \PYG{l+m+mi}{5}\PYG{p}{)} \PYG{o}{+} \PYG{n}{Fraccion}\PYG{p}{(}\PYG{l+m+mi}{6}\PYG{p}{,} \PYG{l+m+mi}{5}\PYG{p}{)} \PYG{o}{==} \PYG{n}{Fraccion}\PYG{p}{(}\PYG{l+m+mi}{2}\PYG{p}{,} \PYG{l+m+mi}{1}\PYG{p}{)}
\PYG{k}{assert} \PYG{n}{Fraccion}\PYG{p}{(}\PYG{l+m+mi}{5}\PYG{p}{,} \PYG{l+m+mi}{12}\PYG{p}{)} \PYG{o}{+} \PYG{n}{Fraccion}\PYG{p}{(}\PYG{l+m+mi}{4}\PYG{p}{,} \PYG{l+m+mi}{19}\PYG{p}{)} \PYG{o}{==} \PYG{n}{Fraccion}\PYG{p}{(}\PYG{l+m+mi}{143}\PYG{p}{,} \PYG{l+m+mi}{228}\PYG{p}{)}
\PYG{k}{assert} \PYG{n}{Fraccion}\PYG{p}{(}\PYG{l+m+mi}{3}\PYG{p}{,} \PYG{l+m+mi}{2}\PYG{p}{)} \PYG{o}{+} \PYG{n}{Fraccion}\PYG{p}{(}\PYG{l+m+mi}{8}\PYG{p}{,} \PYG{l+m+mi}{11}\PYG{p}{)} \PYG{o}{==} \PYG{n}{Fraccion}\PYG{p}{(}\PYG{l+m+mi}{49}\PYG{p}{,} \PYG{l+m+mi}{22}\PYG{p}{)}

\PYG{c+c1}{\PYGZsh{} Probar la simplificacion al inicio}
\PYG{k}{assert} \PYG{n}{Fraccion}\PYG{p}{(}\PYG{l+m+mi}{8}\PYG{p}{,} \PYG{l+m+mi}{4}\PYG{p}{)} \PYG{o}{==} \PYG{n}{Fraccion}\PYG{p}{(}\PYG{l+m+mi}{2}\PYG{p}{,} \PYG{l+m+mi}{1}\PYG{p}{)}

\PYG{n}{f1} \PYG{o}{=} \PYG{n}{Fraccion}\PYG{p}{(}\PYG{l+m+mi}{2}\PYG{p}{,} \PYG{l+m+mi}{4}\PYG{p}{)}
\PYG{k}{assert} \PYG{n}{f1} \PYG{o}{==} \PYG{n}{Fraccion}\PYG{p}{(}\PYG{l+m+mi}{1}\PYG{p}{,} \PYG{l+m+mi}{2}\PYG{p}{)}

\PYG{n}{f1}\PYG{o}{.}\PYG{n}{numerador} \PYG{o}{=} \PYG{l+m+mi}{2}
\PYG{k}{assert} \PYG{n}{f1} \PYG{o}{==} \PYG{n}{Fraccion}\PYG{p}{(}\PYG{l+m+mi}{1}\PYG{p}{,} \PYG{l+m+mi}{1}\PYG{p}{)}\PYG{p}{,} \PYG{l+s+sa}{f}\PYG{l+s+s1}{\PYGZsq{}}\PYG{l+s+si}{\PYGZob{}}\PYG{n}{f1}\PYG{l+s+si}{\PYGZcb{}}\PYG{l+s+s1}{ no es igual a }\PYG{l+s+si}{\PYGZob{}}\PYG{n}{Fraccion}\PYG{p}{(}\PYG{l+m+mi}{1}\PYG{p}{,} \PYG{l+m+mi}{1}\PYG{p}{)}\PYG{l+s+si}{\PYGZcb{}}\PYG{l+s+s1}{\PYGZsq{}}

\PYG{n}{f1}\PYG{o}{.}\PYG{n}{denominador} \PYG{o}{=} \PYG{l+m+mi}{8}
\PYG{k}{assert} \PYG{n}{f1} \PYG{o}{==} \PYG{n}{Fraccion}\PYG{p}{(}\PYG{l+m+mi}{1}\PYG{p}{,} \PYG{l+m+mi}{8}\PYG{p}{)}

\PYG{n+nb}{print}\PYG{p}{(}\PYG{l+s+s1}{\PYGZsq{}}\PYG{l+s+s1}{TODO OK}\PYG{l+s+s1}{\PYGZsq{}}\PYG{p}{)}
\end{sphinxVerbatim}

\sphinxstepscope


\chapter{Paquetes y módulos}
\label{\detokenize{my-modules:paquetes-y-modulos}}\label{\detokenize{my-modules::doc}}
\sphinxAtStartPar
Hasta aquí hemos ejecutado nuestro código en un solo archivo.

\begin{sphinxadmonition}{note}{no es exactamente así}

\sphinxAtStartPar
En realidad cuando usamos algo como \sphinxcode{\sphinxupquote{from random import ranint}}
estamos usando (importando) código que esta en otros archivos que no vemos (pero
podríamos, \sphinxhref{https://github.com/python/cpython/blob/main/Lib/random.py}{aquí esta el modulo interno de python random.py}%
\begin{footnote}[30]\sphinxAtStartFootnote
\sphinxnolinkurl{https://github.com/python/cpython/blob/main/Lib/random.py}
%
\end{footnote}).
\end{sphinxadmonition}

\sphinxAtStartPar
En la medida que el código que hacemos crece, es necesario mantener un orden.
Es por esto que conviene empaquetar el código que hacemos. Esto incluso nos permite
reutilizarlo en el futuro.

\begin{sphinxadmonition}{note}{reutilizar y compartir}

\sphinxAtStartPar
Además de reutilizarlo nosotros lo podemos compartir
abiertamente. La comunidad de Python es una de las más grandes en el desarrollo de
software abierto. Al momento de escribir estas líneas, hay alrededor de 400.000 paquetes
abiertos en \sphinxhref{https://pypi.org/}{Pypi}%
\begin{footnote}[31]\sphinxAtStartFootnote
\sphinxnolinkurl{https://pypi.org/}
%
\end{footnote} (\sphinxstyleemphasis{The Python Package Index}). Todo este código
esta disponible para nosotros.
\end{sphinxadmonition}

\sphinxAtStartPar
Podemos pensar a los paquetes Python como carpetas que pueden contener más paquetes
(sub\sphinxhyphen{}carpetas) y modulos (archivos de Python \sphinxcode{\sphinxupquote{.py}}) con funciones y clases para reutilizar.

\sphinxAtStartPar
Para indicar que una carpeta es un paquete alcanza con agregarle un archivo llamado
\sphinxcode{\sphinxupquote{\_\_init\_\_.py}}. Por el momento alcanza con que este archivo este vacío.

\begin{figure}[htbp]
\centering
\capstart

\sphinxincludegraphics[]{graphviz-16754c4d8191cffa9801398f7d26d5cf82f3df74.pdf}
\caption{Estructura de archivos y carpetas}\label{\detokenize{my-modules:pkg-structure}}\end{figure}

\sphinxAtStartPar
De esta forma es posible mantener el codigo ordenado y facil de mantener a medida que crece.


\section{Tarea}
\label{\detokenize{my-modules:tarea}}\begin{itemize}
\item {} 
\sphinxAtStartPar
Crear un repositorio nuevo en GitHub, clonarlo localmente y agregar dos archivos:
\begin{itemize}
\item {} 
\sphinxAtStartPar
\sphinxcode{\sphinxupquote{auto.py}} donde vamos a definir nuestra clase \sphinxcode{\sphinxupquote{Auto}}

\item {} 
\sphinxAtStartPar
\sphinxcode{\sphinxupquote{programa.py}} donde vamos a crear objetos de tipo \sphinxcode{\sphinxupquote{Auto}} y usarlos para probar su funcionalidad.

\item {} 
\sphinxAtStartPar
Finalmente compartir el link del repositorio en el canal del curso.

\end{itemize}

\end{itemize}

\sphinxAtStartPar
Ejemplo

\noindent\sphinxincludegraphics{{class-module-test}.png}

\sphinxstepscope


\chapter{Paquetes externos}
\label{\detokenize{external-packages:paquetes-externos}}\label{\detokenize{external-packages::doc}}
\sphinxAtStartPar
La construcción de paquetes en Python representan una de las mejores prácticas para
distribuir código. Aislar correctamente la solución a un problema nos permite reutilizar
esa porción de código en otros proyectos, y nos permite además compartirlo con la comunidad.
\sphinxhref{https://pypi.org/}{Pypi}%
\begin{footnote}[32]\sphinxAtStartFootnote
\sphinxnolinkurl{https://pypi.org/}
%
\end{footnote} es la biblioteca más usada por la comunidad de Python para
redistribuir paquetes. Cientos de miles de paquetes estan disponibles allí para descarga.

\sphinxAtStartPar
Los paquetes externos son paquetes que no vienen por defecto con Python, pero que se
pueden instalar para ampliar las funcionalidades de Python.

\sphinxAtStartPar
Cada proyecto de software incluye en general un conjunto de paquetes a instalar del cual depende.
Para instalar un paquete externo, se usa la herramienta \sphinxcode{\sphinxupquote{pip}}. Esta herramienta puede usarse
desde la terminal.

\sphinxAtStartPar
Si estas usando Git Bash en Windows, es posible que ya tengas instalado \sphinxcode{\sphinxupquote{pip}}.
Para probar si ya lo tenes instalado poder ejecutar alguna de estas opciones.

\begin{sphinxVerbatim}[commandchars=\\\{\}]
pip \PYGZhy{}\PYGZhy{}version
\PYG{c+c1}{\PYGZsh{} o lo que es lo mismo}
pip \PYGZhy{}V
\end{sphinxVerbatim}

\sphinxAtStartPar
Si no recibis un mensaje de error y podes ver la versión de \sphinxcode{\sphinxupquote{pip}} podes saltearte la
siguiente sección


\section{Instalar \sphinxstyleliteralintitle{\sphinxupquote{pip}}}
\label{\detokenize{external-packages:instalar-pip}}

\subsection{Instalar \sphinxstyleliteralintitle{\sphinxupquote{pip}} en Windows}
\label{\detokenize{external-packages:instalar-pip-en-windows}}
\sphinxAtStartPar
Descargar el script de instalación \sphinxhref{https://bootstrap.pypa.io/get-pip.py}{get\sphinxhyphen{}pip.py}%
\begin{footnote}[33]\sphinxAtStartFootnote
\sphinxnolinkurl{https://bootstrap.pypa.io/get-pip.py}
%
\end{footnote} y
ejecutarlo con Python desde tu terminal.

\begin{sphinxVerbatim}[commandchars=\\\{\}]
python get\PYGZhy{}pip.py
\end{sphinxVerbatim}


\subsection{Instalar \sphinxstyleliteralintitle{\sphinxupquote{pip}} en Linux (Ubuntu)}
\label{\detokenize{external-packages:instalar-pip-en-linux-ubuntu}}
\begin{sphinxVerbatim}[commandchars=\\\{\}]
sudo apt install python3\PYGZhy{}pip
\end{sphinxVerbatim}


\subsection{Instalar paquetes con \sphinxstyleliteralintitle{\sphinxupquote{pip}}}
\label{\detokenize{external-packages:instalar-paquetes-con-pip}}
\sphinxAtStartPar
Una vez instalado pip ya podes instalar paquetes localmente.

\begin{sphinxVerbatim}[commandchars=\\\{\}]
pip install \PYGZlt{}nombre\PYGZus{}paquete\PYGZgt{}
\PYG{c+c1}{\PYGZsh{} opcionalmente se puede indicar la versión exacta a instalar}
pip install \PYGZlt{}nombre\PYGZus{}paquete\PYGZgt{}\PYG{o}{=}\PYG{o}{=}\PYGZlt{}version\PYGZgt{}
\PYG{c+c1}{\PYGZsh{} Instalar una lista de paquetes desde un archivo de texto con los requerimientos}
pip install \PYGZhy{}r requirements.txt
\end{sphinxVerbatim}


\chapter{Entornos virtuales}
\label{\detokenize{external-packages:entornos-virtuales}}
\sphinxAtStartPar
Un desarrollador de software en general trabaja sobre más de un proyecto, y por lo tanto
necesita instalar más de un conjunto de paquetes. Uno para cada proyecto.
Para evitar conflictos entre las versiones usadas en cada proyecto, se recomienda usar
entornos virtuales.
Cada entorno virtual se crea con una versión específica de Python, y permite instalar
un conjunto de paquetes específicos.
Cada proyeto activa y se ejecuta dentro de estos entornos.
Python permite crear y activar estos entornos virtuales con el módulo \sphinxcode{\sphinxupquote{venv}}.


\section{Crear un entorno virtual}
\label{\detokenize{external-packages:crear-un-entorno-virtual}}
\begin{sphinxVerbatim}[commandchars=\\\{\}]
python \PYGZhy{}m venv \PYGZlt{}carpeta\PYGZus{}donde\PYGZus{}se\PYGZus{}creara\PYGZus{}el\PYGZus{}entorno\PYGZgt{}
\end{sphinxVerbatim}


\section{Activar un entorno virtual}
\label{\detokenize{external-packages:activar-un-entorno-virtual}}
\sphinxAtStartPar
La forma de activación de los entornos virtuales depende del sistema operativo.
Para Windows el comando para activate un entorno virtual es

\begin{sphinxVerbatim}[commandchars=\\\{\}]
\PYGZlt{}carpeta\PYGZus{}donde\PYGZus{}se\PYGZus{}creara\PYGZus{}el\PYGZus{}entorno\PYGZgt{}\PYG{l+s+se}{\PYGZbs{}S}cripts\PYG{l+s+se}{\PYGZbs{}a}ctivate.bat
\end{sphinxVerbatim}

\sphinxAtStartPar
Para Linux el comando para activar un entorno es

\begin{sphinxVerbatim}[commandchars=\\\{\}]
\PYG{n+nb}{source} \PYGZlt{}carpeta\PYGZus{}donde\PYGZus{}se\PYGZus{}creara\PYGZus{}el\PYGZus{}entorno\PYGZgt{}/bin/activate
\end{sphinxVerbatim}

\sphinxAtStartPar
Te vas a dar cuenta que el entorno esta activado porque tu terminal va a agregar
el nombre de tu entorno entre parentesis en la linea de tu terminal.
Una vez activado el entorno, el comando \sphinxcode{\sphinxupquote{pip}} instalara los paquetes dentro de este.
Para desactivar el entorno virtual, ejecutar el comando \sphinxcode{\sphinxupquote{deactivate}} (el mismo para
ambos sistemas operativos).


\subsection{Tarea}
\label{\detokenize{external-packages:tarea}}\begin{itemize}
\item {} 
\sphinxAtStartPar
Clonar el repositorio \sphinxhref{https://github.com/avdata99/autos-justicia-cordoba-2022}{autos justicia 2022}%
\begin{footnote}[34]\sphinxAtStartFootnote
\sphinxnolinkurl{https://github.com/avdata99/autos-justicia-cordoba-2022}
%
\end{footnote}
\begin{itemize}
\item {} 
\sphinxAtStartPar
Crear un entorno local, activarlo e instalar \sphinxcode{\sphinxupquote{requirements.txt}}

\item {} 
\sphinxAtStartPar
Si el entorno va a ser una carpeta dentro de la carpeta del proyecto, agregar la
carpeta al archivo \sphinxcode{\sphinxupquote{.gitignore}}

\item {} 
\sphinxAtStartPar
Ejecutar el script \sphinxcode{\sphinxupquote{scrape.py}} y asegurarse de que funcione como se espera.

\item {} 
\sphinxAtStartPar
Analizar el código y proponer algún cambio mediante algún PR

\end{itemize}

\end{itemize}

\sphinxstepscope


\chapter{Paquete \sphinxstyleliteralintitle{\sphinxupquote{requests}}}
\label{\detokenize{requests:paquete-requests}}\label{\detokenize{requests::doc}}
\sphinxAtStartPar
El paquete \sphinxhref{https://pypi.org/project/requests/}{requests}%
\begin{footnote}[35]\sphinxAtStartFootnote
\sphinxnolinkurl{https://pypi.org/project/requests/}
%
\end{footnote} es uno muy usado.
Formalmente podemos decir que es un cliente HTTP. Con el podemos realizar
peticiones de recursos web de una forma muy sencilla.

\sphinxAtStartPar
Veamos este ejemplo:

\begin{sphinxVerbatim}[commandchars=\\\{\}]
\PYG{k+kn}{import} \PYG{n+nn}{requests}

\PYG{n}{url} \PYG{o}{=} \PYG{l+s+s1}{\PYGZsq{}}\PYG{l+s+s1}{https://mendiolaza.gob.ar/}\PYG{l+s+s1}{\PYGZsq{}}
\PYG{n}{response} \PYG{o}{=} \PYG{n}{requests}\PYG{o}{.}\PYG{n}{get}\PYG{p}{(}\PYG{n}{url}\PYG{p}{)}
\PYG{n}{codigo} \PYG{o}{=} \PYG{n}{response}\PYG{o}{.}\PYG{n}{text}

\PYG{n+nb}{print}\PYG{p}{(}\PYG{l+s+sa}{f}\PYG{l+s+s1}{\PYGZsq{}}\PYG{l+s+s1}{Codigo de estado de la respuesta para }\PYG{l+s+si}{\PYGZob{}}\PYG{n}{url}\PYG{l+s+si}{\PYGZcb{}}\PYG{l+s+s1}{: }\PYG{l+s+si}{\PYGZob{}}\PYG{n}{response}\PYG{o}{.}\PYG{n}{status\PYGZus{}code}\PYG{l+s+si}{\PYGZcb{}}\PYG{l+s+s1}{\PYGZsq{}}\PYG{p}{)}
\PYG{n+nb}{print}\PYG{p}{(}\PYG{l+s+sa}{f}\PYG{l+s+s1}{\PYGZsq{}}\PYG{l+s+s1}{Primeros caracteres de la respuesta }\PYG{l+s+si}{\PYGZob{}}\PYG{n}{codigo}\PYG{p}{[}\PYG{p}{:}\PYG{l+m+mi}{180}\PYG{p}{]}\PYG{l+s+si}{\PYGZcb{}}\PYG{l+s+s1}{\PYGZsq{}}\PYG{p}{)}

\PYG{c+c1}{\PYGZsh{} grabar el HTML resultante a disco}
\PYG{n}{f} \PYG{o}{=} \PYG{n+nb}{open}\PYG{p}{(}\PYG{l+s+s1}{\PYGZsq{}}\PYG{l+s+s1}{web.html}\PYG{l+s+s1}{\PYGZsq{}}\PYG{p}{,} \PYG{l+s+s1}{\PYGZsq{}}\PYG{l+s+s1}{w}\PYG{l+s+s1}{\PYGZsq{}}\PYG{p}{)}
\PYG{n}{f}\PYG{o}{.}\PYG{n}{write}\PYG{p}{(}\PYG{n}{codigo}\PYG{p}{)}
\PYG{n}{f}\PYG{o}{.}\PYG{n}{close}\PYG{p}{(}\PYG{p}{)}
\end{sphinxVerbatim}

\sphinxAtStartPar
Ese ejemplo muestra como descargar el contenido de una página web.
El resultado obtenido es el código HTML de la página.
Esto es similar a lo que hacen los navegadores de internet (Google
Chrome o Mozilla Fiefox por ejemplo) cuando accedemos a una página web.

\sphinxAtStartPar
Esto es muy útil para obtener información de una página web y procesarla
con Python. Existen tecnicas para extraer información de un documento HTML
y transformarlo en datos que podemos usar en nuestro programa.

\sphinxAtStartPar
A estas técnicas se les llama \sphinxhref{https://es.wikipedia.org/wiki/Web\_scraping}{Web scraping}%
\begin{footnote}[36]\sphinxAtStartFootnote
\sphinxnolinkurl{https://es.wikipedia.org/wiki/Web\_scraping}
%
\end{footnote}
y son muy usadas.

\sphinxAtStartPar
Pero no toda la web es HTML. Existen muchos otros formatos de datos que
ya está pensados para ser leidos directamente por computadoras.
El formato \sphinxcode{\sphinxupquote{JSON}} es uno de los más usados.

\sphinxAtStartPar
Veamos un ejemplo de un recurso web en formato \sphinxcode{\sphinxupquote{JSON}}:

\begin{sphinxVerbatim}[commandchars=\\\{\}]
\PYG{k+kn}{import} \PYG{n+nn}{requests}

\PYG{c+c1}{\PYGZsh{} JSON con los datos de un usuario de GitHub}
\PYG{n}{url} \PYG{o}{=} \PYG{l+s+s1}{\PYGZsq{}}\PYG{l+s+s1}{https://api.github.com/users/avdata99}\PYG{l+s+s1}{\PYGZsq{}}
\PYG{n}{response} \PYG{o}{=} \PYG{n}{requests}\PYG{o}{.}\PYG{n}{get}\PYG{p}{(}\PYG{n}{url}\PYG{p}{)}
\PYG{n}{data} \PYG{o}{=} \PYG{n}{response}\PYG{o}{.}\PYG{n}{json}\PYG{p}{(}\PYG{p}{)}
\PYG{n+nb}{print}\PYG{p}{(}\PYG{l+s+sa}{f}\PYG{l+s+s1}{\PYGZsq{}}\PYG{l+s+s1}{Datos: }\PYG{l+s+si}{\PYGZob{}}\PYG{n}{data}\PYG{l+s+si}{\PYGZcb{}}\PYG{l+s+s1}{\PYGZsq{}}\PYG{p}{)}

\PYG{c+c1}{\PYGZsh{} De que tipo es data?}
\PYG{n+nb}{print}\PYG{p}{(}\PYG{n+nb}{type}\PYG{p}{(}\PYG{n}{data}\PYG{p}{)}\PYG{p}{)}

\PYG{l+s+sd}{\PYGZdq{}\PYGZdq{}\PYGZdq{}}
\PYG{l+s+sd}{Ejemplo}

\PYG{l+s+sd}{\PYGZob{}}
\PYG{l+s+sd}{    \PYGZsq{}login\PYGZsq{}: \PYGZsq{}avdata99\PYGZsq{},}
\PYG{l+s+sd}{    \PYGZsq{}id\PYGZsq{}: 3237309,}
\PYG{l+s+sd}{    \PYGZsq{}node\PYGZus{}id\PYGZsq{}: \PYGZsq{}MDQ6VXNlcjMyMzczMDk=\PYGZsq{},}
\PYG{l+s+sd}{    \PYGZsq{}avatar\PYGZus{}url\PYGZsq{}: \PYGZsq{}https://avatars.githubusercontent.com/u/3237309?v=4\PYGZsq{},}
\PYG{l+s+sd}{    \PYGZsq{}gravatar\PYGZus{}id\PYGZsq{}: \PYGZsq{}\PYGZsq{},}
\PYG{l+s+sd}{    \PYGZsq{}url\PYGZsq{}: \PYGZsq{}https://api.github.com/users/avdata99\PYGZsq{},}
\PYG{l+s+sd}{    \PYGZsq{}html\PYGZus{}url\PYGZsq{}: \PYGZsq{}https://github.com/avdata99\PYGZsq{},}
\PYG{l+s+sd}{    \PYGZsq{}followers\PYGZus{}url\PYGZsq{}: \PYGZsq{}https://api.github.com/users/avdata99/followers\PYGZsq{},}
\PYG{l+s+sd}{    \PYGZsq{}following\PYGZus{}url\PYGZsq{}: \PYGZsq{}https://api.github.com/users/avdata99/following\PYGZob{}/other\PYGZus{}user\PYGZcb{}\PYGZsq{},}
\PYG{l+s+sd}{    \PYGZsq{}gists\PYGZus{}url\PYGZsq{}: \PYGZsq{}https://api.github.com/users/avdata99/gists\PYGZob{}/gist\PYGZus{}id\PYGZcb{}\PYGZsq{},}
\PYG{l+s+sd}{    \PYGZsq{}starred\PYGZus{}url\PYGZsq{}: \PYGZsq{}https://api.github.com/users/avdata99/starred\PYGZob{}/owner\PYGZcb{}\PYGZob{}/repo\PYGZcb{}\PYGZsq{},}
\PYG{l+s+sd}{    \PYGZsq{}subscriptions\PYGZus{}url\PYGZsq{}: \PYGZsq{}https://api.github.com/users/avdata99/subscriptions\PYGZsq{},}
\PYG{l+s+sd}{    \PYGZsq{}organizations\PYGZus{}url\PYGZsq{}: \PYGZsq{}https://api.github.com/users/avdata99/orgs\PYGZsq{},}
\PYG{l+s+sd}{    \PYGZsq{}repos\PYGZus{}url\PYGZsq{}: \PYGZsq{}https://api.github.com/users/avdata99/repos\PYGZsq{},}
\PYG{l+s+sd}{    \PYGZsq{}events\PYGZus{}url\PYGZsq{}: \PYGZsq{}https://api.github.com/users/avdata99/events\PYGZob{}/privacy\PYGZcb{}\PYGZsq{},}
\PYG{l+s+sd}{    \PYGZsq{}received\PYGZus{}events\PYGZus{}url\PYGZsq{}: \PYGZsq{}https://api.github.com/users/avdata99/received\PYGZus{}events\PYGZsq{},}
\PYG{l+s+sd}{    \PYGZsq{}type\PYGZsq{}: \PYGZsq{}User\PYGZsq{},}
\PYG{l+s+sd}{    \PYGZsq{}site\PYGZus{}admin\PYGZsq{}: False,}
\PYG{l+s+sd}{    \PYGZsq{}name\PYGZsq{}: \PYGZsq{}Andres Vazquez\PYGZsq{},}
\PYG{l+s+sd}{    \PYGZsq{}company\PYGZsq{}: None,}
\PYG{l+s+sd}{    \PYGZsq{}blog\PYGZsq{}: \PYGZsq{}http://andresvazquez.com.ar\PYGZsq{},}
\PYG{l+s+sd}{    \PYGZsq{}location\PYGZsq{}: \PYGZsq{}Mendiolaza, Cordoba, Argentina\PYGZsq{},}
\PYG{l+s+sd}{    \PYGZsq{}email\PYGZsq{}: None,}
\PYG{l+s+sd}{    \PYGZsq{}hireable\PYGZsq{}: None,}
\PYG{l+s+sd}{    \PYGZsq{}bio\PYGZsq{}: None,}
\PYG{l+s+sd}{    \PYGZsq{}twitter\PYGZus{}username\PYGZsq{}: None,}
\PYG{l+s+sd}{    \PYGZsq{}public\PYGZus{}repos\PYGZsq{}: 130,}
\PYG{l+s+sd}{    \PYGZsq{}public\PYGZus{}gists\PYGZsq{}: 18,}
\PYG{l+s+sd}{    \PYGZsq{}followers\PYGZsq{}: 66,}
\PYG{l+s+sd}{    \PYGZsq{}following\PYGZsq{}: 14,}
\PYG{l+s+sd}{    \PYGZsq{}created\PYGZus{}at\PYGZsq{}: \PYGZsq{}2013\PYGZhy{}01\PYGZhy{}10T17:45:49Z\PYGZsq{},}
\PYG{l+s+sd}{    \PYGZsq{}updated\PYGZus{}at\PYGZsq{}: \PYGZsq{}2022\PYGZhy{}01\PYGZhy{}08T22:57:42Z\PYGZsq{}}
\PYG{l+s+sd}{\PYGZcb{}}
\PYG{l+s+sd}{\PYGZdq{}\PYGZdq{}\PYGZdq{}}
\end{sphinxVerbatim}

\sphinxAtStartPar
En este caso podemos ver como el recurso JSON es transformado a diccionario
con la funcion \sphinxcode{\sphinxupquote{json()}} de la librería requests.

\sphinxAtStartPar
Veamos un ejemplo de un recurso web en formato \sphinxcode{\sphinxupquote{txt}}:

\begin{sphinxVerbatim}[commandchars=\\\{\}]
\PYG{l+s+sd}{\PYGZdq{}\PYGZdq{}\PYGZdq{}}
\PYG{l+s+sd}{Descargar un libro en formato texto y analizar }
\PYG{l+s+sd}{algunos detalles generales de su contenido}
\PYG{l+s+sd}{\PYGZdq{}\PYGZdq{}\PYGZdq{}}

\PYG{k+kn}{import} \PYG{n+nn}{requests}
\PYG{k+kn}{import} \PYG{n+nn}{os}

\PYG{n}{path\PYGZus{}libro} \PYG{o}{=} \PYG{l+s+s1}{\PYGZsq{}}\PYG{l+s+s1}{florante.txt}\PYG{l+s+s1}{\PYGZsq{}}
\PYG{k}{if} \PYG{o+ow}{not} \PYG{n}{os}\PYG{o}{.}\PYG{n}{path}\PYG{o}{.}\PYG{n}{exists}\PYG{p}{(}\PYG{n}{path\PYGZus{}libro}\PYG{p}{)}\PYG{p}{:}    
    \PYG{n+nb}{print}\PYG{p}{(}\PYG{l+s+s1}{\PYGZsq{}}\PYG{l+s+s1}{DESCARGANDO LIBRO}\PYG{l+s+s1}{\PYGZsq{}}\PYG{p}{)}
    \PYG{n}{url} \PYG{o}{=} \PYG{l+s+s1}{\PYGZsq{}}\PYG{l+s+s1}{http://www.gutenberg.org/cache/epub/15531/pg15531.txt}\PYG{l+s+s1}{\PYGZsq{}}
    \PYG{n}{req} \PYG{o}{=} \PYG{n}{requests}\PYG{o}{.}\PYG{n}{get}\PYG{p}{(}\PYG{n}{url}\PYG{p}{)}
    \PYG{n}{f} \PYG{o}{=} \PYG{n+nb}{open}\PYG{p}{(}\PYG{n}{path\PYGZus{}libro}\PYG{p}{,} \PYG{l+s+s1}{\PYGZsq{}}\PYG{l+s+s1}{w}\PYG{l+s+s1}{\PYGZsq{}}\PYG{p}{)}
    \PYG{n}{f}\PYG{o}{.}\PYG{n}{write}\PYG{p}{(}\PYG{n}{req}\PYG{o}{.}\PYG{n}{text}\PYG{p}{)}
    \PYG{n}{f}\PYG{o}{.}\PYG{n}{close}\PYG{p}{(}\PYG{p}{)}

\PYG{n}{f} \PYG{o}{=} \PYG{n+nb}{open}\PYG{p}{(}\PYG{n}{path\PYGZus{}libro}\PYG{p}{,} \PYG{l+s+s1}{\PYGZsq{}}\PYG{l+s+s1}{r}\PYG{l+s+s1}{\PYGZsq{}}\PYG{p}{)}
\PYG{n}{text} \PYG{o}{=} \PYG{n}{f}\PYG{o}{.}\PYG{n}{read}\PYG{p}{(}\PYG{p}{)}
\PYG{n}{f}\PYG{o}{.}\PYG{n}{close}\PYG{p}{(}\PYG{p}{)}
\PYG{n+nb}{print}\PYG{p}{(}\PYG{l+s+s1}{\PYGZsq{}}\PYG{l+s+s1}{Leyendo archivo local}\PYG{l+s+s1}{\PYGZsq{}}\PYG{p}{)}

\PYG{n+nb}{print}\PYG{p}{(}\PYG{l+s+s1}{\PYGZsq{}}\PYG{l+s+s1}{Largo del texto }\PYG{l+s+si}{\PYGZob{}\PYGZcb{}}\PYG{l+s+s1}{\PYGZsq{}}\PYG{o}{.}\PYG{n}{format}\PYG{p}{(}\PYG{n+nb}{len}\PYG{p}{(}\PYG{n}{text}\PYG{p}{)}\PYG{p}{)}\PYG{p}{)}
\PYG{n+nb}{print}\PYG{p}{(}\PYG{l+s+s1}{\PYGZsq{}}\PYG{l+s+s1}{Primeras 30 letras: }\PYG{l+s+si}{\PYGZob{}\PYGZcb{}}\PYG{l+s+s1}{\PYGZsq{}}\PYG{o}{.}\PYG{n}{format}\PYG{p}{(}\PYG{n}{text}\PYG{p}{[}\PYG{p}{:}\PYG{l+m+mi}{30}\PYG{p}{]}\PYG{p}{)}\PYG{p}{)}
\PYG{n+nb}{print}\PYG{p}{(}\PYG{l+s+s1}{\PYGZsq{}}\PYG{l+s+s1}{Ultimas 30 letras: }\PYG{l+s+si}{\PYGZob{}\PYGZcb{}}\PYG{l+s+s1}{\PYGZsq{}}\PYG{o}{.}\PYG{n}{format}\PYG{p}{(}\PYG{n}{text}\PYG{p}{[}\PYG{o}{\PYGZhy{}}\PYG{l+m+mi}{30}\PYG{p}{:}\PYG{p}{]}\PYG{p}{)}\PYG{p}{)}

\PYG{c+c1}{\PYGZsh{} analizar letras}
\PYG{n+nb}{print}\PYG{p}{(}\PYG{l+s+s1}{\PYGZsq{}}\PYG{l+s+s1}{\PYGZhy{}\PYGZhy{}\PYGZhy{}\PYGZhy{}\PYGZhy{}\PYGZhy{}\PYGZhy{}\PYGZhy{}\PYGZhy{}\PYGZhy{}\PYGZhy{}\PYGZhy{}\PYGZhy{}\PYGZhy{}\PYGZhy{}\PYGZhy{}\PYGZhy{}\PYGZhy{}\PYGZhy{}\PYGZhy{}\PYGZhy{}\PYGZhy{}\PYGZhy{}\PYGZhy{}\PYGZhy{}\PYGZhy{}\PYGZhy{}\PYGZhy{}\PYGZhy{}\PYGZhy{}\PYGZhy{}}\PYG{l+s+s1}{\PYGZsq{}}\PYG{p}{)}
\PYG{n+nb}{print}\PYG{p}{(}\PYG{l+s+s1}{\PYGZsq{}}\PYG{l+s+s1}{Letras mas usadas}\PYG{l+s+s1}{\PYGZsq{}}\PYG{p}{)}
\PYG{n+nb}{print}\PYG{p}{(}\PYG{l+s+s1}{\PYGZsq{}}\PYG{l+s+s1}{\PYGZhy{}\PYGZhy{}\PYGZhy{}\PYGZhy{}\PYGZhy{}\PYGZhy{}\PYGZhy{}\PYGZhy{}\PYGZhy{}\PYGZhy{}\PYGZhy{}\PYGZhy{}\PYGZhy{}\PYGZhy{}\PYGZhy{}\PYGZhy{}\PYGZhy{}\PYGZhy{}\PYGZhy{}\PYGZhy{}\PYGZhy{}\PYGZhy{}\PYGZhy{}\PYGZhy{}\PYGZhy{}\PYGZhy{}\PYGZhy{}\PYGZhy{}\PYGZhy{}\PYGZhy{}\PYGZhy{}}\PYG{l+s+s1}{\PYGZsq{}}\PYG{p}{)}
\PYG{n}{letras} \PYG{o}{=} \PYG{p}{\PYGZob{}}\PYG{p}{\PYGZcb{}}
\PYG{k}{for} \PYG{n}{letra} \PYG{o+ow}{in} \PYG{n}{text}\PYG{p}{:}
    \PYG{n}{letra} \PYG{o}{=} \PYG{n}{letra}\PYG{o}{.}\PYG{n}{lower}\PYG{p}{(}\PYG{p}{)}
    \PYG{k}{if} \PYG{n}{letra} \PYG{o+ow}{not} \PYG{o+ow}{in} \PYG{n}{letras}\PYG{o}{.}\PYG{n}{keys}\PYG{p}{(}\PYG{p}{)}\PYG{p}{:}
        \PYG{n}{letras}\PYG{p}{[}\PYG{n}{letra}\PYG{p}{]} \PYG{o}{=} \PYG{l+m+mi}{0}
    
    \PYG{n}{letras}\PYG{p}{[}\PYG{n}{letra}\PYG{p}{]} \PYG{o}{+}\PYG{o}{=} \PYG{l+m+mi}{1}

\PYG{n}{letras} \PYG{o}{=} \PYG{n+nb}{sorted}\PYG{p}{(}\PYG{n}{letras}\PYG{o}{.}\PYG{n}{items}\PYG{p}{(}\PYG{p}{)}\PYG{p}{,} \PYG{n}{key}\PYG{o}{=}\PYG{k}{lambda} \PYG{n}{x}\PYG{p}{:} \PYG{n}{x}\PYG{p}{[}\PYG{l+m+mi}{1}\PYG{p}{]}\PYG{p}{,} \PYG{n}{reverse}\PYG{o}{=}\PYG{k+kc}{True}\PYG{p}{)}

\PYG{n+nb}{print}\PYG{p}{(}\PYG{n}{letras}\PYG{p}{[}\PYG{p}{:}\PYG{l+m+mi}{10}\PYG{p}{]}\PYG{p}{)}


\PYG{c+c1}{\PYGZsh{} analizar palabras}
\PYG{n+nb}{print}\PYG{p}{(}\PYG{l+s+s1}{\PYGZsq{}}\PYG{l+s+s1}{\PYGZhy{}\PYGZhy{}\PYGZhy{}\PYGZhy{}\PYGZhy{}\PYGZhy{}\PYGZhy{}\PYGZhy{}\PYGZhy{}\PYGZhy{}\PYGZhy{}\PYGZhy{}\PYGZhy{}\PYGZhy{}\PYGZhy{}\PYGZhy{}\PYGZhy{}\PYGZhy{}\PYGZhy{}\PYGZhy{}\PYGZhy{}\PYGZhy{}\PYGZhy{}\PYGZhy{}\PYGZhy{}\PYGZhy{}\PYGZhy{}\PYGZhy{}\PYGZhy{}\PYGZhy{}\PYGZhy{}}\PYG{l+s+s1}{\PYGZsq{}}\PYG{p}{)}
\PYG{n+nb}{print}\PYG{p}{(}\PYG{l+s+s1}{\PYGZsq{}}\PYG{l+s+s1}{Palabras mas usadas}\PYG{l+s+s1}{\PYGZsq{}}\PYG{p}{)}
\PYG{n+nb}{print}\PYG{p}{(}\PYG{l+s+s1}{\PYGZsq{}}\PYG{l+s+s1}{\PYGZhy{}\PYGZhy{}\PYGZhy{}\PYGZhy{}\PYGZhy{}\PYGZhy{}\PYGZhy{}\PYGZhy{}\PYGZhy{}\PYGZhy{}\PYGZhy{}\PYGZhy{}\PYGZhy{}\PYGZhy{}\PYGZhy{}\PYGZhy{}\PYGZhy{}\PYGZhy{}\PYGZhy{}\PYGZhy{}\PYGZhy{}\PYGZhy{}\PYGZhy{}\PYGZhy{}\PYGZhy{}\PYGZhy{}\PYGZhy{}\PYGZhy{}\PYGZhy{}\PYGZhy{}\PYGZhy{}}\PYG{l+s+s1}{\PYGZsq{}}\PYG{p}{)}
\PYG{n}{palabras} \PYG{o}{=} \PYG{p}{\PYGZob{}}\PYG{p}{\PYGZcb{}}
\PYG{k}{for} \PYG{n}{palabra} \PYG{o+ow}{in} \PYG{n}{text}\PYG{o}{.}\PYG{n}{split}\PYG{p}{(}\PYG{l+s+s1}{\PYGZsq{}}\PYG{l+s+s1}{ }\PYG{l+s+s1}{\PYGZsq{}}\PYG{p}{)}\PYG{p}{:}
    \PYG{n}{palabra} \PYG{o}{=} \PYG{n}{palabra}\PYG{o}{.}\PYG{n}{lower}\PYG{p}{(}\PYG{p}{)}
    \PYG{k}{if} \PYG{n+nb}{len}\PYG{p}{(}\PYG{n}{palabra}\PYG{p}{)} \PYG{o}{\PYGZlt{}} \PYG{l+m+mi}{5}\PYG{p}{:}  \PYG{c+c1}{\PYGZsh{} omitir las palabras cortas}
        \PYG{k}{continue}
    \PYG{k}{if} \PYG{n}{palabra} \PYG{o+ow}{not} \PYG{o+ow}{in} \PYG{n}{palabras}\PYG{o}{.}\PYG{n}{keys}\PYG{p}{(}\PYG{p}{)}\PYG{p}{:}
        \PYG{n}{palabras}\PYG{p}{[}\PYG{n}{palabra}\PYG{p}{]} \PYG{o}{=} \PYG{l+m+mi}{0}
    
    \PYG{n}{palabras}\PYG{p}{[}\PYG{n}{palabra}\PYG{p}{]} \PYG{o}{+}\PYG{o}{=} \PYG{l+m+mi}{1}

\PYG{n}{letras} \PYG{o}{=} \PYG{n+nb}{sorted}\PYG{p}{(}\PYG{n}{palabras}\PYG{o}{.}\PYG{n}{items}\PYG{p}{(}\PYG{p}{)}\PYG{p}{,} \PYG{n}{key}\PYG{o}{=}\PYG{k}{lambda} \PYG{n}{x}\PYG{p}{:} \PYG{n}{x}\PYG{p}{[}\PYG{l+m+mi}{1}\PYG{p}{]}\PYG{p}{,} \PYG{n}{reverse}\PYG{o}{=}\PYG{k+kc}{True}\PYG{p}{)}

\PYG{n+nb}{print}\PYG{p}{(}\PYG{n}{letras}\PYG{p}{[}\PYG{p}{:}\PYG{l+m+mi}{20}\PYG{p}{]}\PYG{p}{)}
\end{sphinxVerbatim}

\sphinxAtStartPar
En este ejemplo nos descargamos un libro completo y analizamos las letras
y las palabras más usadas.


\section{Tarea}
\label{\detokenize{requests:tarea}}\begin{itemize}
\item {} 
\sphinxAtStartPar
Clonar el repositorio \sphinxhref{https://github.com/avdata99/autos-justicia-cordoba-2022}{autos justicia 2022}%
\begin{footnote}[37]\sphinxAtStartFootnote
\sphinxnolinkurl{https://github.com/avdata99/autos-justicia-cordoba-2022}
%
\end{footnote}
\begin{itemize}
\item {} 
\sphinxAtStartPar
Crear un entorno local, activarlo e instalar \sphinxcode{\sphinxupquote{requirements.txt}}

\item {} 
\sphinxAtStartPar
Si el entorno va a ser una carpeta dentro de la carpeta del proyecto, agregar la
carpeta al archivo \sphinxcode{\sphinxupquote{.gitignore}}

\item {} 
\sphinxAtStartPar
Ejecutar el script \sphinxcode{\sphinxupquote{scrape.py}} y asegurarse de que funcione como se espera.

\item {} 
\sphinxAtStartPar
Analizar el código y proponer algún cambio mediante algún PR

\end{itemize}

\end{itemize}


\chapter{Indices y tablas}
\label{\detokenize{index:indices-y-tablas}}\begin{itemize}
\item {} 
\sphinxAtStartPar
\DUrole{xref,std,std-ref}{genindex}

\item {} 
\sphinxAtStartPar
\DUrole{xref,std,std-ref}{modindex}

\item {} 
\sphinxAtStartPar
\DUrole{xref,std,std-ref}{search}

\end{itemize}



\renewcommand{\indexname}{Índice}
\printindex
\end{document}